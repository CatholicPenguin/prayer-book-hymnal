\fancyhead[C]{{\LARGE Holy Mass}}
\fancyhead[RO,LE]{}
\fancyhead[RE,LO]{}
%\fancyhead[RO,LE]{\textit{Liturgy of St. Tikhon}}
%\section{Liturgy of St. Tikhon}\label{tikhon}
\phantomsection
\addcontentsline{toc}{section}{Liturgy of St. Tikhon}

\subby{Collect for Purity}
\begin{rubric}
    The Priest, at the Foot of the Altar, saith the Collect for Purity, as followeth.
\end{rubric}
\lett{O}{mn\smash{í}potens} Deus, cui omne cor patet, et cui omnes affectus animorum cogniti sunt, et quem nihil latet, purifica cogitationes cordium nostrorum, ut per inspirationem Sancti Spiritus te ex animo amemus, et debita veneratione celebramus Nomen tuum sanctum. Per Jesum Christum Dominum nostrum. \textit{Amen.}
\begin{rubric}
Then shall the Priest, turning to the People, rehearse distinctly The Ten Commandments; and the People, still kneeling, shall, after every Commandment, ask God mercy for their transgressions for the time past, and grace to keep the law for the time to come.
\end{rubric}
\begin{rubric}
And \textsc{Note,} that in rehearsing the Ten Commandments, the Priest may omit that part of the Commandment which is inset.
\end{rubric}
\begin{rubric}
	The Decalogue may be omitted. But \textsc{Note,}, That when ever it is omitted, the Minister shall say the Summary of the Law, beginning, \emph{Audite quod Dominus noster Jesus Christus dicit.}
\end{rubric}
%MANUAL ADJUSTMENT:
\vspace{-1ex}
\subby{Ten Commandments}
\begin{center}
	\textsc{Deus h{\ae}c verba ad hunc modum effatus est.:}
\end{center}
%\begin{multicols}{2}
\par\noindent
    Ego sum Dominus Deus tuus. Deos nullos alios habebis pr{\ae}ter me.\par
    \textit{Dómine miserére nostri, et dírige corda nostra ad servándam hanc legem.}
    \par\noindent
    Non facies tibi sculptile, neque ullam similitudinem ullius rei quæ est supra in cœlo, aut infra in terra, aut in aquis sub terra : non adorabis ea nec coles. 
    \par\noindent
    \leftskip=2em
	{\small{Ego enim Deus tuus fortis zelotes sum, visitans iniquitates patrum in filios, in tertiam et quartam generationem eorum qui oderunt me, et faciens misericordiam in millia, his qui diligunt et custodiunt præcepta mea.}}
	\par
	\leftskip=0cm
    \textit{Dómine miserére nostri, et dírige corda nostra ad servándam hanc legem.}
    \par\noindent
    Non assumes nomen Domini Dei tui in vanum;
    \par\noindent
    \leftskip=2em
	{\small{non enim habebit insontem Dominus eum, qui assumpserit nomen Domini Dei frustra.}}
	\par
	\leftskip=0em
    \textit{Dómine miserére nostri, et dírige corda nostra ad servándam hanc legem.}
    \par\noindent
    Memento ut diem Sabbati sanctifices.
    \par\noindent
    \leftskip=2em
	{\small{Sex diebus operaberis, et facies omnia opera tua, septimo autem die Sabbatum Domini Dei tui est : nullun in eo facies opus, tu et filius tuus et filia tua, servus tuus et ancilla tua, jumentum tuum, et advena qui est intra portas tuas. Sex enim diebus fecit Dominus cœlum et terram et mare, et omnia quæ in eis sunt, et requievit die septimo. Idcirco benedixit Dominus Dominus diei Sabbati, et sanctificavit eum.}}
	\par
	\leftskip=0em
    \textit{Dómine miserére nostri, et dírige corda nostra ad servándam hanc legem.}
    \par\noindent
    Honora patrem tuum et matrem tuam;
    \par\noindent
    \leftskip=2em
	{\small{ut sis longævus super terram, quam Dominus Deus tuus dabit tibi.}}
	\par
	\leftskip=0em
    \textit{Dómine miserére nostri, et dírige corda nostra ad servándam hanc legem.}
    \par\noindent
    Non occides.\par
    \textit{Dómine miserére nostri, et dírige corda nostra ad servándam hanc legem.}
    \par\noindent
    Non committes adulterium.\par
    \textit{Dómine miserére nostri, et dírige corda nostra ad servándam hanc legem.}
\par\noindent
    Non furtum facies.\par
    \textit{Dómine miserére nostri, et dírige corda nostra ad servándam hanc legem.}
\par\noindent
    Non loqueris contra proximum tuum falsum testimonium.\par
    \textit{Dómine miserére nostri, et dírige corda nostra ad servándam hanc legem.}
    \par\noindent
    Non concupisces.
    \par\noindent
    \leftskip=2em
	{\small{domum proximi tui, nec desiderabis uxorem ejus, non servum, non ancillam, non bovem, non asinus, nihil denique quod sit alterius.}}
	\par
	\leftskip=0em
	\textit{Dómine miserére nostri, et qu{\ae}sumus has omnes leges in córdibus nostris inscríbas.}
	
%	\end{multicols}
\begin{rubric}
	Then may the Priest say the Summary of the Law, as followeth.
\end{rubric}
\subby{Summary of the Law}
\begin{center}
%1979 Latin:
	{\textsc{Audite quod Dominus noster Jesus Christus dicit:}}
\end{center}
%Vulgate:
\lett{D}{iliges} Dominum Deum tuum ex toto corde tuo, et in tota anima tua, et in tota mente tua. Hoc est maximum, et primum mandatum. Secundum autem simile est huic: Diliges proximum tuum, sicut teipsum. In his duobus mandatis universa lex pendet, et prophetae. 
\begin{rubric}
    Then shall the Priest turn towards the Altar and ascend, saying secretly the \emph{Take away}.
\end{rubric}
\begin{rubric}
    Then, with hands joined upon the Altar, the Priest saith, bowing (kissing the Altar at the middle), the \emph{We pray thee}.
\end{rubric}
\begin{rubric}
    At a solemn Mass, the Celebrant, before reading the Introit, shall bless the incense with the \emph{Be thou blessed}.
\end{rubric}
\begin{rubric}
	Receiving the thurible from the Deacon, he censeth the Altar, saying nothing. Then the Deacon taketh the thurible from the Celebrant and censeth him only.
\end{rubric}
\begin{rubric}
%Added clarity that the Priest should be in the middle of the Altar for the Kyrie.
    Then shall the Priest move to the Epistle side and, signing himself with the sign of the Cross, begin the Introit: which finished, at the middle of the Altar, with joined hands, he saith alternately with the Ministers the \emph{Kyrie},
\end{rubric}
\subby{Kyrie}
\begin{multicols}{3}
	℣. Kyrie, eléison.

℟. Kyrie, eléison.

℣. Kyrie, eléison.

\columnbreak
℟. Christe, eléison.

℣. Christe, eléison.

℟. Christe, eléison.

\columnbreak
℣. Kyrie, eléison.

℟. Kyrie, eléison.

℣. Kyrie, eléison.
\end{multicols}


\subby{Gloria in Excelsis}
\begin{rubric}
    Then the Priest shall, at the middle of the Altar, extend and join his hands and---bowing his head slightly---say, if it is to be said,
\end{rubric}
\lett{G}{l\smash{ó}ria} in excélsis Deo. Et in terra pax homínibus bon{\ae} voluntátis. Laudámus te. Benedícimus te. Adorámus te.\margrub{Bow head.} Glorificámus te. Grátias ágimus tibi\margrub{Bow head.} propter magnam glóriam tuam. Dómine Deus, Rex c{\ae}léstis, Deus Pater omnípotens.\par
    Dómine Fili unigénite, Jesu Christe. Dómine Deus, Agnus Dei, Fílius Patris. Qui tollis peccáta mundi, miserére nobis. Qui tollis peccáta mundi, súscipe deprecatiónem nostram.\margrub{Bow head.} Qui sedes ad déxteram Patris, miserére nobis.\par
    Quóniam tu solus Sanctus. Tu solus Dóminus. Tu solus Altíssimus, Jesu Christe. Cum Sancto Spíritu {\ding{64}} in glória Dei Patris. Amen.

\begin{rubric}
    Then shall the Priest kiss the Altar in the middle, turn to the People, extend his hands, and say,
\end{rubric}
℣. Dóminus vobíscum.

℟. Et cum spíritu tuo.

℣. Orémus.
\subby{Collect}
\begin{rubric}
	Then shall the Priest move to the Epistle corner of the Altar and pray the Collect(s) of the Day.
\end{rubric}
\subby{Epistle}
\begin{rubric}
%CHECK:
	The Epistle for the Day shall then be read by the Subdeacon, first saying, \emph{Epistola scripta est in Capitulo \emph{--} et incipit ad Versum \emph{--}.}
\end{rubric}
\begin{rubric}
	The Epistle ended, he shall say, \emph{Hic explicit Epistola}, the people responding, \emph{Deo gratias}.
\end{rubric}
\begin{rubric}
    The Gradual and Alleluia (or Tract) and (if provided) Sequence is here chanted by the Choir.
\end{rubric}
\subby{Gospel}
\begin{multicols}{2}
\begin{rubric}
    These being ended, if it be a Solemn Mass, the Deacon shall then place the book of the Gospels on the middle of the Altar, and the Celebrant shall bless the incense with the \emph{Be thou blessed}, as above.
\end{rubric}
\begin{rubric}
	Then shall the Deacon, kneeling before the Altar, say with joined hands the \emph{Cleanse my heart} continuing to the \emph{Bid, sir}, with the priest saying \emph{The Lord be in thy}.
\end{rubric}
\begin{rubric}
    Having received the blessing, the Deacon kisseth the hand of the Celebrant. And going with the other Ministers, with the incense and the lights, to the place of the Gospel, he standeth with joined hands, saying what is below.
\end{rubric}
\begin{rubric}
    If, however, the Priest celebrate without Deacon and Subdeacon, when the book has been carried to the other corner of the Altar, he boweth in the middle, and with joined hands, saith the \emph{Cleanse my heart} and the \emph{Bid, Lord} and \emph{The Lord be in my}.
\end{rubric}
\end{multicols}
\begin{rubric}
	Then, all the People standing, the Priest, turning to the Book, shall say with joined hands,
\end{rubric}

℣. Dóminus vobíscum.

℟. Et cum spíritu tuo.

℣. Inítium (vel, Sequéntia) {\ding{66}} sancti Evangélii secúndum \textit{N.}

℟. Glória tibi, Dómine.
\begin{rubric}
    Then shall the Priest sign the book with the thumb of his right hand at the beginning of the Gospel text which he is to read, then himself on the forehead, the mouth, and the breast: and while the Ministers respond, he censeth the book thrice, then readeth the Gospel with joined hands.
\end{rubric}
\begin{rubric}
    At the end of the Gospel, the Ministers respond,
\end{rubric}

℟. Laus tibi, Christe.
\begin{rubric}
    Then shall the Subdeacon carry the book to the Priest, who kisseth the Gospel text, saying: \emph{Through the words of the Gospel may our sins be blotted out.} Then the Priest is censed by the Deacon.\par
    \textsc{Note,} In Masses of the Dead, \emph{Cleanse} is said, but a blessing is not asked, lights are not carried, and the Celebrant doth not kiss the book.
\end{rubric}
\begin{rubric}
    Then, in the middle of the Altar, extending, raising, and joining his hands, the Priest shall say, if it is to be said, \emph{I believe in one God}, proceeding with joined hands.
\end{rubric}
\subby{Nicene Creed}\label{NiceneCreed}
\lett{C}{redo} in unum Deum,\margrub{Bow head to Cross.} Patrem omnipoténtem, factórem c{\ae}li et terr{\ae}, visibílium ómnium et invisibílium.\par
    Et in unum Dóminum Jesum Christum,\margrub{Bow head to Cross.} Fílium Dei unigénitum. Et ex Patre natum ante ómnia sǽcula. Deum de Deo, lumen de lúmine, Deum verum de Deo vero. Génitum, non factum, consubstantiálem Patri: per quem ómnia facta sunt. Qui propter nos hómines et propter nostram salútem descéndit de c{\ae}lis. \inrub{Everyone genuflects.} Et incarnátus est de Spíritu Sancto ex María Vírgine: Et homo factus est. \inrub{Everyone rises.} Crucifíxus étiam pro nobis: sub Póntio Piláto passus, et sepúltus est. Et resurréxit tértia die, secúndum Scriptúras. Et ascéndit in c{\ae}lum: sedet ad déxteram Patris. Et íterum ventúrus est cum glória judicáre vivos et mórtuos: cujus regni non erit finis.\par
    Et in Spíritum Sanctum, Dóminum et vivificántem: qui ex Patre procédit. Qui cum Patre et Fílio simul adorátur\margrub{Bow head to Cross.} et conglorificátur: qui locútus est per Prophétas. Et unam sanctam cathólicam et apostólicam Ecclésiam. Confíteor unum baptísma in remissiónem peccatórum. Et exspécto resurrectiónem mortuórum. {\ding{64}} Et vitam ventúri s{\ae}culi. Amen.
\begin{rubric}
Then shall be declared unto the People what Holy-days, or Fasting-days, are in the week following to be observed; and (if occasion be) shall Notice be given of the Communion, and of the Banns of Matrimony, and other matters to be published.
\end{rubric}

\begin{rubric}
Here, or immediately after the Creed, may be said the Bidding Prayer, or other authorized Prayers and intercessions.
\end{rubric}

\begin{rubric}
	Then followeth the Sermon.
\end{rubric}

\begin{rubric}
    Then shall the Priest ascend the Altar, kiss it, turn towards the People, and say,
\end{rubric}

℣. Dóminus vobíscum.

℟. Et cum spíritu tuo.

℣. Orémus.
\subsec{Offertory}
\begin{multicols}{2}    
\begin{rubric}
    The Priest then saith the Offertory Verse.\par
    \textsc{Note,} At High Mass, a hymn may be sung while the Priest prepareth the Oblations in the Offertory, and while the Collection is taken if there be one.
\end{rubric}
\begin{rubric}
    Then shall the Deacon present the Paten with the Bread to the Celebrant, which he then offereth, saying: \emph{Receive, O holy Father}.
\end{rubric}

\begin{rubric} 
    Then, making a cross with the same Paten, he shall place the Bread upon the Corporal (or the Paten with the Bread upon the Corporal).
\end{rubric}
\begin{rubric}
	Then the Deacon shall minister the Wine, and the Subdeacon the water in the Cup. The Priest (blessing with the sign of the Cross the water to be mixed in the Cup, unless it be a Requiem Mass) shall say: \emph{O God, who didst wonderfully}.
\end{rubric}
\begin{rubric}
    Then shall the Priest receive the Cup and offer it, saying, \emph{We offer unto thee}.
\end{rubric}
\begin{rubric}
    Then shall he make the sign of the Cross with the Cup, place it upon the Corporal, and cover it with the Pall. Then, with hands joined upon the Altar, he saith, bowing slightly, \emph{In a humble spirit}.
\end{rubric}
\begin{rubric}
    Then shall the Priest---standing upright---extend his hands, raise them, and join them; and lifting his eyes to heaven and lowering them immediately, he saith, \emph{Come, O Sanctifier}.
\end{rubric}
\begin{rubric}
    If he be celebrating solemnly, the Priest blesseth incense, saying, \emph{Through the intercession}.
\end{rubric}
\begin{rubric}
    Receiving the Thurible from the Deacon, he censeth the Oblations, in the manner prescribed in the General Rubrics, saying: \emph{May this incense}.\par
    Then he censeth the Altar, saying: \emph{Let my prayer}.\par
    While he returneth the Thurible to the Deacon, he saith, \emph{The Lord kindle}.\par
    Then the Priest is censed by the Deacon, and afterwards the others in order.
\end{rubric}
\begin{rubric}
    Meanwhile, the Priest washeth his hands saying, \emph{I will wash}.\par
    \textsc{Note,} In Requiem Masses, and during Passiontide in Masses of the Season, \emph{Glory be} is omitted.
\end{rubric}
\begin{rubric}
    Then bowing slightly, in the middle of the Altar with hands joined upon it, he saith, \emph{Receive, O holy Trinity}.
\end{rubric}
\end{multicols}
\begin{rubric}
    Then shall the Priest kiss the Altar and, turning to the People, extend and join his hands, and say, raising his voice a little,
\end{rubric}

℣. Oráte, fratres: ut meum ac vestrum sacrifícium acceptábile fiat apud Deum Patrem omnipoténtem.

℟. Suscípiat Dóminus sacrifícium de mánibus tuis ad laudem et glóriam nominis sui, ad utilitátem quoque nostram, totiúsque Ecclési{\ae} su{\ae} sanct{\ae}.
\begin{rubric}
    The Priest saith in a low voice: \emph{Amen.}
\end{rubric}
\begin{rubric}
	Then shall the Priest, with hands extended, immediately (without \emph{Let us pray}), add the Secret Prayers.\par
	\textsc{Note,} When these are ended, he saith in a loud voice,
\end{rubric}
℣. Throughout all ages, world without end.\par
℟. Amen.
\begin{rubric}
    Then shall the Deacon say,
\end{rubric}
\centerline{Oremus pro statu universalis Ecclesi{\ae} Christi.}
\begin{rubric}
    The Priest then saith, with hands extended, the following.
\end{rubric}
\lett{O}{mn\smash{í}potens} sempiterne Deus, qui per sanctum Apostolum tuum nos docuisti facere orationes, obsecrationes, et gratiarum actiones pro omnibus hominibus: Supplices te rogamus ut clementer (eleemosynas atque) oblationes nostras accipias, et has preces nostras exaudias, quas offerimus Divinæ Majestati tuæ: Supplicantes ut veritatis, unitatis, et concordiæ spiritum Catholicæ Ecclesiæ tuæ perpetuo inspires: Et præsta ut omnes qui sanctum Nomen tuum confitentur, in sancti verbi tui veritate consentiant, et in unitate et pia charitate concordes vivant.

\needspace{4\baselineskip}
\lett{I}{nsuper} te rogamus ut corda omnium Christianorum Dominatorum domineris et disponas, ut recte ac sine personarum acceptione jus dicant, quo scelera et nequitia corrigantur, et vera tua religio, virtusque, stabiliantur.
\needspace{4\baselineskip}
\lett{D}{a} gratiam, Pater c{\oe}lestis, omnibus Episcopis et Parochis, præcipue famulum tuum \textit{N.} Metropolitano nostro (et \textit{N.} Episcopo nostro), ut tam vita quam doctrina sua verum vivumque verbum tuum annuntient, et sancta tua Sacramenta recte et rite ministrent.
\needspace{4\baselineskip}
\lett{E}{t} universo populo tuo tribue gratiam tuam; et præcipue huic congregationi præsenti; ut humili animo et debita reverentia audiant et accipiant sanctum verbum tuum: et tibi fideliter serviant in sanctitate et justitia omnibus diebus vitæ suæ.
\needspace{4\baselineskip}
\lett{S}{upplices} etiam te rogamus, Domine, ut pro bonitate tua eos omnes consoleris et adjuves, qui in hac temporali vita tribulatione, mœstitia, inopia, morbo, aliisve rebus adversis laborant.
\needspace{4\baselineskip}
\lett{B}{enedicimus} quoque sancto Nomini tuo propter omnes famulos tuos in fide et timore tuo defunctos; %1979 Latin:
te rogantes ut dones eisdem incrementum perpetuum in caritate ministerioque tuo; et gratiam nobis concedas qua, bona eorum exempla secuti, præcipue Beátæ Maríæ semper Vírginis, Genetrícis Jesu Christi, Fílii tui, Dómini et Dei nostri, etiam Sanctorum Patriárchum, Prophetarum, Apostolorum, Martyrumque, nos una cum illis cœlestis regni tui fiamus participes.
\begin{rubric}
	 The Priest placeth both hands upon the Altar, outside the Corporal, and continueth,
\end{rubric}
\needspace{4\baselineskip}
\lett{H}{oc,} Pater, largiri digneris, propter Jesum Christum unicum nostrum Mediatorem atque Advocatum.
\secondline{℟. Amen.}
%This is a difficult situation to discern. The Prayer Book Traditions reflects the uncertainty as to where to place the Confession. The current American placement is reflective of Cranmer's Protestant theology and liturgical sense, since the origins come from placing the Confession close to the reception of Communion (which was immediately after the Words of Institution). The 1928 reflects a more catholic understanding of the liturgy, including the epiklesis and the other integral parts of the Eucharistic Sacrifice before the Reception of Holy Communion. This understanding was more present in the 1549 which placed the Confession in a congruous place as the Roman tradition: after the Sacrifice and before the Communion of the Faithful. While this is obscured here due to the Byzantine devotion imposed, this is only a temporary imposition and the overall shape of the liturgy (due to the Orthodox theology present here) is better preserved in its Catholic and even Prayer Book form by placing the Confession immediately before the `Ecce Agnus Dei', which is where the Second Confiteor would go in the Roman Rite.


\subsec{Sursum Corda}
\begin{rubric}
    Then shall the Priest begin the Preface, with both hands placed apart on the Altar.
\end{rubric}
℣. Dóminus vobíscum.

℟. Et cum spíritu tuo.

℣. Sursum corda.\margrub{He raises his hands a little.}

℟. Habémus ad Dóminum.

℣. Grátias agámus Dómino, Deo nostro.\margrub{He joins his hands before his breast, \& bows his head.}

℟. Dignum et justum est.\margrub{He separates his hands.}
\subby{Preface}
\lett{I}{t} is very meet, right, and our bounden duty, that we should at all times, and in all places, give thanks unto thee, O Lord, Holy Father, Almighty, Everlasting God.
\begin{rubric}
    The proper preface is then here said (p. \pageref{prefaces})---if there be one---concluding with the following.
\end{rubric}
\label{PrefaceEnding}
\lett{T}{herefore} with Angels and Archangels, and with all the company of heaven, we laud and magnify thy glorious Name; evermore praising thee, and saying:\margrub{He joins his hands \& bows.}%\textsuperscript{\alph{margcount}}

%\hspace{0.98\textwidth}
%\rlap{\parbox{\marginparwidth}{\raggedright \itshape\scriptsize\textsuperscript{\alph{margcount}}%
%He joins his hands \& bows.}} % Use \parbox for multiline text

%\stepcounter{margcount}
%\stepcounter{latinrubric}

\subby{Sanctus}
\lett{S}{anctus, Sanctus, Sanctus,} Dóminus, Deus Sábaoth. Pleni sunt c{\ae}li et terra glória tua. Hosánna in excélsis. Benedíctus,\margrub{He stands upright \& signs himself.} {\ding{64}} qui venit in nómine Dómini. Hosánna in excélsis.
\clearpage
\checkoddpage
\ifoddpage \thispagestyle{empty}
~\clearpage\fi
\thispagestyle{empty}
\includegraphics[width=98mm, height=135mm]{Canon.eps}
\setcounter{margcount}{1}
\clearpage
%\reversemarginpar
\subby{Canon of the Mass}
\begin{rubric}
    The Priest, extending, raising somewhat, and joining his hands, raising his eyes towards heaven, and immediately lowering them, shall begin the Canon, as followeth.
\end{rubric}
%1979 Latin:
Omnis gloria tibi Deus omnipotens, Pater noster cœlestis, qui pro misericordiæ tuæ pietate unicum Filium tuum Jesum Christum dedisti, ut mortem in Cruce pro nostra redemptione pateretur; qui ibi (propria sui ipsius oblatione semel facta) plenum, perfectum, et sufficiens sacrificium, oblationem, et satisfactionem pro totius mundi peccatis fecit; et instituit, et in sancto Evangelio suo nobis præcepit observare, pretiosæ mortis sacrificique illius memoriam, usque dum rediret, perpetuam:
\subbysub{Words of Institution}\noindent
In enim qua nocte tradebatur,\margrub{He takes \& holds the Bread.} accepit Panem;\margrub{He looks up to heaven \& bows his head.} Et tibi gratias {\ding{64}} agens, fregit, deditque discipulis suis, dicens, Accipite, et manducate.\margrub{He holds the Bread between both of his thumbs \& forefingers.}
\begin{center}
\large{Hoc est enim Corpus meum, quod pro vobis datur: Hoc facite in meam commemorationem.}
\end{center}
\begin{rubric}
    Then shall the Priest immediately genuflect, briefly elevate the Bread for the People to see, replace it upon the corporal (or Paten), genuflect again, and then immediately continue,\par
    \textsc{Note,} From henceforth, the Priest doth not disjoin his forefingers \& thumbs until the ablutions.
\end{rubric}\par\noindent
Simili modo posteaquam cœnatum est,\margrub{The Priest uncovers the Cup, holds it with both hands, \& bows his head.} accipiens Calicem;\margrub{He holds the Cup in his left hand.} item tibi gratias {\ding{64}} agens, dedit illis, dicens, Bibite ex eo omnes;\margrub{He holds the Cup slightly raised.}
\begin{center}
\large{Hic est enim Sanguis meus Novi Testamenti, qui pro vobis et pro multis effunditur in remissionem peccatorum:}
\end{center}
\begin{rubric}
    Then shall the Priest place the Cup back upon the corporal and immediately say,
\end{rubric}\par\noindent
    \centerline{Hoc facite, quotiescumque bibetis, in meam commemorationem.}
\begin{rubric}
    Then shall the Priest immediately genuflect, briefly elevate the Cup, replace the Cup upon the corporal, cover it, genuflect again, and continue with hands extended.
\end{rubric}
\subbysub{Recollection}
%1979 Latin:
\lett{U}{nde,} Domine caelestis Pater, secundum institutionem dilectissimi Filii tui Salvatoris nostri Iesu Christi, nos humiles servi tui celebramus et hic facimus coram divina Maiestate tua, de his tuis sanctis {\ding{64}} donis, quae nunc tibi offerimus, illam quam Filius tuus celebrare nos praecepit memoriam; memores et eius tam beatae passionis et pretiosae mortis, necnon et mirabilis resurrectionis, sed et gloriosae ascensionis; tibi gratias agentes ex animo propter innumerabilia beneficia nobis inde collata.
\subbysub{Invocation}
\begin{rubric}
	Then shall the Priest uncover the Cup, bow profoundly, join his hands upon the Altar, and continue,
\end{rubric}
%Epiklesis of St. James through the 1929 Scottish BCP:
%Crosses supplied from the Anglican Missal.
%Moving the Pall from precedent in the LAP Missal:
\lett{E}{xaudi} nos, indignos servos tuos, quaesumus, misericors Pater,\margrub{He stands upright and imposes his hands over the Gifts.} et mitte tuum Sanctum Spiritum super nobis atque super his don {\ding{64}} is et creaturis tuis, pan {\ding{64}} is et vi {\ding{64}} ni, ut, benedictae et sanctificatae virtute vitali ejus, fiant Cor {\ding{64}} pus et Sang {\ding{64}} uis dilecti Filii tui, ut omnis qui percipit eundem sanctificetur in corpore animaque, et custodiatur in vitam aeternam. 

\begin{rubric}
	Then shall the Priest cover the Cup, genuflect, and continue, with hands extended,
\end{rubric}

\par
\needspace{4\baselineskip}
\lett{E}{t} rogamus supplices paternam tuam bonitatem, ut hoc nostrum laudis et gratiarum sacrificium benignus accipias: humillime supplicantes, ut propter merita et mortem Filii tui Jesu Christi, et per fidem in sanguine ipsius, et nos et Universa Ecclesia tua peccatorum remissionem et cætera omnia passionis ejus beneficia consequamur.
\subbysub{Commemoration for the Departed}
\lett{M}{em\smash{é}nto} étiam, Dómine, famulórum famularúmque tuárum \emph{N.} et \emph{N.}, qui nos præcessérunt cum signo fídei, et dórmiunt in somno pacis.
\begin{rubric}
    Then shall the Priest join his hands, pray for the dead he hath in mind, then extend his hands and say,
\end{rubric}\par\noindent
Ipsis, Dómine, et ómnibus in Christo quiescéntibus locum refrigérii, lucis, et pacis, ut indúlgeas, deprecámur.
%The English & American Missals lacks this addition, being only in the Anglican Missal tradition:
%\lett{A}{nd} vouchsafe to grant some part and fellowship with thy holy Apostles and Martyrs: with John, Stephen, Matthias, Barnabas, Ignatius, Alexander, Marcellinus, Peter, Felicitas, Perpetua, Agatha, Lucy, Agnes, Cecilia, Anastasia, and with all thy Saints: within whose fellowship we beseech thee admit us.
%\lett{R}{emember} Lord, also the souls of thy servants and handmaidens, which are gone before us with the mark of faith, and rest in the sleep of peace. \inrub{The Priest now prays for the souls of the dead.} We beseech thee, O Lord, that unto them, and unto all such as rest in Christ, thou wilt grant a place of refreshing, of light, and of peace. And vouchsafe to give unto us some portion and fellowship with thy holy Apostles and Martyrs; with John, Stephen, Matthias, Barnabas, Ignatius, Alexander, Marcellinus, Peter, Felicitas, Perpetua, Agatha, Lucia, Agnes, Cecilia, Anastasia, and with all thy Saints; within whose fellowship we beseech thee to admit us.
\subbysub{Oblation}
\lett{E}{t} hic tibi, Domine, offerimus et exhibemus nosmetipsos, animas et corpora nostra, tibi hostiam rationabilem, sanctam, et viventem; supplices te rogantes, ut quotquot hujus sacræ Communionis participes facti sumus, digne percipiant pretiosum Cor {\ding{64}} pus et Sang {\ding{64}} uinem\margrub{After signing the Gifts, he signs himself.} Filii tui Jesu Christi, omni benedictione cœlesti et gratia tua repleamur, et sint unum corpus cum eo, ut ipse in illis et illi in ipso maneant.

%\subbysub{Commemorations}
%Liturgy of St. Tikhon:
%\lett{B}{e} mindful also, O Lord, of thy servants who are gone before us with the sign of faith, and who rest in the sleep of peace, \textit{N.} and \textit{N.} To them O Lord, and to all who rest in Christ grant we pray thee a place of refreshment, light, and peace. To us sinners also, thy servants, confiding in the multitude of thy mercies, grant some lot and partnership with thy holy Apostles and Martyrs: John, Stephen, Matthias, Barnabas, Ignatius, Alexander, Marcellinus, Peter, Felicitas, Perpetua, Agatha, Lucia, Agnes, Cecilia, Anastasia, and with all thy Saints, into whose company we pray thee of thy mercy to admit us.
\par
\needspace{4\baselineskip}
\lett{E}{t} quamvis propter multiplicia peccata nostra non digni simus qui ullum sacrificium tibi offeramus, hanc tamen debitam oblationem servitutis nostræ, non æstimator meriti sed veniæ, quæsumus, largitor accipias;\margrub{He joins his hands \& bows his head profoundly.}
\par
\needspace{4\baselineskip}
\lett{P}{er} Jesum Christum Dominum nostrum,\margrub{He genuflects} per {\ding{64}} quem et cum {\ding{64}} quo est tibi Deo Patri Omnipotenti, in unitate Spiritus {\ding{64}} Sancti, omnis hon {\ding{64}} or et glo {\ding{64}} ria\margrub{He uncovers the Cup, elevates the Host \& Cup to the height of his breast then replacing them on the Altar, covers the Cup, genuflects.} per omnia sæcula sæculorum.\margrub{He joins his hands.}\par
℟. Amen.


\subby{Lord's Prayer}\par\noindent
Oremus. Præcéptis salutáribus móniti, et divína institutióne formáti audémus dícere:\margrub{He extends his hands.}

\lett{P}{ater} noster, qui es in c{\oe}lis, Sanctificetur Nomen tuum. Adveniat regnum tuum. Fiat voluntas tua, Sicut in c{\oe}lo, et in terra. Panem nostrum quotidianum da nobis hodie. Et dimitte nobis debita nostra, Sicut et nos dimittimus debitoribus nostris. Et ne nos inducas in tentationem: Sed libera nos a malo: Quia tuum est regnum, Potentia, et gloria, In sæcula sæculorum. Amen.
\begin{rubric}
    Unless the Priest consecrated on the Paten, he takes the Paten between the fore and middle fingers of his right hand, and holds it upright upon the Altar.
\end{rubric}
\needspace{4\baselineskip}
\lett{L}{\smash{í}bera} nos, quǽsumus, Dómine, ab ómnibus malis, prætéritis, præséntibus et futúris: et intercedénte beáta et gloriósa semper Vírgine Dei Genetríce María, cum beátis Apóstolis tuis Petro et Paulo, atque Andréa, et ómnibus Sanctis,\margrub{He signs himself (with the Paten, if he consecrated on the Corporal).} da propítius pacem in diébus nostris: ut, ope misericórdiæ tuæ adjúti, et a peccáto simus semper líberi et ab omni perturbatióne secúri.
\begin{rubric}
    If the Host be not already on the Paten, the Priest is to slide the Paten underneath the Host. He then shall uncover the Cup, genuflect, and break the Host in half over the Cup, saying,
\end{rubric}\par\noindent
Per eúndem Dóminum nostrum Jesum Christum, Fílium tuum.
\begin{rubric}
    The Priest placeth the half in his right hand upon the Paten. He then breaketh a particle off from the half in his left hand, saying,
\end{rubric}\par\noindent
Qui tecum vivit et regnat in unitáte Spíritus Sancti Deus.
\begin{rubric}
    Then shall the Priest join the other half, which he holdeth in his left hand, to the half laid upon the Paten, and retaining the small particle in his right hand over the Cup, which he holdeth with his left by the knop below the cup, say in an audible voice, or sing,
\end{rubric}\par\noindent
Per ómnia sǽcula sæculórum.\par
℟. Amen.\par
℣. Pax\margrub{With the same particle, he signs thrice over the Cup.} {\ding{64}} Dómini sit {\ding{64}} semper vobís {\ding{64}} cum.\par
℟. Et cum spíritu tuo.
\begin{rubric}
    The Priest, putting the same particle into the Cup, saith, in the secret voice:
\end{rubric}
\lett{H}{{\ae}c} commíxtio, et consecrátio Córporis et Sánguinis Dómini nostri Jesu Christi, fiat accipiéntibus nobis in vitam ætérnam. Amen.
\subby{Agnus Dei}
\begin{rubric}
    Then shall the Priest cover the Cup, genuflect, and---bowing to the Sacrament---join his hands and strike his breast thrice, saying in an audible voice:
\end{rubric}
\lett{A}{gnus} Dei, qui tollis peccáta mundi: miserére nobis.

\secondline{Agnus Dei, qui tollis peccáta mundi: miserére nobis.}
\thirdline{Agnus Dei, qui tollis peccáta mundi: dona nobis pacem.}
\begin{rubric}
	In Requiem Masses, \emph{miserére nobis} is replaced by \emph{dona eis réquiem} and \emph{dona nobis pacem} by \emph{dona eis réquiem sempitérnam}.
\end{rubric}
%This is the original location of the Prayer Book Confession, mirroring the placement of the Roman Confession. However, in order to emphasise Reformed theology, the later English Prayer Books placed reception of Holy Communion to immediately after the Words of Institution, requiring the Confession to be placed before the Canon. Even though the American tradition developed a more Catholic Eucharistic rite, the Reformed placement of the Confession remained. Here, this is corrected in order to better express our Orthodox Catholic theology.
\subsec{Confession}
\begin{rubric}
    Then the Deacon turneth to the People and saith to those who come to receive the Holy Communion,
\end{rubric}
\lett{V}{os} quos vere et serio peccatorum vestrorum p{\oe}nitet, qui erga proximos veram habetis charitatem, qui vitam novam instituere decrevistis, mandatis Dei obsequendo, et in viis ejus posthac ambulando: Cum fide accedite, ut hoc sanctum percipiatis Sacramentum ad vestram consolationem; et, reverenter genuflexi, humilem vestram Deo Omnipotenti confessionem facite.
\begin{rubric}
Then shall this General Confession be made, by the Priest, bowing, and all those who are minded to receive the Holy Communion, humbly kneeling.
\end{rubric}
\needspace{4\baselineskip}
\lett{O}{mnipotens} Deus, Pater Domini nostri Jesu Christi, Conditor omnium rerum, Omnium hominum judex: Confitemur et deploramus multiplicia peccata et delicta nostra, Quæ subinde impie admisimus, Cogitatione, verbo, et opere, Contra Divinam Majestatem tuam, Provocantes adversus nos justissimam iram et indignationem tuam. Serio nos pœnitet, Et ex animo dolemus ob has prævaricationes nostras: Quanim recordatio nobis acerba est, Onus intolerabile. Miserere nostri, Miserere nostri, Pater misericors; Propter Filium tuum Jesum Christum Dominum nostrum, Quod præteritum est nobis condona: Et concede ut semper posthac Tibi in novitate vitæ serviamus et placeamus, Ad honorem et gloriam Nominis tui; per Jesum Christum Dominum nostrum. Amen. 
\begin{rubric}
	Then shall the Priest (Bishop if he be present) stand up and, turning to the People, say,
\end{rubric}
\needspace{4\baselineskip}
\lett{O}{mnipotens} Deus, Pater noster cœlestis, qui pro magna misericordia sua omnibus ex animo pœnitentibus, ad se cum vera fide conversis, peccatorum remissionem est pollicitus: Misereatur vestri, et dimittat vobis {\ding{64}} omnia peccata vestra: liberet vos ab omni malo, conservet et confirmet in omni bono, et ad vitam perducat æternam; per Jesum Christum Dominum nostrum.

℟. Amen.
\subsec{Comfortable Words}
\begin{rubric}
Then shall the Priest, facing the People, say,
\end{rubric}\noindent
\begin{center}
	\textsc{Audite quam consola toriis verbis omnes ad se veraciter conversos alloquitur Christus Salvator noster.}
\end{center}
\par\noindent
Venite ad me, omnes qui laboratis et onerati estis, et ego reficiam vos. \vr{Matt. 11:28}
\par\noindent
    Sic Deus dilexit mundum, ut Filium suum unigenitum daret, ut omnis qui credit in eum non pereat, sed habeat vitam æternam. \vr{John 3:16}
    \par\noindent
    \begin{center}
		\textsc{Audite etiam quid dicat Sanctus Paulus.}
	\end{center}
\par\noindent
    Fidelis sermo, et omni acceptione dignus, quod Christus Jesus venit in hunc mundum peccatores salvos facere. \inrub{1 Tim. 1:15}
\par\noindent
    \begin{center}
		\textsc{Audite etiam quid dicat Sanctus Joannes.}
	\end{center}
    \par\noindent
    Si quis peccaverit, Advocatum habemus apud Patrem, Jesum Christum justum, et ipse est propitiatio pro peccatis nostris. \inrub{1 John 2:1-2}
\begin{rubric}
    Then the Priest, turning to the Altar, bowing with hands joined upon the Altar, saith in the secret voice,
\end{rubric}
\lett{D}{\smash{ó}mine} Jesu Christe, qui dixísti Apóstolis tuis: Pacem relínquo vobis, pacem meam do vobis: ne respícias peccáta mea, sed fidem Ecclésiæ tuæ; eámque secúndum voluntátem tuam pacificáre et coadunáre dignéris: Qui vivis et regnas Deus per ómnia sǽcula sæculórum. Amen.
\begin{rubric}
    If the \emph{Pax} be given, the Priest is to kiss the Altar and---giving the \emph{Pax}---say: \emph{Pax tecum.} with the response \emph{Et cum spiritu tuo.}
\end{rubric}
\subby{Prayer of Humble Access}
\begin{rubric}
    The Priest alone then saith,
\end{rubric}
\lett{N}{on} justitiæ nostræ, misericors Domine, sed multitudinis magnarum miserationum tuarum fiducia, ad hanc Mensam tuam accedere audemus. Non sumus digni qui vel micas sub Mensa tua colligamus. Tu autem idem ille es Dominus, cui proprium est semper misereri : Tribuas igitur nobis, benigne Domine, Carnem dilecti Filii tui Jesu Christi ita manducare, et Sanguinem ejus bibere, ut corpora nostra immunda per Corpus ejus mundentur, et animæ per pretiosissimum ejus Sanguinem laventur, et nos perpetuo habitemus in eo, et ipse in nobis.\par
℟. Amen.
\subby{Communion of the Priest}
\begin{multicols}{2}
\begin{rubric}
    Then the Priest shall say the \emph{Dómine Jesu Christe} and \emph{Percéptio Córporis tui}.
\end{rubric}
\begin{rubric}
    He then genuflecteth and saith: \emph{Panem cæléstem accípiam, et nomen Dómini invocábo.}.
\end{rubric}
\begin{rubric}
    Then, bowing slightly, the Priest shall take both parts of the Host between the thumb and forefinger of his left hand, and place the Paten between the same forefinger and middle finger, and striking his breast three times with his right hand, say thrice, devoutly and humbly, raising his voice slightly:
\end{rubric}\par\noindent
    Dómine, non sum dignus, \inrub{Proceeding in the secret voice:} ut intres sub tectum meum: sed tantum dic verbo, et sanábitur ánima mea.
\begin{rubric}
After signing himself with his right hand with the Host over the Paten, he saith,
\end{rubric}\par\noindent
Corpus Dómini nostri Jesu Christi custódiat ánimam meam in vitam ætérnam. Amen.
\begin{rubric}
And bowing, he reverently taketh both parts of the Host. After consuming the Host, he is to place the Paten down upon the Corporal, and raising himself, join his hands, and remain still for a short time in meditation on the Most Holy Sacrament.
\end{rubric}
\begin{rubric}
	Then he shall uncover the Cup, genuflect, collect the fragments (if there be any), and cleanse the Paten over the Cup, saying meanwhile:
\end{rubric}
\par\noindent
Quid retríbuam Dómino pro ómnibus, quæ retríbuit mihi? Cálicem salutáris accípiam, et nomen Dómini invocábo. Laudans invocábo Dóminum, et ab inimícis meis salvus ero.
\begin{rubric}
    Taking the Cup in his right hand and signing himself with it, he saith,
\end{rubric}\par\noindent
Sanguis Dómini nostri Jesu Christi custódiat ánimam meam in vitam ætérnam. Amen.
\begin{rubric}
    Holding the Paten under the Cup with his left hand, the Priest is to reverently receive the Blood with the particle.
\end{rubric}
\begin{rubric}
	Having received, if there be any to be communicated, let him communicate them before he purifieth himself.
\end{rubric}
\end{multicols}
\subsec{Communion of the People}
\begin{rubric}
    If there be any to be communicated, the Priest shall genuflect and place the consecrated particles in a Ciborium (or if there are few to be communicated, on the Paten), unless from the beginning they had been placed in a Ciborium.
\end{rubric}
\begin{rubric}
    If the Priest is to administer Communion from the reserved Sacrament, he openeth the Tabernacle, genuflecteth, taketh out the Ciborium, and placeth it upon the Corporal.
\end{rubric}
\begin{rubric}
    Then shall the Priest genuflect, take the Ciborium with his left hand (or the Cup, if he consecrated on the Paten), and take one particle with his right hand takes, which he holdeth between his thumb and forefinger slightly raised above the Ciborium (Cup); and turning to the people in the midst of the Altar, he saith in the clear voice:
\end{rubric}
℣. Ecce Agnus Dei, ecce, qui tollit peccáta mundi.

℟. Dómine, non sum dignus, ut intres sub tectum meum, sed tantum dic verbo, et sanábitur ánima mea.

℟. Dómine, non sum dignus, ut intres sub tectum meum, sed tantum dic verbo, et sanábitur ánima mea.

℟. Dómine, non sum dignus, ut intres sub tectum meum, sed tantum dic verbo, et sanábitur ánima mea.
\begin{rubric}
%This rubric is adapted from the Antiochian Service Book (Eastern Rite).
	After the Priest returneth to the Altar, the Pre-Communion Hymn (p. \pageref{byzantine}) is to then be sung as communicants approach the Altar Rail.\par
	\textsc{Note,} The Pre-Communion Hymn may be said or sung any time after the Prayer of Humble Access.
\end{rubric}
\begin{rubric}\label{communionrubrics}
    Only Orthodox Christians in good standing and properly disposed may receive the Eucharist.\par
    \textsc{Note,} Each communicant shall receive the Communion kneeling.
\end{rubric}
\begin{rubric}
    If they are to communicate, the Priest first communicateth the Sacred Ministers, and then the other Priests and Clerics in choir. (Priest and Deacons shall wear a stole either of white colour or of the same colour as the administering Priest.) And last of all, he proceedeth to communicate the others, beginning with those on the Epistle side.
\end{rubric}
\begin{rubric}
    If the Body and Blood of Christ are to be administered by intinction, then the Priest, when giving the Sacrament to each one, intincteth the Host into the Cup; then, making with the Host the sign of the Cross over the Cup, he placeth the Host on the tongue of each communicant, while saying:
\end{rubric}
\needspace{4\baselineskip}
\lett{C}{orpus} Domini nostri Jesu Christi, quod pro te datum est, etiam Sanguis ejus, qui pro te effusus est. Accipe eos in memoriam quod Christus mortuus est pro te, et in corde tuo, per fidem, vescere illo cum gratiarum actione.
\begin{rubric}
    But if both kinds are to be administered separately, the Priest, when giving the Body, shall make with the Host the sign of the Cross over the Ciborium and place the Host on the tongue of each communicant, while saying:
\end{rubric}
\needspace{4\baselineskip}
\lett{C}{orpus} Domini nostri Jesu Christi, quod pro te datum est, custodiat corpus et animam tuam in vitam æternam. Accipe et manduca hoc in memoriam quod Christus mortuus est pro te, et in corde tuo, per fidem, vescere illo cum gratiarum actione.
\begin{rubric}
    And then, while administering the Cup to each communicant, the Priest, keeping hold of the Cup, shall say:
\end{rubric}
\needspace{4\baselineskip}
\lett{S}{anguis} Domini nostri Jesu Christi, qui pro te effusus est, custodiat corpus et animam tuam in vitam æternam. Bibe hoc in memoriam quod Sanguis Christi effusus est pro te, et gratias age.
\begin{rubric}
    The Communion ended, the Priest shall return to the Altar, saying nothing, nor doth he give the Blessing since he will give it at the end of Mass.    
\end{rubric}
\begin{rubric}
    When the faithful have received Holy Communion, the Deacon and Priest return the Blessed Sacrament to the Tabernacle. The Priest shall then consume the Precious Blood in the Cup.\par
    \textsc{Note,} When prepared for reservation, the Host shall be touched with the Blood.
\end{rubric}
\begin{rubric}
    Then shall the Priest say,
\end{rubric}
\lett{Q}{uod} ore súmpsimus, Dómine, pura mente capiámus: et de múnere temporáli fiat nobis remédium sempitérnum.
\begin{rubric}
    Meanwhile, he presenteth the Cup to the Minister, who shall pour in a little wine, wherewith he purifieth himself. Then he continueth,    
\end{rubric}
\lett{C}{orpus} tuum, Dómine, quod sumpsi, et Sanguis, quem potávi, adhǽreat viscéribus meis: et præsta; ut in me non remáneat scélerum mácula, quem pura et sancta refecérunt sacraménta: Qui vivis et regnas in sǽcula sæculórum. Amen.
\begin{rubric}
    Then shall the Priest wash and wipe his fingers and take the ablution. Then he wipeth his mouth and the Cup.
\end{rubric}
\begin{rubric}
	After folding the Corporal, he is to cover the Cup and place it upon the Altar as before. Then he proceedeth with the Mass.
\end{rubric}
\subby{Thanksgiving}
\begin{rubric}
    Standing with his hands joined, the Priest shall read the Communion Antiphon. Then, at the Epistle corner, he alone saith the Thanksgiving.
\end{rubric}
\oremuslatin
\lett{O}{mnipotens} sempiterne Deus, tibi toto cordis affectu gratias agimus, quia nos hæc sancta Mysteria recte accipientes cibo spirituali pretiosissimi Corporis et Sanguinis Filii tui Salvatoris nostri Jesu Christi pascere dignatus es; et per hoc nos certiores facere de gratia et bonitate tua erga nos, et quod sumus vera membra corpori Filii tui mystico, fidelium omnium beatæ societati, incorporata, et hæredes secundum spem æterni regni tui, propter merita pretiosissimæ mortis et passionis dilecti Filii tui. Teque, cœlestis Pater, supplices rogamus, ut gratiæ tuæ subsidiis adjuti in sancta illa societate perseveremus, et ea omnia bona faciamus opera, quæ præparasti ut in illis ambulemus; per Jesum Christum Dominum nostrum, cui sit tecum, in unitate Spiritus Sancti, omnis honor et gloria per omnia sæcula sæculorum.\par
℟. Amen.

\subsec{Dismissal}

\begin{rubric}
    Then, with hands joined before his breast, the Priest shall proceed to the midst of the Altar, kiss the Altar, and turn to the People, with hands extended, saying:
\end{rubric}
℣. Dóminus vobíscum.

℟. Et cum spíritu tuo.
\begin{rubric}
    Then, turning back to the Book, he saith:
\end{rubric}
℣. Orémus.
\begin{rubric}
The Priest then saith the Postcommunion Prayers in the same manner, number, and order as the Collects at the beginning of Mass.
\end{rubric}
\begin{rubric}
    After saying the last Prayer, the Priest is to return to the midst of the Altar, kissing it, and turning toward the people, saying:
\end{rubric}
℣. Dóminus vobíscum.

℟. Et cum spíritu tuo.
    \begin{rubric}
    	{On days when the \emph{Gloria in exclesis} is said,}
    \end{rubric}
    ℣. Ite, Missa est.
    
    ℟. Deo grátias.
    \begin{rubric}
    	{On days when the \emph{Gloria in excelsis} is not said,}
    \end{rubric}
    ℣. Benedicámus Dómino.
    
    ℟. Deo grátias.
\begin{rubric}
In Requiem Masses,
\end{rubric}
    ℣. Requiéscant in pace.
    
    ℟. Amen.
\begin{rubric}
    In Eastertide, in Masses of the Season,
\end{rubric}
    ℣. Ite, Missa est, allelúja, allelúja.
    
    ℟. Deo grátias, allelúja, allelúja.
\begin{rubric}
    Having said the Dismissal, the Priest boweth before the midst of the Altar, and with hands joined thereon, saith in the secret voice: \emph{Pláceat tibi}.
\end{rubric}
\begin{rubric}
    Then---unless it be a Requiem Mass---the Congregation kneeling, the Priest shall kiss the Altar; raise his eyes; extend, raise, and join his hands; bow his head to the Cross; and give the Blessing, saying,
\end{rubric}
\lett{P}{ax} Dei quæ exsuperat omnem sensum, custodiat corda vestra et intelligentias vestras in scientia et amore Dei, et Filii ejus Jesu Christi Domini nostri: Et benedictio Dei Omnipotentis,\margrub{He turns to the people, blessing them once only, even in solemn Masses.} Patris, {\ding{64}} Filii, et Spiritus Sancti sit super vos, et maneat semper vobiscum. \textit{Amen.}
\begin{rubric}
    In Pontifical Masses, the Blessing is given by the Bishop and is threefold, as ordered in the Pontifical.
\end{rubric}
\begin{rubric}
    In Masses of the Dead, the Blessing is not given, but having said \emph{Requiéscant in pace} and \emph{Pláceat tibi}, he kisseth the Altar and readeth the Last Gospel.
\end{rubric}
\subsec{Last Gospel}
\begin{rubric}
    Unless a proper Last Gospel be provided, the Priest shall conclude the Mass with this Last Gospel, at the Gospel corner and with hands joined:
\end{rubric}

℣. Dóminus vobíscum.

℟. Et cum spíritu tuo.

℣. Inítium\margrub{He signs with the sign of the Cross first the Altar or the Book, then himself on the forehead, mouth, \& breast.} {\ding{66}} sancti Evangélii secúndum Joánnem

℟. Glória tibi, Dómine.

In princípio erat Verbum, et Verbum erat apud Deum, et Deus erat Verbum. Hoc erat in princípio apud Deum. Omnia per ipsum facta sunt: et sine ipso factum est nihil, quod factum est: in ipso vita erat, et vita erat lux hóminum: et lux in ténebris lucet, et ténebræ eam non comprehendérunt.
Fuit homo missus a Deo, cui nomen erat Joánnes. Hic venit in testimónium, ut testimónium perhibéret de lúmine, ut omnes créderent per illum. Non erat ille lux, sed ut testimónium perhibéret de lúmine.
Erat lux vera, quæ illúminat omnem hóminem veniéntem in hunc mundum. In mundo erat, et mundus per ipsum factus est, et mundus eum non cognóvit. In própria venit, et sui eum non recepérunt. Quotquot autem recepérunt eum, dedit eis potestátem fílios Dei fíeri, his, qui credunt in nómine ejus: qui non ex sanguínibus, neque ex voluntáte carnis, neque ex voluntáte viri, sed ex Deo nati sunt. \inrub{Everyone genuflects.} Et Verbum caro factum est, \inrub{Everyone rises.} et habitávit in nobis: et vídimus glóriam ejus, glóriam quasi Unigéniti a Patre, plenum gráti{\ae} et veritátis.\par
℟. Deo grátias.
\begin{rubric}
    At High Mass, a hymn may be sung while the Priest and servers leave the sanctuary.
\end{rubric}
\begin{rubric}
    As he departeth from the Altar, the Priest saith for thanksgiving the Antiphon \emph{Let us sing}, with the rest (p. \pageref{CommunionThanksgiving}).
\end{rubric}
\subsec{General Rubrics}\label{GeneralRubrics}
%The Book of Common Prayer has always contained rubrics at the beginning and end of the liturgical text in order for the good order of the Church. Due to the unique situation of the Vicariate, it is fitting that these be accommodated to our use.
%\begin{rubric}
%Notwithstanding anything that is elsewhere enjoined in any Rubrick or Canon, the Priest, in celebrating the Holy Communion, shall wear the vestments proper to the Priest as of 1950 in dissident Rome. That is, Amice, Alb, Cincture, Maniple, Stole, and Chasuble (the latter three of the colour of the day).
%\end{rubric}
\begin{rubric}
    It is expedient for the Priest to have, set up on the Altar, cards with the most common prayers (or at least a convenient pamphlet), including the those only mentioned by their incipit in this Book.\par
\end{rubric}
%\begin{rubric}
%The Priest, the presence of an Ordinary notwithstanding, when he processes or reposes, wears a Canterbury Cap or Biretta.
%\end{rubric}
%\begin{rubric}
%There shall be no celebration of the Lord's Supper without at least one person, other than the Priest, physically present.
%\end{rubric}
\begin{rubric}
The Altar, at the Communion-time---having a fair white linen cloth, a cross, and at least two candles upon it---shall stand in the Sanctuary.
\end{rubric}
\begin{rubric}
The Bread for the Eucharist ought to be provided by the Priest or the members of the Congregation. It must be leavened. That is, before and after being initially baked, it must be made of---and only made of---wheat flour, water, yeast, and salt. The Wine should be made of grapes, without any additives or supplements. The Cup for the Eucharist should be made of precious metal.
\end{rubric}
\begin{rubric}
It is an ancient and laudable rule of the Church to receive this Holy Sacrament fasting. That is, one who desireth to receive the Eucharist must consume neither food nor drink (water excepted) from the midnight before the liturgy until the time of reception. In the case of Masses which occur after noon, it is sufficient that the fasting begin at noon or immediately after one's noon-time meal.
\end{rubric}

%\begin{center}
%	{\textsc{Here Endeth the Liturgy of St. Tikhon.}}
%\end{center}
%\clearpage
%\subby{Exhortations for Holy Communion}
%\begin{rubric}
%	At the time of the celebration of the Communion, the Priest may say this Exhortation. And Note, that the Exhortation shall be said on the First Sunday in Advent, the First Sunday in Lent, and Trinity Sunday.
%\end{rubric}
%\lett{D}{early} beloved in the Lord, ye who mind to come to the holy Communion of the Body and Blood of our Saviour Christ, must consider how Saint Paul exhorteth all persons diligently to try and examine themselves, before they presume to eat of that Bread, and drink of that Cup. For as the benefit is great, if with a true penitent heart and lively faith we receive that holy Sacrament; 
%Addition:
%that is, the very and true Body and Blood of Our Lord Jesus Christ; 
%
%so is the danger great, if we receive the same unworthily. Judge therefore yourselves, brethren, that ye be not judged of the Lord; repent you truly for your sins past; 
%Addition:
%confess them to your spiritual father; 
%
%have a lively and steadfast faith in Christ our Saviour; amend your lives, and be in perfect charity with all men; so shall ye be meet partakers of those holy mysteries. And above all things ye must give most humble and hearty thanks to God, the Father, the Son, and the Holy Ghost, for the redemption of the world by the death and passion of our Saviour Christ, both God and man; who did humble himself, even to the death upon the Cross, for us, miserable sinners, who lay in darkness and the shadow of death; that he might make us the children of God, and exalt us to everlasting life. And to the end that we should always remember the exceeding great love of our Master, and only Saviour, Jesus Christ, thus dying for us, and the innumerable benefits which by his precious blood-shedding he hath obtained for us; he hath instituted and ordained holy mysteries, as pledges of his love, and for a continual remembrance of his death, to our great and endless comfort. To him therefore, with the Father and the Holy Ghost, let us give (as we are most bounden) continual thanks; submitting ourselves wholly to his holy will and pleasure, and studying to serve him in true holiness and righteousness all the days of our life. Amen.
%\begin{rubric}
%	When the Minister giveth warning for the Celebration of the Holy Communion, (which he shall always do upon the Sunday, or some Holy-day, immediately preceding,) he shall read this Exhortation following, or so much thereof as, in his discretion, he may think convenient.
%\end{rubric}
%\lett{D}{early} beloved, on \textit{N.} day next I purpose, through God's assistance, to administer to all such as shall be religiously and devoutly disposed the most comfortable Sacrament of the Body and Blood of Christ; to be by them received in remembrance of his meritorious Cross and Pas-sion; whereby alone we obtain remission of our sins, and are made partakers of the Kingdom of heaven. Wherefore it is our duty to render most humble and hearty thanks to Almighty God, our heavenly Father, for that he hath given his Son our Saviour Jesus Christ, not only to die for us, but also to be our spiritual food and sustenance in that holy Sacrament. Which being so divine and comfortable a thing to them who receive it worthily, and so dangerous to those who will presume to receive it unworthily; my duty is to exhort you, in the mean season to consider the dignity of that holy mystery, and the great peril of the unworthy receiving thereof; and so to search and examine your own consciences, and that not lightly, and after the manner of dissemblers with God; but so that ye may come holy and clean to such a heavenly Feast, in the marriage-garment required by God in holy Scripture, and be received as worthy partakers of that holy Table.
%
%    The way and means thereto is: First, to examine your lives and conversations by the rule of God's commandments; and whereinsoever ye shall perceive yourselves to have offended, either by will, word, or deed, there to bewail your own sinfulness, and to confess yourselves to Almighty God, with full purpose of amendment of life. And if ye shall perceive your offences to be such as are not only against God, but also against your neighbours; then ye shall reconcile yourselves unto them; being ready to make restitution and satisfaction, according to the uttermost of your powers, for all injuries and wrongs done by you to any other; and being likewise ready to forgive others who have offended you, as ye would have forgiveness of your offences at God's hand: for otherwise the receiving of the holy Communion doth nothing else but increase your condemnation. Therefore, if any of you be a blasphemer of God, an hinderer or slanderer of his Word, an adulterer, or be in malice, or envy, or in any other grievous crime; repent you of your sins, or else come not to that holy Table.
%    
%    And because it is requisite that no man should come to the holy Communion, but with a full trust in God's mercy, and with a quiet conscience; 
    %This phrase is omitted so to indicate that all such sinners ought to confess:
    %therefore, if there be any of you, who by this means cannot quiet his own conscience herein, but requireth further comfort or counsel, 
%    let him come to me, or to some other Priest of 
    %Change:
%    Christ's Church, confessing his sins and opening his grief; that he may receive such godly counsel, advice, and absolution of his sins as may tend to the quieting of his conscience, the removing of all scruple and doubtfulness, and the sure benefit of Jesus Christ's promise: the remission of sins and bestowing of the benefits of his Passion.
%\begin{rubric}
%	Or, in case he shall see the People negligent to come to the Holy Communion, instead of the former, he may use this Exhortation.
%\end{rubric}
%\lett{D}{early} beloved brethren, on \textit{N.} I intend, by God's grace, to celebrate the 
%Change:
%Holy Sacrifice of the Mass: 
%
%unto which, in God's behalf, I bid you all who are here present; and beseech you, for the Lord Jesus Christ's sake, that ye will not refuse to come thereto, being so lovingly called and bidden by God himself. Ye know how grievous and unkind a thing it is, when a man hath prepared a rich feast, decked his table with all kind of provision, so that there lacketh nothing but the guests to sit down; and yet they who are called, without any cause, most unthankfully refuse to come. Which of you in such a case would not be moved? Who would not think a great injury and wrong done unto him? Wherefore, most dearly beloved in Christ, take ye good heed, lest ye, withdrawing yourselves from this holy Supper, provoke God's indignation against you. It is an easy matter for a man to say, I will not communicate, because I am otherwise hindered with worldly business. But such excuses are not so easily accepted and allowed before God. If any man say, I am a grievous sinner, and therefore am afraid to come: wherefore then do ye not repent and amend? 
%Addition:
%Wherefore do ye not approach myself or one of Christ's Priests for Confession and Absolution? 
%
%When God calleth you, are ye not ashamed to say ye will not come? When ye should return to God, will ye excuse yourselves, and say ye are not ready? Consider earnestly with yourselves how little such feigned excuses will avail before God. Those who refused the feast in the Gospel, because they had bought a farm, or would try their yokes of oxen, or because they were married, were not so excused, but counted unworthy of the heavenly feast. Wherefore, according to mine office, I bid you in the Name of God, I call you in Christ's behalf, I exhort you, as ye love your own salvation, that ye will be partakers of this holy Communion. And as the Son of God did vouchsafe to yield up his soul by death upon the Cross for your salvation; so it is your duty to receive the Communion in remembrance of the sacrifice of his death, as he himself hath commanded: which if ye shall neglect to do, consider with yourselves how great is your ingratitude to God, and how sore punishment hangeth over your heads for the same; when ye wilfully abstain from the Lord's Table, and separate from your brethren, who come to feed on the banquet of that most heavenly food. These things if ye earnestly consider, ye will by God’s grace return to a better mind: for the obtaining whereof we shall not cease to make our humble petitions unto Almighty God, our heavenly Father.