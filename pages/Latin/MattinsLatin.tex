\fancyhead[RO,LE]{\textit{Mattins}}
\section{Ordo Matutinarum}
℣. Dómine, {\ding{61}} lábia nostra apéries.

℟. Et os nostrum annuntiábit laudem tuam.

℣. Deus, {\ding{64}} in adjutórium nostrum inténde.

℟. Dómine, ad adjuvándum nos festína.

\begin{rubric}
    Hic, omnibus sese erigentibus, dicat Minister.
\end{rubric}

℣. Glória Patri, et Fílio, * et Spirítui Sancto:

℟. Sicut erat in princípio, et nunc, et semper, * et in sǽcula s{\ae}culórum. Amen.

℣. Laudáte Dominum.

℟. Sit Nomen Dómini Benedíctum.

\begin{rubric}
    Tunc dicatur vel cantetur Psalmus hic sequens nisi in Die Sancto Paschæ, cui alia assignatur Antiphona. Item in die undevicesimo cujusque mensis non hoc loco, sed in ordinario Psalmorum cursu, est legendus.
\end{rubric}
\begin{rubric}
    But \textsc{Note}, That on Ash Wednesday and Good Friday the \emph{Venite} may be omitted.
\end{rubric}
\begin{rubric}
    On the days hereafter named, immediately before and after the \emph{Venite} may be sung or said,
\end{rubric}
\par\noindent
\inv{Advent:} Our King and Saviour draweth nigh; * O come, let us adore him.
\par\noindent
\inv{Christmastide:} Alleluia. Unto us a child is born; * O come, let us adore him. Alleluia.
\par\noindent
\inv{Epiphanytide \& Transfiguration:} The Lord hath manifested forth his glory; * O come, let us adore him.
\par\noindent
%Adapted from AOB:
\inv{Septuagesimatide:} Let us come before the presence of the Lord with thanksgiving;  * O come, let us adore him.
\par\noindent
%Adapted from the 1962 BCP:
\inv{Lent:} The goodness of God leadeth to repentance; * O come, let us adore him.
\par\noindent
\inv{Passiontide:} Christ our Lord became obedient unto death; * O come, let us adore him.
\par\noindent
\inv{Eastertide outside the Octave:} Alleluia. The Lord is risen indeed; * O come let us adore him. Alleluia.
\par\noindent
\inv{Ascensiontide:} Alleluia. Christ the Lord ascended into heaven; * O come, let us adore him. Alleluia.
\par\noindent
\inv{Whitsuntide:} Alleluia. The Spirit of the Lord filleth the world; * O come, let us adore him. Alleluia.
\par\noindent
\inv{Trinitytide:} Father, Son, and Holy Ghost, One God; * O come, let us adore him.
\par\noindent
\inv{Feasts of Our Lord \& Our Lady:} The Word was made flesh; * O come, let us adore him.
\par\noindent
%Due to the wider use of Feast Days in the AWRV, it is difficult to decide how to adapt this rubric (Feast Days which have a Proper Epistle and Gospel). Given that Semidouble Feast Days almost never occur, and moving the bar so low to Simplex or Memorial would destroy the desired effect, Semidouble seems to be the best rank. For it would not cast too wide of a net: effectively capturing all Doubles and their Octaves, except when the Octave is meant to not be noticed intensely.
\inv{Other Feast Days of Semidouble or higher:} The Lord is glorious in his saints; * O come, let us adore him.
%For text of the psalms, the Latin Monastic Diurnal is used.
\subby{Venite, exultemus Domino.}

\lett{V}{en\smash{í}te}, exsultémus Dómino, jubilémus Deo, salutári nostro:\par
\secondline{Pr{\ae}occupémus fáciem ejus in confessióne, et in psalmis jubilémus ei.}
\thirdline{Quóniam Deus magnus Dóminus, et Rex magnus super omnes deos, quóniam non repéllet Dóminus plebem suam:}
Quia in manu ejus sunt omnes fines terr{\ae}, et altitúdines móntium ipse cónspicit.\par
Quóniam ipsíus est mare, et ipse fecit illud, et áridam fundavérunt manus ejus\par
Veníte, adorémus, et procidámus ante Deum: plorémus coram Dómino, qui fecit nos,\par
Quia ipse est Dóminus, Deus noster; nos autem pópulus ejus, et oves páscu{\ae} ejus.\par
Hódie, si vocem ejus audiéritis, nolíte obduráre corda vestra, sicut in exacerbatióne secúndum diem tentatiónis in desérto: ubi tentavérunt me patres vestri, probavérunt et vidérunt ópera mea.
\par
Quadragínta annis próximus fui generatióni huic, et dixi; Semper hi errant corde, ipsi vero non cognovérunt vias meas: quibus jurávi in ira mea; Si introíbunt in réquiem meam.

\begin{rubric}
    %1928 BCP Rubric (permission to not use Gloria Patri, except at the end of all of the Psalms, is removed):
    Then shall follow a Portion of the Psalms, according to the Use of this Church. And at the end of every Psalm, and liksewise at the end of the \emph{Venite}, daily Old Testament Canticle, and \emph{Benedictus}, shall be sung or said the \emph{Gloria Patri}.
\end{rubric}
℣. Glória Patri, et Fílio, * et Spirítui Sancto:\par
℟. Sicut erat in princípio, et nunc, et semper, * et in sǽcula s{\ae}culórum. Amen.

\par\noindent
	\centerline{\rule{0.5\textwidth}{0.4pt}}
\par\noindent
\begin{rubric}
    Verses 8-11 of the \emph{Venite} may be replaced with the following.
\end{rubric}
Tóllite hóstias, et introíte in átria ejus: * adoráte Dóminum in átrio sancto ejus.\par
Judicábit orbem terr{\ae} in {\ae}quitáte, * et pópulos in veritáte sua.
\par\noindent
	\centerline{\rule{0.5\textwidth}{0.4pt}}
\par\noindent
\begin{rubric}
%Rubric added to permit the Sarum Invitatory Psalm.
    After the \emph{Venite}, a Hymn may be sung.
\end{rubric}
\begin{rubric}
Then shall be read the First Lesson, according to the Table or Calendar. And \textsc{Note}, That before every Lesson, the Minister shall say, \emph{Here beginneth} such a \emph{Chapter} (or \emph{Verse of} such a \emph{Chapter}) \emph{of} such a Book; and after every Lesson, \emph{Here endeth the First} (or \emph{the Second Lesson.})
\end{rubric}
\vspace{-2ex}
\subby{Te Deum laudamus}
%Reversion to English language brings the Te Deum is conformity to the Monastic Matins use.
\begin{rubric}
%Great reduction of the Office, allowed in the 1928, not reproduced. Additional Canticles supplied.
The \emph{Te Deum} is prayed on all Sundays (outside Advent \& Lent) and Feast Days %This rubric brings the Te Deum to the same standard as the other Festal canticles and brings it closer to Breviary use.
(Double or higher).\par
\textsc{Note,} The Old Testament Canticle (p. \pageref{OT}) may be said instead of the \emph{Te Deum}.
\end{rubric}
\lett{T}{e} Deum laudámus: * te Dóminum confitémur.\par
\secondline{Te {\ae}térnum Patrem * omnis terra venerátur.}
\thirdline{Tibi omnes Ángeli, * tibi C{\ae}li, et univérs{\ae} Potestátes}
Tibi Chérubim et Séraphim * incessábili voce proclámant:\par
Sanctus, Sanctus, Sanctus * Dóminus Deus Sábaoth.\par
Pleni sunt c{\ae}li et terra * majestátis glóri{\ae} tu{\ae}.\par
Te gloriósus * Apostolórum chorus,\par
Te Prophetárum * laudábilis númerus,\par
Te Mártyrum candidátus * laudat exércitus.\par
Te per orbem terrárum * sancta confitétur Ecclésia,\par
Patrem * imméns{\ae} majestátis;\par
Venerándum tuum verum * et únicum Fílium;\par
Sanctum quoque * Paráclitum Spíritum.

\lett{T}{u} Rex glóri{\ae}, * Christe.\par
\secondline{Tu Patris * sempitérnus es Fílius.}
\thirdline{Tu, ad liberándum susceptúrus hóminem: * non horruísti Vírginis úterum.}
Tu, devícto mortis acúleo, * aperuísti credéntibus regna c{\ae}lórum.\par
Tu ad déxteram Dei sedes, * in glória Patris.\par
Judex créderis * esse ventúrus. \par
Te ergo quǽsumus, tuis fámulis súbveni, * quos pretióso sánguine redemísti.\par
{\AE}térna fac cum Sanctis tuis * in glória numerári.\par
\lett{S}{alvum} fac pópulum tuum, Dómine, * et bénedic hereditáti tu{\ae}.\par
\secondline{Et rege eos, * et extólle illos usque in {\ae}térnum.}
\thirdline{Per síngulos dies * benedícimus te.}
Et laudámus nomen tuum in sǽculum, * et in sǽculum sǽculi.\par
Dignáre, Dómine, die isto * sine peccáto nos custodíre.\par
Miserére nostri, Dómine, * miserére nostri.\par
Fiat misericórdia tua, Dómine, super nos, * quemádmodum sperávimus in te.\par
In te, Dómine, sperávi: * non confúndar in {\ae}térnum.

%MANUAL ADJUSTMENT:
\vspace{-2ex}
\subby{Benedicite, omnia opera Domini.}

%MANUAL ADJUSTMENT:
\vspace{-1ex}

\begin{rubric}
The \emph{Benedicite} is prayed on all Sundays in Advent \& Lent, and on Days below Double.\par
\textsc{Note,} The Old Testament Canticle (p. \pageref{OT}) may be said instead of the \emph{Benedicite}.
\end{rubric}

\lett{B}{ened\smash{í}cite,} ómnia ópera Dómini, Dómino: * laudáte et superexaltáte eum in sǽcula.\par
\secondline{Benedícite, Ángeli Dómini, Dómino: * laudáte et superexaltáte eum in sǽcula.}
\lett{B}{ened\smash{í}cite,} c{\ae}li, Dómino:  * laudáte et superexaltáte eum in sǽcula.\par
\secondline{Benedícite, aqu{\ae} omnes, qu{\ae} super c{\ae}los sunt, Dómino: * laudáte et superexaltáte eum in sǽcula.}
\thirdline{Benedícite, omnes virtútes Dómini, Dómino: * laudáte et superexaltáte eum in sǽcula.}
Benedícite, sol et luna, Dómino: * laudáte et superexaltáte eum in sǽcula.\par
Benedícite, stell{\ae} c{\ae}li, Dómino: * laudáte et superexaltáte eum in sǽcula.\par
Benedícite, omnis imber et ros, Dómino: * laudáte et superexaltáte eum in sǽcula.\par
Benedícite, omnes spíritus Dei, Dómino: * laudáte et superexaltáte eum in sǽcula.\par
Benedícite, ignis et {\ae}stus, Dómino: * laudáte et superexaltáte eum in sǽcula.\par
Benedícite, frigus et {\ae}stus, Dómino: * laudáte et superexaltáte eum in sǽcula.\par
Benedícite, rores et pruína, Dómino: * laudáte et superexaltáte eum in sǽcula.\par
Benedícite, gelu et frigus, Dómino: * laudáte et superexaltáte eum in sǽcula.\par
Benedícite, glácies et nives, Dómino: * laudáte et superexaltáte eum in sǽcula.\par
Benedícite, noctes et dies, Dómino: * laudáte et superexaltáte eum in sǽcula.\par
Benedícite, lux et ténebr{\ae}, Dómino: * laudáte et superexaltáte eum in sǽcula.\par
Benedícite, fúlgura et nubes, Dómino: * laudáte et superexaltáte eum in sǽcula.\par

\lett{B}{ened\smash{í}cat} terra Dóminum: * laudet et superexáltet eum in sǽcula.\par
\secondline{Benedícite, montes et colles, Dómino: * laudáte et superexaltáte eum in sǽcula.}
\thirdline{Benedícite, univérsa germinántia in terra, Dómino: * laudáte et superexaltáte eum in sǽcula.}
Benedícite, fontes, Dómino: * laudáte et superexaltáte eum in sǽcula.\par
Benedícite, mária et flúmina, Dómino: * laudáte et superexaltáte eum in sǽcula.\par
Benedícite, cete, et ómnia, qu{\ae} movéntur in aquis, Dómino: * laudáte et superexaltáte eum in sǽcula.\par
Benedícite, omnes vólucres c{\ae}li, Dómino: * laudáte et superexaltáte eum in sǽcula.\par
Benedícite, omnes bésti{\ae} et pécora, Dómino: * laudáte et superexaltáte eum in sǽcula.\par
Benedícite, fílii hóminum, Dómino: * laudáte et superexaltáte eum in sǽcula.\par
\lett{B}{ened\smash{í}cat} Israël Dóminum: * laudet et superexáltet eum in sǽcula.\par
\secondline{Benedícite, sacerdótes Dómini, Dómino: * laudáte et superexaltáte eum in sǽcula.}
\thirdline{Benedícite, servi Dómini, Dómino: * laudáte et superexaltáte eum in sǽcula.}
Benedícite, spíritus, et ánim{\ae} justórum, Dómino: * laudáte et superexaltáte eum in sǽcula.\par
Benedícite, sancti, et húmiles corde, Dómino: * laudáte et superexaltáte eum in sǽcula.\par
\lett{B}{enedic\smash{á}mus} Patrem et Fílium cum Sancto Spíritu: * laudémus et superexaltémus eum in sǽcula.

\begin{rubric}
    {Then shall be read, in like manner, the Second Lesson, taken out of the New Testament, according to the Table or Calendar.}
\end{rubric}
%\begin{rubric}
%    And after that may be sung a Hymn, then shall be sung or said the Hymn following.
%\end{rubric}

%Permission to shorten Benedictus removed.
\subby{Benedictus}\label{Benedictus}

%MANUAL ADJUSTMENT:
\vspace{-2ex}

\begin{rubric}
%Permission for shorter Office removed.
    After which may be sung or said a Hymn and then shall be sung or said the \emph{Benedictus}, as followeth.
\end{rubric}

%MANUAL ADJUSTMENT:
\vspace{-1ex}

\lett{B}{ened\smash{í}ctus} {\ding{64}} Dóminus, Deus Isra{\"e}l: * quia visitávit, et fecit redemptiónem plebi su{\ae}:\par
\secondline{Et eréxit cornu salútis nobis: * in domo David, púeri sui.}
\thirdline{Sicut locútus est per os sanctórum, * qui a sǽculo sunt, prophetárum ejus:}
Salútem ex inimícis nostris, * et de manu ómnium, qui odérunt nos.\par
Ad faciéndam misericórdiam cum pátribus nostris: * et memorári testaménti sui sancti.\par
Jusjurándum, quod jurávit ad Ábraham patrem nostrum, * datúrum se nobis:\par
Ut sine timóre, de manu inimicórum nostrórum liberáti, * serviámus illi.\par
In sanctitáte, et justítia coram ipso, * ómnibus diébus nostris.\par
Et tu, puer, Prophéta Altíssimi vocáberis: * pr{\ae}íbis enim ante fáciem Dómini, paráre vias ejus:\par
Ad dandam sciéntiam salútis plebi ejus: * in remissiónem peccatórum eórum:\par
Per víscera misericórdi{\ae} Dei nostri: * in quibus visitávit nos, óriens ex alto:\par
Illumináre his, qui in ténebris, et in umbra mortis sedent: * ad dirigéndos pedes nostros in viam pacis.

%Permission for Psalm 100 removed.
\vspace{-2ex}
\subby{Apostles' Creed}
%While the Athanasian Creed was excised from the American tradition (during an attempt to remove every Creed), it is present in the Prayer Book Tradition, including the Canadian version. Therefore, it is thus supplied here. Since the Canadian tradition allows the Athanasian Creed on any day of the year, it is allowed likewise here.
\begin{rubric}
    Then shall be said the Apostles' Creed, by the Minister and the People, standing.\par%Permission for alternate words removed.
    \textsc{Note,} the Nicene Creed may be said instead of the Apostles' (p. \pageref{NiceneCreed}).
\end{rubric}
\begin{rubric}
    Upon these Feasts; \emph{Christmas Day}, the \emph{Epiphany}, Saint \emph{Matthias}, \emph{Easter Day}, \emph{Ascension Day}, \emph{Whitsunday}, Saint \emph{John Baptist}, Saint \emph{James}, Saint \emph{Bartholomew}, Saint \emph{Matthew}, Saint \emph{Simon} and Saint \emph{Jude}, Saint \emph{Andrew}, and upon \emph{Trinity Sunday}, shall be sung or said, instead of the Apostle's Creed, the Athanasian Creed (p. \pageref{Ath}).
\end{rubric}
\lett{C}{redo} in Deum, Patrem omnipoténtem, Creatórem c{\ae}li et terr{\ae}. Et in Jesum Christum, Fílium ejus únicum, Dóminum nostrum: qui concéptus est de Spíritu Sancto, natus ex María Vírgine, passus sub Póntio Piláto, crucifíxus, mórtuus, et sepúltus: descéndit ad ínferos; tértia die resurréxit a mórtuis; ascéndit ad c{\ae}los; sedet ad déxteram Dei Patris omnipoténtis: inde ventúrus est judicáre vivos et mórtuos.\par
     Credo in Spíritum Sanctum, sanctam Ecclésiam cathólicam, Sanctórum communiónem, remissiónem peccatórum, carnis {\ding{64}} resurrectiónem, vitam {\ae}térnam. Amen.

\begin{rubric}
    And after that, these Prayers following, the People devoutly kneeling; the Minister first pronouncing,
\end{rubric}
%\begin{rubric}
%    The Minister begins with the \emph{Dóminus vobíscum} only if he be in Major Orders. Otherwise, he shall use the second option.
%\end{rubric}
℣. Dóminus vobíscum.\par
\textit{vel,} Dómine, exáudi oratiónem nostram.

℟. Et cum spíritu tuo.\par
\textit{vel,} Et clamor noster ad te véniat.

℣. Orémus.

℣. Kýrie, eléison.

℟. Christe, eléison.

℣. Kýrie, eléison.

\subby{Lord's Prayer}
\begin{rubric}
    Then the Minister, Clerks, and People---all kneeling---shall say the Lord's Prayer with a loud voice.
\end{rubric}
\lett{P}{ater} noster, qui es in c{\ae}lis, sanctificétur nomen tuum: advéniat regnum tuum: fiat volúntas tua, sicut in c{\ae}lo et in terra. Panem nostrum quotidiánum da nobis hódie: et dimítte nobis débita nostra, sicut et nos dimíttimus debitóribus nostris: et ne nos indúcas in tentatiónem: sed líbera nos a malo. Amen.

\vspace{-0.5ex}
\subby{Preces}
\begin{rubric}
    Then the Minister standing up shall say,
\end{rubric}
℣. Osténde nobis, Dómine, misericórdiam tuam.\par
℟. Et salutáre tuum da nobis.\par
℣. Dómine salvam fac \textit{Civitatem}.\par
℟. Et exáudi nos cum invocámus te.\par
℣. Sacerdótes tui induántur Justítia.\par
℟. Et sancti tui exúltent.\par
℣. Salvum fac Pópulum tuum, Dómine.\par
℟. Et bénedic H{\ae}reditáti tu{\ae}.\par
℣. Da pacem Dómine in diébus nostris.\par
℟. Quóniam tu, Dómine, singuláriter in spe constituísti me.\par
℣. Cor mundum crea in nobis, O Deus.\par
℟. Et Spíritum Sanctum tuum ne áuferas a nobis.
%\elcol{℣. O Lord, show thy mercy upon us.}{℣. Osténde nobis, Dómine, misericórdiam tuam.}
%\elcol{℟. And grant us thy salvation.}{℟. Et salutáre tuum da nobis.}
%\elcol{℣. O Lord, save the \textit{State}.}{℣. Dómine salvam fac \textit{Civitatem}.}
%\elcol{℟. And mercifully hear us when we call upon thee.}{℟. Et exáudi nos cum invocámus te.}
%\elcol{℣. Endue thy Ministers with righteousness.}{℣. Sacerdótes tui induántur Justítia.}
%\elcol{℟. And make thy chosen people joyful.}{℟. Et sancti tui exúltent.}
%\elcol{℣. O Lord, save thy people.}{℣. Salvum fac Pópulum tuum, Dómine.}
%\elcol{℟. And bless thine inheritance.}{℟. Et bénedic H{\ae}reditáti tu{\ae}.}
%\elcol{℣. Give peace in our time, O Lord.}{℣. Da pacem Dómine in diébus nostris.}
%See the notes for this petition in Evensong.
%\elcol{℟. Because there is none other that fighteth for us, but only thou, O God.}{℟. Quia non est álius qui pugnet pro nobis, nisi tu Deus noster.}
%Testing using American version.
%\elcol{℟. For it is thou, Lord, only, that makest us dwell in safety.}{℟. Quóniam tu, Dómine, singuláriter in spe constituísti me.}
%\elcol{℣. O God, make clean our hearts within us.}{℣. Cor mundum crea in nobis, O Deus.}
%\elcol{℟. And take not thy Holy Spirit from us.}{℟. Et Spíritum Sanctum tuum ne áuferas a nobis.}
\begin{rubric}
%1928 Rubric:
Then shall follow the Collect(s) for the Day, except when the Communion Service is read; and then the Collect(s) for the Day shall be omitted here.
\end{rubric}
\vspace{-2ex}
\subby{A Collect for Peace}
\lett{D}{eus,} auctor pacis et amator, quem nosse vivere, cui servire regnare est: Protege ab omnibus impugnationibus supplices tuos; ut qui in tua defensione confidimus, nullius hostilitatis arma timeamus, per potentiam Jesu Christi Domini nostri. \textit{Amen.}
%\vspace{-1ex}
\subby{A Collect for Grace}
\lett{D}{\smash{ó}mine} Sancte, Pater Omnipotens, æterne Deus, qui nos ad principium hujus diei pervenire fecisti; Tua nos hodie salva virtute; et concede ut in hac die ad nullum declinemus peccatum, nec ullum incurramus periculum; sed semper ad tuam justitiam faciendam omnis nostra actio tuo moderamine dirigatur. Per Jesum Christum Dominum nostrum. \textit{Amen.}
%Conclusion restored from the Breviary tradition:
%Rubric here already specified in the rubrics page.
%\begin{rubric}
%    The Minister begins with the \emph{Dóminus vobíscum} only if he be in Major Orders. Otherwise, he shall use the second option.
%\end{rubric}
%\vspace{-1ex}
\subby{Conclusion}
%It was difficult to determine whether to use `my' or `our' for those not in Major Orders. Since both are present in the English liturgy, and the beginning uses `our' we decided to remain consistent and use `our'.
℣. Dóminus vobíscum.\par
\textit{vel,} Dómine, exáudi oratiónem nostram.

℟. Et cum spíritu tuo.\par
\textit{vel,} Et clamor noster ad te véniat.

℣. Benedicámus Dómino (allelúja, allelúja.)

℟. Deo grátias (allelúja, allelúja.)

℣. Fidélium ánim{\ae} {\ding{64}} per misericórdiam Dei requiéscant in pace.

℟. Amen.
 \begin{rubric}
     The following Prayers shall be omitted here when the Great Litany is said, and may be omitted when the Holy Communion is to follow.
 \end{rubric}
 %The American Tradition experiences a growth of options after the Third Collect. To accommodate this growth, longer than the English Prayer Book, the Minister is given great liberty to remove any of the following prayers. While this may not be the best method to prune an overgrown tree, we do not believe ourselves to be best-suited to choose an alternative.
\begin{rubric}
    And \textsc{Note}, That the Minister may here end the Morning Prayer with such intercessions taken out of this Book, as he shall think fit, or with the Grace.
\end{rubric}
\vspace{-2ex}
\subby{After the Third Collect}
\begin{rubric}
    A Prayer for one's nation (p. \pageref{prayers}) is here said.
\end{rubric}
\vspace{-2ex}
\subbysub{A Prayer for the Clergy and People}
\lett{O}{mnipotens} sempiterne Deus, qui facis mirabilia magna solus; Prætende super famulos tuos Pontifices nostros, et Parochos, et super cunctos Greges illis commissos, Spiritum gratiæ salutaris; et ut in veritate tibi complaceant, perpetuum eis rorem tuæ benedictionis infunde. Hoc quæsumus Domine, donare digneris, in honorem Advocati et Mediatoris nostri Jesu Christi. \textit{Amen.}
\begin{rubric}
%Move the rubric from Prayers to here:
    Additional Prayers (p. \pageref{prayers}) may be said here.
\end{rubric}
\vspace{-2ex}
\subbysub{Oratio pro omnibus hominibus cujuscunque conditionis}
\lett{D}{eus,} humani generis Creator et Conservator, supplices te rogamus pro omnibus hominibus cujuscunque conditionis, ut iis vias tuas, et omnibus gentibus salutare tuum, ostendere digneris. Et præcipue pro bono statu Ecclesiæ Catholicæ; quam bono Spiritu tuo, quæsumus, ita dirigas et gubernes, ut omnes qui se Christianos profitentur, in viam veritatis ducti, in unitate spiritus, in vinculo pacis, et in justitia vitæ fidem teneant. Postremo paternæ bonitati tuæ omnes illos commendamus qui quovis modo, in mente, corpore, aut rebus, anguntur aut laborant; [* eos præsertim pro quibus orationes nostræ poscuntur,] ut eos, prout variæ eorum requirunt necessitates, consolari et relevare digneris, et fortitudinem in patiendo, et felicem e pressuris exitum concedens. Quod propter Jesum Christum te rogamus. \textit{Amen.}
\subbysub{Gratiarum Actio generalis}
\begin{rubric}
    The General Thanksgiving may be said by the Congregation with the Minister.
\end{rubric}
\lett{O}{mnipotens} Deus, Pater omnium misericordiarum, nos indigni famuli tui humillime et ex animo tibi gratias agimus pro omni bonitate tua et benignitate erga nos et omnes homines: (eos præcipue qui jam volunt tibi confiteri et gratias agere pro misericordia tua nuper in se collata.) Benedicimus tibi pro creatione et conservatione nostra et omnibus hujusce vitæ bonis; sed maxime pro amore tuo inæstimabili, quo mundum per Dominum nostrum Jesum Christum redemisti; insuper pro gratiæ instrumentis et spe gloriæ. Quæ omnia beneficia tua da nobis ita, ut justum est, sentire, ut cordibus vere gratis, laudem tuam non loquendo tantum, sed vivendo annuntiemus; dum servitio tuo penitus devoti, coram te in sanctitate et justitia omnibus diebus vitæ nostræ ambulemus. Per Jesum Christum Dominum nostrum, cui tecum, in unitate Spiritus Sancti, sit omnis honor et gloria, per omnia sæcula sæculorum. \textit{Amen.}
\begin{rubric}
%Move the rubric from Thanksgivings to here:
    The Thanksgivings (p. \pageref{thanksgiving}) may here be offered.
\end{rubric}
\vspace{-2ex}
\subbysub{Oratio Sancti Chrysostomi.}
\lett{O}{mnipotens} Deus, qui nobis gratiam dedisti, ut hoc tempore concordes te communi supplicatione invocemus; et duobus vel tribus congregatis in Nomine tuo te quod rogaverint daturum polliceris: Vota, quæsumus, Domine, et petitiones famulorum tuorum, jam prout illis maxime expediat, prosequere; nobis in hoc sæculo veritatis tuæ cognitionem et in futuro vitam largiens æternam. \textit{Amen.}
\subbysub{2 ad Corinthios, cap. xiii.}
\lett{G}{ratia} Domini nostri Jesu Christi, et charitas Dei, et communicatio Sancti Spiritus, sit cum omnibus nobis in sæcula sæculorum. \textit{Amen.}

\begin{center}
    \textsc{Finis Ordinis Matutinarum per totum Annum.}
\end{center}