\fancyhead[C]{{\LARGE Daily Office}}
\fancyhead[RO,LE]{\textit{Fore-Office}}
\section{Fore-Office}
\selectlanguage{latin}
\begin{secrubric}
    In principio Matutinarum Minister clara voce legat unam aliquam sive plures ex Sacræ Scripturæ Sententiis quæ consequuntur. Et postea dicat quod infra has Sententias scriptum est.
\end{secrubric}
\begin{secrubric}
    %This rubric is very intentionally placed in obedience (and good sense) of the Church that prayers to the saints ought not to be omitted. Without this rubric, the bare minimum could result in an Office without any intercession of the saints. Given how quickly parishes (sadly) and laymen (sensibly) will scale down their liturgy to the bare minimum, it seems necessary to have this rubric as a safeguard.
    On any day, save a Day of Fasting or Abstinence, or on any day when the Litany or Holy Communion is immediately to follow, the Minister may, at his discretion, pass at once from the Sentences to the Lord's Prayer and Angelic Salutation, which may never be omitted.
\end{secrubric}
\begin{multicols}{2}
\begin{inhead}
General
\end{inhead}\noindent
%Provided here are the original opening sentences of the 1662 BCP. It is a difficult balance, as discussed in the Preface, balancing this American Prayer Book with the needs, desires, and practices of Western Rite people and parishes. The Opening Sentences themselves are very English. The English Reformation saw a trend of making changeable parts of the liturgy unchangeable. This is seen excellently in the replacement of seasonal antiphons with standardised collects.
%The American tradition, along with the Canadian and elsewhere, eventually saw an attempt to make the liturgy more changeable according to the liturgical season, seen here in the Sentences. In America, this involved moving the standard Sentences (of a penitential character) to Lent.
%The Opening Sentences come from the Sacerdotal Versicles which introduced Lauds. Cranmer accommodated this to the Fore-Office's penitential nature, which was itself both a replacement for the introduction to the Hours and also good sense to begin worship and sacrifice with a confession of sins unto almighty God. Later on, notably in the American 1928 and Proposed 1928, the Confession became optional and its character as an introduction to the Office as a whole was re-emphasised, notably with seasonal collects. While the Prayer Book Tradition gets its distinctive beauty and efficacy by reduced options, propers, and complicated rubrics, it seems fitting to continue this emphasis by supplying from the Sarum the proper Opening Sentences for general kinds of Feast Days and for the Feast Days in the Proper.
%Given its pedigry and that other Western Rite material has provided a `General' section of the Opening Sentences, and this is actively used in AWRV parishes, I do not think it an innovation to provide it here. And given the beauty and interest among AWRV clergy and laymen in these aesthetic elements of the English tradition, it makes sense to me to make these the `General' Sentences. I will say, however, by way of suggestion and not rubric, that these should not be used if the Confession \& Absolution are omitted but rather the seasonal prayers.
	Si impius egerit p{\oe}nitentiam pro omnibus peccatis suis qu{\ae} operatus est, et custodierit omnia pr{\ae}cepta mea, et fecerit judicium et justiciam, vita vivet et non morietur. Omnium iniquitatum ejus quas operatus est, non recordabor: dicit Dominus.\vr{Ezech. 18:27}
	
	Iniquitatem meam agnosco, et peccatum meum contra me est semper.\vr{Psa. 51:3}
	
	Averte faciem tuam a peccatis meis: et omnes iniquitates meas dele.\vr{Psa. 51:9}
	
	Scindite corda vestra, et non vestimenta vestra, et convertimini ad Dominum Deum vestrum, quia benignus et misericors est, patiens et mult{\ae} clementi{\ae}, qui se ab inferendo malo contineat.\vr{J{\oe}l 2:13}
	
	Tui Domini Dei nostri est misericordia et propiciatio, quia recessimus a te, et non audivimus vocem Domini Dei nostri, ut ambularemus in lege ejus.\vr{Dan 9:9-10}
	
	Corripe nos, Domine, veruntamen in judicio et non in furore tuo, ne forte ad nihilum redigas nos.\vr{Jer 10:24; Ps 6:1}
	
	P{\oe}nitentiam agite; appropinquat enim regnum c{\oe}lorum.\vr{Mt 3:2}
	
	Surgam, et ibo ad patrem meum, et dicam ei: Pater, peccavi in c{\oe}lum et coram te. Jam non sum dignus vocari filius tuus.\vr{Lk 15:18-19}
	
	Non intres in judicium cum servo tuo, Domine, quia non justificabitur in conspectu tuo omnis vivens.\vr{Ps 143:2}
	
	Si nos peccati expertes esse dicimus, fallimus nos ipsos, nec est in nobis veritas.\vr{1 Jn 1:8-9}
\begin{inhead}
Morning
\end{inhead}\noindent
    %From 1928 Morning Prayer:
    The \textsc{Lord} is in his holy temple: let all the earth keep silence before him.\vr{Hab 2:20}\par
    I was glad when they said unto me, We will go into the house of the \textsc{Lord}.\vr{Ps 122:1}\par
    Let the words of my mouth, and the meditation of my heart, be alway acceptable in thy sight, O \textsc{Lord}, my strength and my redeemer.\vr{Ps 19:14}\par
    O send out thy light and thy truth, that they may lead me, and bring me unto thy holy hill, and to thy dwelling.\vr{Ps 43:3}\par
    Thus saith the high and lofty One that inhabiteth eternity, whose name is Holy; I dwell in the high and holy place, with him also that is of a contrite and humble spirit, to revive the spirit of the humble, and to revive the heart of the contrite ones.\vr{Is 57:15}\par
    The hour cometh, and now is, when the true worshippers shall worship the Father in spirit and in truth: for the Father seeketh such to worship him.\vr{Jn 4:23}\par
    Grace be unto you, and peace, from God our Father, and from the Lord Jesus Christ.\vr{Phil 1:2}\par
\begin{inhead}
Evening
\end{inhead}\noindent
%From 1928 Evening Prayer:
    The \textsc{Lord} is in his holy temple: let all the earth keep silence before him.\vr{Hab 2:20}\par
    \textsc{Lord}, I have loved the habitation of thy house, and the place where thine honour dwelleth.\vr{Ps 26:8}\par
    Let my prayer be set forth in thy sight as the incense; and let the lifting up of my hands be an evening sacrifice.\vr{Ps 141:2}\par
    O worship the \textsc{Lord} in the beauty of holiness; let the whole earth stand in awe of him.\vr{Ps 96:9}\par
    Let the words of my mouth, and the meditation of my heart, be alway acceptable in thy sight, O \textsc{Lord}, my strength and my redeemer.\vr{Ps 19:14-15}\par
%From 1918 Morning Prayer:
    Seek ye the Lord while he may be found, call ye upon him while he is near: let the wicked forsake his way and the unrighteous man his thoughts: and let him return unto the Lord, and he will have mercy upon him: and to our God, for he will abundantly pardon.\vr{Is 55:6-7}
%For the seasonal schema, the first option is from the 1928 Morning, the second from 1928 Evening Prayer, and the third from either the 1962 or 1918. If the Canadian is the same as the American, another is chosen.

\begin{inhead}
Advent
\end{inhead}\noindent
    Repent ye, for the Kingdom of heaven is at hand.\vr{Mt 3:2}
    \par
    Watch ye, for ye know not when the master of the house cometh, at even, or at midnight, or at the cock-crowing, or in the morning: lest coming suddenly he find you sleeping.\vr{Mk 13:35-6}
    \par
    Prepare ye the way of the \textsc{Lord}, make straight in the desert a highway for our God.\vr{Is 40:3}

\begin{inhead}
Christmas
\end{inhead}\noindent
    Behold, I bring you good tidings of great joy, which shall be to all people. For unto you is born this day in the city of David a Saviour, which is Christ the Lord.\vr{Lk 2:10-11}
    \par
    Behold, the tabernacle of God is with men, and he will dwell with them, and they shall be his people, and God himself shall be with them, and be their God.\vr{Rev 21:3}
    \par
    %From the Proposed 1928:
    Herein was the love of God manifested in us, that God hath sent his only begotten Son into the world, that we might live through him. \vr{1 Jn 4:9}
    \par

\begin{inhead}
Epiphany
\end{inhead}\noindent
%From 1928 Morning Prayer
    From the rising of the sun even unto the going down of the same my Name shall be great among the Gentiles; and in every place incense shall be offered unto my Name, and a pure offering: for my Name shall be great among the heathen, saith the \textsc{Lord} of hosts.\vr{Mal 1:11}
    \par
%From 1928 Evening Prayer
    And the Gentiles shall come to thy light, and kings to the brightness of thy rising.\vr{Is 60:3}\par
%From 1962 Morning Prayer
    The earth shall be filled with the knowledge of the glory of the \textsc{Lord}, as the waters cover the sea.\vr{Hab 2:14}
\begin{inhead}
Lent
\end{inhead}\noindent
%To avoid overlap, we are avoiding any Lenten Sentences which are already provided in `General'.
%From Sarum (Lent I & II):
    He hath delivered me from the snare of the hunter, and from the sharp word.\vr{Ps 91:3 LXX}\par
    He shall deliver thee from the snare of the hunter. And from the noisome pestilence.\vr{Ps 91:3}

\begin{inhead}
Passiontide
\end{inhead}\noindent
%From 1962 Canadian:
    God commendeth his love toward us, in that, while we were yet sinners, Christ died for us.\vr{Rom 5:8}\par
%From the English Proposed 1928 BCP
    Is it nothing to you, all ye that pass by? behold, and see if there be any sorrow like unto my sorrow which is done unto me, wherewith the \textsc{Lord} hath afflicted me.\vr{Lam 1:12}\par
%From Sarum (Passion Sunday):
Draw nigh unto my soul and save it. O deliver me because of mine enemies.\vr{Ps 69:18}

\begin{inhead}
Eastertide
\end{inhead}\noindent
    He is risen. The Lord is risen indeed.\vr{Mk. 16:6}\par
    Thanks be to God, which giveth us the victory through our Lord Jesus Christ.\vr{1 Cor. 15:57}\par
    If ye then be risen with Christ, seek those things which are above, where Christ sitteth on the right hand of God.\vr{Col 3:1}

\begin{inhead}
Ascensiontide
\end{inhead}\noindent
    Seeing that we have a great High Priest, that is passed into the heavens, Jesus the Son of God, let us come boldly unto the throne of grace, that we may obtain mercy, and find grace to help in time of need.\vr{Heb 4:14,16}\par
    
    Christ is not entered into the holy places made with hands, which are the figures of the true; but into heaven itself, now to appear in the presence of God for us.\vr{Heb 9:24}\par

    %From Sarum (Ascension Thursday):
    I ascend to my Father, and your Father. And to my God, and your God. Alleluia.\vr{Jn 20:17}

\begin{inhead}
Whitsuntide
\end{inhead}\noindent
    Ye shall receive power, after that the Holy Ghost is come upon you: and ye shall be witnesses unto me both in Jerusalem, and in all Jud{\ae}a, and in Samaria, and unto the uttermost part of the earth.\vr{Acts 1:8}\par
    
    There is a river, the streams whereof shall make glad the city of God, the holy place of the tabernacles of the Most High.\vr{Ps 46:4}\par

    %From 1962 Morning Prayer (which uses the ASV here, interestingly):
    The love of God hath been shed abroad in our hearts through the Holy Spirit which was given unto us.\vr{Rom 5:5}
\begin{inhead}
Trinity Sunday
\end{inhead}\noindent
    Holy, holy, holy, Lord God Almighty, which was, and is, and is to come.\vr{Rev 4:8}
    \par
    Holy, holy, holy, is the \textsc{Lord} of hosts: the whole earth is full of his glory. \vr{Is 6:3}
    \par
%From 1962 Morning Prayer:
    God is love; and he that abideth in love abideth in God and God in him.\vr{1 Jn 4:16}

%There has been an attempt to avoid novelty in this Prayer Book, especially in the Sentences where novelty has become common. However, the trajectory of the American Prayer Book tradition seems both good and clear: seasonal Sentences based on the feast day. Therefore, the Opening Sentences (Sacerdotal Versicles) from the Sarum Office's Common is thus provided.
\begin{inhead}
Apostle
\end{inhead}\noindent
Thou hast given an heritage unto those that fear thy Name, O Lord.\vr{Ps 61:5}

\begin{inhead}
Martyr
\end{inhead}\noindent
Thou hast crowned him, O Lord, with glory and worship.\vr{Ps 8:5}

\begin{inhead}
Bishop, Confessor, or Doctor
\end{inhead}\noindent
The righteous shall flourish like a palm-tree: and shall spread abroad like a cedar in
Libanus.\vr{Ps 92:11}

\begin{inhead}
Abbot or Monk
\end{inhead}\noindent
O ye holy and humble men of heart, bless ye the Lord: praise and exalt him above all for ever.\vr{Pr. of Az. 65}

\begin{inhead}
Virgin
\end{inhead}\noindent
In thy grace and in thy beauty, go forth, ride prosperously and reign.\vr{Ps 45:4}

\begin{inhead}
Matron
\end{inhead}\noindent
God is in the midst of her, therefore shall she not be removed : God shall help her, and that right early.\vr{Ps 46:5}

\begin{inhead}
Blessed Virgin Mary
\end{inhead}\noindent
Thou art the holy Mother of God, O Mary, ever Virgin.\vr{cf. Is 7:14}

\begin{inhead}
Blessed Sacrament
\end{inhead}\noindent
I have eaten my honeycomb with my honey. I have drunk my wine with my milk. (Alleluia.)\vr{Song 5:1}

\begin{inhead}
Dedication of a Church
\end{inhead}\noindent
%From the Sarum Versicle for the Anniversary of the Dedication of a Church
	My house shall be called the house of prayer.\vr{Mt. 21:13}
\end{multicols}
\subby{Exhortation to Penitence}
\lett{D}{ilectissimi} fratres, Sacra Scriptura multis in locis nos commonefacit, ut multiplicia peccata nostra et iniquitatem nostram agnoscamus et confiteamur; nec ea in conspectu Dei Omnipotentis, Patris nostri cœlestis, audeamus dissimulare; sed animo demisso, humili, contrito, et obedienti confiteamur; quatenus, per infinitam bonitatem et misericordiam ejus, indulgentiam consequamur. Et quanquam in omni tempore peccata nostra coram Deo humiliter agnoscenda sunt, tunc tamen præcipue hoc agendum est cum in publicum cœtum ideo convocati sumus, ut pro magnis ab eo collatis beneficiis gratias referamus, ut ejus laudes dignas annuntiemus, et sanctissimum ejus Verbum audiamus, et ut quicquid animorum status aut corporis requirit ab eo postulemus. Quapropter vos omnes, qui adestis, rogo et obsecro ut, puro corde, et voce humili, ad thronum gratiæ cœlestis mecum accedatis, ita me præeunte dicentes:
\begin{inhead}
    or,
\end{inhead}
\par%
\noindent%
%From the English Proposed 1928. The American tradition is horribly confused on the point of the Exhortation to Penitence (even more than its general confusion regarding penitence!). Each edition has a different approach. It is understandable that the same long exhortation is undesirable. However, removing the exhortation for a short sentence seems to betray a loss of the plot. Not only does this shorter exhortation meet the challenge sought by the 1928 (more variety, not as long as the longer exhortation but not too short) even better, but it incorporates a greater sense of the communion of the saints both on earth and heaven. It lays out clearly and briskly the purposes of worship, the Office as the Morning and Evening sacrifice offered by every Christian, and is simply beautiful. It is currently in wide use among Christians in the UK and even in these United States. For these reasons, and its presence in the Prayer Book Tradition of today and yesteryear, I include it.
\needspace{3\baselineskip}
\lett{B}{eloved,} we are come together in the presence of Almighty God and of the whole company of heaven to offer unto him through our Lord Jesus Christ our worship and praise and thanksgiving; to make confession of our sins; to pray, as well for others as for ourselves, that we may know more truly the greatness of God's love and shew forth in our lives the fruits of his grace; and to ask on behalf of all men such things as their well-being doth require. Wherefore let us kneel in silence, and remember God's presence with us now.

\begin{rubric}%It seems too bold to omit the possibility of the novel introductory sentence entirely, so a longer and a shorter exhortation are provided with the sentence provided in the rubric.
	If the exhortation be omitted, to be said in its place: \emph{Let us humbly confess our sins unto Almighty God.}
\end{rubric}

\subby{General Confession}
\begin{rubric}
    To be said by the whole Congregation, after the Minister, all kneeling.
\end{rubric}
%The American edition corrupts the grammar of this prayer (replacing `them' with `those'). It's a shame it would be too jarring to change. The same is even more true for the Lord's Prayer.
\needspace{3\baselineskip}

\lett{A}{lmighty} and most merciful Father; We have erred, and strayed from thy ways like lost sheep. We have followed too much the devices and desires of our own hearts. We have offended against thy holy laws. We have left undone those things which we ought to have done; And we have done those things which we ought not to have done; And there is no health in us. But thou, O Lord, have mercy upon us, miserable offenders. Spare thou those, O God, who confess their faults. Restore thou those who are penitent; According to thy promises declared unto mankind in Christ Jesus our Lord. And grant, O most merciful Father, for his sake; That we may hereafter live a godly, righteous, and sober life, To the glory of thy holy Name. Amen.
\subby{Declaration of Absolution}
\begin{rubric}
    To be made by the Priest alone, standing; the People still kneeling.
\end{rubric}
\begin{rubric}
    But \textsc{Note}, That the Priest, at his discretion, may use, instead of what follows, the Absolution from the Order for the Holy Communion.
\end{rubric}
\lett{A}{lmighty} God, the Father of our Lord Jesus Christ, who desireth not the death of a sinner, but rather that he may turn from his wickedness and live, hath given power, and commandment, to his Ministers, to declare and pronounce to his people, being penitent, the Absolution and Remission of their sins. He pardoneth and {\ding{64}} absolveth all those who truly repent, and unfeignedly believe his holy Gospel.
\par
Wherefore let us beseech him to grant us true repentance, and his Holy Spirit, that those things may please him which we do at this present; and that the rest of our life hereafter may be pure and holy; so that at the last we may come to his eternal joy; through Jesus Christ our Lord. \textit{Amen.}
\begin{inhead}
    or,
\end{inhead}
%From 1928 BCP EP.
\lett{T}{he} Almighty and merciful Lord grant you Absolution {\ding{64}} and Remission of all your sins, true repentance, amendment of life, and the grace and consolation of the Holy Spirit. Amen.
%The American tradition does not supply a lay `absolution', but in our times it is greatly needful. Therefore the 1662 is supplied instead of borrowing from Compline. And only one provided. We don't need too many options.
\begin{rubric}
    If no Priest be present, the Minister saying the Service shall read the following, that Minister and the People still kneeling.
\end{rubric}
\lett{G}{rant,} we beseech thee, merciful Lord, to thy faithful people pardon and peace, that they may be cleansed from all their sins, and serve thee with a quiet mind. Through Jesus Christ thy Son our Lord; who liveth and reigneth with thee in the unity of the Holy Spirit, one God, world without end. \textit{Amen.}

\subby{Lord's Prayer \& Angelic Salutation}
\begin{rubric}
    Then the Minister shall kneel, and say the Lord's Prayer \& Angelic Salutation; the People still kneeling, and repeating them with him, both here, and wheresoever else they are used in Divine Service.
\end{rubric}
%The traditional Office, before the English liturgical reform, was prefaced with the Pater Noster, Ave Maria, and Credo. The addition of the Ave Maria not only restores the Catholic and Western practice but also ensures that the intercession of the saints will be present in every Office (not unrubrically truncated, at least).
\needspace{3\baselineskip}
\lett{P}{ater} noster, qui es in cœlis, Sanctificetur Nomen tuum. Adveniat regnum tuum. Fiat voluntas tua, Sicut in cœlo, et in terra. Panem nostrum quotidianum da nobis hodie. Et dimitte nobis debita nostra, Sicut et nos dimittimus debitoribus nostris. Et ne nos inducas in tentationem; Sed libera, nos a malo: Quia tuum est regnum, Potentia, et gloria, In sæcula sæculorum. Amen.
\par\noindent
\needspace{3\baselineskip}
\lett{A}{ve} María, grátia plena; Dóminus tecum: benedícta tu in muliéribus, et benedíctus fructus ventris tui Jesus. Sancta María, Mater Dei, ora pro nobis peccatóribus, nunc et in hora mortis nostræ. Amen.
