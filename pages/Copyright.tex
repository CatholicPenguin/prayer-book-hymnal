\frontmatter
\setcounter{page}{1}
\phantomsection
\addcontentsline{toc}{chapter}{Copyright}
\fancyhead[C]{{\LARGE Copyright}}

\noindent
\copyright 2025 Apologia Anglicana, LLC\par\noindent
Edited by Augustine Watson (Augustine@ApologiaAnglicana.org), with the irreplaceable assistance of Ryan May.
The Prayer Book Hymnal, published by Apologia Anglicana, LLC, is licenced under the \href{https://creativecommons.org/licenses/by-sa/4.0/}{Creative Commons Attribution-ShareAlike 4.0 International} (\url{https://creativecommons.org/licenses/by-sa/4.0/}) licence.

\begin{center}
	\textsc{Please note that this is not currently an official text of the Antiochian Western Rite Vicariate.}
\end{center}

While great care was taken to use only works in the public domain in these United States of America, there were a few cases where this was not possible. We are thankful to have such great giants upon whose shoulders we stand. These works are used in this Book according to fair use.\par
Davis, C. Lance, ed. \textit{The Anglican Office Book}. Chester: Whithorn Press, 2023.\par\noindent
\textit{This was indispensable, for the adapted Office portions of the Feasts of Our Lady of Walsingham and St. Joseph of Arimathea.}\par
%Services for the Martyrdom of S. Charles, King and Martyr, transcribed by Mr Thomas J. W. Mason (\url{https://anglicanhistory.org/charles/texts.html})\par\noindent
%\textit{This was indispensable, for the adapted Mass portions of the Feast of Tsar St. Nicholas.}\par


And a special thanks must be given to GregoBase (\url{https://gregobase.selapa.net/}), without which the notation for the Invitatory and Office Hymns simply would not be feasible. I am especially thankful to the transcribers Albert Bloomfield, Benedictine monk, BruceL, Mateusz Ciesiółka, Dominique CROCHU, pcouderc, Pierre François, Harpa Dei, HdP, Andrew Hinkley, Nguyễn Thiện Tuấn Hoàng, Joerg Hudelmaier, Jakub Jelínek, MarkM, marteo, Br Michael, Br Michael OSB, rcarey, and Christopher Tatum.
\clearpage
\footnotesize
\section*{Devotional Hymn Copyright}
\begin{multicols}{2}

\hcopyrightsection{A Mighty Fortress is Our God}
\par\noindent
Words: Martin Luther, 1529. Translation Frederic Henry Hedge, 1853.
\par\noindent
Music: `Ein Feste Burg (Isorhythmic)' Martin Luther, 1529.  Setting: ``Common Service Book'' (ULCA), 1917, alt.
\par\noindent
Copyright: public domain. This score is a part of the Open Hymnal Project, 2013 Revision.
\par\noindent
Music source: `Common Service Book with Hymnal', ULCA 1918 Hymn 195, alt. to match syncopation of Hedge translation
\par\noindent
Lyric source: Presbyterian Hymnal, Revised, 1911 Hymn 122.

\hcopyrightsection{Alleluia, Sing to Jesus!}
\par\noindent
Words: William Chatterton Dix, 1867. 
\par\noindent
Music: `HyFrydol' Rowland H. Prichard, 1830.  Setting: ``The English Hymnal'', 1906.
\par\noindent
Arrangement: English hymnal, 1906 assumes RV Williams.
\par\noindent
Copyright: public domain. This score is a part of the Open Hymnal Project, 2006 Revision.
\par\noindent
Music source: `The English Hymnal' 1906, Hymn 563. 

\hcopyrightsection{All Creatures of Our God and King}
\par\noindent
Words: Francis of Assisi circa 1225. Translated by William H. Draper, 1919. 
\par\noindent
Music: `Lasst Uns Erfreuen' from Geistliche Kirchengesäng, Köln, 1623. Setting: Ralph Vaughan Williams, 1906.
\par\noindent
copyright: public domain. This score is a part of the Open Hymnal Project, 2005 Revision.
\par\noindent
Music source: `Lutheran Worship' Hymnal, 1982 Hymn 436.

\hcopyrightsection{Amazing Grace}
\par\noindent
Words: John Newton, 1779.  last verse author unknown, before 1829.
\par\noindent
Music: `New Britain' James P. Carrell and David L. Clayton, 1831.  Setting: Edwin Othello Excell, 1900.
\par\noindent
Copyright: public domain. This score is a part of the Open Hymnal Project, 2006 Revision.
\par\noindent
Music source: ``Joy to the World'', 1915, Hymn 209 Ed. E. O. Excell  explicit about arrangement.  Christian Classics Ethereal Library (\url{ccel.org}) Hymnary  Think setting from ``Make His praise glorious'', 1900.  Ed. Edwin Othello Excell.

\hcopyrightsection{And did those feet in ancient time}
\par\noindent
Words: First (Original) Verse, Will­iam Blake, c. 1804. Second Verse, Kenneth G.H. Bryant, 2017.
\par\noindent
Music: Je­ru­sa­lem (Par­ry) Charles H. H. Par­ry, 1916.
\par\noindent
Arrangement: Augustine Watson, 2025.
\par\noindent
Copyright: \href{https://creativecommons.org/publicdomain/zero/1.0/}{CC0 Public Domain Dedication} (\href{https://imslp.org/wiki/Jerusalem_(Parry,_Charles_Hubert_Hastings)}{Bryant dedicated his words to the public domain}).

\hcopyrightsection{Be Thou My Vision}
\par\noindent
Words: Attr. Dallan Forgaill, 8th Century.  Translated by Mary Byrne, 1905 and Eleanor Hull, 1912. 
\par\noindent
Music: `Slane' Traditional Irish.  Setting: Mark Hamilton Dewey, 2007.
\par\noindent
Copyright: public domain. This score is a part of the Open Hymnal Project, 2008 Revision. 
\par\noindent
All portions of the setting that were not already public domain were released to the public domain by the arranger on 27 July 2007. He already had released the parts and the versification (except for a few changes in the third verse, which he released to the public domain in 2007) to the public domain in 2006.

\hcopyrightsection{Come Thou Fount of Every Blessing}
\par\noindent
Words: Robert Robinson, 1758.  Music: `Nettleton' Asahel Nettleton, 1812. 
\par\noindent
Setting: ``The Evangelical Hymnal'', 1921.
\par\noindent
Copyright: public domain. This score is a part of the Open Hymnal Project, 2010 Revision.
\par\noindent
Music source: The Evangelical Hymnal, 1921, Hymn 256.  Almost the same as The Presbyterian Hymnal, 1874 Hymn 94.

\hcopyrightsection{Crown Him with Many Crowns}
\par\noindent
Words: Verses 1, 4, 5, 6, \& 9: Matthew Bridges, The Passion of Jesus, 1852. Verses 2 \& 3: Godfrey Thring, Hymns and Sacred Lyrics, 1874. 
\par\noindent
Music: `Diademata' George J. Elvey, 1868.  Setting: ``Appendix to Hymns Ancient and Modern'', 1869.
\par\noindent
Copyright: public domain. This score is a part of the Open Hymnal Project, 2008 Revision.
\par\noindent
Music source: ``Appendix to Hymns Ancient and Modern'', 1869 Hymn 318. Ed. William H. Monk.  Unknown if the arrangement is Elvey's or Monk's.

\hcopyrightsection{Dear Christians, One and All Rejoice}
\par\noindent
Words: Martin Luther, 1523.  Translated by Richard Massie, 1854, alt. 
\par\noindent
Music: `Nun Freut Euch' attr. Martin Luther from Etlich Christlich Lider, Wittenberg, 1524. 
\par\noindent
Setting: Johann Hermann Schein, 1627.
\par\noindent
Copyright: public domain. This score is a part of the Open Hymnal Project, 2009 Revision.
\par\noindent
Music source: ``The Hymns of Martin Luther'' by Bacon, 1883.  Setting originally from ``Cantional oder Gesangbuch Augburgischer Konfession'' by Johann Hermann Schein.

\hcopyrightsection{Faith of Our Fathers}
\par\noindent
Words: Frederick W. Faber, 1849. Refrain by James G. Walton, 1874. 
\par\noindent
Music: `St. Catherine' Henri F. Hemy (1818-1888).  Setting: James G. Walton, 1874.
\par\noindent
Copyright: public domain. This score is a part of the Open Hymnal Project, 2006 Revision.
\par\noindent
Music source: The Evangelical Hymnal, 1921  Hymn 408.

\hcopyrightsection{I Need Thee Every Hour}
\par\noindent
Words: Annie Sherwood Hawks, 1872.
\par\noindent
Music: `I Need Thee Every Hour' Robert Lowry, 1872.  Setting: ``Pentecostal Hymns, No. 2'', 1898.
\par\noindent
Copyright: public domain. This score is a part of the Open Hymnal Project, 2012 Revision.
\par\noindent
Music source: ``Pentecostal Hymns, No. 2'' 1898 page 155.   ABC file contributed to the Open Hymnal by Samuel Cantrell, 18 Jan 2012.

\hcopyrightsection{I Sing the Mighty Power of God}
\par\noindent
Words: Isaac Watts, 1709.
\par\noindent
Music: `Ellacombe' from Gesangbuch der Herzogl. Hofkapelle, Wurttemberg, 1784. Setting: ``Amore Dei'', 1897.
\par\noindent
Copyright: public domain. This score is a part of the Open Hymnal Project, 2014 Revision.
\par\noindent
Music and Lyrics source: Hymnal ``Amore Dei'' (published 1897)  Hymn 115.  ABC file contributed to the Open Hymnal by Tobin Strong, 09 Jan 2014.

\hcopyrightsection{I Vow to Thee My Country}
\par\noindent
Words: Cecil Spring-Rice, 1859-1918.
\par\noindent
Music: `Thaxted' Gustav Holst, 1918.
\par\noindent
Arrangement: Augustine Watson, 2025, based on ``The Winchester Hymn Supplement: with Tunes'', 1928, in consultation with the arrangement by Paul Hayward, 2011.
\par\noindent
Copyright: \href{https://creativecommons.org/publicdomain/zero/1.0/}{CC0 Public Domain Dedication}.
\par\noindent
Music source: ``The Winchester Hymn Supplement: with Tunes'', 1928.

\hcopyrightsection{If God Had Not Been on Our Side}
\par\noindent
Words: Martin Luther, 1524.  Translation composite. 
\par\noindent
Music: `Wär Gott nicht mit uns diese Zeit (1537)' from Walter's Hymnal, 1537. 
\par\noindent
Setting: ``Evangelical Lutheran Hymn-Book'', 1931.
\par\noindent
Copyright: public domain. This score is a part of the Open Hymnal Project, 2009 Revision. Translation is public domain per Project Wittenberg: http://www.iclnet.org/pub/resources/text/wittenberg/hymns/ourside.txt

\hcopyrightsection{Immaculate Mary}
\par\noindent
Words: ``Catholic Church Hymnal'', 1905. Verses added by Augustine Watson, 2025 (put into the public domain).
\par\noindent
Music: Traditional tune.
\par\noindent
Setting: ``Catholic Church Hymnal'', 1905.
\par\noindent
Copyright: \href{https://creativecommons.org/publicdomain/zero/1.0/}{CC0 Public Domain Dedication}.

\hcopyrightsection{Immortal, Invisible, God Only Wise}
\par\noindent
Words: Walter Chalmers Smith, 1876. 
\par\noindent
Music: `St. Denio' or `Joanna' or `Palestrina' traditional Welsh found in ``Caniadau y Cyssegr'' by John Roberts, 1839.
\par\noindent
Setting: ``Caniadau y Cyssegr a'r Teulu'', 1878, alt.
\par\noindent
Copyright: public domain. This score is a part of the Open Hymnal Project, 2005 Revision.
\par\noindent
Music source: Episcopal Hymnal, 1940, Hymn 301.  Also in English Hymnal, 1906 Hymn 407 (page missing in my copy). Setting from ``Caniadau y Cyssegr a'r Teulu'', 1878 hymn 442, alt.

\hcopyrightsection{It Is Well with My Soul}
\par\noindent
Words: Horatio G. Spafford, 1873.
\par\noindent
Music and Setting: `It Is Well' or `Ville Du Havre' Philip P. Bliss, 1876.
\par\noindent
Copyright: public domain. This score is a part of the Open Hymnal Project, 2010 Revision.
\par\noindent
Music source: The Evangelical Hymnal, 1921 Hymn 208.

\hcopyrightsection{Joyful, Joyful, We Adore Thee}
\par\noindent
Words: Henry J. van Dyke, 1907. 
\par\noindent
Music: `Ode to Joy' Ludwig van Beethoven; Adapted by Edward Hodges, 1824. 
\par\noindent
Setting: ``The Methodist Hymnal'', 1905.
\par\noindent
Copyright: public domain. This score is a part of the Open Hymnal Project, 2005 Revision.
\par\noindent
Music source: The Methodist Hymnal, 1905 Hymn 160.

\hcopyrightsection{Let All Mortal Flesh Keep Silence}
\par\noindent
Words: from Liturgy of St. James, 4th Century.  Translated by Gerard Moultrie, 1864. 
\par\noindent
Music: `Picardy' traditional French.  Setting: ``The English Hymnal'', 1906, alt.
\par\noindent
Copyright: public domain. This score is a part of the Open Hymnal Project, 2008 Revision.
\par\noindent
Music source: The Episcopal Hymnal, 191x Hymn 339 and English Hymnal 1906 Hymn 318.

\hcopyrightsection{Lift High the Cross}
\par\noindent
Words: George W. Kitchin (1827-1912). Modified by Michael R. Newbolt, 1916. 
\par\noindent
Music: `Crucifier' Sydney H. Nicholson, 1916.  Setting: ``Hymns Ancient and Modern'', 1922.
\par\noindent
Copyright: public domain. This score is a part of the Open Hymnal Project, 2006 Revision.
\par\noindent
Music source: `Lutheran Worship' Hymnal, 1982 Hymn 311.  From ``Hymns Ancient and Modern'', Standard Edition Hymn 745, 1922.

\hcopyrightsection{Look Down, O Lord, from Heaven Behold}
\par\noindent
Words: Martin Luther, 1524. Translated by Frances Elizabeth Cox, 1864. 
\par\noindent
Music: `Ach Gott vom Himmel' from Erfurt Enchiridion, 1524.  Setting: Karl August Haupt, 1869. This tune is also used in Klug's 1543 hymnal, and that hymnal is sometimes incorrectly referenced as the source
\par\noindent
Copyright: public domain. This score is a part of the Open Hymnal Project, 2009 Revision.
\par\noindent
Music source: ``The Hymns of Martin Luther'' by Bacon, 1883.

\hcopyrightsection{Lord, Keep Us Steadfast in Thy Word}
\par\noindent
Words: Martin Luther, 1541.  Translated by Catherine Winkworth, 1863. 
\par\noindent
Music: `Erhalt Uns, Herr, bei deinem Wort' from Klug's ``Geistliche Lieder auffs new gebessert'', 1543. 
\par\noindent
Setting: Hans Leo Hassler, unknown date.
\par\noindent
Copyright: public domain. This score is a part of the Open Hymnal Project, 2008 Revision.
\par\noindent
Music source: Evangelical Lutheran Hymn Book (LCMS), Edition of 1931 Hymn 274.

\hcopyrightsection{Now Thank We All Our God}
\par\noindent
Words: Martin Rinkart, c.1636. Translated Catherine Winkworth, 1856.  Music: `Nun Danket' Johann Crüger, 1647. 
\par\noindent
Setting: ``Evangelical Lutheran Hymn-Book'', 1931.
\par\noindent
Copyright: public domain. This score is a part of the Open Hymnal Project, 2006 Revision.
\par\noindent
Music source: `Evangelical Lutheran Hymn-Book', Edition of 1931 Hymn 64.

\hcopyrightsection{Our God, Our Help in Ages Past}
\par\noindent
Words: Isaac Watts, 1719.  Music: `St. Anne' William Croft, 1708.
\par\noindent
Setting: composite found in ``The Lutheran Hymnary'', 1913.
\par\noindent
Copyright: public domain. This score is a part of the Open Hymnal Project, 2005 Revision.
\par\noindent
Music source: `The Lutheran Hymnary', 1913, Hymn 261. Arrangement is composite, first half from ``Hymns Ancient and Modern'', 1869, Hymn 231. ed  William Henry Monk. Second half is from ``The Hymnal Companion to the Book of Common Prayer'', 1890, Hymn 279. ed. Charles Vincent, D.J. Wood, John Stainer

\hcopyrightsection{Out of the Deep I Cry to Thee}
\par\noindent
Words: Martin Luther, 1524. Translated by Arthur Tozer Russell (1806-1874). 
\par\noindent
Music: `Aus Tiefer Not (Luther)' or `Af Dybsens Nød' Martin Luther from Erfurt Enchiridion, 1524. 
\par\noindent
Setting: Johann Sebastian Bach, 1725.
\par\noindent
Copyright: public domain. This score is a part of the Open Hymnal Project, 2009 Revision.
\par\noindent
Music source: ``The Hymns of Martin Luther'' by Bacon, 1883.

\hcopyrightsection{Praise God from Whom All Blessing Flow}
\par\noindent
Words: Thomas Ken, 1674.  Music: `Old 100th' Genevan Psalter, attr. Louis Bourgeois, c. 1551.  
\par\noindent
Setting: Sternhold and Hopkins' Psalter, 1561.
\par\noindent
Copyright: public domain. This score is a part of the Open Hymnal Project, 2006 Revision.
\par\noindent
Music source:ccel from Sternhold and Hopkins' Psalter 1561.

\hcopyrightsection{Praise My Soul the King of Heaven}
\par\noindent
Words: Henry F. Lyte, 1834. 
\par\noindent
Music: `Praise My Soul' or `Lauda Anima' or `St. Paul' John Goss, 1869.  Setting: ``The Choral Hymnal'', 1888.
\par\noindent
Copyright: public domain. This score is a part of the Open Hymnal Project, 2006 Revision.
\par\noindent
Music Source: ``The Plymouth Hymnal'', 1894 Hymn 219.  ``The Choral Hymnal'', 1888 page 54.

\hcopyrightsection{Praise to the Lord, the Almighty}
\par\noindent
Words: Joachim Neander, 1680. Translated by Catherine Winkworth, 1863. 
\par\noindent
Music: `Lobe den Herren' from Ander Theil des Erneuerten Gesangbuch, 1665.
\par\noindent
Setting: William Sterndale Bennett, 1863, alt. 
\par\noindent
Copyright: public domain. This score is a part of the Open Hymnal Project, 2006 Revision.
\par\noindent
Music source: ``The Chorale Book for England'', 1863 Hymn 9.  Ed. Winkworth (words) and Bennett (music)., slightly alt. per ``Evangelical Lutheran Hymn-Book'', 1931.

\hcopyrightsection{That Men a Godly Life Might Live}
\par\noindent
Words: Martin Luther, 1524.  Translated by Richard Massie, 1854, alt. 
\par\noindent
Music: `Dies sind die heil`gen zehu Gebot' or `In Gottes Namen fahren wir' circa 1200s found in Erfurt Enchiridion, 1524.  Setting: Michael Praetorius, 1609.
\par\noindent
Copyright: public domain. This score is a part of the Open Hymnal Project, 2009 Revision.

\hcopyrightsection{The Church's One Foundation}
\par\noindent
Words: Samuel John Stone, 1866.  Music: `Aurelia' Samuel Sebastian Wesley, 1864. Verse 4 modified by Augustine Watson, 2025.
\par\noindent
Setting: ``Order of worship for the Reformed Church in the United States'', 1866.
\par\noindent
Copyright: \href{https://creativecommons.org/publicdomain/zero/1.0/}{CC0 Public Domain Dedication}.
\par\noindent
Music source: `Lutheran Worship' Hymnal, 1982 Hymn 289. Almost the same as ``Order of worship for the Reformed Church in the United States'', 1866 Hymn 441

\hcopyrightsection{The King of Love}
\par\noindent
Words: Henry W. Baker, 1868.
\par\noindent
Music: Irish Melody, harm. from ``The English Hymnal'', 1906.
\par\noindent
Copyright: public domain.
\par\noindent
Music source: Hymnary.org

\hcopyrightsection{Though in the Midst of Life We Be}
\par\noindent
Words: v.1 Medieval sequence, vs 2,3 Martin Luther, 1524.  Translated by Richard Massie, 1854, alt. 
\par\noindent
Music: `Mitten wir im Leben Sind' Medieval sequence altered by Martin Luther. Found in Walter's Geistliche Gesangbüchlein, 1524.  Setting: Erythraeus, 1608.
\par\noindent
Copyright: public domain. This score is a part of the Open Hymnal Project, 2009 Revision.

\hcopyrightsection{To Avert from Men God's Wrath}
\par\noindent
Words: Latin c. 1400, sometimes attr. John Hus.  Translated Christian Ignatius Latrobe, 1789. 
\par\noindent
Music: `Gethsemane' or `Petra' Richard Redhead, 1853. 
\par\noindent
Setting: ``The Church Hymnal, Revised and Enlarged'' (Episcopal), 1896.
\par\noindent
Copyright: public domain. This score is a part of the Open Hymnal Project, 2011 Revision.
\par\noindent
Music source: ``The Church Hymnal, Revised and Enlarged'' (Episcopal), 1896  Ed. Charles Hutchins Hymn 93. 
\par\noindent
Lyrics source:  ``Hymnal and Liturgies of the Moravian Church'', 1920.

\hcopyrightsection{Ye Watchers and Ye Holy Ones}
\par\noindent
Words: John Athelstan Laurie Riley, 1906.
\par\noindent
Music: Geistliche Kirchengesäng, 1623; harm. Ralph Vaughan Williams, 1906.
\par\noindent
Copyright: public domain.
\par\noindent
Music source: Hymnary.org

%Advent:
\hcopyrightsection{Come, Thou Long Expected Jesus}
\par\noindent
Words: Charles Wesley, 1745. 
\par\noindent
Music and Setting: `Jefferson'  from ``Southern Harmony'', 1835, alt.
\par\noindent
Copyright: public domain.  This score is a part of the Open Hymnal Project, 2008 Revision.
\par\noindent
Music source: Southern Harmony, 1835, Hymn 42.  Setting heavily altered for congregational use.


\hcopyrightsection{O Come O Come Emmanuel}
\par\noindent
Words: various, combined by unknown author approx 12th Century, Translated by John Mason Neale, 1851. Second and seventh verses translated by H. S. Coffin (modified by Augustine Watson), 1916. Last verses by Augustine Watson, 2025.
\par\noindent
Music: `Veni Emmanuel' 15th Century French processional.
\par\noindent
Setting: ``Common Service Book'' (ULCA), 1917. 
\par\noindent
Arrangement: Augustine Watson, 2025.
\par\noindent
Copyright: \href{https://creativecommons.org/publicdomain/zero/1.0/}{CC0 Public Domain Dedication}.
\par\noindent
Music source: ULCA Hymnal, 1917 Hymn 1. PECUSA Hymnal (1920), Hymn 44.

\hcopyrightsection{On Jordan's Bank the Baptist's Cry}
\par\noindent
Words: Charles Coffin, 1736. st. 1-3 translated by John Chandler, 1837; st 4-5 translator unknown.  
\par\noindent
Music: `Puer Nobis Nascitur' Michael Praetorius, 1609. 
\par\noindent
Setting: George Ratcliffe Woodward for ``The English Hymnal'', 1906.
\par\noindent
Copyright: public domain. This score is a part of the Open Hymnal Project, 2006 Revision.
\par\noindent
Music source: `The English Hymnal', 1906 Hymn 14.

\hcopyrightsection{Saviour of the Nations Come}
\par\noindent
Words: Ambrose of Milan, c. 397.  Translated to German by Martin Luther, 1524. Translated from German to English by William M. Reynolds, 1851. 
\par\noindent
Music: `Nun Komm, Der Heiden Heiland' from Walter's Geistliche Gesangbüchlein, 1524.
\par\noindent
Setting: ``Mehrstimmiges ChoralBuch'', 1906.
\par\noindent
Copyright: public domain. This score is a part of the Open Hymnal Project, 2007 Revision.
\par\noindent
Music source: ``Mehrstimmiges ChoralBuch'', 1906 Hymn \#135, page 109 Ed. Karl Brauer. slightly altered. The Evangelical Lutheran Hymn Book, 1931, Hymn 141.

%Christmas:
\hcopyrightsection{A Great and Mighty Wonder}
\par\noindent
Words: Germanus of Constantinople (634-734).  Translated by John Mason Neale, 1862. 
\par\noindent
Music: `Es Ist Ein Ros Entsprungen (Rhythmic)' German from Köln, 1599.  Setting: Michael Praetorius, 1609.
\par\noindent
Copyright: public domain. This score is a part of the Open Hymnal Project, 2010 Revision.
\par\noindent
Music source: The English Hymnal, 1906.  Hymn 19

\hcopyrightsection{All Praise to Jesus' Hallowed Name}
\par\noindent
Words: verse 1, ancient German.  verses 2-7, Martin Luther, 1524.  Translated by Richard Massie, 1854, alt. 
\par\noindent
Music: `Gelobet Seist Du' ancient German found in Walter's Geistliche Gesangbüchlein, 1524. 
\par\noindent
Setting: Karl August Haupt, 1869.
\par\noindent
Copyright: public domain. This score is a part of the Open Hymnal Project, 2009 Revision.
\par\noindent
Music source: ``The Hymns of Martin Luther by Leonard Woolsey Bacon 1883, p. 20.

\hcopyrightsection{Angels We Have Heard on High}
\par\noindent
Words: French Carol; Translated by James Chadwick, 1862. 
\par\noindent
Music: `Gloria' French carol melody.  Setting: Edward (or Edwin) S. Barnes, before 1916.
\par\noindent
Copyright: public domain. This score is a part of the Open Hymnal Project, 2005 Revision.
\par\noindent
Music source: `Lutheran Worship' Hymnal, 1982 Hymn 55. ``Carols Old And Carols New'', 1916 Carol 181.

\hcopyrightsection{Away in a Manger}
\par\noindent
Words: stanzas 1,2 anonymous published Philadelphia, 1885.  stanza 3 John T. MacFarland (1851-1913). 
\par\noindent
Music: `Mueller' James R. Murray, 1887.  Setting: ``Hymnal for American Youth'', 1919.
\par\noindent
Copyright: public domain. This score is a part of the Open Hymnal Project, 2006 Revision.
\par\noindent
Music source: ``Hymnal for American Youth'', Hymn 84 1919 ed. H. Augustine Smith. Music reputed to be first published in ``Dainty Songs for Little Lads and Lasses'' by James R. Murray, 1887, hymn 8.

\hcopyrightsection{God Rest Ye Merry Gentlemen}
\par\noindent
Words: Traditional English. 
\par\noindent
Music: `God Rest Ye Merry Gentlemen' Traditional English.  Setting: ``Carols Old And Carols New'', 1918.
\par\noindent
Copyright: public domain. This score is a part of the Open Hymnal Project, 2006 Revision.
\par\noindent
Music source: `Carols Old And Carols New', 1918 carol 722.

\hcopyrightsection{Good King Wenceslas}
\par\noindent
Words: John M. Neale, 1853.
\par\noindent
Music: `Tempus Adest Floridum' 13th Century spring carol; first published in the Swedish Piae Cantones, 1582.
\par\noindent
Setting: Carols Old And Carols New, 1916.
\par\noindent
Copyright: public domain. This score is a part of the Open Hymnal Project, 2006 Revision.

\hcopyrightsection{Hark! The Herald Angels Sing}
\par\noindent
Words: Charles Wesley, 1739, alt. 
\par\noindent
Music: `Mendelssohn' from `Festgesang' Felix Mendelssohn, 1840.  Setting: William H. Cummings, 1857.
\par\noindent
Copyright: public domain. This score is a part of the Open Hymnal Project, 2005 Revision.
\par\noindent
Music source: `Lutheran Worship' Hymnal, 1982 Hymn 49.

\hcopyrightsection{In the Bleak Mid-Winter}
\par\noindent
Words: Christina Georgina Rossetti, 1872, alt. 
\par\noindent
Music and Setting: `Cranham' Gustav Theodore Holst, 1906, alt. 
\par\noindent
Copyright: public domain. This score is a part of the Open Hymnal Project, 2011 Revision.
\par\noindent
Music source: ``The English Hymnal, 1906. 

\hcopyrightsection{It Came upon a Midnight Clear}
\par\noindent
Words: Edmund H. Sears, 1849.  Music: `Carol' Richard S. Willis, 1861.  
\par\noindent
Setting: ``Order of worship for the Reformed Church in the United States'', 1866.
\par\noindent
Copyright: public domain. This score is a part of the Open Hymnal Project, 2005 Revision.
\par\noindent
Music source: ``Order of worship for the Reformed Church in the United States'', 1866 Hymn 63. Supposed to be found in ``Church Corals and Choir Studies'' 1850 by Willis, but I don't see it in that book. Some sources say 1850, some 1859, some 1861.  Willis was editor for ``The Musical World'' magazine.  Probably first appeared there.

\hcopyrightsection{Joy to the World}
\par\noindent
Words: Isaac Watts, 1719.
\par\noindent
Music: `Antioch' pieced together from ``Messiah'' George F. Handel, 1741.  Setting: Lowell Mason, 1836.
\par\noindent
Copyright: public domain. This score is a part of the Open Hymnal Project, 2005 Revision.
\par\noindent
Music source: `Lutheran Worship' Hymnal, 1982 Hymn 53.

\hcopyrightsection{Lo, How A Rose E'er Blooming}
\par\noindent
Words: verses 1-2, 15th Century German.  Translated by Theodore Baker, 1894. Verses 3,4 Fridrich Layriz (1808-1859).  Translated by Harriet Reynolds Krauth, 1875. Verse 5, 15th Century German.  Translated by John C. Mattes, 1914.
\par\noindent
Music: `Es Ist Ein Ros Entsprungen (Rhythmic)' German from Köln, 1599.  Setting: Michael Praetorius, 1609.
\par\noindent
Copyright: public domain. This score is a part of the Open Hymnal Project, 2010 Revision.
\par\noindent
Music source: The English Hymnal, 1906.  Hymn 19

\hcopyrightsection{O Come, All Ye Faithful}
\par\noindent
Words:  John F. Wade, circa 1743. v.1-3, 6 Translated by Frederick Oakeley, 1841; v. 4, 5 Translated by William T. Brooke (1848-1917). 
\par\noindent
Music: `Adeste Fideles' or `Portuguese Hymn' John F. Wade, 1743.  Setting: ``A Hymnal'' (Episcopal), 1916.
\par\noindent
Copyright: public domain. This score is a part of the Open Hymnal Project, 2007 Revision.
\par\noindent
Music source: ``A Hymnal'' (Episcopal), 1916 Hymn 72, alt.

\hcopyrightsection{O Word of God the Father}
\par\noindent
Words: Bradford Littlejohn.
\par\noindent
Music: `Thaxted', Holst, 1921.
\par\noindent
Arrangement: Augustine Watson, 2025.
\par\noindent
Copyright: 2024 Bradford Littlejohn. Permission received for use in this hymnal.  All Rights Reserved.

\hcopyrightsection{Silent Night}
\par\noindent
Words: Josef Mohr, 1818. stanzas 1,3 Translated by John Freeman Young, 1863. Stanzas 2,4 translator anonymous.
\par\noindent
Music: `Stille Nacht' Franz Xaver Gruber, 1818.  Setting: ``Concordia Kinderchöre'', 1908.
\par\noindent
Copyright: public domain. This score is a part of the Open Hymnal Project, 2005 Revision.
\par\noindent
Music source: `Lutheran Worship' Hymnal, 1982 Hymn 68.
\par\noindent
Music source: ``Concordia Kinderchöre'', 1908 Hymn 42 page 54.

\hcopyrightsection{The First Noel}
\par\noindent
Words: Traditional English carol, possibly dating from as early as the 13th Century.
\par\noindent
Music: `The First Noel' Traditional English carol, possibly dating from as early as the 13th Century.
\par\noindent
Setting: ``The Methodist Sunday School Hymnal'', 1911.
\par\noindent
Copyright: public domain. This score is a part of the Open Hymnal Project, 2006 Revision.
\par\noindent
Music source: The Methodist Sunday School Hymnal, 1911 Hymn 66.

\hcopyrightsection{To Shepherds as They Watched by Night}
\par\noindent
Words: Martin Luther, 1543. translated by Richard Massie, 1854. 
\par\noindent
Music: `Vom Himmel Hoch' traditional German from Schumann's Geistliche Lieder, Leipzig, 1539. 
\par\noindent
Setting: ``Common Service Book'' (ULCA), 1917.
\par\noindent
Copyright: public domain. This score is a part of the Open Hymnal Project, 2009 Revision.
\par\noindent
Music source: `Common Service Book with Hymnal', ULCA 1918 Hymn 19.

\hcopyrightsection{What Child is This?}
\par\noindent
Words: William Chatterton Dix, 1865. 
\par\noindent
Music: `Greensleeves' 16th Century English Traditional. 
\par\noindent
Setting: traditional from ``The Sunday School Hymnal and Service Book'', 1871.
\par\noindent
Copyright: public domain. This score is a part of the Open Hymnal Project, 2005 Revision.
\par\noindent
Music source: `Lutheran Worship' Hymnal, 1982 Hymn 61. very tiny changes from ``The Sunday school hymnal and service book'', 1871 edited by Charles Lewis Hutchins Carol 9 (after hymns).

%Epiphany:
\hcopyrightsection{As with Gladness Men of Old}
\par\noindent
Words: William Chatterton Dix, 1860. 
\par\noindent
Music: `Dix' or `Treuer Heiland, Wir Sind Heir' Conrad Kocher, 1838. Abridged by William Henry Monk, 1861.
\par\noindent
Setting: Conrad Kocher, 1838, alt. by William Henry Monk, 1861, alt. for ``The English Hymnal'', 1906.
\par\noindent
Copyright: public domain. This score is a part of the Open Hymnal Project, 2008 Revision.
\par\noindent
Music source: The English Hymnal, 1906 Hymn 39. Music from ``Stimmen aus dem Reiche Gottes'', 1838 by Kocher, Hymn 201 page 250. Adapted by William Henry Monk from the original (removed two measures and changed parts of the arrangement) for ``Hymns Ancient and Modern, 1861. Also almost just like the Bristol Tune Book of 1863, Hymn 172.

\hcopyrightsection{Brightest and Best of the Sons of Morning}
\par\noindent
Words: James P. Harding, 1892.
\par\noindent
Music: Morning Star.
\par\noindent
Copyright: public domain.
\par\noindent
Music source: The English Hymnal, 1940 Hymn 46.

\hcopyrightsection{Lord, Who at Cana's Wedding Feast}
\par\noindent
Words: Adelaide Thrupp, 1853. Music: `St. Ursula' Frederick Westlake, 1863. 
\par\noindent
Setting: ``The Church Hymnal, Revised and Enlarged'' (Episcopal), 1896.
\par\noindent
Copyright: public domain. This score is a part of the Open Hymnal Project, 2011 Revision.
\par\noindent
Music source: ``The Church Hymnal, Revised and Enlarged'' (Episcopal), 1896 Ed. Charles Hutchins Hymn 237. Words compared against ``The Church Hymnal, Revised and Enlarged'', 1896.

\hcopyrightsection{We Three Kings}
\par\noindent
Words: John Henry Hopkins, Jr., 1857.
\par\noindent
Music: John Henry Hopkins, Jr., 1857, alt.
\par\noindent
Setting: ``Cyber Hymnal''.
\par\noindent
Arrangement: Augustine Watson, 2025.
\par\noindent
Copyright: \href{https://creativecommons.org/publicdomain/zero/1.0/}{CC0 Public Domain Dedication}.
\par\noindent
Music source: ``Cyber Hymnal''.

%Septuagesimatide
\hcopyrightsection{Maker of Earth, To Thee Alone}
\par\noindent
Words: Laeta mundi, conditor. C. Coffin. Translated by J. M. Neale.
\par\noindent
Music: Dunfermline. ``Scottish Psalter'', 1615.
\par\noindent
Copyright: public domain.
\par\noindent
Music source: ``The English Hymnal: with tunes'', 1906, Hymn 64.

\hcopyrightsection{O Love, How Deep, How Broad}
\par\noindent
Words: O Amor quam ecstaticus. \nth{15} century. Translated by B. Webb.
\par\noindent
Music: Eisenach. J. H. Schein. J. S. Bach.
\par\noindent
Copyright: public domain.
\par\noindent
Music source: ``The English Hymnal: with tunes'', 1906, Hymn 459.

\hcopyrightsection{Praise to the Holiest in the Height}
\par\noindent
Words: J. H. Newman.
\par\noindent
Music: Richmond. Adapted from T. Haweis by S. Webbe (the younger).
\par\noindent
Copyright: public domain.
\par\noindent
Music source: ``The English Hymnal: with tunes'', 1906, Hymn 471.

\hcopyrightsection{There is a Book Who Runs May Read}
\par\noindent
Words: J. Keble.
\par\noindent
Music: St. Flavian. Adapted from Psalm 132 in `Day's Psalter', 1563.
\par\noindent
Copyright: public domain.
\par\noindent
Music source: ``The English Hymnal: with tunes'', 1906, Hymn 497.

%Lent:
\hcopyrightsection{All Glory, Laud, and Honour}
\par\noindent
Words: Theodulf of Orleans, circa 820. Translated by John Mason Neale, 1851.  
\par\noindent
Music: `Valet Will Ich Dir Geben' or `St. Theodulph' Melchior Teschner, 1615. 
\par\noindent
Setting: Presbyterian Hymnal, 1911.
\par\noindent
Copyright: public domain. This score is a part of the Open Hymnal Project, 2006 Revision.
\par\noindent
Music source: Presbyterian Hymnal, Revised, 1911 Hymn 216.

\hcopyrightsection{Lord Who Throughout These Forty Days}
\par\noindent
Words: Claudia F. Hernaman, 1873.  Music: `St. Flavian' Day's Psalter, 1563. 
\par\noindent
Setting: ``The Church Hymnal, Revised and Enlarged'' (Episcopal), 1905.
\par\noindent
Copyright: public domain. This score is a part of the Open Hymnal Project, 2008 Revision.
\par\noindent
Music source: Episcopal Hymnal, 1905 Hymn 78.

\hcopyrightsection{O Jesu Christ From Thee Began}
\par\noindent
Words: \nth{9} century text. 
\par\noindent
Music: Plaistow.
\par\noindent
Copyright: Public domain.
\par\noindent
Music source: ``The English Hymnal: with tunes'', 1906, Hymn 69.


\hcopyrightsection{O Sacred Head, Now Wounded}
\par\noindent
Words: Bernard of Clairvaux, 1153. Translated by James W. Alexander, 1830. 
\par\noindent
Music: `Passion Chorale' or `Herzlich Tut Mich Verlangen' Hans Leo Hassler, 1601. Adapted by J.S. Bach, 1729.  
\par\noindent
Setting: Johann Sebastian Bach, 1729.
\par\noindent
Copyright: public domain. This score is a part of the Open Hymnal Project, 2007 Revision.
\par\noindent
Music source: The Episcopal Hymnal, 1916, Hymn 158.


\hcopyrightsection{When I Survey the Wondrous Cross}
\par\noindent
Words: Isaac Watts, 1707.
\par\noindent
Music: `Duke Street' John Hatton, 1793.  Setting: ``Finest of the Wheat No. 3'' Hymnal, 1904.
\par\noindent
Copyright: public domain. This score is a part of the Open Hymnal Project, 2014 Revision.
\par\noindent
Music source: ``Finest of the Wheat No. 3'' Hymnal, 1904 Hymn 257.  ABC file contributed to the Open Hymnal by Tobin Strong, 20 Dec 2013. Composite from ``The Anglican hymn Book'', 1871 hymn 127 and ``Hymns Ancient and Modern'', 1869 Hymn 101.

%Easter:
\hcopyrightsection{At the Lamb's High Feast}
\par\noindent
Words: Latin, circa 6th Century. Translated by Robert Campbell, 1849. 
\par\noindent
Music: `Sonne der Gerechtigkeit' Czech, Kirchengeseng, 1566.  Setting: Brian J. Dumont, 31 Dec 2009.
\par\noindent
Copyright: Words and Music, public domain. Setting: Copyright 2009 Brian J. Dumont. This setting may be freely reproduced or published for Christian worship, provided it is not altered, and this notice is on each copy. All other rights reserved.   This score is a part of the Open Hymnal Project, 2010 Revision.
\par\noindent
Music source: tune found in many places, arrangement by BJD

\hcopyrightsection{Christ the Lord Is Risen Today}
\par\noindent
Words: Stanzas 1-7, Charles Wesley, 1739. Stanzas 8-10, 14th Century; translated in Lyra Davidica. 
\par\noindent
Music: `Llanfair' Robert Williams, 1817.  Setting: John Roberts, 1837.
\par\noindent
Copyright: public domain. This score is a part of the Open Hymnal Project, 2008 Revision.
\par\noindent
Music source: Lutheran Worship, 1982, Hymn 137 (later modified to more closely match John Robert's original setting).

\hcopyrightsection{Hail Thee, Festival Day}
\par\noindent
Words: Venantius Honorius Fotunatus; translation from `The English Hymnal', 1906, alt.
\par\noindent
Music: Ralph Vaughan Williams, 1906.
\par\noindent
Copyright: public domain.
\par\noindent
Music source: `Songs of Praise', 1925, Hymn 445. PECUSA Hymnal, 1940, Hymn 86.

\hcopyrightsection{The Strife is O'er, the Battle Done}
\par\noindent
Words: from Symphonia Sirenum Selectarum, Kˆln, 1695; translated by Francis Pott, 1861.
\par\noindent
Music: `Victory' or  `Palestrina' Giovanni P. da Palestrina, 1591.
\par\noindent
Setting: William Henry Monk, 1861.
\par\noindent
Copyright: public domain.  This score is a part of the Open Hymnal Project, 2011 Revision.
\par\noindent
Music source: ``Hymns Ancient and Modern'', 1861 Hymn 114.  Words checked against the same source.


%Ascensiontide
\hcopyrightsection{See, The Lord Ascends in Triumph}
\par\noindent
Words: Christopher Wordsworth, 1862, alt.
\par\noindent
Music: 'Rex Gloriae' Henry Thomas Smart, 1868. 
\par\noindent
Setting: "Appendix to Hymns Ancient and Modern", 1869.
\par\noindent
Copyright: public domain.  This score is a part of the Open Hymnal Project, 2011 Revision.
\par\noindent
Music source: ``Appendix to Hymns Ancient and Modern'', 1869 Hymn 293.  Text is the same as ``Appendix to Hymns Ancient and Modern'' with the exception of the title.

\hcopyrightsection{Sing We Triumphant Hymns of Praise}
\par\noindent
Words: St. Bede the Venerable. Translated by B. Webb.
\par\noindent
Music: Tugwood. Nicholas Gatty.
\par\noindent
Copyright: public domain.
\par\noindent
Music source: ``The English Hymnal: with tunes'', 1906, Hymn 146.

%Whitsun
\hcopyrightsection{Blest Joys for Mighty Wonders Wrought}
\par\noindent
Words: Beata nobis gaudia. St. Hilary of Poitiers, \nth{4} century. Translated by Neale.
\par\noindent
Music: T. B. Southgate.
\par\noindent
Arrangement: Augustine Watson, 2025.
\par\noindent
Copyright: \href{https://creativecommons.org/publicdomain/zero/1.0/}{CC0 Public Domain Dedication}
\par\noindent
Music source: ``The Armagh Hymnal'', 1915, Hymn 46.

\hcopyrightsection{Let the Holy Spirit's Grace}
\par\noindent
Words: Nobis Sancti Spiritus. Benedict XII, \nth{14} century. Translated by Neale.
\par\noindent
Music: Regina Clementiae. Harleian MS. 978. Mode I.
\par\noindent
Arrangement: Augustine Watson, 2025.
\par\noindent
Copyright: \href{https://creativecommons.org/publicdomain/zero/1.0/}{CC0 Public Domain Dedication}
\par\noindent
Music source: ``The Armagh Hymnal'', 1915, Hymn 47.


%Trinity Sunday:
\hcopyrightsection{God the Father Be Our Stay}
\par\noindent
Words: 15th Century Litany, adapted by Martin Luther, 1524.  Translated by Richard Massie, 1854, alt. 
\par\noindent
Music: `Gott Der Vater, Wohn Uns Bei' from Walter's Geistliche Gesangbüchlein, 1524.
\par\noindent
Setting: composite from Landgraf Moritz, 1612 and ``Evangelical Lutheran Hymn-Book'', 1931.
\par\noindent
Copyright: public domain. This score is a part of the Open Hymnal Project, 2009 Revision.

\hcopyrightsection{Holy God, We Praise Thy Name}
\par\noindent
Words: attr. Ignaz Franz, 1774. Translated by Clarence A. Walworth, 1858. 
\par\noindent
Music: `Te Deum' or `Hursley' or `Grosser Gott, wir Loben Dich' from Katholisches Gesangbuch, Maria Theresa, 1774. Setting: ``Hymns Ancient and Modern'', 1869, alt.
\par\noindent
Copyright: public domain. This score is a part of the Open Hymnal Project, 2005 Revision.
\par\noindent
Music source: ``Hymns Ancient and Modern'', 1869 Hymn 11

\hcopyrightsection{Holy, Holy, Holy}
\par\noindent
Words: Reginald Heber, 1826. 
\par\noindent
Music: `Nicaea' John Bacchus Dykes, 1861.  Setting: ``Hymns Ancient and Modern'', 1869.
\par\noindent
Copyright: public domain. This score is a part of the Open Hymnal Project, 2005 Revision.
\par\noindent
Music source: Hymns Ancient and Modern, 1869 hymn 135.  Tune from Hymns Ancient and Modern, 1861.

\hcopyrightsection{I Bind unto Myself Today}
\par\noindent
Words: attributed to St. Patrick of Ireland (circa 387-466). Paraphrased by Cecil F. Alexander, 1889. 
\par\noindent
Music: `St. Patricks Breastplate' Charles V. Stanford, 1902.  Setting: ``The English Hymnal'', 1906.
\par\noindent
Copyright: public domain. This score is a part of the Open Hymnal Project, 2006 Revision.
\par\noindent
Music source: `The English Hymnal', 1906 Hymn 212.

\hcopyrightsection{Isaiah, Mighty Seer, in Days of Old}
\par\noindent
Words: Martin Luther, 1526 as the Sanctus of the German Mass, after Is 6:1-4.  Translation composite. 
\par\noindent
Music: `Jesaia Dem Propheten das Geschah' Martin Luther, 1526 in the German Mass.  Setting: Erythraeus, 1608.
\par\noindent
Copyright: public domain. This score is a part of the Open Hymnal Project, 2009 Revision. Translation is public domain per Project Wittenberg: http://www.iclnet.org/pub/resources/text/wittenberg/hymns/isaiah.txt

\hcopyrightsection{We All Believe in One True God}
\par\noindent
Words: Medieval text expanded by Martin Luther, 1524.  Translation composite. 
\par\noindent
Music: `Wir Glauben all an Einen Gott, Schoepfer' or `Apostolic Creed' Medieval tune altered by Martin Luther. Found in Walter's Geistliche Gesangbüchlein, 1524. 
\par\noindent
Setting: ``Eisenach Kirchenconserenz'', G.v. Tucher et. al. , Stuttgart, 1854.
\par\noindent
Copyright: public domain. This score is a part of the Open Hymnal Project, 2009 Revision. Translation is public domain per Project Wittenberg: http://www.iclnet.org/pub/resources/text/wittenberg/hymns/believe.txt

%Transfiguration:
\hcopyrightsection{O Wondrous Type, O Vision Fair}
\par\noindent
Words: From the ``Sarum Breviary'', 1495.  Translated by John Mason Neale, 1851.
\par\noindent
Music: `Deo Gracias' or `Agincourt' traditional English, circa 1415.  Setting: Charles Winfred Douglas, 1918, alt.
\par\noindent
Copyright: public domain. This score is a part of the Open Hymnal Project, 2011 Revision.
\par\noindent
Music source: ``A Hymnal'' (Episcopal), 1918 Hymn 439, alt by BJD 21 Mar 2011.  Translation found in Hymns Ancient \& Modern, 1861.

\hcopyrightsection{The Mouth of Fools Doth God Confess}
\par\noindent
Words: Martin Luther, 1524. Translated by Richard Massie, 1854, alt. 
\par\noindent
Music: `Es spricht der Unweisen Mund' from Walter's Geistliche Gesangbüchlein, 1524.  Setting: Michael Praetorius, 1610.
\par\noindent
Copyright: public domain. This score is a part of the Open Hymnal Project, 2009 Revision.

\hcopyrightsection{Tis Good, Lord, to Be Here}
\par\noindent
Words: Joseph Armitage Robinson, 1888.
\par\noindent
Music: `Potsdam' adapted from Johann Sebastian Bach, 1750, by John Goss, 1854. Setting: John Goss, 1854.
\par\noindent
Copyright: public domain. This score is a part of the Open Hymnal Project, 2011 Revision.
\par\noindent
Music source: ``The Church Psalter and Hymn Book'', 1863 hymns 212-214 (rev. from 1854 edition).  Mus. Dir and arrangements by John Goss. Words match those in ``Hymns Ancient And Modern'', 1904, Hymn 251

\end{multicols}

\normalsize