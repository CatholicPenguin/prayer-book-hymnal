\phantomsection
\addcontentsline{toc}{section}{Proper of Season}
\fancyhead[C]{\LARGE Proper of Season}

%\begin{secrubric}
%	Except during Advent \& Lent, the preceding Sunday's propers may be used for a weekday Feria.
%\end{secrubric}

\subby{First Sunday of Advent}\label{AdventI}
\fancyhead[RO,LE]{\textit{Advent I}}
\fancyhead[RE,LO]{}
%\begin{inhead}
%    {First Class Semidouble}
%\end{inhead}

\subbysub{I Evensong}\label{FirstAdventEvensong}

\gregorioscore{resources/gabc/ProperSeason/FirstAdventEvensong.gabc}

℣. Drop down, ye heavens, from above, and let the skies pour down righteousness.

℟. Let the earth open, and let them bring forth salvation.\\

\properantiphon{Mag.}{Behold, the Name of the Lord {\dag} cometh from afar: and his glory filleth all the earth.}

%Creator of the stars of night,
% 
% thy people's everlasting light,
% 
% Jesu, Redeemer, save us all,
% 
% And hear thy servants when they call.
%
%2. Thou, grieving that the ancient curse
% 
% Should doom to death a universe,
% 
% Hast found the med'cine, full of grace,
% 
% To save and heal a ruin'd race.
%
%3. Thou cam'st, the Bridegroom of the bride,
% 
% As drew the world to ev'ning-tide;
% 
% Proceeding from a virgin shrine,
% 
% The spotless victim all divine.
%
%4. At whose dread name, majestic now,
% 
% All knees must bend, all hearts must bow;
% 
% And things celestial thee shall own,
% 
% And things terrestrial, Lord alone.
%
%5. O thou whose coming is with dread
% 
% To judge and doom the quick and dead,
% 
% Preserve us, while we dwell below,
% 
% From every insult of the foe.
%
%6. To God the Father, God the Son,
% 
% And God the Spirit, Three in One,
% 
% Laud, honour, might, and glory be
% 
% From age to age eternally. Amen.

\subbysub{Mattins}

\invitatoryhymn\label{FirstAdventInvitatory}

\gregorioscore{resources/gabc/ProperSeason/FirstAdventInvitatory.gabc}

\officehymn\label{FirstAdventMattins}

\gregorioscore{resources/gabc/ProperSeason/FirstAdventMattins.gabc}

℣. The voice of one crying in the wilderness, Prepare ye the way of the Lord.

℟. Make his paths straight.

\properantiphon{Ben.}{The Holy Ghost {\dag} shall come upon thee, Mary; fear not, thou shalt bear in thy womb the Son of God, alleluia.}

%A thrilling voice by Jordan rings,
%
%rebuking guilt and darksome things:
%
%vain dreams of sin and visions fly;
%
%Christ in his might shines forth on high.
%
%2. Now let each torpid soul arise,
%
%that sunk in guilt and wounded lies;

%see! the new Star's refulgent ray
%
%shall chase disease and sin away.
%
%3. The Lamb descends from heav'n above
%
%to pardon sin with freest love:
%
%for such indulgent mercy shewn
%
%with tearful joy our thanks we own.
%
%4. That when again he shines reveal'd,
%
%and trembling worlds to terror yield.
%
%He give not sin its just reward,
%
%but in his love protect and guard.
%
%5. To God the Father, God the Son,
%
%And God the Spirit, Three in One,
%
%Laud, honour, might, and glory be
%
%From age to age eternally. Amen.

\subbysub{II Evensong}

\begin{rubric}
	II Evensong as in I Evensong, except for the following Antiphon.
\end{rubric}

\properantiphon{Mag.}{Fear not, Mary, {\dag} for thou hast found favour with the Lord: behold, thou shalt conceive, and bring forth a Son, alleluia.}


\subby{Advent Ferial Office}
\begin{rubric}
	On Advent Ferias, the Invitatory Hymn is of the First Sunday of Advent. The Office Hymn and Versicle is taken from Sunday (p. \pageref{SundayMattinsWinter}). The Antiphon is of the Day.
\end{rubric}


\subby{The First Week of Advent}
\feastday{Advent I Week}

\begin{multicols}{2}
\subsubsec{Monday}
\antiphon{Ben.}{The Angel of the Lord {\dag} announced unto Mary, and she conceived by the Holy Ghost, alleluia.}

\antiphon{Mag.}{Lift up thine eyes, {\dag} O Jerusalem, and see the power of the King: behold, the Saviour cometh to loose thee from thy chain.}

\subsubsec{Tuesday}
\antiphon{Ben.}{Before they came together {\dag} Mary was found with child of the Holy Ghost, alleluia.}

\antiphon{Mag.}{Seek ye the Lord {\dag} while he may be found: call ye upon him while he is near, alleluia.}

\subsubsec{Wednesday}
\antiphon{Ben.}{Out of Sion {\dag} shall go forth the law, and the word of the Lord from Jerusalem.}

\antiphon{Mag.}{After me {\dag} cometh one mightier than I, the latchet of whose shoes I am not worthy to unloose.}

\subsubsec{Thursday}
\antiphon{Ben.}{Blessed art thou {\dag} among women, and blessed is the fruit of thy womb.}

\antiphon{Mag.}{I will wait {\dag} upon the Lord my Saviour: and I will look for him while he is near, alleluia.}

\subsubsec{Friday}
\antiphon{Ben.}{Lo, there cometh one {\dag} that is both God and Man, of the house of David, to sit upon the throne, alleluia.}

\antiphon{Mag.}{Out of Egypt {\dag} have I called my Son: he shall come to save his people.}

\subsubsec{Saturday}
\antiphon{Ben.}{Fear not, {\dag} O Sion, behold thy God cometh, alleluia.}

\end{multicols}


\subby{Second Sunday of Advent}
\feastday{Advent II}
%\begin{inhead}
%{Second Class Semidouble}
%\end{inhead}

\begin{rubric}
	The Hymns \& Versicles are of the First Sunday of Advent (p. \pageref{AdventI}), with the following Antiphons.
\end{rubric}

\properantiphon{Mag.}{Come, O Lord, in peace; {\dag} visit us with thy salvation, that we may rejoice before thee with a perfect heart.}

\properantiphon{Ben.}{He shall sit upon the throne {\dag} of David, and of his kingdom there shall be no end, alleluia.}

\properantiphon{Mag.}{Blessed art thou, {\dag} O Mary, for thou hast believed the Lord: and there shall be a performance in thee of those things which were told thee from the Lord, alleluia.}


\subby{The Second Week of Advent}
\feastday{Advent II Week}

\begin{multicols}{2}
\subsubsec{Monday}
\antiphon{Ben.}{From heaven there cometh {\dag} the Lord, the Ruler, and in his hand are honour and dominion.}

\antiphon{Mag.}{Behold the King shall come, {\dag} the Lord of the earth, and he shall take away the yoke of our captivity.}

\subsubsec{Tuesday}
\antiphon{Ben.}{The Lord shall arise upon thee, {\dag} O Jerusalem, and his glory shall be seen upon thee.}

\antiphon{Mag.}{The voice of one crying {\dag} in the wilderness, Prepare ye the way of the Lord, make straight in the desert a highway for our God.}

\subsubsec{Wednesday}
\antiphon{Ben.}{Behold I send my messenger, {\dag} and he shall prepare my way before thy face.}

\antiphon{Mag.}{Sion, {\dag} thou shalt be renewed, and shalt see thy righteous One, he that cometh unto thee.}

\subsubsec{Thursday}
\antiphon{Ben.}{Thou, O Lord, {\dag} art he that is to come, for whom we look, to save thy people.}

\antiphon{Mag.}{He that cometh after me {\dag} is preferred before me, the latchet of whose shoes I am not worthy to unloose.}

\subsubsec{Friday}
\antiphon{Ben.}{Say to them: {\dag} Ye that are of a fearful heart, be strong: for behold the Lord our God shall come.}

\antiphon{Mag.}{Sing unto the Lord {\dag} a new song; and his praise from the end of the earth.}

\subsubsec{Saturday}
\antiphon{Ben.}{The Lord shall set up an ensign {\dag} for the nations, and shall assemble the outcasts of Israel.}

\end{multicols}

\subby{Magnificat Antiphons before Christmas}\label{OAntiphons}
\fancyhead[RO,LE]{\textit{Advent Antiphons}}
\fancyhead[RE,LO]{}
\begin{rubric}
    The Magnificat Antiphon is always taken from here on these days.
\end{rubric}
%\begin{multicols}{2}
\subbysub{16 December: O Sapientia}
\lett{O}{Wisdom,} {\dag} which camest out of the mouth of the Most High, and reachest from one end to another, mightily and sweetly ordering all things: Come and teach us the way of prudence.
\subbysub{17 December: O Adonai}
\lett{O}{Adonai,} {\dag} and Leader of the house of Israel, who appearedst in the bush to Moses in a flame of fire, and gavest him the law in Sinai: Come and redeem us with an outstretched arm.
\subbysub{18 December: O Radix Jesse}
\lett{O}{Root of Jesse,} {\dag} which standest for an ensign of the people, at whom kings shall shut their mouths, unto whom the Gentiles shall seek: Come and deliver us, and tarry not.
\subbysub{19 December: O Clavis David}
\lett{O}{Key of David,} {\dag} and Sceptre of the house of Israel; that openest and no man shutteth, and shuttest and no man openeth: Come, and bring the prisoners out of the prison-house, them that sit in darkness and the shadow of death.
\subbysub{20 December: O Oriens}
\lett{O}{Day-spring,} {\dag} Brightness of the Light everlasting, and Sun of righteousness: Come and enlighten them that sit in darkness and the shadow of death.
\subbysub{21 December: O Rex gentium}
\lett{O}{King of Nations,} {\dag} and their Desire; the Cornerstone, who makest both one: Come and save mankind, whom thou formedst of clay.
\subbysub{22 December: O Emmanuel}
\lett{O}{Emmanuel,} {\dag} our King and Lawgiver, the Desire of all nations and their Salvation: Come and save us, O Lord our God.
\subbysub{23 December: O Virgo virginum}
\lett{O}{Virgin of Virgins,} {\dag} how shall this be? for neither before thee was any seen like thee, nor shall there be after. Daughters of Jerusalem, why marvel ye at me? The thing which ye behold is a divine mystery.
%\end{multicols}


\subby{Third Sunday of Advent}
\feastday{Advent III}
%\begin{inhead}
%{Second Class Semidouble}
%\end{inhead}

\begin{rubric}
	The Hymns \& Versicles are of the First Sunday of Advent (p. \pageref{AdventI}), with the following Antiphons.
\end{rubric}

\properantiphon{Mag.}{Before me {\dag} there was no God formed, neither shall there be after me: unto me every knee shall bow, and me shall every tongue confess.}

\properantiphon{Ben.}{Now when John had heard {\dag} in the prison the works of Christ, he sent two of his disciples, and said unto him: Art thou he that should come, or do we look for another?}

\properantiphon{Mag.}{Art thou he {\dag} that should come, or do we look for another? Go and shew John those things which ye do see: the blind receive their sight, the dead are raised, and the poor have the gospel preached unto them, alleluia.}


\subby{The Third Week of Advent}
\feastday{Advent III Week}

\begin{multicols}{2}
\subsubsec{Monday}
\antiphon{Ben.}{There shall come forth {\dag} a rod out of the stem of Jesse, and all the earth shall be filled with the glory of the Lord: and all flesh shall see the salvation of God.}

\antiphon{Mag.}{All generations {\dag} shall call me blessed: for God hath regarded the lowliness of his handmaiden.}

\subsubsec{Tuesday}
\antiphon{Ben.}{Thou, Bethlehem, {\dag} in the land of Juda, shalt not be the least: for out of thee shall come a Governor, that shall rule my people Israel.}

\properantiphon{Mag.}{Awake, awake, {\dag} arise, O Jerusalem: loose thyself from the bands of thy neck, O captive daughter of Sion.}

\subsubsec{Ember Wednesday}
\properantiphon{Ben.}{The Angel {\dag} Gabriel was sent to Mary, a Virgin espoused to Joseph.}

\properantiphon{Mag.}{Behold the handmaid of the Lord; {\dag} be it unto me according to thy word.}

\subsubsec{Thursday}
\antiphon{Ben.}{Be ye watchful {\dag} in your hearts, for the Lord our God is nigh at hand.}

\properantiphon{Mag.}{Rejoice ye {\dag} with Jerusalem; and be glad with her, all ye that love her, forever.}

\subsubsec{Ember Friday}
\properantiphon{Ben.}{For lo, as soon {\dag} as the voice of thy salutation sounded in mine ears, the babe leaped in my womb for joy, alleluia.}

\properantiphon{Mag.}{This is the witness {\dag} which John bare, saying: He that cometh after me is preferred before me.}

\subsubsec{Ember Saturday}
\properantiphon{Ben.}{How shall this be, {\dag} O Angel of God, seeing I know not a man? Hearken, O Virgin Mary: the Holy Ghost shall come upon thee, and the power of the Highest shall overshadow thee.}

\end{multicols}


\subby{Fourth Sunday of Advent}
\feastday{Advent IV}
\fancyhead[RE,LO]{}
%\begin{inhead}
%{Second Class Semidouble}
%\end{inhead}


\begin{rubric}
	In the Daily Office, the Hymns \& Versicles are of the First Sunday of Advent (p. \pageref{AdventI}), with the following Antiphon.
\end{rubric}

\properantiphon{Ben.}{Hail, Mary, {\dag} thou that art full of grace, the Lord is with thee; blessed art thou among women, alleluia.}


\subby{The Fourth Week of Advent}
\feastday{Advent IV Week}

\begin{multicols}{2}
\subsubsec{Monday}
\antiphon{Ben.}{Thus saith the Lord {\dag} your God: Repent ye, and turn again; for the kingdom of heaven is at hand, alleluia.}

\subsubsec{Tuesday}
\antiphon{Ben.}{Awake, awake, {\dag} put on strength, O arm of the Lord.}

\subsubsec{Wednesday}
\antiphon{Ben.}{I will place salvation {\dag} in Sion, and in Jerusalem my glory, alleluia.}

\subsubsec{Thursday}
\antiphon{Ben.}{Comfort ye, {\dag} comfort ye my people, saith your God.}

\subsubsec{Friday}
\antiphon{Ben.}{Behold all things are fulfilled {\dag} which were spoken by the Angel of the Virgin Mary.}

\end{multicols}


\subby{Christmas Eve}
\feastday{Christmas Eve}
%\begin{inhead}
%{First Class Vigil}
%\end{inhead}

\begin{rubric}
	In Mattins, the Hymn is of the First Sunday of Advent (p. \pageref{AdventI}), with the following.
\end{rubric}

℣. To-morrow the iniquity of the earth shall be blotted out. 

℟. And the Saviour of the world shall reign over us.

\properantiphon{Ben.}{The Saviour {\dag} of the world shall arise as the sun: and shall come down into the Virgin's womb, as the showers upon the grass, alleluia.}


\subby{Christmas Day}\label{ChristmasDay}
\feastday{Christmas}
%\begin{inhead}
%{First Class Double, Third Class Octave}
%\end{inhead}

\subbysub{I Evensong}\label{ChristmasEvensong}

\gregorioscore{resources/gabc/ProperSeason/ChristmasEvensong.gabc}

℣. To-morrow the iniquity of the earth shall be blotted out.

℟. And the Saviour of the world shall reign over us.

\properantiphon{Mag.}{When the sun hath risen {\dag} in the heavens, ye shall see the King of kings proceeding from the Father, as a bridegroom out of his chamber.}

\subbysub{Mattins}

\invitatoryhymn

\begin{rubric}
	The Invitatory Hymn is as in I Evensong, with the following ending.
\end{rubric}

\gregorioscore{resources/gabc/ProperSeason/ChristmasInvitatory.gabc}

\officehymn\label{ChristmasMattins}

\gregorioscore{resources/gabc/ProperSeason/ChristmasMattins.gabc}

℣. The Lord declared, alleluia.

℟. His salvation, alleluia.

\properantiphon{Ben.}{Glory {\dag} to God in the highest, and on earth peace to men of good will, alleluia, alleluia.}

\subbysub{II Evensong}

\begin{rubric}
	The Office Hymn is from I Evensong with the following Versicle and Antiphon.
\end{rubric}

℣. The Lord declared, alleluia.

℟. His salvation, alleluia.

\properantiphon{Mag.}{To-day {\dag} the Christ is born; to-day hath a Saviour appeared: to-day on earth Angels are singing, Archangels rejoicing: today the righteous exult and say. Glory to God in the highest, alleluia.}


\subby{26 December. St. Stephen}
\feastday{St. Stephen}
%\begin{inhead}
%{26 December}\par
%{Second Class Double, Simple Octave}
%\end{inhead}

\subbysub{Mattins}

\begin{rubric}
	The Invitatory Hymn is from the First Common of a Martyr Bishop (p. \pageref{CommonMartyrBishopI}).
\end{rubric}

\officehymn\label{StephenMattins}

\gregorioscore{resources/gabc/ProperSeason/StephenMattins.gabc}

℣. Devout men carried Stephen to his burial.

℟. And made great lamentation over him.

\properantiphon{Ben.}{And Stephen, {\dag} full of faith and power, did great wonders and miracles among the people.}

\subbysub{Evensong}\label{StephenEvensong}

\gregorioscore{resources/gabc/ProperSeason/StephenEvensong.gabc}

℣. Stephen saw the heavens opened.

℟. He saw, and entered in: how blessed is the man unto whom heaven stood open!

\properantiphon{Mag.}{Devout men {\dag} carried Stephen to his burial, and made great lamentation over him.}


\subby{27 December. St. John}
\feastday{St. John}
%\begin{inhead}
%{27 December}\par
%{Second Class Double, Simple Octave}
%\end{inhead}

\subbysub{Mattins}

\begin{rubric}
	The Invitatory Hymn is from the Common of Apostles (p. \pageref{CommonApostles}).
\end{rubric}

\officehymn\label{JohnMattins}

\gregorioscore{resources/gabc/ProperSeason/JohnMattins.gabc}

℣. This is that disciple which testifieth of these things.

℟. And we know that his testimony is true.

\properantiphon{Ben.}{This is the same John, {\dag} who leaned on the Lord's bosom at the last supper: the blessed Apostle, unto whom were revealed the secret things of heaven.}

\subbysub{Evensong}

\begin{rubric}
	The Office Hymn is of Mattins with the following Versicle \& Antiphon.
\end{rubric}

℣. Right worthy of honour is blessed John the Apostle.

℟. Who leaned on the Lord's bosom at the last supper.

\properantiphon{Mag.}{There went {\dag} this saying abroad among the brethren, that that disciple should not die: yet Jesus said not unto him, He shall not die; but, If I will that he tarry till I come.}


\subby{28 December. Holy Innocents}
\feastday{Holy Innocents}
%\begin{inhead}
%{28 December}\par
%{Second Class Double, Simple Octave}
%\end{inhead}

\subbysub{Mattins}

\invitatoryhymn\label{HolyInnocentsInvitatory}

\gregorioscore{resources/gabc/ProperSeason/HolyInnocentsInvitatory.gabc}

\officehymn\label{HolyInnocentsMattins}

\gregorioscore{resources/gabc/ProperSeason/HolyInnocentsMattins.gabc}

℣. Herod was exceeding wroth, and slew many children.

℟. In Bethlehem Judah, the city of David.

\properantiphon{Ben.}{These are they {\dag} which were not defiled with women; for they are virgins: and they follow the Lamb whithersoever he goeth.}

\subbysub{Evensong}

\begin{rubric}
	The Office Hymn is of Mattins with the following Versicle \& Antiphon.
\end{rubric}

℣. Beneath the altar all the Saints do cry aloud.

℟. Shall not our blood be avenged, O our God.

\properantiphon{Mag.}{Then were innocent children {\dag} slain instead of Christ by a wicked ruler; the very sucklings were put to death: spotless, they follow the Lamb himself, and say for ever: Glory be to thee, O Lord.}


\subby{Sunday within the Nativity Octave}
\feastday{Christmas I}
%\begin{inhead}
%{Semidouble}
%\end{inhead}

\begin{rubric}
	In the Daily Office, the Hymns are of Christmas Day (p. \pageref{ChristmasDay}), with the following.
\end{rubric}

℣. All the ends of the world have seen, alleluia.

℟. The salvation of our God, alleluia.

\properantiphon{Mag. \& Ben.}{While all things {\dag} were in quiet silence, and that night was in the midst of her swift course, thine Almighty Word, O Lord, leaped down out of thy royal throne, alleluia.}

\properantiphon{Mag.}{The Child Jesus {\dag} increased in wisdom and stature in the sight of God and man.}


\subby{1 January. Circumcision of Our Lord}
\feastday{Circumcision}
%\begin{inhead}
%{1 January}\par
%{Second Class Double}
%\end{inhead}

\begin{rubric}
	In the Daily Office, the Hymns are of Christmas Day (p. \pageref{ChristmasDay}), with the following.
\end{rubric}

℣. The Lord declared, alleluia.

℟. His salvation, alleluia.

\properantiphon{Mag.}{God, {\dag} for his great love wherewith he loved us, hath sent his own Son in the likeness of sinful flesh, alleluia.}\\

℣. The Word was made flesh, alleluia.

℟. And dwelt among us, alleluia.

\properantiphon{Ben.}{A great and wondrous mystery {\dag} is made known to us this day; a new thing is wrought in both natures: God is made man; that which was, remained, and that which was not, he assumed; suffering no confusion, nor yet division.}\\

℣. The Lord declared, alleluia.

℟. His salvation, alleluia.

\properantiphon{Mag.}{Great {\dag} is the mystery of the inheritance: the womb of her that knew not man is become the temple of the Godhead: by taking flesh of her, he was no way defiled: all the nations shall gather, saying: Glory be to thee, O Lord.}


\subby{2 January. Most Holy Name of Jesus}\label{MostHolyName}
\feastday{Most Holy Name}
\fancyhead[RE,LO]{2 January}
%\begin{inhead}
%{2 January}\par
%{Second Class Double}
%\end{inhead}

\subbysub{I Evensong}\label{HolyNameEvensong}

\gregorioscore{resources/gabc/ProperSeason/HolyNameEvensong.gabc}

℣. Blessed be the Name of the Lord, alleluia.

℟. From this time forth for evermore, alleluia.

\properantiphon{Mag.}{For he that is mighty {\dag} hath magnified me, and holy is his Name, alleluia.}

\subbysub{Mattins}

\invitatoryhymn\label{HolyNameInvitatory}

\gregorioscore{resources/gabc/ProperSeason/HolyNameInvitatory.gabc}

\officehymn\label{HolyNameMattins}

\gregorioscore{resources/gabc/ProperSeason/HolyNameMattins.gabc}

℣. Our help standeth in the Name of the Lord, alleluia. 

℟. Who hath made heaven and earth, alleluia.

\properantiphon{Ben.}{He gave himself {\dag} that he might deliver his people, and get him a perpetual Name, alleluia.}

\subbysub{II Evensong}

\begin{rubric}
	The Office Hymn \& Versicle are of I Evensong with the following Antiphon.
\end{rubric}

\properantiphon{Mag.}{Thou shalt call {\dag} his Name Jesus, for he shall save his people from their sins, alleluia.}

%MANUAL ADJUSTMENT:
\clearpage
\subby{Second Sunday after Nativity}
\feastday{Christmas II}
\fancyhead[RE,LO]{}
%\begin{inhead}
%{The Sunday between Circumcision \& Epiphany, if there be one.}\par
%{Semidouble}
%\end{inhead}

\properantiphon{Mag.}{Behold, the Angel of the Lord {\dag} appeareth in a dream to Joseph, saying: Arise, and take the young Child and his Mother, and flee into Egypt, and be thou there until I bring thee word; for it shall come to pass, that Herod will seek the young Child, to destroy him.}

\properantiphon{Ben.}{But when Herod was dead, {\dag} behold an Angel of the Lord appeareth in a dream to Joseph in Egypt, saying: Arise, and take the young Child and his Mother, and go into the land of Israel; for they are dead which sought the young Child's life.}

\properantiphon{Mag.}{When Joseph heard {\dag} that Archelaus did reign in Jud{\ae}a in the room of his father Herod, he was afraid to go thither; and being warned in a dream, he turned aside into the parts of Galilee: and he came and dwelt in a city which was called Nazareth; that it might be fulfilled which was spoken by the prophets, He shall be called a Nazarene.}


\subby{2 January. Octave Day of St. Stephen}
\fancyhead[RO,LE]{\textit{Stephen Octave Day}}
\fancyhead[RE,LO]{2 January}
%\begin{inhead}
%    {Simple\\
%2 January}
%\end{inhead}

\begin{rubric}
	The propers are from the Feast Day, except for that which followeth.
\end{rubric}

\collect
\lett{A}{lmighty} and everlasting God, who, in the blood of the blessed Levite Stephen, didst consecrate the first-fruits of the Martyrs: grant, we beseech thee; that he may ever intercede for us, who prayed even for his persecutors to our Lord Jesus Christ, thy Son. Who with thee liveth.


\subby{3 January. Octave Day of St. John}
\fancyhead[RO,LE]{\textit{John Octave Day}}
\fancyhead[RE,LO]{3 January}
%\begin{inhead}
%    {Simple\\
%3 January}
%\end{inhead}

\begin{rubric}
	The propers are from the Feast Day, except for that which followeth.
\end{rubric}

\antiphon{Mag.}{This is the same John {\dag} who learned on the Lord's bosom at the last supper: the blessed Apostle, unto whom were revealed the secret things of heaven.}

\collect
\lett{M}{erciful} Lord, we beseech thee to cast thy bright beams of light upon thy Church: that it being enlightened by the doctrine of thy blessed Apostle and Evangelist Saint John may so walk in the light of thy truth, that it may at length attain to the light of everlasting life. Through.

\secret
\lett{R}{eceive,} O Lord, the gifts which we offer unto thee on the solemnity of him, in whose advocacy we trust for deliverance. Through.

\postcommunion
\lett{O}{God,} who hast refreshed us with heavenly meat and drink, we humbly beseech thee: that we may be defended by the prayers of him, in whose memory we have received the same. Through.
\needspace{4\baselineskip}
\begin{rubric}
	If the Postcommunion of this Mass shall have been already said for some other Saint, the Postcommunion for St. John shall be that of his Feast before the Latin Gate, 6 May, as followeth.
\end{rubric}
\lett{W}{e} beseech thee, O Lord: that we, who have been refreshed with heavenly bread, may be nourished unto life eternal. Through.



\subby{4 January. Octave Day of Holy Innocents}
\fancyhead[RO,LE]{\textit{Innocents Octave Day}}
\fancyhead[RE,LO]{4 January}
%\begin{inhead}
%    {Simple\\
%4 January}
%\end{inhead}

\begin{rubric}
	The propers are from the Feast Day, except for that which followeth.\par
	\textsc{Note,} The \emph{Gloria in excelsis} is said.
\end{rubric}

℣. Herod was exceeding wroth, and slew many children.

℟. In Bethlehem Judah, the city of David.\\

\antiphon{Mag.}{These are they {\dag} which were not defiled with women; for they are virgins: and they follow the Lamb whithersoever he goeth.}

\begin{rubric}
	If a Commemoration only is made of the Octave, the Prayers (of the Feast) will be as followeth.
\end{rubric}

\collect
\lett{O}{almighty} God, who out of the mouths of babes and sucklings hast ordained strength, and madest infants to glorify thee by their deaths: mortify and kill all vices in us; and so strengthen us by thy grace, that by the innocency of our lives, and constancy of our faith even unto death, we may glorify thy holy name. Through.

\begin{rubric}
	Commemoration is made of St. Titus (p. \pageref{TitusCollect}).
\end{rubric}

\secret
\lett{M}{ay} the devout prayers of thy Saints never fail us, O Lord: that they may render our gifts acceptable unto thee, and ever obtain for us thy pardon. Through.

\begin{rubric}
	Commemoration is made of St. Titus (p. \pageref{TitusSecret}).
\end{rubric}

\postcommunion
\lett{W}{e} beseech thee, O Lord: that the gifts which we have offered and received may, through the prayers of the Saints, effectually avail for our succour both in this life, and in that which is to come. Through.

\begin{rubric}
	Commemoration is made of St. Titus (p. \pageref{TitusPostcommunion}).
\end{rubric}

\begin{rubric}
	\textsc{Note,} In Lent, either the Ferial Day's propers or Ash Wednesday's propers may be said.
\end{rubric}

\begin{rubric}
	The Office Hymn \& Versicles of the Feast of the Epiphany of Our Lord shall be used for the foregoing Sundays and Ferial Days until I Evensong of Septuagesima, exclusive.
\end{rubric}


\subby{4 January. St. Titus}\label{Titus}
\fancyhead[RO,LE]{\textit{Titus}}
\fancyhead[RE,LO]{4 January}
%\begin{inhead}
%    {Simple\\
%4 January}
%\end{inhead}

\begin{rubric}
	The propers are from the First Common of a Confessor Bishop (p. \pageref{CommonConfessorBishopI}), except for the Gospel \& Prayers as followeth.
\end{rubric}

\collect\label{TitusCollect}
\lett{O}{God,} who didst adorn blessed Titus, thy Confessor and Bishop, with apostolic virtues: grant, through his merits and intercession; that, living justly and godly in this world, we may be found worthy to attain unto the heavenly country. Through.

\readingcitation{Gospel}{Luke 10:1}
\lett{A}{t that time:} The Lord appointed seventy also: and sent them two and two before his face into every city and place, whither he himself would come. Therefore said he unto them, The harvest truly is great, but the labourers are few: pray ye therefore the Lord of the harvest, that he would send forth labourers into his harvest. Go your ways: behold, I send you forth as lambs among wolves. Carry neither purse, nor scrip, nor shoes: and salute no man by the way. And into whatsoever house ye enter, first say, Peace be to this house. And if the son of peace be there, your peace shall rest upon it: if not, it shall turn to you again. And in the same house remain, eating and drinking such things as they give: for the labourer is worthy of his hire. Go not from house to house. And into whatsoever city ye enter, and they receive you, eat such things as are set before you: And heal the sick that are therein, and say unto them, The kingdom of God is come nigh unto you.

\secret\label{TitusSecret}
\lett{W}{e} beseech thee, O Lord, that we remembering with gladness the merits of thy Saints, may in all places feel the succour of their intercession. Through.

\postcommunion\label{TitusPostcommunion}
\lett{G}{rant,} we beseech thee, almighty God: that we, shewing forth our thankfulness for the gifts which we have received, may, at the intercession of blessed Titus, thy Confessor and Bishop, obtain yet more abundant mercies. Through.


\subby{5 January. Vigil of Epiphany}
\fancyhead[RO,LE]{\textit{Epiphany Vigil}}
\fancyhead[RE,LO]{5 January}
%\begin{inhead}
%    {Second Class Vigil\\
%5 January}
%\end{inhead}

\begin{rubric}
	The Daily Office propers are as on the Feast of the Circumcision, except that which followeth.
\end{rubric}

\begin{rubric}
	\textsc{Note,} Commemoration is made of St. Telesphorus with the Prayers of the next Mass.
\end{rubric}

\antiphon{Mag.}{The Child Jesus {\dag} increased in wisdom and stature in the sight of God and man.}

\antiphon{Ben.}{While all things {\dag} were in quiet silence, and that night was in the midst of her swift course, thine Almighty Word, O Lord, leaped down out of thy royal throne, alleluia.}

\introit
\lett{W}{hile} all things were in in quiet silence, and night was in the midst of her swift course, thine almighty Word, O Lord, leaped down from heaven out of thy royal throne. \textit{Ps.} The Lord is King, and hath put on put glorious apparel: the Lord hath put on his apparel, and girded himself with strength.

\collect
\lett{A}{lmighty} and everlasting God, direct our actions according to thy good pleasure: that in the name of thy well-beloved Son we may be made worthy to abound in good works. Who liveth and reigneth with thee.

\begin{rubric}
	Commemoration of St. Telesphorus (p. \pageref{TelesphorusCollect}) and for St. Mary after Christmas (p. \SPMaryChristmas) is said.
\end{rubric}

\readingcitation{Epistle}{Galatians 4:1}
\lett{B}{rethren:} The heir, as long as he is a child, differeth nothing from a servant, though he be lord of all; But is under tutors and governors until the time appointed of the father. Even so we, when we were children, were in bondage under the elements of the world: But when the fulness of the time was come, God sent forth his Son, made of a woman, made under the law, To redeem them that were under the law, that we might receive the adoption of sons. And because ye are sons, God hath sent forth the Spirit of his Son into your hearts, crying, Abba, Father. Wherefore thou art no more a servant, but a son; and if a son, then an heir of God through Christ.

\gradall{Thou art fairer than the children of men: full of grace are thy lips. ℣. My heart is inditing of a good matter, I speak of the things which I have made unto the King: my tongue is the pen of a ready writer.}{Alleluia, alleluia. ℣. The Lord is King, and hath put on glorious apparel: the Lord hath put on his apparel, and girded himself with strength. Alleluia.}

\readingcitation{Gospel}{Matthew 2:19}
\lett{A}{t that time:} When Herod was dead, behold, an angel of the Lord appeareth in a dream to Joseph in Egypt, Saying, Arise, and take the young child and his mother, and go into the land of Israel: for they are dead which sought the young child's life. And he arose, and took the young child and his mother, and came into the land of Israel. But when he heard that Archelaus did reign in Judaea in the room of his father Herod, he was afraid to go thither: notwithstanding, being warned of God in a dream, he turned aside into the parts of Galilee: And he came and dwelt in a city called Nazareth: that it might be fulfilled which was spoken by the prophets, He shall be called a Nazarene.

\offertory{God hath made the round world so sure: that it cannot be moved: ever since the world began, hath thy seat, O God, been prepared, thou art from everlasting.}

\secret
\lett{G}{rant,} we beseech thee, almighty God: that the gift which we offer in the sight of thy majesty, may obtain for us grace to serve thee with all godliness, and bring us in the end to everlasting felicity. Through.

\begin{rubric}
	Commemoration of St. Telesphorus (p. \pageref{TelesphorusSecret}) and for St. Mary after Christmas (p. \SPMaryChristmas) is said.
\end{rubric}

\communion{Take the young Child and his Mother and go into the land of Israel: for they are ead which sought the young Child's life.}

\postcommunion
\lett{M}{ay} the operation of this mystery, Lord, avail for the cleansing of our sins, and for the fulfilment of our godly desires. Through.

\begin{rubric}
	Commemoration of St. Telesphorus (p. \pageref{TelesphorusPostcommunion}) and for St. Mary after Christmas (p. \SPMaryChristmas) is said.
\end{rubric}


\subby{5 January. St. Telesphorus}
\fancyhead[RO,LE]{\textit{Telesphorus}}
\fancyhead[RE,LO]{5 January}
%\begin{inhead}
%    {Memorial\\
%5 January}
%\end{inhead}

\begin{rubric}
	The propers are from the Second Common of a Martyr Bishop (p. \pageref{CommonMartyrBishopII}), with the Prayers below.
\end{rubric}

\collect\label{TelesphorusCollect}
\lett{O}{God,} who makest us glad with the yearly solemnity of blessed Telesphorus, thy Martyr and Bishop: mercifully grant; that, as we now celebrate his birthday so we may likewise rejoice in his protection. (Through.)

\secret\label{TelesphorusSecret}
\lett{S}{anctify,} O Lord, the gifts which we dedicate to thee: that at the intercession of blessed Telesphorus, thy Martyr and Bishop, they may obtain for us thy gracious favour. (Through.)

\postcommunion\label{TelesphorusPostcommunion}
\lett{W}{e} beseech thee, O Lord our God, that like as we, whom thou hast refreshed by the partaking of thy sacred gift, do offer unto thee our worship: so by the intercession of blessed Telesphorus thy Martyr and Bishop, we may perceive the benefit of the same. (Through.)\\


\subby{Epiphany of Our Lord}\label{EpiphanyDay}
\feastday{Epiphany}
\fancyhead[RE,LO]{}
%\begin{inhead}
%{First Class Double, Second Class Octave}
%\end{inhead}

\subbysub{I Evensong}\label{EpiphanyEvensong}

\gregorioscore{resources/gabc/ProperSeason/EpiphanyEvensong.gabc}

℣. The kings of Tharsis and of the isles shall give presents.

℟. The kings of Arabia and Saba shall bring gifts.

\properantiphon{Mag.}{The wise men, {\dag} beholding the star, said one to another, This is the sign of a mighty King; forth fare we, and let us seek him: and let us offer him gifts, gold, incense, and myrrh, alleluia.}

\subbysub{Mattins}

\begin{rubric}
	The Invitatory Hymn is as in I Evensong.
\end{rubric}

\officehymn\label{EpiphanyMattins}

\gregorioscore{resources/gabc/ProperSeason/EpiphanyMattins.gabc}

	℣. O worship the Lord, alleluia.

	℟. All ye Angels of his, alleluia.
	
	\properantiphon{Ben.}{To-day {\dag} the Church is joined to her heavenly Bridegroom; because in Jordan Christ hath washed away her offences: the wise men with their offerings hasten to the royal marriage, and the guests are regaled with water made wine, alleluia.}

\subbysub{II Evensong}

\begin{rubric}
	The Office Hymn \& Versicle are of I Evensong, with the following Antiphon.
\end{rubric}

\properantiphon{Mag.}{Now do we celebrate {\dag} a holy day adorned by three miracles: to-day a star led the wise men to the manger; to-day water was made wine at the wedding feast; to-day Christ vouchsafed to be baptized of John in Jordan that he might save us, alleluia.}


\begin{rubric}
	Within the Octave of the Epiphany, no Feast is kept except an occurrent I Double, on which Commemoration of the Octave is always made.
	
	During the Octave all is said as on the Feast, except the Ants. on Benedictus and Magnificat appointed for each day.
	
	The Office of Sunday within the Octave is said as set forth below, on whatever day in the Octave it may fall.
	
	After Sunday, the days in the Octave continue according to their number, as if Sunday had not intervened. For example, if Sunday should fall on the third day in the Octave, on the following day the Ants. on Benedictus and Magnificat of the third day in the Octave are said, and so thenceforth the others.
\end{rubric}


\subby{Within the Octave of the Epiphany}
\feastday{Within Epiphany Octave}

%MANUAL ADJUSTMENT:
%\begin{multicols}{2}
\subsubsec{Day II, Semidouble}
\antiphon{Ben.}{From the east {\dag} there came wise men to Bethlehem, to worship the Lord: and when they had opened their treasures, they presented unto him precious gifts: gold as to a mighty King, incense as to the true God, and myrrh to foreshew his burial, alleluia.}

\antiphon{Mag.}{When the wise men saw the star {\dag} they rejoiced with exceeding great joy: and when they were come into the house, they presented unto the Lord gold, frankincense, and myrrh, alleluia.}

\subsubsec{Day III, Semidouble}
\antiphon{Ben.}{Three are the gifts {\dag} which the wise men presented unto the Lord: gold, frankincense, and myrrh, to the Son of God, to the mighty King, alleluia.}

\antiphon{Mag.}{Light of light, {\dag} thou, O Christ, hast appeared, unto whom the wise men present their gifts, alleluia, alleluia, alleluia.}
%\end{multicols}
%\clearpage
%MANUAL ADJUSTMENT:
%\begin{multicols}{2}
\subsubsec{Day IV, Semidouble}
\antiphon{Ben.}{We have seen his star {\dag} in the East, and are come with gifts to worship the Lord.}

\antiphon{Mag.}{Herod inquired {\dag} of the wise men, What sign have ye seen, concerning him that is born King? We have seen a star shining, the brightness whereof enlightens the world.}

\subsubsec{Day V, Semidouble}
\antiphon{Ben.}{Many nations {\dag} shall come from afar, bearing their gifts, alleluia.}

\antiphon{Mag.}{All they from Sheba {\dag} shall come: they shall bring gold and incense, alleluia, alleluia.}

\subsubsec{Day VI, Semidouble}
\antiphon{Ben.}{They that despised thee {\dag} shall come unto thee, and shall bow themselves down at the soles of thy feet.}

\antiphon{Mag.}{The wise men, being warned {\dag} in dreams by an Angel, departed into their own country another way.}

\subsubsec{Saturday within the Octave}
℣. We have seen his star in the east, alleluia.

℟. And are come with gifts to worship the Lord, alleluia.

\antiphon{Mag.}{The Child Jesus {\dag} tarried behind in Jerusalem; and Joseph and his mother knew not of it, supposing him to have been in the company; and they sought him among their kinsfolk and acquaintance.}

%\end{multicols}

\begin{rubric}
	When the Octave Day of the Epiphany falls on Sunday, the Office of the Sunday in the Octave is said on the preceding Saturday, and I Evensong of Sunday is said on Friday with Commemoration of the preceding day in the Octave.
	
	But if a I Double occur on this Saturday, the Office of Sunday is anticipated on the nearest day on which only the Offce of the Octave is to be said, and in the Office of the Feast Commemoration is made of the occurrent day within the Octave. In the Office of Sunday all is said as noted below.
\end{rubric}


\subby{Sunday within the Octave of the Epiphany}
\feastday{Epiphany I}
%\begin{inhead}
%{Semidouble}
%\end{inhead}

\begin{rubric}
	The Office Hymns are from the Feast of the Epiphany (p. \pageref{EpiphanyDay}), except for the following.
\end{rubric}

℣. All they from Sheba shall come, alleluia.

℟. They shall bring gold and incense, alleluia.

\properantiphon{Ben.}{The Child Jesus {\dag} tarried behind in Jerusalem; and Joseph and his mother knew not of it, supposing him to have been in the company; and they sought him among their kinsfolk and acquaintance.}\\

%MANUAL ADJUSTMENT:
\clearpage

℣. We have seen his star in the east, alleluia.

℟. And are come with gifts to worship the Lord, alleluia.

\properantiphon{Mag.}{Son, {\dag} why hast thou thus dealt with us? behold, thy father and I have sought thee sorrowing. How is it that ye sought me? wist ye not that I must be about my Father's business?}


\subby{Baptism of Our Lord}

\centerline{\small{(Octave Day of the Epiphany)}}

\feastday{Baptism of Our Lord}
%\begin{inhead}
%{Greater Double}
%\end{inhead}

\begin{rubric}
	The Daily Office propers are as in the Feast of the Epiphany (p. \pageref{EpiphanyDay}).
\end{rubric}


\begin{rubric}
	From the Octave of Epiphany until the Saturday before the I Sunday in Lent, when the Office is of the Feria, ail is said as in the Daily Hymns except that which is appointed as Proper. The Collect is that of the preceding Sunday.
\end{rubric}


\subby{Second Sunday after Epiphany}\label{EpiphanyII}
\feastday{Epiphany II}
%\begin{inhead}
%{Semidouble}
%\end{inhead}

\begin{rubric}
	The Office Hymns \& Versicles are from the Daily Hymns (p. \pageref{DailyHymns}), with the following Antiphons.
\end{rubric}

\properantiphon{Mag.}{God hath holpen {\dag} his servant Israel, as he promised to Abraham and to his seed: and hath exalted the humble for ever and ever.}

\properantiphon{Ben.}{There was a marriage {\dag} in Cana of Galilee, and Jesus was there with Mary his Mother.}

\properantiphon{Mag.}{And when they wanted wine, {\dag} Jesus bade them to fill the water-pots with water; and it was turned into wine, alleluia.}


\subby{Third Sunday after Epiphany}\label{epiphany}
\feastday{Epiphany III}
%\begin{inhead}
%{Semidouble}
%\end{inhead}

\properantiphon{Mag.}{God hath holpen {\dag} his servant Israel, as he promised to Abraham and to his seed: and hath exalted the humble for ever and ever.}

\properantiphon{Ben.}{When Jesus {\dag} was come down from the mountain, behold there came a leper and worshipped him, saying, Lord, if thou wilt, thou canst make me clean. And Jesus put forth his hand and touched him, saying: I will; be thou clean.}

\properantiphon{Mag.}{Lord, if thou wilt, {\dag} thou canst make me clean: then saith Jesus, I will; be thou clean.}


\subby{Fourth Sunday after Epiphany}
\feastday{Epiphany IV}
%\begin{inhead}
%{Semidouble}
%\end{inhead}

\properantiphon{Mag.}{God hath holpen {\dag} his servant Israel, as he promised to Abraham and to his seed: and hath exalted the humble for ever and ever.}

\properantiphon{Ben.}{And when Jesus {\dag} was entered into a ship, behold, there arose a great tempest in the sea: and his disciples awoke him, saying: Lord, save us, we perish.}

\properantiphon{Mag.}{Save us, Lord, {\dag} we perish: rebuke the winds and the sea, O God, and make a great calm.}


\subby{Fifth Sunday after Epiphany}\label{EpiphanyV}
\feastday{Epiphany V}
%\begin{inhead}
%{Semidouble}
%\end{inhead}

\properantiphon{Mag.}{God hath holpen {\dag} his servant Israel, as he promised to Abraham and to his seed: and hath exalted the humble for ever and ever.}

\properantiphon{Ben.}{Sir, {\dag} didst not thou sow good seed in thy field? from whence then hath it tares? He said unto them, An enemy hath done this.}

\properantiphon{Mag.}{Gather ye together {\dag} first the tares, and bind them in bundles to burn them: but gather the wheat into my barn, saith the Lord.}


\subby{Sixth Sunday after Epiphany}
\feastday{Epiphany VI}
%\begin{inhead}
%{Semidouble}
%\end{inhead}

\properantiphon{Mag.}{God hath holpen {\dag} his servant Israel, as he promised to Abraham and to his seed: and hath exalted the humble for ever and ever.}

\properantiphon{Ben.}{The kingdom of heaven {\dag} is like unto a grain of mustard seed, which is the least of all seeds; but when it is grown, it is the greatest among herbs.}

\properantiphon{Mag.}{The kingdom of heaven {\dag} is like unto leaven, which a woman took and hid in three measures of meal, until the whole was leavened.}

%MANUAL ADJUSTMENT:
\vspace{-1.5ex}

\subby{Septuagesima Sunday}
\fancyhead[RO,LE]{\textit{Septuagesima}}
\fancyhead[RE,LO]{Sunday}
%\begin{inhead}
%    {Second Class Semidouble}
%\end{inhead}

\properantiphon{Mag.}{The Lord said {\dag} unto Adam, Of the tree which is in the midst of the garden thou shalt not eat: for in the day that thou eatest thereof, thou shalt surely die.}

\properantiphon{Ben.}{The kingdom {\dag} of heaven is like unto a man that is an householder, which went out early in the morning to hire labourers into his vineyard, saith the Lord.}

\properantiphon{Mag.}{The householder {\dag} saith unto the labourers, Why stand ye here all the day idle? They say unto him, Because no man hath hired us. Go ye also into the vineyard, and whatsoever is right, I will give you.}

%MANUAL ADJUSTMENT:
\vspace{-0.5ex}

\subby{Septuagesimatide Ferial Office}
\begin{rubric}
On Ferias from Septuagesima Sunday until Ash Wednesday, exclusive, when the Office is not of the Feria, it is always commemorated at Mattins and Evensong on Double Feasts, even of the I Class, and on days within Octaves.
\end{rubric}
\begin{rubric}
	The \emph{Benedictus} Antiphons are said on Ferias as set forth in the Psalter, through Tuesday after Quinquagesima.
\end{rubric}
\begin{rubric}
	\textsc{Note,} No notice is taken of an occurrent Vigil, either in the Office of a Feast, or of a day within an Octave, or of a Feria. The Office of St. Mary on Saturday is not said.
\end{rubric}

%MANUAL ADJUSTMENT:
\vspace{-1.5ex}

\subby{Antiphons within Septuagesima Week}
\fancyhead[RO,LE]{Septuagesima Week}
\fancyhead[RE,LO]{}
%\begin{multicols}{2}

\subsubsec{Septuagesima Monday, Feria}
\antiphon{Mag.}{These last {\dag} have wrought but one hour, and thou hast made them equal unto us, which have borne the burden and heat of the day.}


\subsubsec{Septuagesima Tuesday, Feria}
\antiphon{Mag.}{The householder answered and said, {\dag} Friend, I do thee no wrong: didst not thou agree with me for a penny? Take that thine is, and go thy way.}


\subsubsec{Septuagesima Wednesday, Feria}
\antiphon{Mag.}{Take that thine is {\dag} and go thy way: because I am good, saith the Lord.}


\subsubsec{Septuagesima Thursday, Feria}
\antiphon{Mag.}{Is it not lawful {\dag} for me to do what I will? Is thine eye evil, because I am good? saith the Lord.}


\subsubsec{Septuagesima Friday, Feria}
\antiphon{Mag.}{So the last {\dag} shall be first, and the first shall be last: for many are called, but few are chosen.}

%\end{multicols}


\subby{Sexagesima Sunday}
\feastday{Sexagesima}
%\begin{inhead}
%{Second Class Semidouble}
%\end{inhead}

\properantiphon{Mag.}{The Lord {\dag} said unto Noah: The end of all flesh is come before me: make thee an ark of gopher wood, that therein the seed of all flesh may be saved.}

\properantiphon{Ben.}{When much people {\dag} were gathered together to Jesus, and were come to him out of every city, he spake by a parable: A sower went out to sow his seed.}

\properantiphon{Mag.}{Unto you it is given {\dag} to know the mysteries of the kingdom of heaven: but to others in parables, said Jesus to his disciples.}


\subby{Antiphons within Sexagesima Week}
\fancyhead[RO,LE]{Sexagesima Week}
\fancyhead[RE,LO]{}
%\begin{multicols}{2}

\subsubsec{Sexagesima Monday, Feria}
\antiphon{Mag.}{If ye seek {\dag} the summit of true honour, hasten to yon heavenly country with what speed ye may.}


\subsubsec{Sexagesima Tuesday, Feria}
\antiphon{Mag.}{The seed {\dag} is the word of God, but Christ is the Sower: every one that heareth him shall abide for ever.}


\subsubsec{Sexagesima Wednesday, Feria}
\antiphon{Mag.}{But that {\dag} which fell on the good ground are they which in an honest and good heart receive the word; and bring forth fruit with patience.}


\subsubsec{Sexagesima Thursday, Feria}
\antiphon{Mag.}{Some seed {\dag} fell on good ground, and brought forth fruit; some an hundred-fold, and some sixtyfold.}


\subsubsec{Sexagesima Friday, Feria}
\antiphon{Mag.}{They who keep the word of God {\dag} with an honest and perfect heart bring forth fruit with patience.}
%\end{multicols}


\subby{Quinquagesima Sunday}
\feastday{Quinquagesima}
%\begin{inhead}
%{Second Class Semidouble}
%\end{inhead}

\properantiphon{Mag.}{Mighty Abraham, {\dag} the father of our faith, offered a burnt offering upon the altar, instead of his son.}

\properantiphon{Ben.}{Behold, we go up {\dag} to Jerusalem, and all things that are written concerning the Son of man shall be accomplished: for he shall be delivered unto the Gentiles, and shall be mocked, and spitted on; and they shall scourge him, and put him to death; and the third day he shall rise again.}

\properantiphon{Mag.}{And Jesus stood, {\dag} and commanded him to be brought unto him, and asked him, saying: What wilt thou that I shall do unto thee? Lord, that I may receive my sight. And Jesus said unto him, Receive thy sight, thy faith hath saved thee. And straightway he received his sight, and followed him, glorifying God.}


\subby{Antiphons within Quinquagesima Week}
\fancyhead[RO,LE]{Quinquagesima Week}
\fancyhead[RE,LO]{}
%\begin{multicols}{2}

\subsubsec{Quinquagesima Monday, Feria}
\antiphon{Mag.}{And they which went before {\dag} rebuked him that he should hold his peace: but he cried so much the more, Have mercy on me, thou Son of David.}


\subsubsec{Quinquagesima Tuesday, Feria}
\antiphon{Mag.}{Have mercy on me, {\dag} thou Son of David. What wilt thou that I shall do unto thee? Lord, that I may receive my sight.}

%\end{multicols}


\begin{rubric}
	If on the following Ash Wednesday there occur a I or II Double, it is transferred to the first unhindered day. Greater and lesser Doubles and Memorials are only commemorated on the other days of Lent.
\end{rubric}


\subby{Ash Wednesday}
\fancyhead[RO,LE]{\textit{Ash Wednesday}}
\fancyhead[RE,LO]{}
%\begin{inhead}
%    {Greater Privileged Feria}
%\end{inhead}

\begin{rubric}
	On this day all Octaves cease until Holy Sabbath. On this and other Ferias through None of the following Saturday all is said as in the Psalter throughout the Year except that which is appointed as proper.
\end{rubric}

\begin{rubric}
	The Daily Office propers are of the Ferial Day, except the following Antiphons.
\end{rubric}

\properantiphon{Ben.}{When ye fast, {\dag} be not, as the hypocrites, of a sad countenance.}

\properantiphon{Mag.}{Lay up for yourselves {\dag} treasures in heaven, where neither moth nor rust doth corrupt.}



\subby{Thursday after Ash Wednesday}
\fancyhead[RO,LE]{\textit{Ash Wednesday}}
\fancyhead[RE,LO]{Thursday}
%\begin{inhead}
%    {Greater Feria}
%\end{inhead}

\begin{rubric}
	The Daily Office propers are of the Ferial Day, except the following Antiphons and Collects.
\end{rubric}

\begin{paracol}{2}

\sloppy

\begin{inhead}
	Mattins
\end{inhead}

\antiphon{Ben.}{Lord, my servant {\dag} lieth at home sick of the palsy, grievously tormented. Verily I say unto thee, I will come and heal him.}

\lett{O}{God,} who art offended by sin, and reconciled by penitence: mercifully regard the prayers of thy suppliant people, and turn away the scourge of thy wrath, which for our sins we have justly deserved. Through.

\switchcolumn

\begin{inhead}
	II Evensong
\end{inhead}

\antiphon{Mag.}{Lord, I am not worthy {\dag} that thou shouldest enter under my roof: but speak the word only and my servant shall be healed.}

\lett{S}{pare} us, O Lord, spare thy people: that they who are justly chastised by thy scourges, may be relieved by thy tender mercy. Through.

\fussy

\end{paracol}

\introit
\lett{W}{hen} I called upon the Lord, he heard my voice from the battle that was against me, yea, even God, that endureth for ever, shall bring them down: O cast thy burden upon the Lord, and he shall nourish thee. \textit{Ps.} Hear my prayer, O God, and hide not thyself from my petition: take heed unto me and hear me.

\collect
\lett{O}{God,} who art wroth with them that sin against thee, and sparest them that are penitent: mercifully look upon the prayers of thy people which call upon thee; and turn away the scourges of thy wrath which for our sins we justly deserve. Through.

\begin{rubric}
    \nth{2} Collect is \emph{Of Saints} (p. \SPSaints) \& \nth{3} \emph{Of the Living and Departed} (p. \SPLiving).
\end{rubric}

\readingcitation{Epistle}{Isaiah 38:1}
\lett{I}{n those days:} Was Hezekiah sick unto death. And Isaiah the prophet the son of Amoz came unto him, and said unto him, Thus saith the \divineName{Lord}, Set thine house in order: for thou shalt die, and not live. Then Hezekiah turned his face toward the wall, and prayed unto the \divineName{Lord}, And said, Remember now, O \divineName{Lord}, I beseech thee, how I have walked before thee in truth and with a perfect heart, and have done that which is good in thy sight. And Hezekiah wept sore. Then came the word of the \divineName{Lord} to Isaiah, saying, Go, and say to Hezekiah, Thus saith the \divineName{Lord}, the God of David thy father, I have heard thy prayer, I have seen thy tears: behold, I will add unto thy days fifteen years. And I will deliver thee and this city out of the hand of the king of Assyria: and I will defend this city, saith the Lord almighty.

\gradual{O cast thy burden upon the Lord, and he shall nourish thee. ℣. When I called upon the Lord, he heard my voice from the battle that was against me.}

\readingcitation{Gospel}{Matthew 8:5}
\lett{A}{t that time:} When Jesus was entered into Capernaum, there came unto him a centurion, beseeching him, And saying, Lord, my servant lieth at home sick of the palsy, grievously tormented. And Jesus saith unto him, I will come and heal him. The centurion answered and said, Lord, I am not worthy that thou shouldest come under my roof: but speak the word only, and my servant shall be healed. For I am a man under authority, having soldiers under me: and I say to this man, Go, and he goeth; and to another, Come, and he cometh; and to my servant, Do this, and he doeth it. When Jesus heard it, he marvelled, and said to them that followed, Verily I say unto you, I have not found so great faith, no, not in Israel. And I say unto you, That many shall come from the east and west, and shall sit down with Abraham, and Isaac, and Jacob, in the kingdom of heaven. But the children of the kingdom shall be cast out into outer darkness: there shall be weeping and gnashing of teeth. And Jesus said unto the centurion, Go thy way; and as thou hast believed, so be it done unto thee. And his servant was healed in the selfsame hour.

\offertory{Unto thee, O Lord, will I lift up my soul: my God, I have put my trust in thee, O let me not be confounded: neither let mine enemies triumph over me: for all they that hope in thee shall not be ashamed.}

\secret
\lett{W}{e} beseech thee, O Lord, favourably to regard these our sacrifices: that they may be profitable for our devotion and set forward our salvation. Through.

\begin{rubric}
    \nth{2} Secret is \emph{Of Saints} (p. \pageref{SPSaints}) \& \nth{3} \emph{Of the Living and Departed} (p. \pageref{SPLivingDeparted}).
\end{rubric}

\communion{Thou shalt be pleased with the sacrifice of righteousness, with the burnt-offerings and oblations upon thine altar, O Lord.}

\postcommunion
\lett{W}{e} humbly beseech thee, almighty God: that, as we have received the blessing of this heavenly gift; so it may be made to us thy sacrament, and avail to our salvation. Through.

\begin{rubric}
    \nth{2} Postcommunion is \emph{Of Saints} (p. \SPSaints) \& \nth{3} \emph{Of the Living and Departed} (p. \pageref{SPLivingDeparted}).
\end{rubric}

\textsc{Priest.} Let us pray.\par
\textsc{Deacon.} Humble your heads before God.\par
\begin{rubric}
    The Priest then prays the following:
\end{rubric}
\lett{S}{pare,} O Lord, spare thy people: that they who are justly chastised by thy scourges, may by thy mercy be relieved. Through.



\subby{Friday after Ash Wednesday}
\fancyhead[RO,LE]{\textit{Ash Wednesday}}
\fancyhead[RE,LO]{Friday}
%\begin{inhead}
%    {Greater Feria}
%\end{inhead}

\begin{rubric}
	The Daily Office propers are of the Ferial Day, except the following Antiphons and Collects.
\end{rubric}

\begin{paracol}{2}

\sloppy

\begin{inhead}
	Mattins
\end{inhead}

\antiphon{Ben.}{When thou doest thine alms, {\dag} let not thy left hand know what thy right. hand doeth.}

\lett{W}{e} beseech thee, O Lord, to accompany with thy bounteous favour the fast upon which we have entered: that the observance which we shew forth in our bodies, we may be able also to practise with sincerity of heart. Through.

\switchcolumn

\begin{inhead}
	II Evensong
\end{inhead}

\antiphon{Mag.}{But thou, when thou prayest, {\dag} enter into thy closet, and when thou hast shut thy door, pray to thy Father.}

\lett{D}{efend} thy people, O Lord, and mercifully cleanse them from all their sins: for no adversity will harm them over whom iniquity hath no dominion. Through.

\fussy

\end{paracol}

\introit
\lett{T}{he} Lord heard, and had mercy upon me: the Lord became my helper. \textit{Ps.} I will magnify thee, O Lord, for thou hast set me up: and not made my foes to triumph over me.

\collect
\lett{W}{e} beseech thee, O Lord, to further with thy gracious favour the fasts which we have begun: that as we keep this observance in the flesh, so we may have strength to perform the same in singleness of heart. Through.

\begin{rubric}
    \nth{2} Collect is \emph{Of Saints} (p. \pageref{SPSaints}) \& \nth{3} \emph{Of the Living and Departed} (p. \pageref{SPLivingDeparted}).
\end{rubric}

\readingcitation{Epistle}{Isaiah 58:1}
\lett{T}{hus saith the Lord God:} Cry aloud, spare not, lift up thy voice like a trumpet, and shew my people their transgression, and the house of Jacob their sins. Yet they seek me daily, and delight to know my ways, as a nation that did righteousness, and forsook not the ordinance of their God: they ask of me the ordinances of justice; they take delight in approaching to God. Wherefore have we fasted, say they, and thou seest not? wherefore have we afflicted our soul, and thou takest no knowledge? Behold, in the day of your fast ye find pleasure, and exact all your labours. Behold, ye fast for strife and debate, and to smite with the fist of wickedness: ye shall not fast as ye do this day, to make your voice to be heard on high. Is it such a fast that I have chosen? a day for a man to afflict his soul? is it to bow down his head as a bulrush, and to spread sackcloth and ashes under him? wilt thou call this a fast, and an acceptable day to the \divineName{Lord}? Is not this the fast that I have chosen? to loose the bands of wickedness, to undo the heavy burdens, and to let the oppressed go free, and that ye break every yoke? Is it not to deal thy bread to the hungry, and that thou bring the poor that are cast out to thy house? when thou seest the naked, that thou cover him; and that thou hide not thyself from thine own flesh? Then shall thy light break forth as the morning, and thine health shall spring forth speedily: and thy righteousness shall go before thee; the glory of the \divineName{Lord} shall be thy reward. Then shalt thou call, and the \divineName{Lord} shall answer; thou shalt cry, and he shall say, Here I am. For I the \divineName{Lord} thy God am merciful.

\gradual{One thing have I desired of the Lord, which I will require, even that I may dwell in the house of the Lord. ℣. To behold the fair beauty of the Lord, and to hide me in his holy temple.}

\tract{O Lord, deal not with us after our sins: nor reward us according to our wickednesses. ℣. O Lord, remember not our old sins: but have mercy upon us, and that soon, for we are come to great misery. \inrub{Here genuflect.} ℣. Help us, O God of our salvation: and for the glory of thy name, O Lord, deliver us: and be merciful unto our sins, for thy name's sake.}

\readingcitation{Gospel}{Matthew 5:43}
\lett{A}{t that time:} Jesus said unto his disciples: Ye have heard that it hath been said, Thou shalt love thy neighbour, and hate thine enemy. But I say unto you, Love your enemies, bless them that curse you, do good to them that hate you, and pray for them which despitefully use you, and persecute you; That ye may be the children of your Father which is in heaven: for he maketh his sun to rise on the evil and on the good, and sendeth rain on the just and on the unjust. For if ye love them which love you, what reward have ye? do not even the publicans the same? And if ye salute your brethren only, what do ye more than others? do not even the publicans so? Be ye therefore perfect, even as your Father which is in heaven is perfect. Take heed that ye do not your alms before men, to be seen of them: otherwise ye have no reward of your Father which is in heaven. Therefore when thou doest thine alms, do not sound a trumpet before thee, as the hypocrites do in the synagogues and in the streets, that they may have glory of men. Verily I say unto you, They have their reward. But when thou doest alms, let not thy left hand know what thy right hand doeth: That thine alms may be in secret: and thy Father which seeth in secret himself shall reward thee openly.

\offertory{Quicken me, O Lord, according to thy word: that I may know thy testimonies.}

\secret
\lett{G}{rant,} we beseech thee, O Lord, that this sacrifice of Lenten observance which we offer: may both render our souls acceptable unto thee, and make us more readily to serve thee in continence. Through.

\begin{rubric}
    \nth{2} Secret is \emph{Of Saints} (p. \pageref{SPSaints}) \& \nth{3} \emph{Of the Living and Departed} (p. \pageref{SPLivingDeparted}).
\end{rubric}

\communion{Serve the Lord in fear, and rejoice unto him with reverence: lay hold on discipline, lest ye perish from the right way.}

\postcommunion
\lett{P}{our} forth upon us, O Lord, the Spirit of thy charity: that as thou hast fulfilled us with one heavenly bread, so of thy goodness thou wouldest make us to be of one heart and mind. Through . . . in the unity of the same.

\begin{rubric}
    \nth{2} Postcommunion is \emph{Of Saints} (p. \pageref{SPSaints}) \& \nth{3} \emph{Of the Living and Departed} (p. \pageref{SPLivingDeparted}).
\end{rubric}

\textsc{Priest.} Let us pray.\par
\textsc{Deacon.} Humble your heads before God.\par
\begin{rubric}
    The Priest then prays the following:
\end{rubric}
\lett{D}{efend,} O Lord, thy people, and mercifully cleanse them from all their sins: that no adversity may harm them, over whom iniquity hath no dominion. Through.


\subby{Saturday after Ash Wednesday}
\fancyhead[RO,LE]{\textit{Ash Wednesday}}
\fancyhead[RE,LO]{Saturday}
%\begin{inhead}
%    {Greater Feria}
%\end{inhead}

\begin{rubric}
	In the Daily Office, the propers are of the Ferial Day, except the following Antiphon.
\end{rubric}

\antiphon{Ben.}{Yet they seek me daily, {\dag} and delight to know my ways.}

\introit
\lett{T}{he} Lord heard, and had mercy upon me: the Lord became my helper. \textit{Ps.} I will magnify thee, O Lord, for thou hast set me up: and not made my foes to triumph over me.

\collect
\lett{A}{ssist} us, O Lord, in these our supplications: and grant; that like as this solemn fast hath been ordained for the safety and healing of our bodies and our souls, so we may with devout observance celebrate the same. Through.

\begin{rubric}
    \nth{2} Collect is \emph{Of Saints} (p. \pageref{SPSaints}) \& \nth{3} \emph{Of the Living and Departed} (p. \pageref{SPLivingDeparted}).
\end{rubric}

\readingcitation{Epistle}{Isaiah 58:9}
\lett{T}{hus saith the Lord God:} If thou take away from the midst of thee the yoke, the putting forth of the finger, and speaking vanity; And if thou draw out thy soul to the hungry, and satisfy the afflicted soul; then shall thy light rise in obscurity, and thy darkness be as the noonday: And the \divineName{Lord} shall guide thee continually, and satisfy thy soul in drought, and make fat thy bones: and thou shalt be like a watered garden, and like a spring of water, whose waters fail not. And they that shall be of thee shall build the old waste places: thou shalt raise up the foundations of many generations; and thou shalt be called, The repairer of the breach, The restorer of paths to dwell in. If thou turn away thy foot from the sabbath, from doing thy pleasure on my holy day; and call the sabbath a delight, the holy of the \divineName{Lord}, honourable; and shalt honour him, not doing thine own ways, nor finding thine own pleasure, nor speaking thine own words: Then shalt thou delight thyself in the \divineName{Lord}; and I will cause thee to ride upon the high places of the earth, and feed thee with the heritage of Jacob thy father: for the mouth of the \divineName{Lord} hath spoken it.

\gradual{One thing have I desired of the Lord, which I will require, even that I may dwell in the house of the Lord. ℣. To behold the fair beauty of the Lord, and to visit his temple.}


\readingcitation{Gospel}{Mark 6:47}
\lett{A}{t that time:} When even was come, the ship was in the midst of the sea, and Jesus alone on the land. And he saw them toiling in rowing; for the wind was contrary unto them: and about the fourth watch of the night he cometh unto them, walking upon the sea, and would have passed by them. But when they saw him walking upon the sea, they supposed it had been a spirit, and cried out: For they all saw him, and were troubled. And immediately he talked with them, and saith unto them, Be of good cheer: it is I; be not afraid. And he went up unto them into the ship; and the wind ceased: and they were sore amazed in themselves beyond measure, and wondered. For they considered not the miracle of the loaves: for their heart was hardened. And when they had passed over, they came into the land of Gennesaret, and drew to the shore. And when they were come out of the ship, straightway they knew him, And ran through that whole region round about, and began to carry about in beds those that were sick, where they heard he was. And whithersoever he entered, into villages, or cities, or country, they laid the sick in the streets, and besought him that they might touch if it were but the border of his garment: and as many as touched him were made whole.

\offertory{Quicken me, O Lord, according to thy word: that I may know thy testimonies.}

\secret
\lett{A}{ccept,} O Lord, the sacrifice which thou hast ordained to be a worthy propitiation unto thee: and grant, we beseech thee, that we being cleansed by the operation of the same, may offer unto thee the acceptable devotion of our hearts. Through.

\begin{rubric}
    \nth{2} Secret is \emph{Of Saints} (p. \pageref{SPSaints}) \& \nth{3} \emph{Of the Living and Departed} (p. \pageref{SPLivingDeparted}).
\end{rubric}

\communion{Serve the Lord in fear, and rejoice unto him with reverence: lay hold on discipline, lest ye perish from the right wav.}

\postcommunion
\lett{O}{Lord,} who hast quickened us with the gift of heavenly life: we beseech thee, that those things which in this present life are to us a mystery, may be our succour unto life eternal. Through.

\begin{rubric}
    \nth{2} Postcommunion is \emph{Of Saints} (p. \pageref{SPSaints}) \& \nth{3} \emph{Of the Living and Departed} (p. \pageref{SPLivingDeparted}).
\end{rubric}

\textsc{Priest.} Let us pray.\par
\textsc{Deacon.} Humble your heads before God.\par
\begin{rubric}
    The Priest then prays the following:
\end{rubric}
\lett{M}{ay} thy faithful people, O God, be strengthened by thy gifts: that they receiving the same may seek them the more, and seeking them may obtain them everlastingly. Through.


\subby{First Sunday of Lent}\label{LentI}
\fancyhead[RO,LE]{\textit{Lent I}}
\fancyhead[RE,LO]{}
%\begin{inhead}
%    {First Class Semidouble}
%\end{inhead}

\subbysub{I Evensong}\label{FirstLentEvensong}

\gregorioscore{resources/gabc/ProperSeason/FirstLentEvensong.gabc}

℣. God shall give his Angels charge over thee.

℟. To keep thee in all thy ways.

\properantiphon{Mag.}{Then shalt thou call, {\dag} and the Lord shall answer: thou shalt cry, and he shall say, Here am I.}

\subbysub{Mattins}

%MANUAL ADJUSTMENT:
%\invitatoryhymn\label{FirstLentInvitatory}

\begin{inhead}
	Invitatory Hymn\label{FirstLentInvitatory}
\end{inhead}

\gregorioscore{resources/gabc/ProperSeason/FirstLentInvitatory.gabc}

\officehymn\label{FirstLentMattins}

\gregorioscore{resources/gabc/ProperSeason/FirstLentMattins.gabc}

℣. God shall give his Angels charge over thee.

℟. To keep thee in all thy ways.

\properantiphon{Ben.}{Then was Jesus {\dag} led up of the Spirit into the wilderness to be tempted of the devil: and when he had fasted forty days and forty nights, he was afterward an hungred.}

\subbysub{II Evensong}

\begin{rubric}
	The Office Hymn \& Versicle are of I Evensong, with the following Antiphon.
\end{rubric}

\properantiphon{Mag.}{Behold, now {\dag} is the accepted time; behold, now is the day of salvation: let us therefore in all things approve ourselves as the ministers of God, in much patience, in watchings, in fastings, and by love unfeigned.}


\subby{Lenten Ferial Office}
\begin{rubric}
	On Lenten Ferias, the Hymns and Versicles are taken from the First Sunday of Lent. The Antiphon and Collect are of the Day.
\end{rubric}

\subby{Ember Wednesday in Lent}
\feastday{Lenten Emberday}
\fancyhead[RE,LO]{Ember Wednesday}

\properantiphon{Ben.}{This crooked {\dag} and perverse generation seeketh after a sign: and there shall no sign be given it, but the sign of the prophet Jonas.}

\properantiphon{Mag.}{For as Jonas {\dag} was three days and three nights in the whale's belly, so shall the Son of Man be in the heart of the earth.}


\subby{Ember Friday in Lent}
\feastday{Lenten Emberday}
\fancyhead[RE,LO]{Ember Friday}
%\bcpfeast{Ember Friday in Lent}{Ember Friday}{Lenten Emberday}

\properantiphon{Ben.}{An Angel of the Lord {\dag} went down from heaven, and troubled the water, and one was healed.}

\properantiphon{Mag.}{He that made me whole, {\dag} the same commanded me: Take up thy bed, and walk in peace.}

\subby{Ember Saturday in Lent}
\feastday{Lenten Emberday}
\fancyhead[RE,LO]{Ember Saturday}
%\bcpfeast{Ember Saturday in Lent}{Ember Saturday}{Lenten Emberday}

\properantiphon{Ben.}{And Jesus taketh his disciples, {\dag} and goeth up into a mountain, and was transfigured before them.}


\subby{Second Sunday of Lent}
\fancyhead[RO,LE]{\textit{Lent II}}
\fancyhead[RE,LO]{}
%\begin{inhead}
%    {First Class Semidouble}
%\end{inhead}

\begin{rubric}
	The Daily Office propers are as in the First Sunday of Lent (p. \pageref{LentI}), except for the following Antiphons.
\end{rubric}

\properantiphon{Mag.}{Tell the vision {\dag} which ye have seen to no man, until the Son of Man be risen again from the dead.}

\properantiphon{Ben.}{Jesus went thence, {\dag} and departed into the coasts of Tyre and Sidon: and behold, a woman of Canaan came out of the same coasts, and cried, saying: Have mercy on me, thou Son of David.}

\properantiphon{Mag.}{Jesus said {\dag} unto the woman of Canaan, It is not meet to take the children's bread, and to cast it to dogs. Truth, Lord; yet the dogs eat of the crumbs which fall from their master's table. Then Jesus answered, O woman, great is thy faith; be it unto thee even as thou wilt.}


\subby{Third Sunday of Lent}
\fancyhead[RO,LE]{\textit{Lent III}}
\fancyhead[RE,LO]{}
%\begin{inhead}
%    {First Class Semidouble}
%\end{inhead}

\begin{rubric}
	The Daily Office propers are as in the First Sunday of Lent (p. \pageref{LentI}), except for the following Antiphons.
\end{rubric}

\properantiphon{Mag.}{But the father {\dag} said to his servants, Bring forth the best robe and put it on him; and put a ring on his hand, and shoes on his feet.}

\properantiphon{Ben.}{When a strong man armed {\dag} keepeth his palace, all his goods are in peace.}

\properantiphon{Mag.}{A certain woman {\dag} of the company lifted up her voice and cried, Blessed is the womb that bare thee, and the paps which thou hast sucked. But Jesus answered, Yea, rather, blessed are they that hear the word of God, and keep it.}


\subby{Fourth Sunday of Lent}
\fancyhead[RO,LE]{\textit{Lent IV}}
\fancyhead[RE,LO]{}
%\begin{inhead}
%    {First Class Semidouble}
%\end{inhead}

\begin{rubric}
	The Daily Office propers are as in the First Sunday of Lent (p. \pageref{LentI}), except for the following Antiphons.
\end{rubric}

\properantiphon{Mag.}{Woman, {\dag} hath no man condemned thee? No man, Lord. Neither do I condemn thee; go and sin no more.}

\properantiphon{Ben.}{When Jesus {\dag} lifted up his eyes, and saw a great company come unto him, he saith unto Philip: Whence shall we buy bread that these may eat? And this he said to prove him; for he himself knew what he would do.}

\properantiphon{Mag.}{And Jesus {\dag} went up into a mountain, and there he sat with his disciples.}


\subby{Passion Sunday}
\fancyhead[RO,LE]{\textit{Passion Sunday}}
\fancyhead[RE,LO]{}
%\begin{inhead}
%    {First Class Semidouble}
%\end{inhead}

\subbysub{I Evensong}\label{PassionSundayEvensong}

\gregorioscore{resources/gabc/ProperSeason/PassionSundayEvensong1.gabc}

\begin{rubric}
	During the following stanza, all kneel.
\end{rubric}

\gregorioscore{resources/gabc/ProperSeason/PassionSundayEvensongPenultimateVerse.gabc}

\begin{rubric}
	\textsc{Note,} The following stanza is never changed.
\end{rubric}

\gregorioscore{resources/gabc/ProperSeason/PassionSundayEvensongUltimateVerse.gabc}

℣. Deliver me, O Lord, from the evil man.

℟. And preserve me from the wicked man.

\properantiphon{Mag.}{I am one {\dag} that bear witness of myself, and the Father that sent me beareth witness of me.}

\subbysub{Mattins}

%Tune from Lauds
\invitatoryhymn\label{PassionSundayInvitatory}

\gregorioscore{resources/gabc/ProperSeason/PassionSundayInvitatory.gabc}

\officehymn\label{PassionSundayMattins}

\gregorioscore{resources/gabc/ProperSeason/PassionSundayMattins.gabc}

℣. Deliver me from mine enemies, O God.

℟. Defend me from them that rise up against me.

\properantiphon{Ben.}{Jesus said {\dag} unto the multitude of the Jews, and to the chief priests: He that is of God heareth God's words; ye therefore hear them not, because ye are not of God.}

\subbysub{II Evensong}

\begin{rubric}
	The Office Hymn \& Versicle are of I Evensong, with the following Antiphon.
\end{rubric}

\properantiphon{Mag.}{Your father Abraham {\dag} rejoiced to see my day: and he saw it and was glad.}


\subby{Passiontide Ferial Office}
\begin{rubric}
	On Passiontide Ferias, the Hymns and Versicles are taken from Passion Sunday. The Antiphon and Collect are of the Day.
\end{rubric}


\subby{Palm Sunday}
\fancyhead[RO,LE]{\textit{Palm Sunday}}
\fancyhead[RE,LO]{}
%\begin{inhead}
%    {First Class Semidouble}
%\end{inhead}

\begin{rubric}
	The Daily Office propers are as in Passion Sunday, except for the following Antiphons.
\end{rubric}

\properantiphon{Mag.}{Righteous Father, {\dag} the world hath not known thee: but I have known thee, because thou hast sent me.}

\properantiphon{Ben.}{The multitudes {\dag} which came together for the feast day cried unto the Lord: Blessed is he that cometh in the Name of the Lord. Hosanna in the highest.}

\properantiphon{Mag.}{For it is written, {\dag} I will smite the shepherd, and the sheep shall be scattered. But after I am risen again, I will go before you into Galilee: there shall ye see me, saith the Lord.}


\subby{Easter Day}\label{EasterDay}
\fancyhead[RO,LE]{\textit{Easter Day}}
\fancyhead[RE,LO]{}
%\begin{inhead}
%    {First Class Double, First Class Octave}
%\end{inhead}

\subbysub{I Evensong}

\begin{rubric}
	If I Evensong of Easter Day be not said during the service of Easter Even, then it is said as usual with the following Antiphon, omitting the Hymn and Versicle.
\end{rubric}

\properantiphon{Mag.}{In the end of the sabbath, {\dag} as it began to dawn toward the first day of the week, came Mary Magdalene and the other Mary to see the sepulchre, alleluia.}

\subbysub{Mattins}

\invitatoryhymn\label{EasterInvitatory}

\gregorioscore{resources/gabc/ProperSeason/EasterInvitatory.gabc}

\officehymn\label{EasterMattins}

\gregorioscore{resources/gabc/ProperSeason/EasterMattins.gabc}

℣. This is the day which the Lord hath made, alleluia.

℟. We will rejoice and be glad in it, alleluia.

\properantiphon{Ben.}{And very early in the morning {\dag} the first day of the week, they came unto the sepulchre at the rising of the sun, alleluia.}

\subbysub{II Evensong}\label{EasterEvensong}

\gregorioscore{resources/gabc/ProperSeason/EasterEvensong.gabc}

℣. This is the day which the Lord hath made, alleluia.

℟. We will rejoice and be glad in it, alleluia.

\properantiphon{Mag.}{And when they looked, {\dag} they saw that the stone was rolled away: for it was very great, alleluia.}

\begin{rubric}
	During Easter Week, in the Daily Office, the Office Hymns \& Versicles are from Easter Day.
\end{rubric}


\subby{Antiphons within Easter Week}
\fancyhead[RO,LE]{Easter Week}
\fancyhead[RE,LO]{}
\begin{multicols}{2}

\subbysub{Easter Monday}
\properantiphon{Ben.}{Jesus himself {\dag} drew near to his disciples in the way, and went with them: but their eyes were holden that they should not know him: and he rebuked them, saying, O fools and slow of heart to believe all that the prophets have spoken, alleluia.}

\properantiphon{Mag.}{What manner of communications {\dag} are these that ye have one to another, as ye walk, and are sad? alleluia, alleluia.}

\subbysub{Easter Tuesday}
\properantiphon{Ben.}{Jesus stood {\dag} in the midst of his disciples and said unto them, Peace be unto you, alleluia, alleluia.}

\properantiphon{Mag.}{Behold my hands {\dag} and my feet, that it is I myself, alleluia.}

\subbysub{Easter Wednesday}
\properantiphon{Ben.}{Cast the net {\dag} on the right side of the ship, and ye shall find, alleluia.}

\properantiphon{Mag.}{Jesus saith unto his disciples, {\dag} Bring of the fish which ye have now caught. Simon Peter went up, and drew the net to land, full of great fishes, alleluia.}

\subbysub{Easter Thursday}
\properantiphon{Ben.}{Mary stood {\dag} at the sepulchre weeping, and seeth two Angels in white, sitting, and the napkin that was about the head of Jesus, alleluia.}

\properantiphon{Mag.}{They have taken {\dag} away my Lord, and I know not where they have laid him. If thou have borne him hence, tell me, alleluia: and I will take him away, alleluia.}

\subbysub{Easter Friday}
\properantiphon{Ben.}{The eleven disciples, {\dag} when they saw the Lord in Galilee, worshipped him, alleluia.}

\properantiphon{Mag.}{All power {\dag} is given unto me in heaven and in earth, alleluia.}

\subbysub{Easter Saturday}
\properantiphon{Ben.}{They ran both together, {\dag} and the other disciple did outrun Peter, and came first to the sepulchre, alleluia.}

\properantiphon{Mag.}{The same day {\dag} at evening, being the first day of the week, when the doors were shut where the disciples were assembled, came Jesus and stood in the midst, and saith unto them, Peace be unto you, alleluia.}
\end{multicols}


\subby{Low Sunday}
\feastday{Easter I}
%\begin{inhead}
%    {First Class Double, Octave Day of Easter}
%\end{inhead}

\begin{rubric}
	In the Daily Office, the Hymns are of Easter Day (p. \pageref{EasterDay}), with the Versicles and Antiphons as followeth.
\end{rubric}

℣. Abide with us, Lord, alleluia. 

℟. For it is toward evening, alleluia.

\properantiphon{Mag.}{The same day {\dag} at evening, being the first day of the week, when the doors were shut where the disciples were assembled, came Jesus and stood in the midst, and saith unto them, Peace be unto you, alleluia.}\\

%MANUAL ADJUSTMENT:
\clearpage
℣. In thy resurrection, O Christ, alleluia.

℟. Let heaven and earth rejoice, alleluia.

\properantiphon{Ben.}{The same day {\dag} at evening, being the first day of the week, when the doors were shut where the disciples were assembled, came Jesus and stood in the midst, and saith unto them, Peace be unto you, alleluia.}\\

℣. Abide with us, Lord, alleluia. 

℟. For it is toward evening, alleluia.

\properantiphon{Mag.}{After eight days, {\dag} when the doors were shut, the Lord entered, and said unto them, Peace be unto you, alleluia, alleluia.}


\subby{Eastertide Ferial Office}
\begin{rubric}
	On Ferias in Eastertide, the Hymns are as in Easter Day (p. \pageref{EasterDay}). The Versicles are as followeth. The Antiphon is proper.
\end{rubric}

\begin{inhead}
	Mattins
\end{inhead}

℣. In thy resurrection, O Christ, alleluia.

℟. Let heaven and earth rejoice, alleluia.

\begin{inhead}
	Evensong
\end{inhead}

℣. Abide with us, Lord, alleluia. 

℟. For it is toward evening, alleluia.

\subby{The First Week after the Easter Octave}
\feastday{Easter I Week}

\begin{multicols}{2}
\subbysub{Monday}
\antiphon{Ben.}{When Jesus was risen {\dag} early the first day of the week, he appeared first to Mary Magdalene, out of whom he had cast seven devils, alleluia.}

\antiphon{Mag.}{Peace be unto you, it is I, {\dag} alleluia: be not afraid, alleluia.}

\subbysub{Tuesday}
\antiphon{Ben.}{I go before you {\dag} into Galilee: there shall ye see me, as I said unto you, alleluia, alleluia.}

\antiphon{Mag.}{Reach hither thy finger, {\dag} and examine the print of the nails, alleluia: and be not faithless, but believing, alleluia.}

\subbysub{Wednesday}
\antiphon{Ben.}{I am the true vine, {\dag} alleluia: and ye are the true branches, alleluia}

\antiphon{Mag.}{Because thou hast seen me, {\dag} Thomas, thou hast believed: blessed are they that have not seen, and yet have believed, alleluia.}

\subbysub{Thursday}
\antiphon{Ben.}{My heart burns {\dag} within me; I desire to behold my Lord: I seek, and find not where they have laid him, alleluia, alleluia.}

\antiphon{Mag.}{I did put my finger {\dag} into the print of the nails, and thrust my hand into his side, and said, My Lord and my God, alleluia.}

\subbysub{Friday}
\antiphon{Ben.}{There came unto the tomb {\dag} Mary Magdalene and the other Mary, to see the sepulchre, alleluia.}

\begin{rubric}
	At Evensong, unless a Double Feast or some Octave occur on the following day, the Office is of Saint Mary, and this Office is said on the following Saturday. The same order is observed on the other Saturdays through the Saturday before Easter V.
\end{rubric}
	
\end{multicols}


\subby{Second Sunday after Easter}\label{EasterII}
\feastday{Easter II}
%\begin{inhead}
%    {Semidouble}
%\end{inhead}

\begin{rubric}
	In the Daily Office, the Hymns are of Easter Day (p. \pageref{EasterDay}), with the following.
\end{rubric}

℣. Abide with us, Lord, alleluia. 

℟. For it is toward evening, alleluia.

\properantiphon{Mag.}{I am {\dag} the Shepherd of the sheep: I am the way, the truth, and the life: I am the Good Shepherd; and I know my sheep, and am known of mine, alleluia, alleluia.}\\

℣. In thy resurrection, O Christ, alleluia.

℟. Let heaven and earth rejoice, alleluia.

\properantiphon{Ben.}{I am {\dag} the Shepherd of the sheep: I am the way, the truth, and the life: I am the Good Shepherd; and I know my sheep, and am known of mine, alleluia, alleluia.}\\

℣. Abide with us, Lord, alleluia. 

℟. For it is toward evening, alleluia.

\properantiphon{Mag.}{I am {\dag} the Good Shepherd, who feed my sheep; and I lay down my life for my sheep, alleluia.}


\subby{The Second Week after the Easter Octave}
\feastday{Easter II Week}

\begin{multicols}{2}
\subsubsec{Monday}
\antiphon{Ben.}{Go ye into the world, {\dag} alleluia, and teach all nations, alleluia.}

\antiphon{Mag.}{The Good Shepherd {\dag} layeth down his life for the sheep, alleluia.}

\subsubsec{Tuesday}
\antiphon{Ben.}{Go ye into the world {\dag} and teach all nations, baptizing them in the Name of the Father, and of the Son, and of the Holy Ghost, alleluia.}

\antiphon{Mag.}{He that is an hireling, {\dag} whose own the sheep are not, seeth the wolf coming, and leaveth the sheep, and fleeth: and the wolf catcheth them, and, scattereth the sheep, alleluia.}

\subsubsec{Wednesday}
\antiphon{Ben.}{Go unto my brethren {\dag} and say unto them, alleluia: that they go into Galilee, alleluia: there shall they see me, alleluia, alleluia, alleluia.}

\antiphon{Mag.}{As the Father knoweth me, {\dag} even so know I the Father: and I lay down my life for the sheep, alleluia.}

\subsubsec{Thursday}
\antiphon{Ben.}{Art thou only {\dag} a stranger, and hast not heard concerning Jesus, how they delivered him to be condemned to death? alleluia.}

\antiphon{Mag.}{Other sheep I have, {\dag} which are not of this fold: them also I must bring, and they shall hear my voice; and there shall be one fold and one Shepherd, alleluia.}

\subsubsec{Friday}
\antiphon{Ben.}{Ought not Christ {\dag} to have suffered these things, and to enter into his glory? alleluia.}

\end{multicols}


\subby{Patronage of St. Joseph}
\feastday{Patronage of St. Joseph}
%\begin{inhead}
%    {Wednesday after the Second Sunday after Easter}\par
%    {First Class Double, Simple Octave}
%\end{inhead}

\subbysub{I Evensong}\label{PatronageEvensong}

\gregorioscore{resources/gabc/ProperSeason/JosephEvensong.gabc}

℣. I do give praise unto thy name, alleluia.

℟. For thou art my defender and helper, alleluia.

%From the Breviarium Monasticum, using KJV and Coverdale translation:
\properantiphon{Mag.}{When Mary {\dag} the mother of Jesus was espoused to Joseph, before they came together, she was found with child of the Holy Ghost, alleluia.}

\subbysub{Mattins}

\begin{rubric}
	The Invitatory Hymn is of I Evensong.
\end{rubric}

\officehymn\label{PatronageMattins}

\gregorioscore{resources/gabc/ProperSeason/JosephMattins.gabc}

℣. Thou hast given me the defence of thy salvation, alleluia.

℟. And thy right hand also shall hold me up, alleluia.

\properantiphon{Ben.}{Joseph, thou son of David, {\dag} fear not to take unto thee Mary thy wife: for that which is conceived in her is of the Holy Ghost, alleluia.}

\subbysub{II Evensong}

\begin{rubric}
	The Office Hymn is of I Evensong, with the following Versicle \& Antiphon.
\end{rubric}

℣. I sat down under his shadow with great delight, alleluia.

℟. And his fruit was sweet to my taste, alleluia.

\properantiphon{Mag.}{Son, {\dag} why hast thou thus dealt with us? Behold, thy father and I have sought thee sorrowing, alleluia.}


\subby{Third Sunday after Easter}
\feastday{Easter III}
%\begin{inhead}
%    {Semidouble}
%\end{inhead}

\begin{rubric}
	In the Daily Office, the Hymns are of Easter Day (p. \pageref{EasterDay}), with the following.
\end{rubric}

℣. Abide with us, Lord, alleluia. 

℟. For it is toward evening, alleluia.

\properantiphon{Mag.}{A little while, {\dag} and ye shall not see me, saith the Lord: and again, a little while, and ye shall see me, because I go to the Father, alleluia, alleluia.}\\

℣. In thy resurrection, O Christ, alleluia.

℟. Let heaven and earth rejoice, alleluia.

\properantiphon{Ben.}{A little while, {\dag} and ye shall not see me, saith the Lord: and again, a little while, and ye shall see me, because I go to the Father, alleluia, alleluia.}\\

℣. Abide with us, Lord, alleluia. 

℟. For it is toward evening, alleluia.

\properantiphon{Mag.}{Verily I say unto you, {\dag} that ye shall weep and lament, but the world shall rejoice: and ye shall be sorrowful, but your sorrow shall be turned into joy, alleluia.}

\subby{The Third Week after the Easter Octave}
\feastday{Easter III Week}

\begin{multicols}{2}
\subsubsec{Monday}
\antiphon{Ben.}{And beginning {\dag} at Moses and all the prophets, he expounded unto them the scriptures concerning himself, alleluia.}

\antiphon{Mag.}{Your sorrow {\dag} shall be turned into joy, alleluia: and your joy no man taketh from you, alleluia, alleluia.}

\subsubsec{Tuesday}
\antiphon{Ben.}{And they constrained him, {\dag} saying, Abide with us, O Lord, for it is toward evening, alleluia.}

\antiphon{Mag.}{Sorrow {\dag} hath filled your heart: and your joy no man taketh from you, alleluia, alleluia.}

\subsubsec{Wednesday}
\antiphon{Ben.}{Abide with us, {\dag} for it is toward evening, and the day is far spent, alleluia.}

\antiphon{Mag.}{Your sorrow, {\dag} alleluia, shall be turned into joy, alleluia.}

\subsubsec{Thursday}
\antiphon{Ben.}{And he went in {\dag} to tarry with them: and it came to pass, as he sat at meat with them, he took bread, and blessed it, and brake, and gave to them, alleluia, alleluia.}

\antiphon{Mag.}{Verily, verily, I say unto you, {\dag} I will see you again, and your heart shall rejoice: and your joy no man taketh from you, alleluia.}

\subsubsec{Friday}
\antiphon{Ben.}{They knew the Lord Jesus, {\dag} alleluia, in breaking of bread, alleluia.}

\end{multicols}


\subby{Fourth Sunday after Easter}
\feastday{Easter IV}
%\begin{inhead}
%    {Semidouble}
%\end{inhead}

\begin{rubric}
	In the Daily Office, the Hymns are of Easter Day (p. \pageref{EasterDay}), with the following.
\end{rubric}

℣. Abide with us, Lord, alleluia. 

℟. For it is toward evening, alleluia.

\properantiphon{Mag.}{Now I go my way {\dag} to him that sent me; and none of you asketh me, Wither goest thou? alleluia, alleluia.}\\

℣. In thy resurrection, O Christ, alleluia.

℟. Let heaven and earth rejoice, alleluia.

\properantiphon{Ben.}{Now I go my way {\dag} to him that sent me; and none of you asketh me, Wither goest thou? alleluia, alleluia.}\\

℣. Abide with us, Lord, alleluia. 

℟. For it is toward evening, alleluia.

\properantiphon{Mag.}{Now I go my way {\dag} to him that sent me: but because I have said these things unto you, sorrow hath filled your heart, alleluia.}


\subby{The Fourth Week after the Easter Octave}
\feastday{Easter IV Week}

\begin{multicols}{2}
\subsubsec{Monday}
\antiphon{Ben.}{Did not our heart {\dag} burn within us concerning Jesus, while he talked with us by the way? alleluia.}

\antiphon{Mag.}{I tell you the truth; {\dag} it is expedient for you that I go away: for if I go not away, the Comforter will not come unto you, alleluia.}

\subsubsec{Tuesday}
\antiphon{Ben.}{Peace be unto you, it is I, {\dag} alleluia: be not afraid, alleluia.}

\antiphon{Mag.}{When the Comforter, {\dag} the Spirit of truth, is come, he will convince the world of sin, and of righteousness, and of judgement, alleluia.}

\subsubsec{Wednesday}
\antiphon{Ben.}{A spirit {\dag} hath not flesh and bones, as ye see me have: believe ye therefore, alleluia.}

\antiphon{Mag.}{I have yet {\dag} many things to say unto you, but ye cannot bear them now: howbeit, when he, the Spirit of truth, is come, he will guide you into all truth, alleluia.}

\subsubsec{Thursday}
\antiphon{Ben.}{The disciples {\dag} offered the Lord a piece of broiled fish, and of an honeycomb, alleluia, alleluia.}

\antiphon{Mag.}{For he shall not speak of himself; {\dag} but whatsoever he shall hear, that shall he speak: and he will shew you things to come, alleluia.}

\subsubsec{Friday}
\antiphon{Ben.}{These are the words {\dag} which I spake unto you, while I was yet with you, alleluia, alleluia.}

\end{multicols}


\subby{Rogation Sunday}
\feastday{Easter V}
%\begin{inhead}
%    {Semidouble}
%\end{inhead}

\begin{rubric}
	In the Daily Office, the Hymns are of Easter Day (p. \pageref{EasterDay}), with the following.
\end{rubric}

℣. Abide with us, Lord, alleluia. 

℟. For it is toward evening, alleluia.

\properantiphon{Mag.}{Hitherto {\dag} have ye asked nothing in my Name ask, and ye shall receive, alleluia.}\\

℣. In thy resurrection, O Christ, alleluia.

℟. Let heaven and earth rejoice, alleluia.

\properantiphon{Ben.}{Hitherto {\dag} have ye asked nothing in my Name ask, and ye shall receive, alleluia.}\\

℣. Abide with us, Lord, alleluia. 

℟. For it is toward evening, alleluia.

\properantiphon{Mag.}{Ask, {\dag} and ye shall receive, that your joy may be full: for the Father himself loveth you, because ye have loved me, and have believed, alleluia.}


\subby{Rogation Monday}
\feastday{Rogation Monday}
%\begin{inhead}
%    {Greater Feria}
%\end{inhead}

\begin{rubric}
	In the Daily Office, the Hymns are of Easter Day (p. \pageref{EasterDay}), with the following.
\end{rubric}

℣. In thy resurrection, O Christ, alleluia.

℟. Let heaven and earth rejoice, alleluia.

\properantiphon{Ben.}{Ask, {\dag} and ye shall receive; seek, and ye shall find; knock, and it shall be opened unto you, alleluia.}\\

℣. Abide with us, Lord, alleluia. 

℟. For it is toward evening, alleluia.

\properantiphon{Mag.}{For the Father {\dag} himself loveth you, because you have loved me, and have believed, alleluia.}


\subby{Rogation Tuesday}
\feastday{Rogation Tuesday}
%\begin{inhead}
%    {Feria}
%\end{inhead}

\begin{rubric}
	The Hymns are of Easter Day (p. \pageref{EasterDay}), with the following.
\end{rubric}

℣. In thy resurrection, O Christ, alleluia.

℟. Let heaven and earth rejoice, alleluia.

\properantiphon{Ben.}{It behoved Christ to suffer, {\dag} and to rise again from the dead, alleluia.}\\

℣. Abide with us, Lord, alleluia. 

℟. For it is toward evening, alleluia.

\properantiphon{Mag.}{I came forth from the Father, {\dag} and am come into the world: again I leave the world, and go to the Father, alleluia.}


\subby{Vigil of Ascension}
\feastday{Ascension Vigil}
%\begin{inhead}
%    {Vigil}
%\end{inhead}

\begin{rubric}
	The Hymns are of Easter Day (p. \pageref{EasterDay}), with the following.
\end{rubric}

℣. In thy resurrection, O Christ, alleluia.

℟. Let heaven and earth rejoice, alleluia.

\properantiphon{Ben.}{Father, the hour is come: {\dag} glorify thy Son with the glory which I had with thee before the world was, alleluia.}

\introit
\lett{W}{ith} a voice of singing declare ye this, and let it be heard, alleluia: utter it even unto the end of the earth: the Lord hath delivered his people, alleluia, alleluia. \textit{Ps.} O be joyful in God, all ye lands: sing praises unto his name, make his praise to be glorious.

\collect
\lett{O}{Lord,} from whom all good things do come, grant to us thy humble servants: that by thy holy inspiration we may think those things that be good; and by thy merciful guiding may perform the same. Through.

\lett{G}{rant,} we beseech thee, almighty God: that we, who in our affliction do put our trust in thy mercy; may ever be defended by thy protection against all adversities. (Through.)

\lett{G}{rant,} we beseech thee, O Lord God, that we thy servants may enjoy perpetual health of mind and of body: and, at the glorious intercession of blessed Mary ever Virgin, be delivered from present sadness, and rejoice in everlasting gladness. Through.

\readingcitation{Epistle}{Ephesians 4:7}
\lett{B}{rethren:} Unto every one of us is given grace according to the measure of the gift of Christ. Wherefore he saith, When he ascended up on high, he led captivity captive, and gave gifts unto men. (Now that he ascended, what is it but that he also descended first into the lower parts of the earth? He that descended is the same also that ascended up far above all heavens, that he might fill all things.) And he gave some, apostles; and some, prophets; and some, evangelists; and some, pastors and teachers; For the perfecting of the saints, for the work of the ministry, for the edifying of the body of Christ: Till we all come in the unity of the faith, and of the knowledge of the Son of God, unto a perfect man, unto the measure of the stature of the fulness of Christ.

\alleluia{Alleluia, alleluia. ℣. Christ is risen, and hath shewed light unto us, whom he hath redeemed with his blood. Alleluia. ℣. I came forth from the Father, and am come into the world: again I leave the world, and go to the Father. Alleluia.}

\readingcitation{Gospel}{John 17:1}
\lett{A}{t that time:} Jesus lifted up his eyes to heaven, and said: Father, the hour is come; glorify thy Son, that thy Son also may glorify thee: As thou hast given him power over all flesh, that he should give eternal life to as many as thou hast given him. And this is life eternal, that they might know thee the only true God, and Jesus Christ, whom thou hast sent. I have glorified thee on the earth: I have finished the work which thou gavest me to do. And now, O Father, glorify thou me with thine own self with the glory which I had with thee before the world was. I have manifested thy name unto the men which thou gavest me out of the world: thine they were, and thou gavest them me; and they have kept thy word. Now they have known that all things whatsoever thou hast given me are of thee. For I have given unto them the words which thou gavest me; and they have received them, and have known surely that I came out from thee, and they have believed that thou didst send me. I pray for them: I pray not for the world, but for them which thou hast given me; for they are thine. And all mine are thine, and thine are mine; and I am glorified in them. And now I am no more in the world, but these are in the world, and I come to thee.

\offertory{O praise the Lord our God, ye people, and make the voice of his praise to be heard: who holdeth our soul in life, and suffereth not out feet to slip: praised be the Lord, who hath not cast out my prayer, nor turned his mercy from me, alleluia.}

\secret
\lett{R}{eceive,} O Lord, the prayers of thy faithful people, together with the offering of these sacrifices: that through these observances of our bounden devotion, we may attain unto heavenly glory. Through.

\lett{W}{e} beseech thee, O Lord, that these our oblations may both loose the bonds of our iniquity, and obtain for us the gifts of thy loving-kindness. (Through.)

\lett{T}{hrough} thy mercy, O Lord, and the intercession of blessed Mary ever Virgin, may this oblation avail for our prosperity and peace both now and ever. Through.

\communion{O sing unto the Lord, alleluia: sing unto the Lord, and praise his name: be telling of his salvation from day to day, alleluia, alleluia.}

\postcommunion
\lett{G}{rant} unto us, O Lord, that we who have been fulfilled with the strength of thy heavenly table: may both desire those things which be right and obtain those things which we desire. Through.

\lett{W}{e} beseech thee, O Lord, to further with thy gracious favour these our supplications: that we, receiving thy gifts in our tribulation, may increase in thy love by the consolation of the same. (Through.)

\lett{G}{rant,} we beseech thee, O Lord: that we who have received these aids to our salvation may at all times and in all places be protected by the advocacy of blessed Mary ever Virgin; in whose honour we have made these offerings to thy majesty. Through.


\subby{Ascension Thursday}\label{Ascension}
\feastday{Ascension Thursday}
%\begin{inhead}
%    {First Class Double, Third Class Octave}
%\end{inhead}

\subbysub{I Evensong}

\begin{rubric}
	The Office Hymn is of Mattins, with the following Versicle \& Antiphon.
\end{rubric}

℣. God is gone up with a merry noise, alleluia.

℟. And the Lord with the sound of the trump, alleluia.

\properantiphon{Mag.}{Father, {\dag} I have manifested thy Name unto the men which thou gavest me: and now I pray for them, not for the world, because I come to thee, alleluia.}

\subbysub{Mattins}

\invitatoryhymn\label{AscensionInvitatory}

\gregorioscore{resources/gabc/ProperSeason/AscensionInvitatory.gabc}

\officehymn\label{AscensionMattins}

\gregorioscore{resources/gabc/ProperSeason/AscensionMattins.gabc}

℣. The Lord hath prepared, alleluia.

℟. His seat in heaven, alleluia.

\properantiphon{Ben.}{I ascend {\dag} unto my Father, and your Father: and to my God, and your God, alleluia.}

\subbysub{II Evensong}

\begin{rubric}
	The Office Hymn is of Mattins, with the following.
\end{rubric}

℣. God is gone up with a merry noise, alleluia.

℟. And the Lord with the sound of the trump, alleluia.

\properantiphon{Mag.}{O King of glory, {\dag} thou Lord of Sabaoth, who triumphing to-day hast ascended above all heavens, leave us not comfortless; but send on us the promise of the Father, even the Spirit of truth, alleluia.}


\begin{rubric}
	During the Octave of the Ascension, the Office is said daily or Commemoration made. All is said as on the Feast.
\end{rubric}


\subsec{Sunday within the Ascension Octave}
\feastday{Ascension Sunday}
%\begin{inhead}
%    {Semidouble}
%\end{inhead}

\begin{rubric}
	The Hymns are of Ascension Thursday (p. \pageref{Ascension}), with the following Versicles \& Antiphons.
\end{rubric}

℣. The Lord hath prepared, alleluia.

℟. His seat in heaven, alleluia.

\properantiphon{Mag.}{When the Comforter is come, {\dag} whom I will send unto you from the Father, even the Spirit of truth, which proceedeth from the Father, he shall testify of me, alleluia.}\\

℣. When Christ ascended up on high, alleluia.

℟. He led captivity captive, alleluia.

\properantiphon{Ben.}{When the Comforter is come, {\dag} whom I will send unto you from the Father, even the Spirit of truth, which proceedeth from the Father, he shall testify of me, alleluia.}\\

℣. The Lord hath prepared, alleluia.

℟. His seat in heaven, alleluia.

\properantiphon{Mag.}{These things have I told you, {\dag} that when the time shall come, ye may remember that I told you of them, alleluia.}


\subby{Vigil of Whitsunday}
\feastday{Whitsunday Vigil}
%\begin{inhead}
%    {First Class Vigil}
%\end{inhead}

\begin{rubric}
	From this day through Trinity Sunday, if a I Double or II Double occur, it is transferred to a day after Trinity Sunday. Other Feasts and Memorials are commemorated, except during the Triduum of Whitsunday. No notice is taken of other Octaves during this time.
\end{rubric}

\begin{rubric}
	At Mattins and the Hours, all is said as on the preceding Sunday, without Commemoration of the Ascension.
\end{rubric}

\begin{rubric}
	Mattins (or None) having been said in Choir, the Priest and Ministers, clad in vestments of violet colour, go up to the Altar and make a reverence, and the Priest kisses it in the middle. Then the Prophecies are read without title, the candles of the Altar remaining unlighted until the beginning of the Mass, as on Holy Saturday. The Priest reads them in a low voice at the Epistle corner of the Altar. At the end of the Prophecies the Collects are said without \emph{Let us bow the knee.}
\end{rubric}

\readingcitation{Prophecy I}{Genesis 22:1}
\lett{I}{n those days:} God did tempt Abraham, and said unto him, Abraham: and he said, Behold, here I am. And he said, Take now thy son, thine only son Isaac, whom thou lovest, and get thee into the land of Moriah; and offer him there for a burnt offering upon one of the mountains which I will tell thee of. And Abraham rose up early in the morning, and saddled his ass, and took two of his young men with him, and Isaac his son, and clave the wood for the burnt offering, and rose up, and went unto the place of which God had told him. Then on the third day Abraham lifted up his eyes, and saw the place afar off. And Abraham said unto his young men, Abide ye here with the ass; and I and the lad will go yonder and worship, and come again to you. And Abraham took the wood of the burnt offering, and laid it upon Isaac his son; and he took the fire in his hand, and a knife; and they went both of them together. And Isaac spake unto Abraham his father, and said, My father: and he said, Here am I, my son. And he said, Behold the fire and the wood: but where is the lamb for a burnt offering? And Abraham said, My son, God will provide himself a lamb for a burnt offering: so they went both of them together. And they came to the place which God had told him of; and Abraham built an altar there, and laid the wood in order, and bound Isaac his son, and laid him on the altar upon the wood. And Abraham stretched forth his hand, and took the knife to slay his son. And the angel of the \divineName{Lord} called unto him out of heaven, and said, Abraham, Abraham: and he said, Here am I. And he said, Lay not thine hand upon the lad, neither do thou any thing unto him: for now I know that thou fearest God, seeing thou hast not withheld thy son, thine only son from me. And Abraham lifted up his eyes, and looked, and behold behind him a ram caught in a thicket by his horns: and Abraham went and took the ram, and offered him up for a burnt offering in the stead of his son. And Abraham called the name of that place Jehovahjireh: as it is said to this day, In the mount of the \divineName{Lord} it shall be seen. And the angel of the \divineName{Lord} called unto Abraham out of heaven the second time, And said, By myself have I sworn, saith the \divineName{Lord}, for because thou hast done this thing, and hast not withheld thy son, thine only son: That in blessing I will bless thee, and in multiplying I will multiply thy seed as the stars of the heaven, and as the sand which is upon the sea shore; and thy seed shall possess the gate of his enemies; And in thy seed shall all the nations of the earth be blessed; because thou hast obeyed my voice. So Abraham returned unto his young men, and they rose up and went together to Beersheba; and Abraham dwelt at Beersheba.

\begin{rubric}
	After this and the other Prophecies the Response \emph{Thanks be to God} is not made: then the Priest says:
\end{rubric}
\letuspray
\lett{O}{God,} who in the deed of Abraham thy servant hast given a pattern of obedience to mankind: grant us so to conquer the perversity of our desires, that we may in all things fulfil the righteousness of thy commandments. Through.

\readingcitation{Prophecy II}{Exodus 14:24}
\lett{I}{n those days:} It came to pass, that in the morning watch the \divineName{Lord} looked unto the host of the Egyptians through the pillar of fire and of the cloud, and troubled the host of the Egyptians, And took off their chariot wheels, that they drave them heavily: so that the Egyptians said, Let us flee from the face of Israel; for the \divineName{Lord} fighteth for them against the Egyptians. And the \divineName{Lord} said unto Moses, Stretch out thine hand over the sea, that the waters may come again upon the Egyptians, upon their chariots, and upon their horsemen. And Moses stretched forth his hand over the sea, and the sea returned to his strength when the morning appeared; and the Egyptians fled against it; and the \divineName{Lord} overthrew the Egyptians in the midst of the sea. And the waters returned, and covered the chariots, and the horsemen, and all the host of Pharaoh that came into the sea after them; there remained not so much as one of them. But the children of Israel walked upon dry land in the midst of the sea; and the waters were a wall unto them on their right hand, and on their left. Thus the \divineName{Lord} saved Israel that day out of the hand of the Egyptians; and Israel saw the Egyptians dead upon the sea shore. And Israel saw that great work which the \divineName{Lord} did upon the Egyptians: and the people feared the \divineName{Lord}, and believed the \divineName{Lord}, and his servant Moses. Then sang Moses and the children of Israel this song unto the \divineName{Lord}, and spake, saying:

\tract{I will sing unto the \divineName{Lord}, for he hath triumphed gloriously: the horse and his rider hath he thrown into the sea. The \divineName{Lord} is my strength and song, and he is become my salvation. ℣. He is my God, and I will prepare him an habitation; my father's God, and I will exalt him. ℣. The \divineName{Lord} is a man of war: the \divineName{Lord} is his name.}

\letuspray
\lett{O}{God,} who by the light of thy new Covenant hast made manifest thy wonders wrought in former times, shewing in the Red Sea a pattern of the sacred font, and in the deliverance of thy people from bondage in Egypt foreshadowing the sacraments of thy Christian people: grant that all nations, being admitted by the merit of their faith to the privilege of Israel, may be regenerated by the partaking of thy Holy Spirit. Through . . . in the unity of the same Holy Ghost.

\readingcitation{Prophecy III}{Deuteronomy 31:22}
\lett{I}{n those days:} Moses wrote this song, and taught it the children of Israel. And he gave Joshua the son of Nun a charge, and said, Be strong and of a good courage: for thou shalt bring the children of Israel into the land which I sware unto them: and I will be with thee. And it came to pass, when Moses had made an end of writing the words of this law in a book, until they were finished, That Moses commanded the Levites, which bare the ark of the covenant of the \divineName{Lord}, saying, Take this book of the law, and put it in the side of the ark of the covenant of the \divineName{Lord} your God, that it may be there for a witness against thee. For I know thy rebellion, and thy stiff neck: behold, while I am yet alive with you this day, ye have been rebellious against the \divineName{Lord}; and how much more after my death? Gather unto me all the elders of your tribes, and your officers, that I may speak these words in their ears, and call heaven and earth to record against them. For I know that after my death ye will utterly corrupt yourselves, and turn aside from the way which I have commanded you; and evil will befall you in the latter days; because ye will do evil in the sight of the \divineName{Lord}, to provoke him to anger through the work of your hands. And Moses spake in the ears of all the congregation of Israel the words of this song, until they were ended.

\tract{Give ear, O ye heavens, and I will speak; and hear, O earth, the words of my mouth. ℣. My doctrine shall drop as the rain, my speech shall distil as the dew. ℣. As the small rain upon the tender herb, and as the showers upon the grass: Because I will publish the name of the \divineName{Lord}. ℣. Ascribe ye greatness unto our God. He is the Rock, his work is perfect: for all his ways are judgment. ℣. A God of truth and without iniquity, just and right is he.}

\letuspray
\lett{O}{God,} the glory of the faithful and the life of the just, who through Moses thy servant hast instructed us also in the chanting of thy sacred song: accomplish in all nations the work of thy mercy, granting them felicity, and delivering them from terror; that those things which were uttered for punishment may be turned into an everlasting remedy. Through.

\readingcitation{Prophecy IV}{Isaiah 4:1}
\lett{I}{n that day} seven women shall take hold of one man, saying, We will eat our own bread, and wear our own apparel: only let us be called by thy name, to take away our reproach. In that day shall the branch of the \divineName{Lord} be beautiful and glorious, and the fruit of the earth shall be excellent and comely for them that are escaped of Israel. And it shall come to pass, that he that is left in Zion, and he that remaineth in Jerusalem, shall be called holy, even every one that is written among the living in Jerusalem: When the Lord shall have washed away the filth of the daughters of Zion, and shall have purged the blood of Jerusalem from the midst thereof by the spirit of judgment, and by the spirit of burning. And the \divineName{Lord} will create upon every dwelling place of mount Zion, and upon her assemblies, a cloud and smoke by day, and the shining of a flaming fire by night: for upon all the glory shall be a defence. And there shall be a tabernacle for a shadow in the daytime from the heat, and for a place of refuge, and for a covert from storm and from rain.

\tract{My wellbeloved hath a vineyard in a very fruitful hill. ℣. And he fenced it, and gathered out the stones thereof, and planted it with the choicest vine, and built a tower in the midst of it. ℣. And also made a winepress therein: for the vineyard of the \divineName{Lord} of hosts is the house of Israel.}

\letuspray
\lett{A}{lmighty} and everlasting God, who through thy only Son hast revealed thyself to be the husbandman of thy Church, who dost mercifully purge every branch that bringeth forth fruit in the true vine, even the same thy Christ, to the intent that it may bring forth more fruit: let not the thorns of sin prevail against thy faithful people whom by the Font of baptism thou hast brought like a vine out of Egypt; that being fortified by the sanctifying power of thy Spirit, they may be enriched with everlasting fruit. Through the same . . . in the unity of the same.

\readingcitation{Prophecy V}{Baruch 3:9}
\lett{H}{ear,} O Israel, the commandments of life: give ear to understand wisdom. How happeneth it, O Israel, that thou art in thine enemies' land, that thou art waxen old in a strange country, that thou art defiled with the dead, that thou art counted with them that go down into the grave? Thou hast forsaken the fountain of wisdom. For if thou hadst walked in the way of God, thou shouldest have dwelled in peace for ever. Learn where is wisdom, where is strength, where is understanding; that thou mayest know also where is length of days, and life, where is the light of the eyes, and peace. Who hath found out her place? and who hath come into her treasuries? Where are the princes of the heathen, and such as ruled the beasts that are upon the earth; they that had their pastime with the fowls of the air, and they that hoarded up silver and gold, wherein men trust; and of whose getting there is no end? For they that wrought in silver, and were so careful, and whose works are past finding out, they are vanished and gone down to the grave, and others are come up in their steads. Younger men have seen the light, and dwelt upon the earth: but the way of knowledge have they not known, neither understood they the paths thereof: neither have their children laid hold of it: they are far off from their way. It hath not been heard of in Canaan, neither hath it been seen in Teman. The sons also of Agar that seek understanding, which are in the land, the merchants of Merran and Teman, and the authors of fables, and the searchers out of understanding; none of these have known the way of wisdom, or remembered her paths. O Israel, how great is the house of God! and how large is the place of his possession! great, and hath none end; high, and unmeasurable. There were the giants born that were famous of old, great of stature, and expert in war. These did not God choose, neither gave he the way of knowledge unto them: so they perished, because they had no wisdom, they perished through their own foolishness. Who hath gone up into heaven, and taken her, and brought her down from the clouds? Who hath gone over the sea, and found her, and will bring her for choice gold? There is none that knoweth her way, nor any that comprehendeth her path. But he that knoweth all things knoweth her, he found her out with his understanding: he that prepared the earth for evermore hath filled it with fourfooted beasts: he that sendeth forth the light, and it goeth; he called it, and it obeyed him with fear: and the stars shined in their watches, and were glad: when he called them, they said, Here we be; they shined with gladness unto him that made them. This is our God, and there shall none other be accounted of in comparison of him. He hath found out all the way of knowledge, and hath given it unto Jacob his servant, and to Israel that is beloved of him. Afterward did she appear upon earth, and was conversant with men.

\letuspray
\lett{O}{God,} who by the mouths of the Prophets hast commanded us to leave things temporal, and to strive after things eternal: grant unto thy servants; that we, knowing the things which thou commandest, may by thy heavenly inspiration be enabled to perform the same. Through.

\readingcitation{Prophecy VI}{Ezekiel 37:}
\lett{I}{n those days:} The hand of the \divineName{Lord} was upon me, and carried me out in the spirit of the \divineName{Lord}, and set me down in the midst of the valley which was full of bones, And caused me to pass by them round about: and, behold, there were very many in the open valley; and, lo, they were very dry. And he said unto me, Son of man, can these bones live? And I answered, O Lord \divineName{God}, thou knowest. Again he said unto me, Prophesy upon these bones, and say unto them, O ye dry bones, hear the word of the \divineName{Lord}. Thus saith the Lord \divineName{God} unto these bones; Behold, I will cause breath to enter into you, and ye shall live: And I will lay sinews upon you, and will bring up flesh upon you, and cover you with skin, and put breath in you, and ye shall live; and ye shall know that I am the \divineName{Lord}. So I prophesied as I was commanded: and as I prophesied, there was a noise, and behold a shaking, and the bones came together, bone to his bone. And when I beheld, lo, the sinews and the flesh came up upon them, and the skin covered them above: but there was no breath in them. Then said he unto me, Prophesy unto the wind, prophesy, son of man, and say to the wind, Thus saith the Lord \divineName{God}; Come from the four winds, O breath, and breathe upon these slain, that they may live. So I prophesied as he commanded me, and the breath came into them, and they lived, and stood up upon their feet, an exceeding great army. Then he said unto me, Son of man, these bones are the whole house of Israel: behold, they say, Our bones are dried, and our hope is lost: we are cut off for our parts. Therefore prophesy and say unto them, Thus saith the Lord \divineName{God}; Behold, O my people, I will open your graves, and cause you to come up out of your graves, and bring you into the land of Israel. And ye shall know that I am the \divineName{Lord}, when I have opened your graves, O my people, and brought you up out of your graves, And shall put my spirit in you, and ye shall live, and I shall place you in your own land.

\letuspray
\lett{O}{Lord} God of hosts, who restorest those things that are broken down, and preservest those things that thou restorest: increase the peoples that shall be regenerated in the sanctification of thy name; that all who are washed in holy baptism may ever be guided by thy inspiration. Through.

    \par\noindent
	\centerline{\rule{0.25\textwidth}{0.4pt}}
	\par\noindent

\begin{rubric}
	These ended, the Celebrant receives a violet Cope, and, while he descends to the Font, the following is sung:
\end{rubric}

\tract{Like as the hart desireth the water-brooks: so longeth my soul after thee, O God. ℣. My soul is athirst for God, yea, even for the living God: when shall I come to appear before the presence of God? ℣. My tears have been my meat day and night, while they daily say unto me: Where is now thy God?}

\begin{rubric}
	Then the Priest, before he enters for the blessing of the Font, says near the Font:
\end{rubric}

\elcol{℣. The Lord be with you.}{℣. Dóminus vobíscum.}
\elcol{℟. And with thy spirit.}{℟. Et cum spíritu tuo.}
\elcol{℣. Let us pray.}{℣. Orémus.}

\needspace{4\baselineskip}
\lett{G}{rant,} we beseech thee, almighty God: that we, who observe the solemnity of the gift of the Holy Ghost, being inflamed with heavenly desires, may thirst after the fountain of life. Through ... in the unity of the same.

\begin{rubric}
	Then he proceeds to the blessing of the Font, as on Holy Saturday.
\end{rubric}

    \par\noindent
	\centerline{\rule{0.25\textwidth}{0.4pt}}
	\par\noindent

\begin{rubric}
	But when there are no Fonts, the sixth Prophecy with its Collect being finished, the Celebrant lays aside the Chasuble, and prostrates himself before the Altar with the Ministers; and all kneeling, the Litany is sung by two Cantors in the middle of the choir, both choirs responding together. When they come to \emph{℣. We sinners beseech thee,} the Priest and Ministers rise and go to the Sacristy, where they put on red vestments, while the lights around the Altar are lighted.
\end{rubric}

\begin{rubric}
	At the end of the Litany \emph{Kyrie, eléison} is sung solemnly for the Mass, and repeated as usual. Meanwhile the Priest with the Ministers proceeds to the Altar, and makes the Confession, Prayer of Humble Access, and Ten Commandments (and/or Summary of the Law); then ascending, he kisses it, and censes it as usual. \emph{Kyrie, eléison} being ended, he begins solemnly \emph{Glory be to God on high}, and the bells are rung.
\end{rubric}

\begin{rubric}
	Afterwards the Priest says:
\end{rubric}

\elcol{℣. The Lord be with you.}{℣. Dóminus vobíscum.}
\elcol{℟. And with thy spirit.}{℟. Et cum spíritu tuo.}
\elcol{℣. Let us pray.}{℣. Orémus.}

\collect
\lett{G}{rant,} we beseech thee, almighty God: that the splendour of thy brightness may shine forth upon us; and that the light of thy light may, by the illumination of the Holy Spirit, strengthen the hearts of them who through thy grace are born again. Through ... in the unity of the same.

\begin{rubric}
	This Collect only is said, even if a Commemoration has been made in the Office.
\end{rubric}

%MANUAL ADJUSTMENT:
\vspace{-1ex}

\readingcitation{Epistle}{Acts 19:1}
\lett{I}{n those days:} It came to pass, that, while Apollos was at Corinth, Paul having passed through the upper coasts came to Ephesus: and finding certain disciples, He said unto them, Have ye received the Holy Ghost since ye believed? And they said unto him, We have not so much as heard whether there be any Holy Ghost. And he said unto them, Unto what then were ye baptized? And they said, Unto John's baptism. Then said Paul, John verily baptized with the baptism of repentance, saying unto the people, that they should believe on him which should come after him, that is, on Christ Jesus. When they heard this, they were baptized in the name of the Lord Jesus. And when Paul had laid his hands upon them, the Holy Ghost came on them; and they spake with tongues, and prophesied. And all the men were about twelve. And he went into the synagogue, and spake boldly for the space of three months, disputing and persuading the things concerning the kingdom of God.

\alleluia{Alleluia. ℣. O give thanks unto the Lord, for he is gracious: and his mercy endureth for ever.}

\begin{rubric}
	\emph{Alleluia} is not repeated, but there immediately follows:
\end{rubric}

\tract{O praise the Lord, all ye heathen: praise him, all ye nations. ℣. For his merciful kindness is ever more and more towards us: and the truth of the Lord endureth for ever.}

\begin{rubric}
	At the Gospel, lights are not carried, but incense only.
\end{rubric}

\readingcitation{Gospel}{John 14:15}
\lett{A}{t that time:} Jesus said unto his disciples: If ye love me, keep my commandments. And I will pray the Father, and he shall give you another Comforter, that he may abide with you for ever; Even the Spirit of truth; whom the world cannot receive, because it seeth him not, neither knoweth him: but ye know him; for he dwelleth with you, and shall be in you. I will not leave you comfortless: I will come to you. Yet a little while, and the world seeth me no more; but ye see me: because I live, ye shall live also. At that day ye shall know that I am in my Father, and ye in me, and I in you. He that hath my commandments, and keepeth them, he it is that loveth me: and he that loveth me shall be loved of my Father, and I will love him, and will manifest myself to him.

\offertory{O send forth thy Spirit, and they shall be made, and thou shalt renew the face of the earth: the glorious majesty of the Lord shall endure for ever, alleluia.}

\secret
\lett{S}{anctify,} we beseech thee, O Lord, the gifts which we offer: and cleanse our hearts by the enlightening of the Holy Spirit. Through ... in the unity of the same Holy Ghost.

\communion{In the last day of the feast, Jesus said: He that believeth on me, out of his belly shall flow rivers of living water: but this spake he of the Spirit, which they that believe on him should receive, alleluia, alleluia.}

\postcommunion
\lett{P}{our} thy Holy Spirit upon us, O Lord, and cleanse our hearts: that they may be made fruitful by the inward sprinkling of his dew. Through... in the unity of the same Holy Spirit.


\subby{Whitsunday}\label{Whitsunday}
\feastday{Whitsunday}
%\begin{inhead}
%    {First Class Double, First Class Octave}
%\end{inhead}

\subbysub{I Evensong}\label{WhitsundayEvensong}

\gregorioscore{resources/gabc/ProperSeason/WhitsundayEvensong.gabc}

℣. They were all filled with the Holy Ghost, alleluia.

℟. And began to speak, alleluia.

\properantiphon{Mag.}{I will not leave you {\dag} comfortless, alleluia: I go; but I will come to you, alleluia: and your heart shall rejoice, alleluia.}

\subbysub{Mattins}

\invitatoryhymn\label{WhitsundayInvitatory}

\gregorioscore{resources/gabc/ProperSeason/WhitsundayInvitatory.gabc}

\officehymn\label{WhitsundayMattins}

\gregorioscore{resources/gabc/ProperSeason/WhitsundayMattins.gabc}

℣. They were all filled with the Holy Ghost, alleluia.

℟. And began to speak, alleluia.

\properantiphon{Ben.}{Receive ye {\dag} the Holy Ghost: whosesoever sins ye remit, they are remitted unto them, alleluia.}

\subbysub{II Evensong}

\begin{rubric}
	The Office Hymn is of I Evensong, with the following Versicle \& Antiphon.
\end{rubric}

℣. The Apostles did speak with other tongues, alleluia.

℟. The wonderful works of God, alleluia.

\properantiphon{Mag.}{To-day {\dag} are fulfilled the days of Pentecost, alleluia: to-day the Holy Spirit appeared in fire to the disciples, and bestowed upon them his manifold graces: sending them into all the world, to preach the gospel, and to testify: He that believeth and is baptised shall be saved, alleluia.}


\subby{Whit-Monday}
\feastday{Whit-Monday}
%\begin{inhead}
%    {First Class Double}
%\end{inhead}

\begin{rubric}
	In the Daily Office, the Hymns \& Versicles are of Whitsunday (p. \pageref{Whitsunday}), with the following Antiphons.
\end{rubric}

\properantiphon{Ben.}{God so loved the world, {\dag} that he gave his only-begotten Son, that whosoever believeth in him should not perish, but have everlasting life, alleluia.}

\properantiphon{Mag.}{If a man love me, {\dag} he will keep my saying, and my Father will love him: and we will come unto him, and make our abode with him, alleluia.}


\subby{Whit-Tuesday}
\feastday{Whit-Tuesday}
%\begin{inhead}
%    {First Class Double}
%\end{inhead}

\begin{rubric}
	In the Daily Office, the Hymns \& Versicles are of Whitsunday (p. \pageref{Whitsunday}), with the following Antiphons.
\end{rubric}

\properantiphon{Ben.}{I am the door, {\dag} saith the Lord: by me if any man enter in, he shall be saved, and shall find pasture, alleluia.}

\properantiphon{Mag.}{Peace {\dag} I leave with you, my peace I give unto you: not as the world giveth, give I unto you, alleluia.}


\subby{Ember Wednesday in Whitsuntide}
\feastday{Whitsun Emberday}
\fancyhead[RE,LO]{Ember Wednesday}
%\begin{inhead}
%    {Semidouble}
%\end{inhead}

\begin{rubric}
	In the Daily Office, the Hymns \& Versicles are of Whitsunday (p. \pageref{Whitsunday}), with the following Antiphons.
\end{rubric}

\properantiphon{Ben.}{I am the living bread, {\dag} saith the Lord, which came down from heaven, alleluia, alleluia.}

\properantiphon{Mag.}{I am the living bread, {\dag} which came down from heaven: if any man eat of this bread, he shall live for ever: and the bread that I will give is my flesh, which I will give for the life of the world, alleluia.}


\subby{Whit-Thursday}
\feastday{Whit-Thursday}
\fancyhead[RE,LO]{}
%\begin{inhead}
%    {Semidouble}
%\end{inhead}


\begin{rubric}
	In the Daily Office, the Hymns \& Versicles are of Whitsunday (p. \pageref{Whitsunday}), with the following Antiphons.
\end{rubric}

\properantiphon{Ben.}{Jesus called unto him {\dag} his twelve disciples, and gave them power and authority over all devils, and to cure diseases: and he sent them to preach the kingdom of God, and to heal the sick, alleluia.}

\properantiphon{Mag.}{The Spirit {\dag} which proceedeth from the Father, alleluia: he shall glorify me, alleluia, alleluia.}


\subby{Ember Friday in Whitsuntide}
\feastday{Whitsun Emberday}
\fancyhead[RE,LO]{Ember Friday}
%\begin{inhead}
%    {Semidouble}
%\end{inhead}

\begin{rubric}
	In the Daily Office, the Hymns \& Versicles are of Whitsunday (p. \pageref{Whitsunday}), with the following Antiphons.
\end{rubric}

\properantiphon{Ben.}{Jesus said: {\dag} But that ye may know that the Son of man hath power on earth to forgive sins, (he saith unto the sick of the palsy) I say unto thee, Arise, take up thy bed, and go thy way into thine house, alleuia.}

\properantiphon{Mag.}{But the Comforter, {\dag} which is the Holy Ghost, whom the Father will send in my Name, he will teach you all things, and bring all things to your remembrance, whatsoever I have said unto you, alleluia.}


\subby{Ember Saturday in Whitsuntide}
\feastday{Whitsun Emberday}
\fancyhead[RE,LO]{Ember Saturday}
%\begin{inhead}
%    {Semidouble}
%\end{inhead}

\begin{rubric}
	In the Daily Office, the Hymns \& Versicles are of Whitsunday (p. \pageref{Whitsunday}), with the following Antiphons.
\end{rubric}

\properantiphon{Ben.}{The love of God is shed {\dag} abroad in our hearts by his Spirit which dwelleth in us, alleluia.}

%MANUAL ADJUSTMENT:
\clearpage
\subby{Trinity Sunday}
\feastday{Trinity Sunday}
\fancyhead[RE,LO]{}
%\begin{inhead}
%    {First Class Double}
%\end{inhead}

\subbysub{I Evensong}

\begin{rubric}
	The Office Hymn is of Saturday in Winter (p. \pageref{SaturdayEvensongWinter}), with the following.
\end{rubric}

℣. Blessed art thou, O Lord, in the firmament of heaven.

℟. And to be praised, and glorified, and magnified for ever.

\properantiphon{Mag.}{Thanks, O God, {\dag} be unto thee, thanks be unto thee, one and very Trinity, one and supreme Deity, holy and onely Unity.}

\subbysub{Mattins}

\invitatoryhymn\label{TrinityInvitatory}

\gregorioscore{resources/gabc/ProperSeason/TrinityInvitatory.gabc}

\officehymn\label{TrinityMattins}

\gregorioscore{resources/gabc/ProperSeason/TrinityMattins.gabc}

℣. Let us bless the Father, and the Son, and the Holy Ghost.

℟. Praise him, and magnify him for ever.

\properantiphon{Ben.}{Blessed be {\dag} the holy Creator and Governor of all things, the holy and undivided Trinity, both now and ever, and to endless ages of ages.}

\subbysub{II Evensong}

\begin{rubric}
	The Office Hymn is of Saturday in Winter (p. \pageref{SaturdayEvensongWinter}), with the following.
\end{rubric}

℣. Blessed art thou, O Lord, in the firmament of heaven.

℟. And to be praised, and glorified, and magnified for ever.

\properantiphon{Mag.}{Thee, O God, {\dag} the Father unbegotten, thee, O only-begotten Son; thee, O Holy Spirit, the Paraclete; holy and undivided Trinity: with our whole heart and mouth we confess thee, we praise thee and bless thee: to thee be glory for ever and ever.}


\subby{The Most Holy Body of Christ}\label{CorpusChristi}
\feastday{Corpus Christi}
%\begin{inhead}
%	{Thursday after Trinity Sunday}\par
%    {First Class Double, Second Class Octave}
%\end{inhead}

\subbysub{I Evensong}\label{CorpusChristiEvensong}

\gregorioscore{resources/gabc/ProperSeason/CorpusChristiEvensong1.gabc}

\begin{rubric}
	If the Office is recited in the presence of the exposed Sacrament, the following stanza is said kneeling:
\end{rubric}

\gregorioscore{resources/gabc/ProperSeason/CorpusChristiEvensong2.gabc}

℣. Thou gavest them bread from heaven, alleluia.

℟. Containing within itself all sweetness, alleluia.

\properantiphon{Mag.}{O how sweet {\dag} is thy Spirit, O Lord, who, that thou mightest shew thy kindness unto thy children, givest them that sweetest bread from heaven, fillest the hungry with good things, and sendest the rich and scornful empty away.}

\subbysub{Mattins}

\invitatoryhymn\label{CorpusChristiInvitatory}

\gregorioscore{resources/gabc/ProperSeason/CorpusChristiInvitatory.gabc}

\officehymn\label{CorpusChristiMattins}

\gregorioscore{resources/gabc/ProperSeason/CorpusChristiMattins.gabc}

℣. He maketh peace in thy borders, alleluia.

℟. And filleth thee with the flour of wheat, alleluia.

\properantiphon{Ben.}{I am {\dag} the living bread, which came down from heaven: if any man eat of this bread, he shall live for ever, alleluia.}

\subbysub{II Evensong}

\begin{rubric}
	The Office Hymn \& Versicle are of I Evensong, with the following Antiphon.
\end{rubric}

\properantiphon{Mag.}{O sacred banquet, {\dag} wherein Christ is received, the memory of his Passion is renewed; the soul with grace is filled, and a pledge of future glory is bestowed, alleluia.}


\subby{Sunday in the Octave of Corpus Christi}
\feastday{Trinity I}
\centerline{\small{(First Sunday after Trinity)}}
%\begin{inhead}
%    {Semidouble}
%\end{inhead}

\begin{rubric}
	The Hymns are of the Feast of the Holy Body of Christ (p. \pageref{CorpusChristi}), with the following Versicles \& Antiphons.
\end{rubric}

℣. He fed them with the finest wheat flour, alleluia.

℟. And with honey out of the stony rock did he satisfy them, alleluia.

\properantiphon{Mag.}{The child Samuel {\dag} ministered unto the Lord before Eli, and the word of the Lord was precious in his sight.}\\

%MANUAL ADJUSTMENT:
\clearpage
℣. He gave them bread from heaven, alleluia.

℟. So man did eat angels' food, alleluia.

\properantiphon{Ben.}{Father Abraham, {\dag} have mercy on me, and send Lazarus, that he may dip the tip of his finger in water, and cool my tongue.}\\

℣. He fed them with the finest wheat flour, alleluia.

℟. And with honey out of the stony rock did he satisfy them, alleluia.

\properantiphon{Mag.}{Son, remember {\dag} that thou in thy lifetime receivedst thy good things, and likewise Lazarus evil things.}


\subby{Compassion of Our Lord Jesus Christ}
\feastday{Divine Compassion}
%\begin{inhead}
%	{Friday after Trinity I}\par
%    {Second Class Double, Simple Octave}
%\end{inhead}

\subbysub{I Evensong}\label{CompassionEvensong}

\gregorioscore{resources/gabc/ProperSeason/CompassionEvensong.gabc}

℣. I am come to send fire on the earth.

℟. And what will I, if it be already kindled?

\properantiphon{Mag.}{Thy rebuke {\dag} hath broken my heart; I am full of heaviness: I looked for some to have pity on me, but there was no man; neither found I any to comfort me.}

%MANUAL ADJUSTMENT:
\clearpage
\subbysub{Mattins}

\invitatoryhymn\label{CompassionInvitatory}

\gregorioscore{resources/gabc/ProperSeason/CompassionInvitatory.gabc}

\officehymn\label{CompassionMattins}

\gregorioscore{resources/gabc/ProperSeason/CompassionMattins.gabc}

℣. Surely he hath borne our griefs.

℟. And carried our sorrows.

\properantiphon{Ben.}{He was wounded {\dag} for our transgressions he was bruised for our iniquities: the chastisement of our peace was upon him, and with his stripes we are healed.}

\subbysub{II Evensong}

\begin{rubric}
	The Office Hymn is as in I Evensong, with the following Versicle \& Antiphon.
\end{rubric}

℣. With joy shall ye draw water.

℟. Out of the wells of salvation.

\properantiphon{Mag.}{But when they came to Jesus, {\dag} and saw that he was dead already, they brake not his legs: but one of the soldiers with a spear pierced his side, and forthwith came there out blood and water.}


\subby{Sundays after Trinity}
\feastday{Trinitytide}

\begin{multicols}{2}
\subsubsec{Second Sunday after Trinity}
\properantiphon{Mag.}{And all Israel knew {\dag} from Dan even to Beersheba, that Samuel was established to be a prophet of the Lord.}

\properantiphon{Ben.}{A certain man {\dag} made a great supper, and bade many; and sent his servants at supper time to say to them that were bidden, Come; for all things are now ready, alleluia.}

\properantiphon{Mag.}{Go out quickly {\dag} into the streets and lanes of the city, and compel them to come in; the poor and the maimed, the halt and the blind, that my house may be filled, alleluia.}

\subsubsec{Third Sunday after Trinity}
\properantiphon{Mag.}{So David prevailed {\dag} over the Philistine with a sling and a stone, in the Name of the Lord.}

\properantiphon{Ben.}{What man of you, {\dag} having an hundred sheep, if he lose one of them, doth not leave the ninety and nine in the wilderness, and go after that which is lost, until he find it? Alleluia.}

\properantiphon{Mag.}{What woman, {\dag} having ten pieces of silver, if she lose one piece, doth not light a candle, and sweep the house, and seek diligently until she find it?}

\subsubsec{Fourth Sunday after Trinity}
\properantiphon{Mag.}{Ye mountains {\dag} of Gilboa, let there be neither dew nor rain upon you; for there the shield of the mighty is vilely cast away, the shield of Saul, as though he had not been anointed with oil. How are the mighty fallen in the midst of the battle! Jonathan was slain upon thy high places! Saul and Jonathan were lovely and exceeding pleasant in their lives, and in their death they were not divided.}

\properantiphon{Ben.}{Be ye therefore {\dag} merciful, as your Father also is merciful, saith the Lord.}

\properantiphon{Mag.}{Judge not, {\dag} that ye be not judged: for with what judgment ye judge, ye shall be judged, saith the Lord.}

\subsubsec{Fifth Sunday after Trinity}
\properantiphon{Mag.}{O Lord, I beseech thee, {\dag} do away the iniquity of thy servant, for I have done very foolishly.}

\properantiphon{Ben.}{And Jesus entered {\dag} into a ship, and sat down, and taught the people, alleluia.}

\properantiphon{Mag.}{Master, {\dag} we have toiled all the night, and have taken nothing: nevertheless, at thy word I will let down the net.}

\subsubsec{Sixth Sunday after Trinity}
\properantiphon{Mag.}{Zadok the priest {\dag} and Nathan the prophet anointed Solomon king in Gihon: and the people came up rejoicing and said, Let the king live for ever.}

\properantiphon{Ben.}{Ye have heard that it was said {\dag} by them of old time, Thou shalt not kill, and whosoever shall kill, shall be in danger of the judgment.}

\properantiphon{Mag.}{If thou bring {\dag} thy gift to the altar, and there rememberest that thy brother hath ought against thee; leave there thy gift before the altar and go thy way; first be reconciled to thy brother, and then come and offer thy gift, alleluia.}

\subsubsec{Seventh Sunday after Trinity}
\properantiphon{Mag.}{Thou hast heard, O Lord, {\dag} the supplication of thy servant, that I might build a temple to thy Name.}

\properantiphon{Ben.}{The multitude being very great, {\dag} and having nothing to eat, Jesus called his disciples unto him, and saith unto them, I have compassion on the multitude, because they have now been with me three days, and have nothing to eat, alleluia.}

\properantiphon{Mag.}{I have compassion {\dag} on the multitude, because they have now been with me three days, and have nothing to eat: and if I send them away fasting, they will faint by the way, alleluia.}

\subsubsec{Eighth Sunday after Trinity}
\properantiphon{Mag.}{When the Lord took up Elijah {\dag} by a whirlwind into heaven, Elisha cried, saying: My father, the chariot of Israel and the horses thereof.}

\properantiphon{Ben.}{Beware {\dag} of false prophets, which come to you in sheep's clothing, but inwardly they are ravening wolves. Ye shall know them by their fruits, alleluia.}

\properantiphon{Mag.}{A good tree {\dag} cannot bring forth evil fruit, neither can a corrupt tree bring forth good fruit: every tree that bringeth not forth good fruit is hewn down, and cast into the fire, alleluia.}

\subsubsec{Ninth Sunday after Trinity}
\properantiphon{Mag.}{Jehoash {\dag} did that which was right in the sight of the Lord all his days, wherein Jehoiada the priest instructed him.}

\properantiphon{Ben.}{The lord said {\dag} unto the steward, How is it that I hear this of thee? Give an account of thy stewardship, alleluia.}

\properantiphon{Mag.}{What shall I do? {\dag} for my lord taketh away from me the stewardship: I cannot dig; to beg I am ashamed: I am resolved what to do, that, when I am put out of the stewardship, they may receive me into their houses.}

\subsubsec{Tenth Sunday after Trinity}
\properantiphon{Mag.}{I beseech thee, O Lord, {\dag} remember now how I have walked before thee in truth and with a perfect heart, and have done that which is good in thy sight.}

\properantiphon{Ben.}{When the Lord was come near to Jerusalem, {\dag} he beheld the city, and wept over it, saying: If thou hadst known, even thou! For the days shall come upon thee, that thine enemies shall cast a trench about thee, that thine enemies shall cast a trench about thee, and compass thee round, and keep thee in on every side, and shall lay thee even with the ground; because thou knewest not the time of thy visitation, alleluia.}

\properantiphon{Mag.}{Is it not written, {\dag} Mine house shall be called an house of prayer for all people? but ye have made it a den of robbers. And he was daily with them, teaching int he temple.}

\subsubsec{Eleventh Sunday after Trinity}
\properantiphon{Ben.}{And the publican, {\dag} standing afar off, would not lift up so much as his eyes unto heaven, but smote upon his breast, saying, God be merciful to me a sinner.}

\properantiphon{Mag.}{This man went down {\dag} to his house justified rather than the other: for every one that exalteth himself shall be abased; and he that humbleth himself shall be exalted.}

\subsubsec{Twelfth Sunday after Trinity}
\properantiphon{Ben.}{When the Lord had passed {\dag} through the coasts of Tyre, he made the deaf to hear and the dumb to speak.}

\properantiphon{Mag.}{He hath done all things well: {\dag} he maketh both the deaf to hear, and the dumb to speak.}

\subsubsec{Thirteenth Sunday after Trinity}
\properantiphon{Ben.}{Master, {\dag} what shall I do to inherit eternal life? He said unto him, What is written in the law? how readest thou? Thou shalt love the Lord thy God with all thy heart, alleluia.}

\properantiphon{Mag.}{A certain man {\dag} went down from Jerusalem to Jericho, and fell among thieves, which stripped him of his raiment, and wounded him, and departed, leaving him half dead.}

\subsubsec{Fourteenth Sunday after Trinity}
\properantiphon{Ben.}{As Jesus passed through {\dag} a certain village, there met him ten men that were lepers, which stood afar off: and they lifted up their voices, and said, Jesus, Master, have mercy on us.}

\properantiphon{Mag.}{And one of them, {\dag} when he saw that he was healed, turned back, and with a loud voice glorified God, alleluia.}

\subsubsec{Fifteenth Sunday after Trinity}
\properantiphon{Ben.}{Be not therefore anxious, {\dag} saying, What shall we eat? or What shall we drink? for your heavenly Father knoweth that ye have need of all these things, alleluia.}

\properantiphon{Mag.}{Seek ye first {\dag} the kingdom of God, and his righteousness; and all these things shall be added unto you, alleluia.}

\subsubsec{Sixteenth Sunday after Trinity}
\properantiphon{Ben.}{Jesus went into a city {\dag} called Nain; and behold, there was a dead man carried out, the only son of his mother.}

\properantiphon{Mag.}{A great prophet {\dag} is risen up among us: and God hath visited his people.}

\subsubsec{Seventeenth Sunday after Trinity}
\properantiphon{Ben.}{And Jesus went {\dag} into the house of one of the the chief Pharisees to eat bread on the sabbath day, behold there was a certain man before him which had the dropsy; and he took him, and healed him, and let him go.}

\properantiphon{Mag.}{When thou art bidden to {\dag} a wedding, sit down in the lowest place; that he that bade thee may say unto thee, Friend, go up higher; then shalt thou have worship in the presence of them that sit at meat with thee, alleluia.}

\subsubsec{Eighteenth Sunday after Trinity}
\properantiphon{Ben.}{Master, {\dag} which is the great commandment in the law? Jesus said unto him, Thou shalt love the Lord thy God with all thy heart, alleluia.}

\properantiphon{Mag.}{What think ye of Christ? {\dag} whose son is he? They say unto him, The son of David. Jesus saith unto them, How then doth David in spirit call him Lord, saying, The Lord said unto my Lord, Sit thou on my right hand?}

\subsubsec{Nineteenth Sunday after Trinity}
\properantiphon{Ben.}{The Lord said {\dag} unto the sick of the palsy, Son, be of good cheer, thy sins be forgiven thee.}

\properantiphon{Mag.}{The sick of the palsy {\dag} therefore took up his bed whereon he lay, glorifying God: and all the people, when they saw it, give praise unto God.}

\subsubsec{Twentieth Sunday after Trinity}
\properantiphon{Ben.}{Tell them which are bidden, {\dag} Behold, I have prepared my dinner; come unto the marriage, alleluia.}

\properantiphon{Mag.}{And when the king came in {\dag} to see the guests, he saw there a man which had not on a wedding garment: and he said unto him, Friend, how camest thou in hither not having a wedding garment?}

\subsubsec{Twenty-First Sunday after Trinity}
\properantiphon{Ben.}{There was a certain nobleman, {\dag} whose son was sick at Capernaum: when he heard that Jesus was come out of Jud{\ae}a into Galilee, he besought him that he would heal his son.}

\properantiphon{Mag.}{So the father knew {\dag} that it was at the same hour in the which Jesus said, Thy son liveth: and himself believed, and his whole house.}

\subsubsec{Twenty-Second Sunday after Trinity}
\properantiphon{Ben.}{Then said the lord {\dag} unto the servant, Pay me that thou owest. The servant therefore fell down and worshipped him, saying, Lord, have patience with me, and I will pay thee all.}

\properantiphon{Mag.}{Thou wicked servant, {\dag} I forgave thee all that debt, because thou desiredst me: shouldest not thou also have had compassion on thy fellow-servant, even as I had pity on thee, alleluia.}

\subsubsec{Twenty-Third Sunday after Trinity}
\properantiphon{Ben.}{Master, {\dag} we know that thou art true, and teachest the way of God in truth, alleluia.}

\properantiphon{Mag.}{Render therefore {\dag} unto Caesar the things which are Caesar's: and unto God the things that are God's, alleluia.}

\subsubsec{Twenty-Fourth Sunday after Trinity}
\properantiphon{Ben.}{For she said within herself, {\dag} If I may but touch the hem of his garment, I shall be whole.}

\properantiphon{Mag.}{But Jesus turned him about, {\dag} and when he saw her, he said, Daughter, be of good comfort; thy faith hath made thee whole, alleluia.}

\subsubsec{Sunday Next before Advent}
\properantiphon{Ben.}{When Jesus then {\dag} lifted up his eyes, and saw a great company come unto him, he saith unto Philip, Whence shall we buy bread, that these may eat? And this he said to prove him: for he himself knew what he would do.}

\properantiphon{Mag.}{Then those men, {\dag} when they had seen the miracle that Jesus did, said among themselves, This is of a truth that prophet that should come into the world.}

\end{multicols}


%After Trinity VIII

\subby{Magnificat Antiphons for Sundays in Trinitytide}
\fancyhead[RO,LE]{Trinitytide Antiphons}
\fancyhead[RE,LO]{}
\begin{multicols}{2}
\begin{inhead}
Sunday within 29 July - 4 August
\end{inhead}\par\noindent
Wisdom hath builded her house, {\dag} she hath hewn out her seven pillars: she hath subdued the nations, and in her own might hath she trodden under the necks of the proud and lofty.

\begin{inhead}
Sunday within 5 - 11 August
\end{inhead}\par\noindent
I dwell {\dag} in high places, and my throne is in a cloudy pillar.
\begin{inhead}
Sunday within 12 - 18 August
\end{inhead}\par\noindent
All wisdom {\dag} cometh from the Lord, and is with him for ever; and is before the ages.

\begin{inhead}
Sunday within 19 - 25 August
\end{inhead}\par\noindent
Wisdom crieth {\dag} aloud in the broad places: Whosoever loveth wisdom, let him turn in hither, and he shall find her; and when he hath found her, happy is he if he hold her fast.
\begin{inhead}
Sunday within 26 - 28 August
\end{inhead}\par\noindent
My son {\dag} keep thy father's commandment, and forsake not the law of thy mother: bind them continually upon thine heart.
\begin{inhead}
Sunday within 29 August - 4 September
\end{inhead}\par\noindent
Now when Job had heard {\dag} the words of the messengers, he endured with patience, saying, Shall we receive good at the Lord's hand, and shall we not receive evil also? In all this Job sinned not with his lips, neither charged God foolishly.
\begin{inhead}
Sunday within 5 - 11 September
\end{inhead}\par\noindent
In all this {\dag} Job sinned not with his lips, neither charged God foolishly.
\begin{inhead}
Sunday within 12 - 18 September
\end{inhead}\par\noindent
Remember not, {\dag} Lord, our offences, nor the offences of our forefathers; neither take thou vengeance of our sins.
\begin{inhead}
Sunday within 19 - 25 September
\end{inhead}\par\noindent
Adonai, {\dag} Lord God Almighty, great and wonderful, who hast given salvation by the hand of a woman; hear, we beseech thee, the prayers of thy servants.
\begin{inhead}
Sunday within 26 - 27 September
\end{inhead}\par\noindent
O Lord, {\dag} the King Almighty, the whole world is in thy power, and there is no man that can gainsay thee.
\begin{inhead}
Sunday within 28 September - 4 October
\end{inhead}\par\noindent
The Lord open your hearts {\dag} in his law and commandments; and may the Lord our God send you peace.
\begin{inhead}
Sunday within 5 - 11 October
\end{inhead}\par\noindent
The sun shone {\dag} upon the shields of gold, and the mountains glistered therewith: and yet the forces of the heathen were discomfited.
\begin{inhead}
Sunday within 12 - 18 October
\end{inhead}\par\noindent
But Israel mourned Judas, {\dag} and made great lamentation for him, saying: How art thou fallen, valiant in battle, that didst deliver the people of the Lord.
\begin{inhead}
Sunday within 19 - 25 October
\end{inhead}\par\noindent
God be gracious unto you, {\dag} and hear your prayers, and be at one with you: and the Lord our God never forsake you in time of trouble.
\begin{inhead}
Sunday within 26 - 28 October
\end{inhead}\par\noindent
Thine is the power, O Lord, {\dag} and thine is the kingdom: thou art high above all nations: give peace in our time, O Lord our God.
\begin{inhead}
Sunday within 29 October - 4 November
\end{inhead}\par\noindent
I saw the Lord also, {\dag} sitting upon a throne, high and lifted up, and his train filled the temple: the whole earth was full of the majesty of his glory.

\begin{inhead}
Sunday on 5 November
\end{inhead}\par\noindent
Behold, O Lord, {\dag} how the city is become desolate that was full of precious treasure; she sitteth sorrowful that was great among the nations: none can comfort her but only thou, O our God.

\begin{inhead}
Sunday within 6 - 12 November
\end{inhead}\par\noindent
Encompass us, {\dag} O Lord, with thine impregnable wall: and with the arms of thy power defend us alway, O our God.
\begin{inhead}
Sunday within 13 - 19 November
\end{inhead}\par\noindent
Thou that upholdest {\dag} the throne of the heavens and beholdest the depths, O Lord, King of kings; that weighest the mountains and holdest the earth in the hollow of thine hand: give ear, O Lord, unto us, in the midst of our groanings.

\begin{inhead}
Sunday within 20 - 26 November
\end{inhead}\par\noindent
I have set watchmen {\dag} upon thy walls, O Jerusalem, which shall never hold their peace day nor night, praising the Name of the Lord.
\end{multicols}

