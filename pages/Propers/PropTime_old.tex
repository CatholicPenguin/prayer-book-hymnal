\phantomsection
\addcontentsline{toc}{section}{Proper of Saints}
\fancyhead[C]{\LARGE Proper of Saints}


\subby{Vigil of St. Andrew}
\fancyhead[RO,LE]{\textit{Andrew Vigil}}
\fancyhead[RE,LO]{29 November}
\begin{inhead}
    {Vigil\\
29 November}
\end{inhead}

\begin{rubric}
	If to-day be Saturday, the anticipated Vigil of St. Andrew is kept, but the commemoration of St. Saturninus is omitted, in which case the \nth{2} is of St. Mary and the \nth{3} against the persecutors of the Church or for the Chief Bishop.
\end{rubric}

\introit
\lett{T}{he} Lord, walking by the sea of Galilee, saw two brethren, Peter and Andrew, and he called them, saying: Follow me: I will make you fishers of men. \textit{Ps.} The heavens declare the glory of God: and the firmament sheweth his handy-work.

\collect
\lett{W}{e} beseech thee, almighty God: that the blessed Apostle Andrew, whose festival we prevent, may implore thy help for us; that we, being absolved from our offences, may likewise be delivered from all dangers. Through.

\begin{rubric}
	In Advent, \nth{2} Collect of the Feria, \nth{3} of St. Saturninus, as in the following Mass.
\end{rubric}
\begin{rubric}
	Outside of Advent, \nth{2} Collect of St. Saturninus, \nth{3} of St. Mary.
\end{rubric}

\readingcitation{Epistle}{Ecclesiasticus 44:22}
%RV:
\lett{I}{n} Isaac did the Lord establish, for Abraham his father's sake, The blessing of all men, and the covenant: And he made it rest upon the head of Jacob; He acknowledged him in his blessings, And gave to him by inheritance, And divided his portions; Among twelve tribes did he part them. And he brought out of him a man of mercy, Which found favour in the sight of all flesh; A man beloved of God and men, even Moses, Whose memorial is blessed. He made him like to the glory of the saints, And magnified him in the fears of his enemies. By his words he caused the wonders to cease; He glorified him in the sight of kings; He gave him commandment for his people, And shewed him part of his glory. He sanctified him in his faithfulness and meekness; He chose him out of all flesh. He made him to hear his voice, And led him into the thick darkness, And gave him commandments face to face, Even the law of life and knowledge, That he might teach Jacob the covenant, And Israel his judgments. He exalted Aaron, a holy man like unto him, Even his brother, of the tribe of Levi. He established for him an everlasting covenant, And gave him the priesthood of the people; He beautified him with comely ornaments, And girded him about with a robe of glory.

%Traditional Text:
%\lett{T}{he} blessing of the Lord was upon the head of the righteous. Therefore did the Lord give him an heritage, and divided his portions among the twelve tribes: and he found favour in the sight of all flesh. And he magnified him so that his enemies stood in fear of him, and by his words he caused the wonders to cease. He made him glorious in the sight of kings, and gave him a commandment for his people, and shewed him his glory. He sanctified him in his faithfulness and meekness, and chose him out of all men. And he gave him commandments before his face, even the law of life and knowledge, and made him to be exalted. An everlasting covenant he made with him, and girded him about with the girdle of righteousness: and the Lord crowned him with the crown of glory.

\gradual{Right honourable are thy friends, O God: right well is their princedom established. ℣. If I tell them: they are more in number than the sand.}

\readingcitation{Gospel}{John 1:35}
\lett{A}{t that time:} John stood, and two of his disciples; And looking upon Jesus as he walked, he saith, Behold the Lamb of God! And the two disciples heard him speak, and they followed Jesus. Then Jesus turned, and saw them following, and saith unto them, What seek ye? They said unto him, Rabbi, (which is to say, being interpreted, Master,) where dwellest thou? He saith unto them, Come and see. They came and saw where he dwelt, and abode with him that day: for it was about the tenth hour. One of the two which heard John speak, and followed him, was Andrew, Simon Peter's brother. He first findeth his own brother Simon, and saith unto him, We have found the Messias, which is, being interpreted, the Christ. And he brought him to Jesus. And when Jesus beheld him, he said, Thou art Simon the son of Jona: thou shalt be called Cephas, which is by interpretation, A stone. The day following Jesus would go forth into Galilee, and findeth Philip, and saith unto him, Follow me. Now Philip was of Bethsaida, the city of Andrew and Peter. Philip findeth Nathanael, and saith unto him, We have found him, of whom Moses in the law, and the prophets, did write, Jesus of Nazareth, the son of Joseph. And Nathanael said unto him, Can there any good thing come out of Nazareth? Philip saith unto him, Come and see. Jesus saw Nathanael coming to him, and saith of him, Behold an Israelite indeed, in whom is no guile! Nathanael saith unto him, Whence knowest thou me? Jesus answered and said unto him, Before that Philip called thee, when thou wast under the fig tree, I saw thee. Nathanael answered and saith unto him, Rabbi, thou art the Son of God; thou art the King of Israel. Jesus answered and said unto him, Because I said unto thee, I saw thee under the fig tree, believest thou? thou shalt see greater things than these. And he saith unto him, Verily, verily, I say unto you, Hereafter ye shall see heaven open, and the angels of God ascending and descending upon the Son of man.

\offertory{Thou hast crowned him with glory and worship: thou hast made him to have dominion of the works of thy hands, O Lord.}

\secret
\lett{W}{e} offer, O Lord, this gift to be hallowed unto thee: whereby, recalling the festival of thy blessed Apostle Andrew, we likewise implore the purification of our souls. Through.

\communion{Andrew saith to Simon his brother: We have found the Messias, which is called the Christ: and he brought him to Jesus.}

\postcommunion
\lett{O}{Lord,} who hast bestowed on us these sacraments, we humbly beseech thee: that, at the intercession of thy blessed Apostle Andrew, the mysteries which we offer for his venerable passion may be profitable for our healing. Through.

\subby{St. Saturninus}
\fancyhead[RO,LE]{\textit{Saturninus}}
\fancyhead[RE,LO]{29 November}
\begin{inhead}
    {Memorial\\
29 November}
\end{inhead}

\begin{rubric}
	The propers are from the Second Common of a Martyr not a Bishop (p. \pageref{CommonMartyrNotBishopII}), except the following Prayers.
\end{rubric}

\collect
\lett{O}{God,} who vouchsafest unto us to rejoice in the birthday of thy blessed Martyr Saturninus: grant, we pray thee; that we may be succoured by his merits. Through.

\secret
\lett{S}{anctify,} O Lord, we beseech thee, the offerings which we dedicate unto thee: and at the intercession of blessed Saturninus, thy Martyr, for their sake graciously regard us. Through.

\postcommunion
\lett{W}{e} beseech thee, O Lord, that we, being sanctified by the receiving of thy sacrament: may, at the intercession of thy Saints, be thereby rendered acceptable unto thee. Through.

\supplement{30 November}{St. Andrew}{}

\begin{rubric}
	For the Feast of St. Andrew, the Hymn and Versicle are from the Common of Apostles (p. \pageref{CommonApostles}), with the antiphons as below.
\end{rubric}

\properantiphon{Mag.}{One of the two {\dag} which followed the Lord was Andrew, Simon Peter's brother, alleluia.}

\properantiphon{Ben.}{Yield up to us {\dag} a man so righteous, restore to us a man so holy: destroy not a man so dear to God, righteous, dutiful, and gentle.}

\properantiphon{Mag.}{When blessed Andrew {\dag} came to the place where the Cross had been prepared, he cried out and said: O goodly Cross, so long desired, and now made ready for my eager spirit; fearless and joyful do I come to thee: therefore do thou also receive me gladly, as his disciple, who did hang upon thee.}


\subby{St. Peter Chrysologus}
\fancyhead[RO,LE]{\textit{Peter Chrysologus}}
\fancyhead[RE,LO]{2 December}
\begin{inhead}
    {Double\\
2 December}
\end{inhead}

\begin{rubric}
	The propers are from the Common of Doctors (p. \pageref{CommonDoctors}), except the Collect, Gradual, and Communion.\par
	\textsc{Note,} Commemoration is made of St. Bibiana and of the Advent Feria.
\end{rubric}

\collect
\lett{O}{God,} who by divine foreshewing wast pleased to choose blessed Peter Chrysologus thy illustrious Doctor to rule and instruct thy Church: grant, we beseech thee; that, as we have had him for a Doctor of life on earth, so we may be found worthy to have him for an intercessor in heaven. Through.

\gradual{Behold a great priest, who in his days pleased God. ℣. There was none found like unto him, who kept the law of the Most High.}

\communion{Lord. thou deliveredst unto me five talents: behold, I have gained beside them five talents more. Well done, thou good and faithful servant, thou hast been faithful over a few things, I will make thee ruler over many things, enter thou into the joy of thy lord.}

\subby{St. Bibiana}
\fancyhead[RO,LE]{\textit{Bibiana}}
\fancyhead[RE,LO]{2 December}
\begin{inhead}
    {Memorial\\
2 December}
\end{inhead}

\begin{rubric}
	The propers are from the Second Common of a Virgin Martyr (p. \pageref{CommonVirginMartyrII}), except for her Prayers below.
\end{rubric}

\collect
\lett{O}{God,} the giver of all good gifts, who in thy handmaid Bibiana didst unite the palm of martyrdom with the flower of virginity: unite by her intercession our hearts in charity with thee; that all perils being done away, we may attain unto everlasting rewards. Through.

\secret
\lett{G}{raciously} receive, O Lord, through the merits of blessed Bibiana, thy Virgin and Martyr, the sacrifices which we offer unto thee: and grant that they may avail for our continual help. Through.

\postcommunion
\lett{O}{Lord} our God, who hast fulfilled us with the bounty of thy heavenly gift: we beseech thee, at the intercession of blessed Bibiana, thy Virgin and Martyr, we may ever live by the partaking of the same. Through.

\subby{St. Barbara}
\fancyhead[RO,LE]{\textit{Barbara}}
\fancyhead[RE,LO]{4 December}
\begin{inhead}
    {Memorial\\
4 December}
\end{inhead}

\begin{rubric}
	The propers are from the First Common of a Virgin Martyr (p. \pageref{CommonVirginMartyrI}).
\end{rubric}


\subby{St. Sabbas of Jud{\ae}a}
\fancyhead[RO,LE]{\textit{Sabbas}}
\fancyhead[RE,LO]{5 December}
\begin{inhead}
    {Memorial\\
5 December}
\end{inhead}

\begin{rubric}
	The propers are from the Common of an Abbot (p. \pageref{CommonAbbots}).
\end{rubric}

\begin{rubric}
	If today be Saturday, the Mass is of St. Mary on Saturday with \nth{2} Collect of the Feria \& \nth{3} of St. Sabbas.
\end{rubric}


\subby{St. Nicholas}
\fancyhead[RO,LE]{\textit{Nicholas}}
\fancyhead[RE,LO]{6 December}
\begin{inhead}
    {Double\\
6 December}
\end{inhead}

\begin{rubric}
	The Daily Office propers are from the First Common of a Confessor Bishop (p. \pageref{CommonConfessorBishopI}).
\end{rubric}

\introit
\lett{T}{he} Lord hath established a covenant of peace with him, and made him a prince: that he should have the dignity of the priesthood for ever. \textit{Ps.} Lord, remember David: and all his trouble.

\collect
\lett{O}{God,} who didst adorn thy blessed Bishop Nicholas with innumerable miracles: grant, we beseech thee; that by his merits and prayers we may be delivered from the fires of hell. Through.

\begin{rubric}
	Commemoration is of the Feria, unless it be Saturday.
\end{rubric}

\readingcitation{Epistle}{Hebrews 13:7}
\lett{B}{rethren:} Remember them which have the rule over you, who have spoken unto you the word of God: whose faith follow, considering the end of their conversation. Jesus Christ the same yesterday, and to day, and for ever. Be not carried about with divers and strange doctrines. For it is a good thing that the heart be established with grace; not with meats, which have not profited them that have been occupied therein. We have an altar, whereof they have no right to eat which serve the tabernacle. For the bodies of those beasts, whose blood is brought into the sanctuary by the high priest for sin, are burned without the camp. Wherefore Jesus also, that he might sanctify the people with his own blood, suffered without the gate. Let us go forth therefore unto him without the camp, bearing his reproach. For here have we no continuing city, but we seek one to come. By him therefore let us offer the sacrifice of praise to God continually, that is, the fruit of our lips giving thanks to his name. But to do good and to communicate forget not: for with such sacrifices God is well pleased. Obey them that have the rule over you, and submit yourselves: for they watch for your souls, as they that must give account.

\gradall{I have found David my servant, with my holy oil have I anointed him: my hand shall hold him fast, and my arm shall strengthen him. ℣. The enemy shall not be able to do him violence, the son of wickedness shall not hurt him.}{Alleluia, alleluia. ℣. The righteous shall flourish like a palm-tree: and shall spread abroad like a cedar in Libanus. Alleluia.}

\readingcitation{Gospel}{Matthew 25:14}
\lett{A}{t that time:} Jesus spake this parable to his disciples: A man travelling into a far country, who called his own servants, and delivered unto them his goods. And unto one he gave five talents, to another two, and to another one; to every man according to his several ability; and straightway took his journey. Then he that had received the five talents went and traded with the same, and made them other five talents. And likewise he that had received two, he also gained other two. But he that had received one went and digged in the earth, and hid his lord's money. After a long time the lord of those servants cometh, and reckoneth with them. And so he that had received five talents came and brought other five talents, saying, Lord, thou deliveredst unto me five talents: behold, I have gained beside them five talents more. His lord said unto him, Well done, thou good and faithful servant: thou hast been faithful over a few things, I will make thee ruler over many things: enter thou into the joy of thy lord. He also that had received two talents came and said, Lord, thou deliveredst unto me two talents: behold, I have gained two other talents beside them. His lord said unto him, Well done, good and faithful servant; thou hast been faithful over a few things, I will make thee ruler over many things: enter thou into the joy of thy lord.

\offertory{My truth and my mercy shall be with him: and in my name shall his horn be exalted.}

\secret
\lett{S}{anctify,} we bescech thee, O Lord God, these gifts which we offer on the solemnity of thy holy Bishop Nicholas: that our life may ever thereby be directed both in prosperity and in adversity. Through.

\begin{rubric}
	Commemoration is of the Feria, unless it be Saturday.
\end{rubric}

\communion{I have sworn once by my holiness: His seed shall endure for ever, and his seat is like as the sun before me, he shall stand fast for evermore as the moon, and as the faithful witness in heaven.}

\postcommunion
\lett{M}{ay} the sacrifices which we have received, O Lord, for the solemnity of thy holy Bishop Nicholas, preserve us by their everlasting protection. Through.

\begin{rubric}
	Commemoration is of the Feria, unless it be Saturday.
\end{rubric}


\subby{St. Ambrose}
\fancyhead[RO,LE]{\textit{Ambrose}}
\fancyhead[RE,LO]{7 December}
\begin{inhead}
    {Greater Double\\
7 December}
\end{inhead}

\begin{rubric}
	The Daily Office propers are from the Common of a Confessor Bishop (p. \pageref{CommonConfessorBishopI}).
\end{rubric}

\introit
\lett{I}{n} the midst of the Church he opened his mouth: and the Lord filled him with the spirit of wisdom and of understanding: he clothed him with a robe of glory. \textit{Ps.} It is a good thing to give thanks unto the Lord: and to sing praises unto thy name, O most Highest.

\collect
\lett{O}{God,} who didst give blessed Ambrose unto thy people to be a minister of everlasting salvation: grant, we beseech thee; that as we have learned of him the doctrine of life on earth, so we may be found worthy to have him for our advocate in heaven. Through.

\begin{rubric}
	Commemoration is of the Feria.
\end{rubric}

\begin{rubric}
	The Epistle is from the Common of Doctors (p. \pageref{CommonDoctors}).
\end{rubric}

\gradall{Behold a great priest, who in his days pleased God. ℣. There was none foun like unto him, who kept the law of the Most High.}{Alleluia, alleluia. ℣. The Lord sware, and will not repent: Thou art a priest for ever after the order of Melchisedech.}

\begin{rubric}
{In Septuagesimatide or Lent, replacing the Alleluia:}
\end{rubric}\par\noindent
\tract{Blessed is the man that feareth the Lord: he hath great delight in his commandments. ℣. His seed shall be mighty upon earth: the generation of the faithful shall be blessed. ℣. Riches and plenteousness shall be in his house: and his righteousness endureth for ever.}

\begin{rubric}
{In Eastertide, replacing the Lesser Alleluia:}
\end{rubric}\par\noindent
\alleluia{Alleluia, alleluia. ℣. The Lord loved him, and adorned him: and clothed him with a robe of glory. Alleluia. ℣. The righteous shall grow as the lily and flourish for ever before the Lord. Alleluia.}

\begin{rubric}
	The Gospel is from the Common of Doctors (p. \pageref{CommonDoctors}).
\end{rubric}

\offertory{My truth and my mercy shall be with him: and in my name shall his horn be exalted.}

\secret
\lett{A}{lmighty} and everlasting God, grant, that the gifts which we present unto thy majesty, may through the intercession of blessed Ambrose, thy Confessor and Bishop, be profitable unto us for everlasting salvation. Through.

\begin{rubric}
	Commemoration is of the Feria.
\end{rubric}

\communion{I have sworn once by my holiness: His seed shall endure for ever, and his seat is like as the sun before me, he shall stand fast for evermore as the moon, and as the faithful witness in heaven.}

\postcommunion
\lett{G}{rant,} we beseech thee, almighty God: that we, receiving the sacraments of our salvation, may ever be aided by the prayer of blessed Ambrose thy Confessor and Bishop; in whose honour we have made these offerings unto thy majesty. Through.

\begin{rubric}
	Commemoration is of the Feria. In Lent, the Last Gospel is of the Feria.
\end{rubric}

%\begin{rubric}
%	If today be Saturday, the Mass is of the anticipated Vigil of the Conception of the Blessed Virgin Mary.
%\end{rubric}

\vspace{3ex}

\supplement{8 December}{Conception}{of the Blessed Virgin Mary}

\begin{rubric}
	The Office Hymn is from the Common of the Blessed Virgin Mary (p. \pageref{CommonBVM}) with the following Versicle and Antiphons.
\end{rubric}
℣. To-day is the Conception of the holy Virgin Mary.

℟. Whose glorious life illumineth all the churches.

\properantiphon{Mag.}{All generations {\dag} shall call me blessed: for he that is mighty hath magnified me, alleluia.}

\properantiphon{Ben.}{The Lord God said {\dag} unto the serpent, I will put enmity between thee and the woman, and between thy seed and her seed; it shall bruise thy head, alleluia.}

\properantiphon{Mag.}{Let us celebrate {\dag} the worshipful Conception of the blessed and glorious Virgin Mary, whose lowliness the Lord regarded when at the word of an Angel she conceived the world's Redeemer, alleluia.}


\subby{St. Melchiades}
\fancyhead[RO,LE]{\textit{Melchiades}}
\fancyhead[RE,LO]{10 December}
\begin{inhead}
    {Memorial\\
10 December}
\end{inhead}

\begin{rubric}
	The propers are from the First Common of a Martyr Bishop (p. \pageref{CommonMartyrBishopI}).
\end{rubric}


\subby{St. Damasus}
\fancyhead[RO,LE]{\textit{Damasus}}
\fancyhead[RE,LO]{11 December}
\begin{inhead}
    {Memorial\\
11 December}
\end{inhead}

\begin{rubric}
	The Daily Office propers are from the First Common of a Confessor Bishop (p. \pageref{CommonConfessorBishopI}).
\end{rubric}

\introit
\lett{L}{et} thy priests, O Lord, be clothed with righteousness, and let thy Saints sing with joyfulness: for thy servant David's sake turn not away the presence of thine Anointed. \textit{Ps.} Lord, remember David: and all his trouble.

\collect
\lett{G}{raciously} hear our prayers, O Lord: and at the intercession of blessed Damasus, thy Confessor and Bishop, mercifully grant us pardon and peace. Through.

\begin{rubric}
	Commemoration is of the Octave \& Feria.
\end{rubric}

\begin{rubric}
	The Epistle is from the Second Common of a Confessor Bishop (p. \pageref{CommonConfessorBishopII}).
\end{rubric}

\gradall{Behold a great priest who in his days pleased God. ℣. There was none found like unto him, who kept the law of the Most High.}{Alleluia, alleluia. ℣. Thou art a priest for ever after the order of Melchisedech. Alleluia.}

\begin{rubric}
	The Gospel is from the Second Common of a Confessor Bishop (p. \pageref{CommonConfessorBishopII}).
\end{rubric}

\communion{I have found David my servant, with my holy oil have I anointed him: my hand shall hold him fast, and my arm shall strengthen him.}

\secret
\lett{G}{rant,} O Lord, that like as thy dedicated people do acknowledge that in tribulation they have been succoured by the merits of thy Saints: so this oblation, which they offer unto thee in honour of the same, may be acceptable in thy sight. Through.

\begin{rubric}
	Commemoration is of the Octave \& Feria.
\end{rubric}

\communion{Lord, thou deliveredst unto me five talents, behold I have gained beside them five talents more. Well done, thou good and faithful servant, thou hast been faithful over a few things, I will make thee ruler over many things, enter thou into the joy of thy Lord.}

\postcommunion
\lett{G}{rant,} we beseech thee, O Lord, that thy faithful people may ever rejoice in the veneration of thy saints: and be defended by their perpetual supplication. Through.

\begin{rubric}
	Commemoration is of the Octave \& Feria.
\end{rubric}


\subby{St. Lucy}
\fancyhead[RO,LE]{\textit{Lucy}}
\fancyhead[RE,LO]{13 December}
\begin{inhead}
    {Greater Double\\
13 December}
\end{inhead}

\begin{rubric}
	The Daily Office propers are from the Common of Virgins (p. \pageref{CommonVirginOnlyI}), except that which followeth.
\end{rubric}

\antiphon{Mag.}{In thy patience {\dag}} thou didst possess thy soul, O Lucy, spouse of Christ: thou didst hate the things which are in the world, and thou shinest among the Angels: resisting unto blood, thou didst vanquish the enemy.

\antiphon{Ben.}{A pillar art thou {\dag} that may not be moved, O Lucy, spouse of Christ: and all the people are waiting until thou receive the crown of life, alleluia.}\\

℣. Full of grace are thy lips.

℟. Because God hath blessed thee for ever.

\antiphon{Mag.}{With such gravity {\dag} was she endued by the Holy Spirit, that the virgin of the Lord remained unmoved.}

\introit
\lett{T}{hou} hast loved righteousness, and hated iniquity: wherefore God, even thy God, hath anointed thee with the oil of gladness above thy fellows. \textit{Ps.} My heart is inditing of a good matter: I speak of the things which I have made unto the King.

\collect
\lett{G}{raciously} hear us, O God of our salvation: that, like as we do rejoice in the festival of blessed Lucy thy Virgin; so we may be instructed in all godly and devout affection. Through.

\begin{rubric}
	Commemoration is of the Octave \& Feria.
\end{rubric}

\begin{rubric}
	The Epistle is from the First Common of a Virgin (p. \pageref{CommonVirginOnlyI}).
\end{rubric}

\gradall{Thou hast loved righteousness, and hated iniquity. ℣. Wherefore God, even thy God, hath anointed thee with the oil of gladness.}{Alleluia, alleluia. ℣. Full of grace are thy lips: because God hath blessed thee for ever. Alleluia.}

\begin{rubric}
	In Votive Masses after Septuagesima, the Tract, and in Eastertide, the Alleluia, is from the First Common of a Virgin (p. \pageref{CommonVirginOnlyI}).
\end{rubric}

\readingcitation{Gospel}{Matthew 13:44}
\lett{A}{t that time:} Jesus spake this parable unto his disciples: The kingdom of heaven is like unto treasure hid in a field; the which when a man hath found, he hideth, and for joy thereof goeth and selleth all that he hath, and buyeth that field. Again, the kingdom of heaven is like unto a merchant man, seeking goodly pearls: Who, when he had found one pearl of great price, went and sold all that he had, and bought it. Again, the kingdom of heaven is like unto a net, that was cast into the sea, and gathered of every kind: Which, when it was full, they drew to shore, and sat down, and gathered the good into vessels, but cast the bad away. So shall it be at the end of the world: the angels shall come forth, and sever the wicked from among the just, And shall cast them into the furnace of fire: there shall be wailing and gnashing of teeth. Jesus saith unto them, Have ye understood all these things? They say unto him, Yea, Lord. Then said he unto them, Therefore every scribe which is instructed unto the kingdom of heaven is like unto a man that is an householder, which bringeth forth out of his treasure things new and old.

\offertory{The Virgins that be her fellows shall be brought unto the King: they that bear her company shall be brought unto thee with joy and gladness: and shall enter into the palace of the Lord the King.}

\secret
\lett{G}{rant,} O Lord, that like as thy dedicated people do acknowledge that in tribulation they have been succoured by the merits of thy Saints: so this oblation, which they offer unto thee in honour of the same, may be acceptable in thy sight. Through.

\begin{rubric}
	Commemoration is of the Octave \& Feria.
\end{rubric}

\communion{Princes have persecuted me without a cause, but my heart standeth in awe of thy word: I am as glad of thy word, as one that findeth great spoils.}

\postcommunion
\lett{O}{Lord,} who hast satisfied thy family with sacred gifts: we beseech thee; that we may at all times be comforted by the intercession of her whose festival we celebrate. Through.

\begin{rubric}
	Commemoration is of the Octave \& Feria.
\end{rubric}

\subby{St. Herman of Alaska}
\fancyhead[RO,LE]{\textit{Herman}}
\fancyhead[RE,LO]{13 December}
\begin{inhead}
    {Memorial\\
13 December}
\end{inhead}

\begin{rubric}
	The propers are from the First Common of a Confessor not Bishop (p. \pageref{CommonConfessorNotBishopI}).
\end{rubric}


\subby{Day VII within the Octave of the Conception}
\fancyhead[RO,LE]{\textit{Conception Octave}}
\fancyhead[RE,LO]{14 December}
\begin{inhead}
    {Semidouble\\
14 December}
\end{inhead}

\begin{rubric}
	Of the Octave, as on December 9. But if Ember Wednesday fall on this day, of the Ember Day, with a Commemoration of the Octave. The Mass is with the \nth{3} Prayer of the Holy Ghost.
\end{rubric}

\begin{rubric}
	If an Ember Day occur on any of the following Feasts, the Commemoration of the Feria is omitted.
\end{rubric}


\subby{Octave Day of the Conception of the Blessed Virgin Mary}
\fancyhead[RO,LE]{\textit{Conception Octave Day}}
\fancyhead[RE,LO]{15 December}
\begin{inhead}
    {Greater Double\\
15 December}
\end{inhead}

\begin{rubric}
	Hymn, Versicle, Antiphon, \& Mass as on the Feast, with Commemoration of the Feria.
\end{rubric}


\subby{St. Eusebius of Vercelli}
\fancyhead[RO,LE]{\textit{Eusebius}}
\fancyhead[RE,LO]{16 December}
\begin{inhead}
    {Memorial\\
16 December}
\end{inhead}

\begin{rubric}
	The propers are from the Second Common of a Martyr Bishop (p. \pageref{CommonMartyrBishopII}), with commemoration of the Feria (and the Collect of St. Mary in Mass).
\end{rubric}


\subby{Expectation of the Blessed Virgin Mary}
\fancyhead[RO,LE]{\textit{Expectation}}
\fancyhead[RE,LO]{18 December}
\begin{inhead}
    {Double\\
18 December}
\end{inhead}
%Is this the same as the Mass of the BVM from Advent to Christmas except for the Gradual?
%\subsubsec{Daily Office Propers}\par\noindent
%%From the Dominican Breviary (https://archive.org/details/BrevariumIuxtaRitumSacriOrdinisPraedicatorum/black-and-white%20version/Breviarium_Cormier_1909_prior_cropped%20%285.18x6.5%29%20pg%201-1577%20%28vol.%201%29/page/n826/mode/1up):
\par\noindent
\textit{Opening Sentence.} Send ye the lamb to the ruler of the land from Sela to the wilderness, unto the mount of the daughter of Zion.\vr{Is. 16:1}\par

\begin{paracol}{2}[]
\sloppy
\begin{inhead}
	I Evensong
\end{inhead}
\begin{hangparas}{1.25em}{1}

Creator of the stars of night,
 
 thy people's everlasting light,
 
 Jesu, Redeemer, save us all,
 
 and hear thy servants when they call.\\

 Thou, grieving that the ancient curse
 
 should doom to death a universe,
 
 hast found the med'cine, full of grace,
 
 to save and heal a ruined race.\\

 Thou cam'st, the Bridegroom of the bride,
 
 as drew the world to ev'ning-tide;
 
 proceeding from a virgin shrine,
 
 the spotless victim all divine.\\

 At whose dread name, majestic now,
 
 all knees must bend, all hearts must bow;
 
 and things celestial thee shall own,
 
 and things terrestrial, Lord alone.\\

 O thou whose coming is with dread
 
 to judge and doom the quick and dead,
 
 preserve us, while we dwell below,
 
 from every insult of the foe.\\

 To God the Father, God the Son,
 
 and God the Spirit, Three in One,
 
 laud, honour, might, and glory be
 
 from age to age eternally. Amen.\\
\end{hangparas}

℣. Hail Mary, full of grace.

℟. The Lord is with thee.

\switchcolumn

\begin{inhead}
	Mattins
\end{inhead}

\begin{hangparas}{1.25em}{1}

A thrilling voice by Jordan rings,

rebuking guilt and darksome things:

vain dreams of sin and visions fly;

Christ in his might shines forth on high.\\

Now let each torpid soul arise,

that sunk in guilt and wounded lies;

see! the new Star's refulgent ray

shall chase disease and sin away.\\

The Lamb descends from heav'n above

to pardon sin with freest love:

for such indulgent mercy shewn

with tearful joy our thanks we own.\\

That when again he shines reveal'd,

and trembling worlds to terror yield.

He give not sin its just reward,

but in his love protect and guard.\\

To God the Father, God the Son,

And God the Spirit, Three in One,

Laud, honour, might, and glory be

From age to age eternally. Amen.\\
\end{hangparas}

℣. The Holy Ghost shall come upon thee.

℟. And the power of the Highest shall overshadow thee.\\

\antiphon{Ben.}{He shall sit upon the throne of David, {\dag} and of his kingdom, for ever.}\par

\fussy
\end{paracol}

\begin{rubric}
	II Evensong as in I Evensong.
\end{rubric}


%%From Bute's Roman Breviary translation (https://archive.org/details/theromanbreviary01unknuoft/page/665/mode/1up):
%O Antiphons will always take precedence.
%\antiphon{Mag.}{The Holy Ghost shall come upon thee, O Mary, * fear not; thou shalt bear in thy womb the Son of God. Alleluia.}\par

%\antiphon{Mag.}{O maiden of maidens, how shall this be, since neither before nor henceforth hath there been, nor shall be such another? Daughters of Jerusalem, why look ye curiously upon me? What ye see is a mystery of God.}\par

\begin{rubric}
	The Mass propers are from the Common of the Blessed Virgin Mary (p. \pageref{CommonBVM}), except for the Collect as followeth.
\end{rubric}

\collect
\lett{O}{God,} who wast pleased that thy Word should take flesh of the womb of the Blessed Virgin Mary at the message of an Angel: grant to thy humble servants; that we who believe her to be truly the Mother of God may be aided by her intercession in thy sight. Through the same.


\subby{Vigil of St. Thomas}
\fancyhead[RO,LE]{\textit{Thomas Vigil}}
\fancyhead[RE,LO]{20 December}
\begin{inhead}
    {Vigil\\
20 December}
\end{inhead}

\begin{rubric}
	If today be Sunday, in the Ember Saturday Mass, Commemoration is made of the anticipated Vigil of St. Thomas and the last Gospel of the Vigil is read at the end, the \nth{3} Collect is of St. Mary.
\end{rubric}

\begin{rubric}
	The Mass propers are from the Vigil of Apostles (p. \pageref{CommonVigilApostles}), with Commemoration of the Feria of Advent and the \nth{3} Collect of St. Mary. But if an Ember Day occur, Commemoration is made of the Vigil in the Mass of the Feria.
\end{rubric}

\supplement{21 December}{St. Thomas}{}

\begin{rubric}
	The Office Hymn and Versicle are from the Common of Apostles (p. \pageref{CommonApostles}), with the following Antiphon.
\end{rubric}

\properantiphon{Mag. \& Ben.}{Because thou hast seen me, {\dag} Thomas, thou hast believed: blessed are they that have not seen, and yet have believed, alleluia.}


\subby{St. Paul the First Hermit}
\fancyhead[RO,LE]{\textit{Paul Hermit}}
\fancyhead[RE,LO]{10 January}
\begin{inhead}
    {Memorial\\
10 January}
\end{inhead}

\begin{rubric}
	The propers are from the Common of Abbots (p. \pageref{CommonAbbots}), except that which followeth.
\end{rubric}

\introit
\lett{T}{he} just shall flourish like a palm-tree: and shall spread abroad like a cedar in Libanus: planted in the house of the Lord: in the courts of the house of our God. \textit{Ps.} It is a good thing to give thanks unto the Lord: and to sing praises unto thy name, O Most Highest.

\collect
\lett{O}{God,} who makest us glad with the yearly solemnity of blessed Paul, thy Confessor: mercifully grant; that, as we now celebrate his birthday, so we may follow the example of his life. Through.

\gradall{The just shall flourish like a palm-tree: and shall spread abroad like a cedar in Libanus in the house of the Lord. ℣. To tell of thy loving-kindness early in the morning, and of thy truth in the night-season.}{Alleluia, alleluia. ℣. The just shall grow as the lily: and flourish for ever before the Lord. Alleluia.}

\readingcitation{Gospel}{Matthew 11:25}
\lett{A}{t that time:} Jesus answered and said: I thank thee, O Father, Lord of heaven and earth, because thou hast hid these things from the wise and prudent, and hast revealed them unto babes. Even so, Father: for so it seemed good in thy sight. All things are delivered unto me of my Father: and no man knoweth the Son, but the Father; neither knoweth any man the Father, save the Son, and he to whomsoever the Son will reveal him. Come unto me, all ye that labour and are heavy laden, and I will give you rest. Take my yoke upon you, and learn of me; for I am meek and lowly in heart: and ye shall find rest unto your souls. For my yoke is easy, and my burden is light.

\offertory{The just shall rejoice in thy strength, O Lord: exceeding glad shall he be of thy salvation: thou hast given him his heart's desire.}

\secret
\lett{G}{rant,} we beseech thee, O Lord, that we who, trusting in this our sacrifice of praise, do offer it before thee to the honour of thy Saints: may by the same be delivered from all evils both in this life and that which is to come. Through.

\communion{The just shall rejoice in the Lord, and put his trust in him: and all they that are true of heart shall be glad.}

\postcommunion
\lett{O}{Lord,} our God, who hast refreshed us with heavenly meat and drink, we humbly beseech thee: that we may be defended by the prayers of him in whose memory we have received the same. Through.


\subby{St. Hyginus of Rome}
\fancyhead[RO,LE]{\textit{Hyginus}}
\fancyhead[RE,LO]{11 January}
\begin{inhead}
    {Memorial\\
11 January}
\end{inhead}

\begin{rubric}
	The propers are from the First Common of a Martyr not a Bishop (p. \pageref{CommonMartyrNotBishopI}).
\end{rubric}


\subby{St. Benedict}
\fancyhead[RO,LE]{\textit{Benedict}}
\fancyhead[RE,LO]{12 January}
\begin{inhead}
    {Memorial\\
12 January}
\end{inhead}

\begin{rubric}
	The propers are from the Common of Abbots (p. \pageref{CommonAbbots}), except for that which followeth.
\end{rubric}

\collect
\lett{O}{God,} by whose gift the blessed Abbot Benedict left all things that he might be made perfect: grant unto all those who have entered upon the path of evangelical perfection, that they may neither look back nor linger in the way; but hastening to thee without stumbling, may lay hold upon eternal life. Through.

\secret
\lett{W}{e} beseech thee, O Lord, that thy holy Abbot Benedict, may may intercede for us: that this sacrifice which we offer and present upon thy holy altar may be profitable unto us for our salvation. Through.

\postcommunion
\lett{L}{et} thy sacrament, O Lord, which we have now received and the prayers of the blessed Abbot Benedict, effectually defend us: that we may both imitate the example of his conversion, and receive the succour of his intercession. Through.


\subby{St. Hilary}
\fancyhead[RO,LE]{\textit{Hilary}}
\fancyhead[RE,LO]{14 January}
\begin{inhead}
    {Double\\
14 January}
\end{inhead}


\begin{rubric}
	The propers are from the Common of Doctors (p. \pageref{CommonDoctors}).\par
	\textsc{Note,} Commemoration is made of St. Felix, with the Prayers from the Mass below.
\end{rubric}


\subby{St. Felix}
\fancyhead[RO,LE]{\textit{Felix}}
\fancyhead[RE,LO]{14 January}
\begin{inhead}
    {Memorial\\
14 January}
\end{inhead}

\begin{rubric}
	The propers are from the Second Common of a Martyr not a Bishop (p. \pageref{CommonMartyrNotBishopII}), except that which followeth.\par
\end{rubric}

\collect
\lett{G}{rant,} we beseech thee, almighty God, that the examples of thy Saints may provoke us to a better life: that as we celebrate their festival so we may imitate their actions. Through.

\secret
\lett{W}{e} beseech thee, O Lord, mercifully to accept this our sacrifice, which we offer unto thee, pleading the merits of blessed Felix, thy Martyr: that the same may avail for our perpetual succour. Through.

\postcommunion
\lett{O}{Lord,} who hast fulfilled us with saving mysteries: we beseech thee that we may be aided by the prayers of blessed Felix thy Martyr, whose festival we celebrate. Through.


%https://archive.org/details/missale-monasticum-propers/mode/2up?view=theater

\subby{St. Maurus}
\fancyhead[RO,LE]{\textit{Maurus}}
\fancyhead[RE,LO]{15 January}
\begin{inhead}
    {Greater Double\\
15 January}
\end{inhead}

\begin{inhead}
	I Evensong
\end{inhead}

\begin{multicols}{2}
\begin{hangparas}{1.25em}{1}
Defender, leader true, thine own companions deem'd

Thy splendour half divine, since all were less esteem'd:

For thy most worthy deeds, Maurus, accept the lays

Wherewith we celebrate thy praise.\\

Born of a noble stock, great honour was his due,

But palaces he spurn'd, and from the world withdrew;

Delights he trampl'd down, estates and robes unpric'd,

To undergo the yoke of Christ.\\

The holy Abbot's grace, before his eyes display'd,

By deeds of equal worth he eagerly portray'd;

The pattern of the life monastic shone in truth

From every action of the youth.\\

Sternly, with sackcloth rough, self-mastery he wrought,

And by the curb of law, unbroken silence sought;

The ever-watchful nights in fervent prayer he spent;

Whole days of fasting underwent\\

Right speedily he flew to do the father's hest,

Dry-shod, the waters deep with fearless feet he press'd;

And safely he return'd with Placidus, set free

Like Peter walking on the sea.\\

To thee, O Trinity, high praise and honour be,

Whose countenance desir'd the heaven-dwellers see:

Grant that the Holy Rule may be our pathway plain

The prize of Maurus to attain.  Amen.\\
\end{hangparas}

℣. The Lord loved him and adorned him.

℟. He clothed him with a robe of glory.
\end{multicols}

\antiphon{Mag.}{O most blessed of men! {\dag} who, rejecting this world, bore the yoke of Holy Rule from tender years so lovingly; and being made obedient even unto death, he denied himself, that he might wholly cling to Christ his Master, alleluia.}

\begin{rubric}
	In Mattins, the Office Hymn \& Versicle are from the First Common of a Confessor not a Bishop (p. \pageref{CommonConfessorNotBishopI}).\par
	The Antiphon is from I Evensong.
\end{rubric}

\begin{rubric}
	In II Evensong, the Office Hymn is of I Evensong, with the Versicle \& Antiphon as followeth.
\end{rubric}

℣. The Lord guided the righteous in right paths.

℟. And shewed him the kingdom of God.\\

\antiphon{Mag.}{To-day holy Maurus, {\dag} lying upon a goat-skin, died happily before the altar; to-day the first-begotten disciple of blessed Benedict, through the guiding of the Holy Rule, came up to Christ, rising untroubled, accompanied by choirs of Angels; today the obedient man, telling his victories, was worthy to be crowned by the Lord, alleluia.}

\introit
\lett{T}{hy} way is in the sea, and thy paths in the great waters: and thy footsteps are not known. Thou leddest thy people like sheep. \textit{Ps.} The waters saw thee, O God, the waters saw thee, and were afraid: the depths also were troubled.

\collect
\lett{O}{God,} who for a pattern of obedience didst cause blessed Maurus to walk dry-shod upon the waters: grant that we may both follow perfectly the example of his virtues, and also be worthy to share in his reward. Through.

\readingcitation{Epistle}{Ecclesiasticus 51:13}
%RV:
\lett{W}{hen} I was yet young, Or ever I went abroad, I sought wisdom openly in my prayer. Before the temple I asked for her, And I will seek her out even to the end. From her flower as from the ripening grape my heart delighted in her: My foot trod in uprightness, From my youth I tracked her out. I bowed down mine ear a little, and received her, And found for myself much instruction. I profited in her: Unto him that giveth me wisdom I will give glory. For I purposed to practice her, And I was zealous for that which is good; And I shall never be put to shame. My soul hath wrestled in her, And in my doing I was exact: I spread forth my hands to the heaven above, And bewailed my ignorances of her. I set my soul aright unto her, And in pureness I found her. I gat me a heart joined with her from the beginning: Therefore shall I not be forsaken. My inward part also was troubled to seek her: Therefore have I gotten a good possession. The Lord gave me a tongue for my reward; And I will praise him therewith.
\par
Draw near unto me, ye unlearned, And lodge in the house of instruction. Say, wherefore are ye lacking in these things, And your souls are very thirsty? I opened my mouth, and spake, Get her for yourselves without money. Put your neck under the yoke, And let your soul receive instruction: She is hard at hand to find. Behold with your eyes, How that I laboured but a little, And found for myself much rest. Get you instruction with a great sum of silver, And gain much gold by her. May your soul rejoice in his mercy, And may ye not be put to shame in praising him. Work your work before the time cometh, And in his time he will give you your reward. 

\gradall{I will bring thy seed from the east, and gather thee from the west. ℣. When thou passest through the waters, I will be with thee; and through the rivers, they shall not overflow thee.}{Alleluia, alleluia. ℣. An obedient man shall speak of victory; Wealth and riches shall be in his house. Alleluia.}

\begin{rubric}
	In Septuagesimatide \& Lent, the Alleluia is omitted and the Tract is said instead.
\end{rubric}

\tract{Blessed is the man that feareth the Lord: he hath great delight in his commandments. ℣. His seed shall be mighty upon earth: the generation of the faithful shall be blessed. ℣. Riches and plenteousness shall be in his house: and his righteousness endureth for ever.}

\readingcitation{Gospel}{Matthew 14:28}
\lett{A}{t that time:} Peter answered Jesus and said, Lord, if it be thou, bid me come unto thee on the water. And he said, Come. And when Peter was come down out of the ship, he walked on the water, to go to Jesus. But when he saw the wind boisterous, he was afraid; and beginning to sink, he cried, saying, Lord, save me. And immediately Jesus stretched forth his hand, and caught him, and said unto him, O thou of little faith, wherefore didst thou doubt? And when they were come into the ship, the wind ceased. Then they that were in the ship came and worshipped him, saying, Of a truth thou art the Son of God.

\offertory{And the places that have been desolate for ages shall be built in thee: thou shalt raise up the foundations of generation and generation: turning the paths into rest. And I will feed thee with the inheritance of thy Father.}

%CHECK TRANSLATION:
\secret
\lett{M}{ay} the sacrifices we offer ascend unto thee, O Lord, as an odour of sweetness; and may the intercession of blessed Maurus, Abbot, intervene for us, that thy propitious power may descend upon us. Through.

\communion{I have chosen you, and ordained you, that ye should go and bring forth fruit, and that your fruit should remain: that whatsoever ye shall ask of the Father in my name, he may give it you.}

%CHECK TRANSLATION:
\postcommunion
\lett{W}{e} implore thy mercy, O Lord our God, that having received the pledges of our salvation and giving thanks for thy help; thou may accept our offerings unto our salvation; sending upon us thy grace to support us; that the celestial blessing, which thou hast brought us by the patronage of blessed Maurus, Abbot, may be perfected by continuing in imitation of him. Through.


\subby{St. Marcellus}
\fancyhead[RO,LE]{\textit{Marcellus}}
\fancyhead[RE,LO]{16 January}
\begin{inhead}
    {Memorial\\
16 January}
\end{inhead}

\begin{rubric}
	The propers are from the Second Common of a Martyr Bishop (p. \pageref{CommonMartyrBishopII}), except for that which followeth.
\end{rubric}

\introit
\lett{T}{he} Lord hath established a covenant of peace with him, and made him a prince: that he should have the dignity of the priesthood for ever. \textit{Ps.} Lord, remember David: and all his trouble.

\collect
\lett{O}{Lord,} we beseech thee favourably to hear the prayers of thy people: that, as we do rejoice in the passion of blessed Marcellus thy Martyr and Bishop, so we may be succoured by his merits. Through.

\gradall{I have found David my servant, with my holy oil have I anointed him: my hand shall hold him fast, and my arm shall strengthen him. ℣. The enemy shall not be able to do him violence, the son of wickedness shall not hurt him.}{Alleluia, alleluia. ℣. Thou art a priest for ever, after the order of Melchisedech. Alleluia.}

\offertory{My truth and my mercy shall be with him: and in my name shall his horn be exalted.}

\secret
\lett{R}{eceive,} O Lord, we beseech thee, the gifts which we duly offer: and by the pleading of the merits of blessed Marcellus thy Martyr and Bishop, grant that they may avail to set forward our salvation. Through.

\communion{Lord, thou deliveredst unto me five talents: behold, I have gained beside them five talents more. Well done, thou good and faithful servant, thou hast been faithful over a few things, I will make thee ruler over many things, enter thou into the joy of thy lord.}

\postcommunion
\lett{O}{Lord,} who hast satisfied thy family with sacred gifts: we beseech thee; that we may at all times be comforted by the intercession of him whose festival we celebrate. Through.


\subby{St. Anthony}
\fancyhead[RO,LE]{\textit{Anthony}}
\fancyhead[RE,LO]{17 January}
\begin{inhead}
    {Double\\
17 January}
\end{inhead}

\begin{rubric}
	The propers are from the Common of Abbots (p. \pageref{CommonAbbots}), except for the Gospel which is from the First Common of a Confessor not a Bishop (p. \pageref{CommonConfessorNotBishopI}).
\end{rubric}


\subby{Chair of St. Peter at Rome}
\fancyhead[RO,LE]{\textit{Peter at Rome}}
\fancyhead[RE,LO]{18 January}
\begin{inhead}
    {Greater Double\\
18 January}
\end{inhead}

\begin{rubric}
	The Daily Office propers are from the Chair of St. Peter at Antioch on 22 February (p. \pageref{CathedraAntioch}).
\end{rubric}

\introit
\lett{T}{he} To Lord hath established a covenant of peace with him, and made him a prince: that he should have the dignity of the priesthood for ever. \textit{Ps.} Lord, remember David: and all his trouble.

\collect
\lett{O}{God,} who didst bestow upon thy blessed Apostle Peter the keys of the kingdom of heaven, and didst appoint unto him the high priesthood of binding and loosing: vouchsafe; that by the help of his intercession we may be delivered from the bonds of our iniquities. Who livest and reignest.

\lett{O}{God,} who by the preaching of the blessed Apostle Paul didst teach the multitude of the Gentiles: grant to us, we beseech thee; that we who celebrate his commemoration may know him to be our advocate with thee. (Through.)

\begin{rubric}
    Commemoration is made of St. Prisca of Rome (p. \pageref{PriscaCollect}).
\end{rubric}

\readingcitation{Epistle}{1 Peter 1:1}
\lett{P}{eter} Peter, an apostle of Jesus Christ, to the strangers scattered throughout Pontus, Galatia, Cappadocia, Asia, and Bithynia, elect according to the foreknowledge of God the Father, through sanctification of the Spirit, unto obedience and sprinkling of the blood of Jesus Christ: Grace unto you, and peace, be multiplied. Blessed be the God and Father of our Lord Jesus Christ, which according to his abundant mercy hath begotten us again unto a lively hope by the resurrection of Jesus Christ from the dead, to an inheritance incorruptible, and undefiled, and that fadeth not away, reserved in heaven for you, who are kept by the power of God through faith unto salvation ready to be revealed in the last time. Wherein ye greatly rejoice, though now for a season, if need be, ye are in heaviness through manifold temptations: that the trial of your faith, being much more precious than of gold that perisheth, though it be tried with fire, might be found unto praise and honour and glory at the appearing of Jesus Christ our Lord.

\gradall{Let them exalt him in the congregation of the people: and praise him in the seat of the elders. ℣. O that men would praise the Lord for his goodness, and declare the wonders that he doeth for the children of men.}{Alleluia, alleluia. ℣. Thou art Peter, and upon this rock I will build my Church. Alleluia.}

\begin{rubric}
	In Septuagesimatide \& Lent, the Alleluia is replaced with the following.
\end{rubric}

\tract{Thou art Peter, and upon this rock I will build my Church. ℣. And the gates of hell shall not prevail against it: and I will give unto thee the keys of the kingdom of heaven. ℣. Whatsoever thou shalt bind on earth shall be bound in heaven. ℣. And whatsoever thou shalt loose on earth shall be loosed in heaven.}

\begin{rubric}
	In Eastertide, the Alleluia is replaced with the following.
\end{rubric}

\alleluia{Alleluia, alleluia. ℣. O that men would praise the Lord for his goodness, and declare the wonders that he doeth for the children of men. Alleluia. ℣. Thou art Peter, and upon this rock I will build my Church. Alleluia.}

\readingcitation{Gospel}{Matthew 16:13}
\lett{A}{t that time:} When Jesus came into the coasts of C{\ae}sarea Philippi, he asked his disciples, saying, Whom do men say that I the Son of man am? And they said, Some say that thou art John the Baptist: some, Elias; and others, Jeremias, or one of the prophets. He saith unto them, But whom say ye that I am? And Simon Peter answered and said, Thou art the Christ, the Son of the living God. And Jesus answered and said unto him, Blessed art thou, Simon Bar-jona: for flesh and blood hath not revealed it unto thee, but my Father which is in heaven. And I say also unto thee, That thou art Peter, and upon this rock I will build my church; and the gates of hell shall not prevail against it. And I will give unto thee the keys of the kingdom of heaven: and whatsoever thou shalt bind on earth shall be bound in heaven: and whatsoever thou shalt loose on earth shall be loosed in heaven.

\offertory{Thou art Peter, and upon this rock I will build my Church: and the gates of hell shall not prevail against it: and I will give unto thee the keys of the kingdom of heaven.}

\secret
\lett{W}{e} beseech thee, O Lord, that the intercession of blessed Peter the Apostle may commend unto thee the prayers and sacrifices of thy Church: that those things which we celebrate for his glory may avail for our pardon. Through.
\needspace{4\baselineskip}
\lett{S}{anctify,} O Lord, through the prayers of thine Apostle Paul, the gifts of thy people: that those things, which by thine institution are pleasing unto thee, may be made more pleasing by his prayer and advocacy. (Through.)

\begin{rubric}
    Commemoration is made of St. Prisca of Rome (p. \pageref{PriscaSecret}).
\end{rubric}

\communion{Thou art Peter, and upon this rock I will build my Church.}

\postcommunion
\lett{M}{ay} the gift, O Lord, which we have offered, make us to rejoice: that as we proclaim thy wonders in thine Apostle Peter; so through him we may receive the abundance of thy loving-kindness. Through.
\needspace{4\baselineskip}
\lett{O}{Lord,} who hast sanctified us with this saving mystery: we beseech thee; that he whom thou hast given to be our advocate and guide may never fail in prayer for us. (Through.)

\begin{rubric}
    Commemoration is made of St. Prisca of Rome (p. \pageref{PriscaPostcommunion}).
\end{rubric}


\subby{St. Prisca of Rome}
\fancyhead[RO,LE]{\textit{Prisca}}
\fancyhead[RE,LO]{18 January}
\begin{inhead}
    {Memorial\\
18 January}
\end{inhead}

\begin{rubric}
	The propers are from the Second Common of a Virgin Martyr (p. \pageref{CommonVirginMartyrII}), except for that which followeth.
\end{rubric}

\collect\label{PriscaCollect}
\lett{G}{rant,} we beseech thee, almighty God: that we who celebrate the heavenly birthday of blessed Prisca, thy Virgin and Martyr; may both rejoice in her yearly solemnity, and profit by the example of so great a faith. Through.

\secret\label{PriscaSecret}
\lett{W}{e} beseech thee, O Lord, that this sacrifice which we offer in remembrance of the birthday of thy Saints may both loose us from the bonds of our iniquity, and obtain for us the gifts of thy mercy. Through.

\postcommunion\label{PriscaPostcommunion}
\lett{O}{Lord,} who hast fulfilled us with saving mysteries: we beseech thee that we may be aided by the prayers of her whose festival we celebrate. Through.


\subby{Sts. Marius, Martha, Audifax, \& Abachum}
\fancyhead[RO,LE]{\textit{Marius \& Company}}
\fancyhead[RE,LO]{19 January}
\begin{inhead}
    {Memorial\\
19 January}
\end{inhead}

\begin{rubric}
	The Daily Office propers are from the First Common of Many Martyrs (p. \pageref{CommonMartyrsI}).
\end{rubric}

\introit
\lett{L}{et} the righteous be glad and rejoice before God: let them also be merry and joyful. \textit{Ps.} Let God arise, and let his enemies be scattered: let them also that hate him flee before him.

\collect
\lett{G}{raciously} hear thy people, O Lord, who call upon thee with the advocacy of thy Saints: and grant us both to rejoice in peace in this temporal life; and to find succour unto life everlasting. Through.

\begin{rubric}
    Commemoration is made of St. Mark of Ephesus (p. \pageref{EphesianCollect}).
\end{rubric}

%MANUAL ADJUSTMENT:
\begin{rubric}
    Commemoration is made of St. Mary in Epiphanytide (p. BCP 622).
\end{rubric}

\begin{rubric}
	The Epistle is from the Third Common of Many Martyrs (p. \pageref{CommonMartyrsIII}).
\end{rubric}

\gradall{The souls of the just are in the hand of God: and there shall no torment of malice touch them. ℣. In the sight of the unwise they seemed to die: but they are in peace.}{Alleluia, alleluia. ℣. Our God is wonderful in his Saints. Alleluia.}

\begin{rubric}
	In Septuagesimatide \& Lent, the Alleluia is replaced with the Tract from the Third Common of Many Martyrs (p. \pageref{CommonMartyrsIII}).
\end{rubric}

\begin{rubric}
	The Gospel is from the first additional Gospel of the Third Common of Many Martyrs (p. \pageref{Matthew243}).
\end{rubric}

\offertory{Our soul is escaped, even as a bird out of the snare of the fowler: the snare is broken, and we are delivered.}

\secret
\lett{R}{egard,} O Lord, the prayers and oblations of thy faithful people: that they may be acceptable unto thee for the festival of thy Saints, and bestow on us the succour of thy mercy. Through.

\begin{rubric}
    Commemoration is made of St. Mark of Ephesus (p. \pageref{EphesianSecret}).
\end{rubric}

%MANUAL ADJUSTMENT:
\begin{rubric}
    Commemoration is made of St. Mary in Epiphanytide (p. BCP 623).
\end{rubric}

\communion{I say unto you, my friends: Be not afraid of them that persecute you.}

\postcommunion
\lett{G}{rant,} we beseech thee, that the intercession of thy Saints may make us acceptable unto thee: that those things which we perform in this temporal celebration we may receive unto eternal salvation. Through.

\begin{rubric}
    Commemoration is made of St. Mark of Ephesus (p. \pageref{EphesianPostcommunion}).
\end{rubric}

%MANUAL ADJUSTMENT:
\begin{rubric}
    Commemoration is made of St. Mary in Epiphanytide (p. BCP 623).
\end{rubric}

%PROVISIONAL:
\subby{St. Mark of Ephesus}
\fancyhead[RO,LE]{\textit{Mark of Ephesus}}
\fancyhead[RE,LO]{19 January}
\begin{inhead}
    {Memorial\\
19 January}
\end{inhead}

\begin{rubric}
	The propers are from the Second Common of a Confessor Bishop (p. \pageref{CommonConfessorBishopII}), except for that which followeth.
\end{rubric}

%Through thee shone uncreated light

%Of the life given to the least,

%And like a giant of great might

%Thou didst advance from the East.\\

%Thou didst illumine the whole world

%With the rays of thy true words,

%O divine Mark, Bishop of Christ,

%Pray for us for that same light.\\

%Thine el'quent lips and honey'd tongue

%Became the mouthpiece of his grace;

%Thy sacred tongue was shown to be

%A pen of wisdom, grace, and peace.\\

%All glory, Lord, to thee we pay

%For the Ephesian which thou gav'st,

%All glory, as is ever meet,

%To Father and to Paraclete. Amen.

\antiphon{Mag.}{O blessed Mark {\dag} clothed with invincible armour, thou didst cast down rebellious pride.}

\antiphon{Ben.}{Thou didst serve {\dag} as the instrument of the Paraclete, and shone forth as the champion of Orthodoxy.}

\antiphon{Mag.}{Rejoice, Mark, {\dag} the boast of the Orthodox and joy to all true Catholics!}

\collect\label{EphesianCollect}
\lett{O}{Lord} Jesu Christ, the only Shepherd and Bishop of our souls; we beseech thee to keep us in thy truth, preserve us by thy grace, and govern us according to thy loving-kindness. That as we imitate the good example of blessed Mark thy Confessor and Bishop, we may grow in love for thy Word and service of thy Church. Who with.

\secret\label{EphesianSecret}
\lett{R}{eceive,} O most holy Father, these gifts now offered unto thee in memory of Saint Mark. That as he was a bulwark and defender of the Catholic Faith, so we may, by these gifts, be given the same firm love of thee. Through.

\postcommunion\label{EphesianPostcommunion}
\lett{A}{bide} in us, O divine Comforter, who have here received thine holy Sacraments. That just as thy faithful servant Mark did contend for the unity of the Body of Christ, so we may be cleansed and knit together by the same. Who with the Father, from whom alone thou dost proceed, and the Son, who sendeth thy power upon us, liveth and reigneth, God, world without end. Amen.


\subby{Sts. Fabian \& Sebastian, Martyrs}
\fancyhead[RO,LE]{\textit{Fabian \& Sebastian}}
\fancyhead[RE,LO]{20 January}
\begin{inhead}
    {Double\\
20 January}
\end{inhead}

\begin{rubric}
	The Daily Office propers are from the First Common of Many Martyrs (p. \pageref{CommonMartyrsI}).
\end{rubric}

\introit
\lett{L}{et} the sorrowful sighing of the prisoners, O Lord, come before thee, reward thou our neighbours seven-fold into their bosom: avenge thou the blood of thy Saints that is shed. \textit{Ps.} O God, the heathen are come into thine inheritance: thy holy temple have they defiled: and made Jerusalem an heap of stones.

\collect
\lett{A}{lmighty} God, mercifully look upon our infirmities: that whereas we are oppressed by the burden of our sins, the glorious intercession of thy blessed Martyrs Fabian and Sebastian may be our succour and defence. Through.

\begin{rubric}
	The Epistle is the fifth optional Epistle of the Third Common of Many Martyrs (p. \pageref{Hebrews1133}).
\end{rubric}

\gradall{God is glorious in his holy ones: fearful in praises, doing wonders. ℣. Thy right hand, O Lord, is become glorious in power: thy right hand hath dashed in pieces the enemy.}{Alleluia, alleluia. ℣. Thy Saints give thanks unto thee, O Lord: they shew the glory of thy kingdom. Alleluia.}

\begin{rubric}
	In Septuagesimatide \& Lent, the Alleluia is replaced with the following.
\end{rubric}

\tract{They that sow in tears, shall reap in joy. ℣. He that now goeth on his way weeping, and beareth forth good seed. ℣. Shall doubtless come again with joy, and bring his sheaves with him.}

\readingcitation{Gospel}{Luke 6:17}
\lett{A}{t that time:} Jesus came down from the mountain, and stood in the plain, and the company of his disciples, and a great multitude of people out of all Jud{\ae}a and Jerusalem, and from the sea coast of Tyre and Sidon, which came to hear him, and to be healed of their diseases; and they that were vexed with unclean spirits: and they were healed. And the whole multitude sought to touch him: for there went virtue out of him, and healed them all. And he lifted up his eyes on his disciples, and said, Blessed be ye poor: for yours is the kingdom of God. Blessed are ye that hunger now: for ye shall be filled. Blessed are ye that weep now: for ye shall laugh. Blessed are ye, when men shall hate you, and when they shall separate you from their company, and shall reproach you, and cast out your name as evil, for the Son of man’s sake. Rejoice ye in that day, and leap for joy: for, behold, your reward is great in heaven.

\offertory{Be glad, O ye righteous, and rejoice in the Lord: and be joyful, all ye that are true of heart.}

\secret
\lett{W}{e} beseech thee, O Lord, mercifully to accept this our sacrifice which we offer unto thee, pleading the merits of thy blessed Martyrs Fabian and Sebastian: that the same may avail for our perpetual succour. Through.

\communion{A multitude of sick folk, and they that were vexed with unclean spirits came to him: for there went virtue out of him, and healed them all.}

\postcommunion
\lett{W}{e} beseech thee, O Lord our God, that like as we whom thou hast refreshed by the partaking of thy sacred gift do offer unto thee our worship: so, by the intercession of thy holy Martyrs Fabian and Sebastian, we may perceive the benefit of the same. Through.


\subby{St. Agnes of Rome}
\fancyhead[RO,LE]{\textit{Agnes}}
\fancyhead[RE,LO]{21 January}
\begin{inhead}
    {Greater Double\\
21 January}
\end{inhead}

\begin{rubric}
	The Daily Office propers are from the Common of a Virgin (p. \pageref{CommonVirginOnlyI}), except for that which followeth.
\end{rubric}

\antiphon{Mag.}{Blessed Agnes, {\dag} in the midst of the flames, stretched out her hands and prayed: I call on thee, O Father transcendent, august, and dread; for by thy holy Son's protection I have escaped the threats of an impious tyrant, and passed unscathed through the foulness of fleshly pollution: and behold, I come to thee, whom I have loved, whom I have sought, whom I have alway desired.}

\antiphon{Ben.}{Lo, that which I desired, {\dag} now I see; that for which I hoped, I now possess; I am united in heaven unto him, whom on earth I loved with a perfect devotion.}\\

℣. Full of grace are thy lips.

℟. Because God hath blessed thee for ever.

\antiphon{Mag.}{While the blessed Agnes {\dag} was standing in the midst of the flames, she stretched out her hands, and prayed unto God: Almighty Lord, worthy of all adoration, fear, and worship: I bless thy holy Name, and I glorify thee for ever and ever.}

\introit
\lett{T}{he} ungodly wait for me to destroy me: O Lord, I will consider thy testimonies: I see that all things come to an end: but thy commandment is exceeding broad. \textit{Ps.} Blessed are those that are undefiled in the way: and walk in the law of the Lord.

\collect
\lett{A}{lmighty} and everlasting God, who dost choose the weak things of the world to confound those things that are strong: mercifully grant; that we who keep the feast of blessed Agnes thy Virgin and Martyr may feel the succour of her intercession in thy sight. Through.

\readingcitation{Epistle}{Ecclesiasticus 51:1}
%RV:
\lett{I}{will} give thanks unto thee, O Lord, O King, And will praise thee, O God my Saviour: I do give thanks unto thy name: For thou wast my protector and helper, And didst deliver my body out of destruction, And out of the snare of a slanderous tongue, From lips that forge lies, And wast my helper before them that stood by; And didst deliver me, according to the abundance of thy mercy, and greatness of thy name, From the gnashings of teeth ready to devour, Out of the hand of such as sought my life, Out of the manifold afflictions which I had; From the choking of a fire on every side, And out of the midst of fire which I kindled not; Out of the depth of the belly of the grave, And from an unclean tongue, And from lying words, The slander of an unrighteous tongue unto the king. My soul drew near even unto death, And my life was near to the grave beneath. They compassed me on every side, And there was none to help me. I was looking for the succour of men, And it was not. And I remembered thy mercy, O Lord, And thy working which hath been from everlasting, How thou deliverest them that wait for thee, And savest them out of the hand of the enemies, O Lord our God.

%\lett{I}{will} give thanks unto thee, O Lord, O King, And will praise thee, O God my Saviour: I do give thanks unto thy name: For thou wast my protector and helper, And didst deliver my body out of destruction, And out of the snare of a slanderous tongue, From lips that forge lies, And wast my helper before them that stood by;  And didst deliver me, according to the abundance of thy mercy, and greatness of thy name, From the gnashings of teeth ready to devour, Out of the hand of such as sought my life, Out of the manifold afflictions which I had; From the choking of a fire on every side, And out of the midst of fire which I kindled not; Out of the depth of the belly of the grave, And from an unclean tongue, And from lying words, The slander of an unrighteous tongue unto the king. My soul drew near even unto death, And my life was near to the grave beneath. And I remembered thy mercy, O Lord, And thy working which hath been from everlasting, How thou deliverest them that wait for thee, And savest them out of the hand of the enemies, O Lord our God.

\gradall{Full of grace are thy lips: because God hath blessed thee for ever. ℣. Because of the word of truth, of meekness, and righteousness: and thy right hand shall teach thee terrible things.}{Alleluia, alleluia. ℣. The five wise virgins took oil in their vessels with their lamps: and at midnight there was a cry made: Behold, the bridegroom cometh: go ye out to meet Christ the Lord. Alleluia.}

\begin{rubric}
{In Septuagesimatide or Lent, replacing the Alleluia:}
\end{rubric}\par\noindent
\tract{Come, Spouse of Christ, receive the crown which the Lord hath prepared for thee for ever: for love of whom thou didst shed thy blood. ℣. Thou hast loved righteousness and hated iniquity: wherefore God, even thy God, hath anointed thee with the oil of gladness above thy fellows. ℣. In thy comeliness and in thy beauty go forth, ride prosperously, and reign.}

\readingcitation{Gospel}{Matthew 25:1}
\lett{A}{t that time:} Jesus spake this parable unto his disciples: The kingdom of heaven shall be likened unto ten virgins, which took their lamps, and went forth to meet the bridegroom. And five of them were wise, and five were foolish. They that were foolish took their lamps, and took no oil with them:  But the wise took oil in their vessels with their lamps. While the bridegroom tarried, they all slumbered and slept. And at midnight there was a cry made, Behold, the bridegroom cometh; go ye out to meet him. Then all those virgins arose, and trimmed their lamps. And the foolish said unto the wise, Give us of your oil; for our lamps are gone out. But the wise answered, saying, Not so; lest there be not enough for us and you: but go ye rather to them that sell, and buy for yourselves. And while they went to buy, the bridegroom came; and they that were ready went in with him to the marriage: and the door was shut. Afterward came also the other virgins, saying, Lord, Lord, open to us. But he answered and said, Verily I say unto you, I know you not. Watch therefore, for ye know neither the day nor the hour wherein the Son of man cometh.

\offertory{The Virgins that be her fellows shall be brought unto the King: they that bear her company shall be brought unto thee with joy and gladness: and shall enter into the palace of the Lord the King.}

\secret
\lett{O}{Lord,} mercifully regard the sacrifices which we offer unto thee: and at the intercession of blessed Agnes, thy Virgin and Martyr, absolve us from the bonds of our sins. Through.

\communion{The five wise virgins took oil in their vessels with their lamps: and at midnight there was a cry made: Behold, the bridegroom cometh: go ye out to meet Christ the Lord.}

\postcommunion
\lett{O}{Lord,} our God, who hast refreshed us with heavenly meat and drink, we humbly beseech thee: that we may be defended by the prayers of her in whose memory we have received the same. Through.

\subby{St. Vincent}
\fancyhead[RO,LE]{\textit{Vincent}}
\fancyhead[RE,LO]{22 January}
\begin{inhead}
    {Memorial\\
22 January}
\end{inhead}

\begin{rubric}
	The propers are from the Second Common of a Martyr not a Bishop (p. \pageref{CommonMartyrNotBishopII}), except for that which followeth.
\end{rubric}

\collect
\lett{A}{ssist} us, O Lord, in our supplications: that we who perceive the guilt of our iniquities, may be delivered through the intercession of thy blessed Martyr Vincent. Through.

%CHECK TRANSLATION:
\secret
\lett{W}{e} beseech thee, O Lord, to accept our service and prayers, cleanse us by thy heavenly mysteries, and graciously hear us. Through.

%CHECK TRANSLATION:
\postcommunion
\lett{W}{e} beseech thee, almighty God; that, by the intercession of thy blessed Martyr Vincent, we who have received heavenly food may be strengthened against all adversity by the same. Through.


\subby{St. Emerentiana}
\fancyhead[RO,LE]{\textit{Emerentiana}}
\fancyhead[RE,LO]{23 January}
\begin{inhead}
    {Memorial\\
23 January}
\end{inhead}

\begin{rubric}
	The propers are from the Second Common of a Virgin and Martyr (p. \pageref{CommonVirginMartyrII}).
\end{rubric}


\subby{St. Timothy}
\fancyhead[RO,LE]{\textit{Timothy}}
\fancyhead[RE,LO]{24 January}
\begin{inhead}
    {Double\\
24 January}
\end{inhead}

\begin{rubric}
	The propers come from the First Common of a Bishop and Martyr (p. \pageref{CommonMartyrBishopI}), except for the Epistle, as below.
\end{rubric}

\readingcitation{Epistle}{1 Timothy 6:11}
\lett{D}{early beloved:} Follow after righteousness, godliness, faith, love, patience, meekness. Fight the good fight of faith, lay hold on eternal life, whereunto thou art also called, and hast professed a good profession before many witnesses. I give thee charge in the sight of God, who quickeneth all things, and before Christ Jesus, who before Pontius Pilate witnessed a good confession; That thou keep this commandment without spot, unrebukeable, until the appearing of our Lord Jesus Christ: Which in his times he shall shew, who is the blessed and only Potentate, the King of kings, and Lord of lords; Who only hath immortality, dwelling in the light which no man can approach unto; whom no man hath seen, nor can see: to whom be honour and power everlasting. Amen.

\supplement{25 January}{Conversion}{of St. Paul the Apostle}

\begin{inhead}
	I Evensong
\end{inhead}

O by thy doctrine,

Paul, thou safe illustrious,

Guide us in virtue, raise our spirits heavenwards;

Till perfect knowledge stream on us abundantly,

And that which only is in part be done away.\\

Glory eternal to the blessed Trinity,

With laud and honour, virtue and supremacy;

Trinal yet Onely, reigning in his majesty

Both now and ever, through the ages infinite. Amen.\\

℣. Thou art a chosen vessel, holy Apostle Paul.

℟. A preacher of the truth throughout all the world.\\

\begin{rubric}
	In Mattins, the Office Hymn is from the Common of the Apostles (p. \pageref{CommonApostles}), with the following Versicle.
\end{rubric}

℣. Thou art a chosen vessel, holy Apostle Paul.

℟. A preacher of the truth throughout all the world.

\begin{rubric}
	II Evensong as in I Evensong.
\end{rubric}


\subby{St. Polycarp of Smyrna}
\fancyhead[RO,LE]{\textit{Polycarp}}
\fancyhead[RE,LO]{26 January}
\begin{inhead}
    {Memorial\\
26 January}
\end{inhead}

\begin{rubric}
	The Daily Office propers are from the First Common of a Martyr Bishop (p. \pageref{CommonMartyrBishopI}), except for the Collect.
\end{rubric}

\introit
\lett{O}{ye} priests of the Lord, bless ye the Lord: O ye holy and humble men of heart, bless ye the Lord. \textit{Ps.} O all ye works of the Lord, bless ye the Lord: praise him, and magnify him for ever.

\collect
\lett{O}{God,} who makest us glad with the yearly solemnity of blessed Polycarp, thy Martyr and Bishop: mercifully grant; that, as we now cebrate his birthday, so we may likewise rejoice in his protection. Through.

\readingcitation{Epistle}{1 John 3:10}
\lett{D}{early beloved:} Whosoever doeth not righteousness is not of God, neither he that loveth not his brother. For this is the message that ye heard from the beginning, that we should love one another. Not as Cain, who was of that wicked one, and slew his brother. And wherefore slew he him? Because his own works were evil, and his brother's righteous. Marvel not, my brethren, if the world hate you. We know that we have passed from death unto life, because we love the brethren. He that loveth not his brother abideth in death. Whosoever hateth his brother is a murderer: and ye know that no murderer hath eternal life abiding in him. Hereby perceive we the love of God, because he laid down his life for us: and we ought to lay down our lives for the brethren.

\gradall{Thou hast crowned him with glory and worship. ℣. Thou hast made him to have dominion of the works of thy hands, O Lord.}{Alleluia, alleluia. ℣. This is a priest whom the Lord hath crowned. Alleluia.}

\begin{rubric}
{In Septuagesimatide or Lent, replacing the Alleluia:}
\end{rubric}\par\noindent
\tract{Blessed is the man that feareth the Lord: he hath great delight in his commandments. ℣. His seed shall be mighty upon earth: the generation of the faithful shall be blessed. ℣. Riches and plenteousness shall be in his house: and his righteousness endureth for ever.}

\readingcitation{Gospel}{Matthew 10:26}
\lett{A}{t that time:} Jesus said to his disciples: There is nothing covered, that shall not be revealed; and hid, that shall not be known. What I tell you in darkness, that speak ye in light: and what ye hear in the ear, that preach ye upon the housetops. And fear not them which kill the body, but are not able to kill the soul: but rather fear him which is able to destroy both soul and body in hell. Are not two sparrows sold for a farthing? and one of them shall not fall on the ground without your Father. But the very hairs of your head are all numbered. Fear ye not therefore, ye are of more value than many sparrows. Whosoever therefore shall confess me before men, him will I confess also before my Father which is in heaven.

\offertory{I have found David my servant, with my holy oil have I anointed him: my hand shall hold him fast, and my arm shall strengthen him.}

\secret
\lett{S}{anctify,} O Lord, the gifts which we dedicate to thee: that at the intercession of blessed Polycarp, thy Martyr and Bishop, they may obtain for us thy gracious favour. Through.

\communion{Thou hast set, O Lord, a crown of pure gold upon his head.}

\postcommunion
\lett{W}{e} beseech thee, O Lord our God, that like as we, whom thou hast refreshed by the partaking of thy sacred gift, do offer unto thee our worship: so, by the intercession of blessed Polycarp, thy Martyr and Bishop, we may perceive the benefit of the same. Through.


\subby{St. John Chrysostom}
\fancyhead[RO,LE]{\textit{Chrysostom}}
\fancyhead[RE,LO]{27 January}
\begin{inhead}
    {Double\\
27 January}
\end{inhead}

\begin{rubric}
	The propers are from the Common of Doctors (p. \pageref{CommonDoctors}), except for that which followeth.
\end{rubric}

\collect
\lett{M}{ultiply,} we beseech thee, O Lord, thy Church with thy heavenly grace: even as thou didst vouchsafe to enlighten her with the glorious merits and doctrine of blessed John Chrysostom, thy Confessor and Bishop. Through.

\gradall{Behold a great priest, who in his days pleased God. ℣. There was none found like unto him, who kept the law of the Most High.}{Alleluia, alleluia. ℣. Blessed is the man that endureth temptation: for when he is tried, he shall receive the crown of life, alleluia.}

\begin{rubric}
{In Septuagesimatide or Lent, replacing the Alleluia:}
\end{rubric}\par\noindent
\tract{Blessed is the man that feareth the Lord: he hath great delight in his commandments. ℣. His seed shall be mighty upon earth: the generation of the faithful shall be blessed. ℣. Riches and plenteousness shall be in his house: and his righteousness remaineth for ever.}

\secret
\lett{M}{ay} the devout prayers of Saint John Chrysostom thy Bishop and Doctor, never fail to succour us, O Lord: that they may render our oblations acceptable in thy sight; and may ever obtain for us thy merciful pardon. Through.

\postcommunion
\lett{W}{e} beseech thee, O Lord, that blessed John Chrysostom, thy Bishop and illustrious Doctor; may stand before thee as our advocate: that these thy sacrifices may avail for our salvation. Through.


\subby{St. Cyril of Alexandria}
\fancyhead[RO,LE]{\textit{Cyril}}
\fancyhead[RE,LO]{28 January}
\begin{inhead}
    {Double\\
28 January}
\end{inhead}

\begin{rubric}
	The propers are from the Common of Doctors (p. \pageref{CommonDoctors}), except for that which followeth.
\end{rubric}

\collect
\lett{O}{God,} who didst make blessed Cyril, thy Confessor and Bishop, an invincible defender of the divine Motherhood of the most blessed Virgin Mary: grant, by his intercession; that we, who believe her to be indeed the Mother of God, may be saved through her maternal protection. Through the same.

\begin{rubric}
    Commemoration is made of St. Agnes (p. \pageref{AgnesCollectII}).
\end{rubric}

\secret
\lett{A}{lmighty} God, graciously look upon our gifts: and at the intercession of blessed Cyril vouchsafe; that we may be found meet worthily to receive in our hearts thine only-begotten Son, Jesus Christ our Lord, co-eternal with thee in thy glory. Who liveth and reigneth with thee.

\begin{rubric}
    Commemoration is made of St. Agnes (p. \pageref{AgnesSecretII}).
\end{rubric}

\postcommunion
\lett{O}{Lord,} who hast refreshed us with divine mysteries, we humbly beseech thee: that, being aided by the example and merits of the blessed Bishop Cyril, we may be enabled worthily to serve the most holy Mother of thine only-begotten Son. Who liveth and reigneth with thee.

\begin{rubric}
    Commemoration is made of St. Agnes (p. \pageref{AgnesPostcommunionII}).
\end{rubric}


\subby{The Second Feast of St. Agnes}
\fancyhead[RO,LE]{\textit{Agnes}}
\fancyhead[RE,LO]{28 January}
\begin{inhead}
    {Memorial\\
28 January}
\end{inhead}

\begin{rubric}
	The propers are from the First Common of a Virgin (p. \pageref{CommonVirginOnlyI}), except for that which followeth.
\end{rubric}

\introit
\lett{A}{ll} the rich among the people shall make their supplication before thee: the Virgins that be her fellows shall be brought unto the King: they that bear her company shall be brought unto thee with joy and gladness. \textit{Ps.} My heart is inditing of a good matter: I speak of the things which I have made unto the King.

\collect\label{AgnesCollectII}
\lett{O}{God,} who makest us glad with the yearly solemnity of blessed Agnes, thy Virgin and Martyr: grant, we beseech thee; that as we venerate her in our service, so we may follow the example of her godly conversation. Through.

\readingcitation{Gospel}{Matthew 13:44}
\lett{A}{t that time:} Jesus spake this parable unto his disciples: The kingdom of heaven is like unto treasure hid in a field; the which when a man hath found, he hideth, and for joy thereof goeth and selleth all that he hath, and buyeth that field. Again, the kingdom of heaven is like unto a merchant man, seeking goodly pearls: Who, when he had found one pearl of great price, went and sold all that he had, and bought it. Again, the kingdom of heaven is like unto a net, that was cast into the sea, and gathered of every kind: Which, when it was full, they drew to shore, and sat down, and gathered the good into vessels, but cast the bad away. So shall it be at the end of the world: the angels shall come forth, and sever the wicked from among the just, And shall cast them into the furnace of fire: there shall be wailing and gnashing of teeth. Jesus saith unto them, Have ye understood all these things? They say unto him, Yea, Lord. Then said he unto them, Therefore every scribe which is instructed unto the kingdom of heaven is like unto a man that is an householder, which bringeth forth out of his treasure things new and old.

\offertory{Full of grace are thy lips, because God hath blessed thee for ever and ever.}

\secret\label{AgnesSecretII}
\lett{L}{et} thy plenteous benediction, we beseech thee, O Lord, come down upon these sacrifices: that it may mercifully work out our sanctification, and make us to rejoice in the solemnity of thy Martyrs. Through.

\communion{The kingdom of heaven is like unto a merchant man, seeking goodly pearls: who when he had found one pearl of great price, gave all that he had, and bought it.}

\postcommunion\label{AgnesPostcommunionII}
\lett{G}{rant,} we beseech thee, O Lord, that the sacrament which we have received in our observance of this yearly festival: may bestow on us thy healing; both in this temporal life and unto life eternal. Through.


\subby{St. Martina}
\fancyhead[RO,LE]{\textit{Martina}}
\fancyhead[RE,LO]{30 January}
\begin{inhead}
    {Memorial\\
30 January}
\end{inhead}

\begin{rubric}
	The propers are from the First Common of a Virgin and Martyr (p.\pageref{CommonVirginMartyrI}).
\end{rubric}


\subby{St. Ignatius of Antioch}
\fancyhead[RO,LE]{\textit{Ignatius}}
\fancyhead[RE,LO]{1 February}
\begin{inhead}
    {Double\\
1 February}
\end{inhead}

\begin{rubric}
	The Daily Office propers are from the First Common of Martyr Bishops (p. \pageref{CommonMartyrBishopI}).
\end{rubric}

\introit
\lett{B}{ut} God forbid that I should glory, save in the Cross of our Lord Jesus Christ: by whom the world is crucified unto me, and I unto the world. \textit{Ps.} Lord, remember David: and all his trouble.

\collect
\lett{A}{lmighty} God, mercifully look upon our infirmities; that whereas we are oppressed by the burden of our sins, the glorious intercession of blessed Ignatius thy Martyr and Bishop may be our succour and defence. Through.

\begin{rubric}
    Commemoration is made of St. Bridget of Ireland, from the First Common of a Virgin (p. \pageref{CommonVirginOnlyI}).
\end{rubric}

\readingcitation{Epistle}{Romans 8:35}
\lett{B}{rethren:} Who shall separate us from the love of Christ? shall tribulation, or distress, or persecution, or famine, or nakedness, or peril, or sword? As it is written, For thy sake we are killed all the day long; we are accounted as sheep for the slaughter. Nay, in all these things we are more than conquerors through him that loved us. For I am persuaded, that neither death, nor life, nor angels, nor principalities, nor powers, nor things present, nor things to come, Nor height, nor depth, nor any other creature, shall be able to separate us from the love of God, which is in Christ Jesus our Lord.

\gradall{Behold a great priest, who in his days pleased God. ℣. There was none found like unto him, who kept the law of the Most High.}{Alleluia, alleluia. ℣. I am crucifed with Christ: I live, yet not I, but Christ liveth in me. Alleluia.}

\begin{rubric}
{In Septuagesimatide or Lent, replacing the Alleluia:}
\end{rubric}\par\noindent
\tract{Thou hast given him his heart's desire: and hast not denied him the request of his lips. ℣. For thou hast prevented him with the blessings of goodness. ℣. Thou hast set a crown of pure gold upon his head.}

\readingcitation{Gospel}{John 12:24}
\lett{A}{t that time:} Jesus said unto his disciples: Verily, verily, I say unto you, Except a corn of wheat fall into the ground and die, it abideth alone: but if it die, it bringeth forth much fruit. He that loveth his life shall lose it; and he that hateth his life in this world shall keep it unto life eternal. If any man serve me, let him follow me; and where I am, there shall also my servant be: if any man serve me, him will my Father honour.

\offertory{Thou hast crowned him with glory and worship: and hast made him to have dominion of the works of thy hands, O Lord.}

\secret
\lett{W}{e} beseech thee, O Lord, mercifully to accept this our sacrifice which we offer unto thee, pleading the merits of blessed Ignatius thy Martyr and Bishop: that the same may avail for our perpetual succour. Through.

\begin{rubric}
    Commemoration is made of St. Bridget of Ireland, from the First Common of a Virgin (p. \pageref{CommonVirginOnlyI}).
\end{rubric}

\communion{I am the wheat of Christ: let me be ground by the teeth of beasts, that I may be found pure bread.}

\postcommunion
\lett{W}{e} beseech thee, O Lord, our God, that as we whom thou hast refreshed by the partaking of thy sacred gift offer unto thee our worship: so by the intercession of blessed Ignatius thy Martyr and Bishop, we may perceive the benefit of the same. Through.

\begin{rubric}
    Commemoration is made of St. Bridget of Ireland, from the First Common of a Virgin (p. \pageref{CommonVirginOnlyI}).
\end{rubric}


%%CHECK MISSAL:
\subby{St. Bridget of Ireland}
\fancyhead[RO,LE]{\textit{Bridget}}
\fancyhead[RE,LO]{1 February}
\begin{inhead}
    {Memorial\\
1 February}
\end{inhead}

\begin{rubric}
	The propers are from the First Common of a Virgin (p. \pageref{CommonVirginOnlyI}).
\end{rubric}

%From Common(?):
%\collect\label{BridgetCollect}
%\lett{H}{ear} us, O God our Saviour: and grant that as we rejoice in the festival of blessed Bridget, thy Virgin, so we may be brought to perfect holiness by the love of piety and devotion. Through.


\supplement{2 February}{Purification}{of the Blessed Virgin Mary}

\begin{rubric}
	The Office Hymn and Versicle are from the Common of the Blessed Virgin Mary (p. \pageref{CommonBVM}), except for the Evensong Versicles as followeth.
\end{rubric}

℣. It was revealed unto Simeon by the Holy Ghost. 

℟. That he should not see death before he had seen the Lord's Christ.


\subby{St. Blaise}
\fancyhead[RO,LE]{\textit{Blaise}}
\fancyhead[RE,LO]{3 February}
\begin{inhead}
    {Memorial\\
3 February}
\end{inhead}

\begin{rubric}
	The propers are from the Second Common of a Martyr Bishop (p. \pageref{CommonMartyrBishopII}).
\end{rubric}


\subby{St. Joseph of Aleppo}
\fancyhead[RO,LE]{\textit{Joseph}}
\fancyhead[RE,LO]{4 February}
\begin{inhead}
    {Memorial\\
4 February}
\end{inhead}

\begin{rubric}
	The propers are from the First Common of a Martyr not a Bishop (p. \pageref{CommonMartyrNotBishopI}).
\end{rubric}

\begin{rubric}
	Commemoration is made of the New Martyrs of Russia, from the Prayers in the First Common of Many Martyrs (p. \pageref{CommonMartyrsI}).
\end{rubric}

\subby{The New Martyrs of Russia}
\fancyhead[RO,LE]{\textit{New Russian Martyrs}}
\fancyhead[RE,LO]{4 February}
\begin{inhead}
    {Memorial\\
4 February}
\end{inhead}

\begin{rubric}
	The propers are from the First Common of Many Martyrs (p. \pageref{CommonMartyrsI}).
\end{rubric}


\subby{St. Agatha}
\fancyhead[RO,LE]{\textit{Agatha}}
\fancyhead[RE,LO]{5 February}
\begin{inhead}
    {Greater Double\\
5 February}
\end{inhead}

\begin{rubric}
	The Office Hymns and Versicles are from the First Common of a Virgin Martyr (p. \pageref{CommonVirginMartyrI}), except for the II Evensong Versicle as below.
\end{rubric}

\antiphon{Mag.}{The blessed Agatha, {\dag} standing in the midst of the prison, with outstretched hands entreated the Lord: O Lord Jesus Christ, my gracious Master, I give thanks unto thee, who hast enabled me to overcome the torments of the executioner: bid me now, O Lord, joyfully to enter into thine unfading glory.}

\antiphon{Ben.}{The multitude {\dag} of the heathen, fleeing to the tomb of the virgin, took thence her veil to defend them from the fire: that the Lord might shew himself a deliverer from the burning, for the merits of Agatha his blessed Martyr.}

℣. Full of grace are thy lips.

℟. Because God hath blessed thee for ever.

\antiphon{Mag.}{The blessed Agatha, {\dag} standing in the midst of the prison, with outstretched hands entreated the Lord: O Lord Jesus Christ, my gracious Master, I give thanks unto thee, who hast enabled me to overcome the torments of the executioner: bid me now, O Lord, joyfully to enter into thine unfading glory.}

\introit
\lett{R}{ejoice} we all in the Lord, keeping feast day in honour of blessed Agatha, the Virgin and Martyr: in whose passion the Angels rejoice, and glorify the Son of God. \textit{Ps.} My heart is inditing of a good matter: I speak of the things which I have made unto the King.

\collect
\lett{O}{God,} who among the manifold works of thy power hast bestowed even upon the weakness of women the victory of martyrdom: mercifully grant; that we, who celebrate the birthday of blessed Agatha, thy Virgin and Martyr, may by her example be drawn nearer unto thee. Through.

\readingcitation{Epistle}{1 Corinthians 1:26}
\lett{B}{rethren:} Ye see your calling: how that not many wise men after the flesh, not many mighty, not many noble, are called: But God hath chosen the foolish things of the world to confound the wise; and God hath chosen the weak things of the world to confound the things which are mighty; And base things of the world, and things which are despised, hath God chosen, yea, and things which are not, to bring to nought things that are: That no flesh should glory in his presence. But of him are ye in Christ Jesus, who of God is made unto us wisdom, and righteousness, and sanctification, and redemption: That, according as it is written, He that glorieth, let him glory in the Lord.

\gradall{God shall help her with his countenance: God is in the midst of her, therefore shall she not be removed. ℣. The rivers of the flood thereof shall make glad the city of God: the holy place of the tabernacle of the Most Highest.}{Alleluia, alleluia. ℣. I will speak of thy testimonies even before kings: and will not be ashamed. Alleluia.}

\begin{rubric}
{In Septuagesimatide or Lent, replacing the Alleluia:}
\end{rubric}\par\noindent
\tract{They that sow in tears, shall reap in joy. ℣. They that now go on their way weeping, and bear forth good seed. ℣. Shall doubtless come again with joy, and bring their sheaves with them.}

\readingcitation{Gospel}{Matthew 19:3}
\lett{A}{t that time:} The Pharisees came unto Jesus, tempting him, and saying unto him: Is it lawful for a man to put away his wife for every cause? And he answered and said unto them, Have ye not read, that he which made them at the beginning made them male and female, And said, For this cause shall a man leave father and mother, and shall cleave to his wife: and they twain shall be one flesh? Wherefore they are no more twain, but one flesh. What therefore God hath joined together, let not man put asunder. They say unto him, Why did Moses then command to give a writing of divorcement, and to put her away? He saith unto them, Moses because of the hardness of your hearts suffered you to put away your wives: but from the beginning it was not so. And I say unto you, Whosoever shall put away his wife, except it be for fornication, and shall marry another, committeth adultery: and whoso marrieth her which is put away doth commit adultery. His disciples say unto him, If the case of the man be so with his wife, it is not good to marry. But he said unto them, All men cannot receive this saying, save they to whom it is given. For there are some eunuchs, which were so born from their mother's womb: and there are some eunuchs, which were made eunuchs of men: and there be eunuchs, which have made themselves eunuchs for the kingdom of heaven's sake. He that is able to receive it, let him receive it.

\offertory{The Virgins that be her fellows shall be brought unto the King: they that bear her company shall be brought unto thee.}

\secret
\lett{R}{eceive,} O Lord, the gifts which we offer on the solemnity of blessed Agatha, thy Virgin and Martyr, through whose advocacy we trust to be delivered. Through.

\communion{He who deigned to heal my every wound, and to restore my breast unto my body, on him do I call, the living God.}

\postcommunion
\lett{M}{ay} the mysteries which we have received be for our succour, O Lord: and at the intercession of blessed Agatha, thy Virgin and Martyr, cause us to rejoice in thy continual protection. Through.


\subby{St. Dorothea}
\fancyhead[RO,LE]{\textit{Dorothea}}
\fancyhead[RE,LO]{6 February}
\begin{inhead}
    {Memorial\\
6 February}
\end{inhead}

\begin{rubric}
	The propers are from the Second Common of a Virgin Martyr (p. \pageref{CommonVirginMartyrII}), except for that which followeth.
\end{rubric}

\collect
\lett{W}{e} beseech thee, O Lord, that as blessed Dorothea, thy Virgin and Martyr, was ever found pleasing unto thee, both by the merit of her chastity, and by her confession of thy power: so she may implore for us thy pardon. Through.

\secret
\lett{G}{raciously} receive, O Lord, through the merits of bressed Dorotnea thy Virgin and Martyr, the sacrifices which we offer unto thee and grant that they may avail for our continual help. Through.

\postcommunion
\lett{O}{Lord} our God, who hast fulfilled us with the bounty of thy heavenly gift: we beseech thee that, at the intercession of blessed Dorothea thy Virgin and Martyr, we may ever live by the partaking of the same. Through.


\subby{St. Romuald}
\fancyhead[RO,LE]{\textit{Romuald}}
\fancyhead[RE,LO]{7 February}
\begin{inhead}
    {Double\\
7 February}
\end{inhead}

\begin{rubric}
	The propers are from the Common of Abbots (p. \pageref{CommonAbbots}).
\end{rubric}


\subby{St. Apollonia of Alexandria}
\fancyhead[RO,LE]{\textit{Apollonia}}
\fancyhead[RE,LO]{9 February}
\begin{inhead}
    {Memorial\\
9 February}
\end{inhead}

\begin{rubric}
	The propers are from the First Common of a Virgin Martyr (p. \pageref{CommonVirginMartyrI}), except for that which followeth.
\end{rubric}

\collect
\lett{O}{God} who among the manifold works of thy power hast bestowed even upon the weakness of women the victory of martyrdom mercifully grant; that we, who celebrate the birthday of blessed Apollonia thy Virgin and Martyr, may by her example be drawn nearer unto thee. Through.

\secret
\lett{R}{eceive,} O Lord, the gifts which we offer on the solemnity of blessed Apollonia thy Virgin and Martyr: through whose advocacy we trust to be delivered. Through.

\postcommunion
\lett{M}{ay} the mysteries which we have received be for our succour, O Lord: and at the intercession of blessed Apollonia thy Virgin and Martyr, cause us to rejoice in thy continual protection. Through.


\subby{St. Scholastica}
\fancyhead[RO,LE]{\textit{Scholastica}}
\fancyhead[RE,LO]{10 February}
\begin{inhead}
    {Greater Double\\
10 February}
\end{inhead}

\begin{rubric}
	The propers are from the First Common of a Virgin Martyr (p. \pageref{CommonVirginMartyrI}), except for that which followeth.
\end{rubric}

\begin{paracol}{2}[]
\sloppy
\begin{inhead}
	I Evensong
\end{inhead}
\begin{hangparas}{1.25em}{1}

Blessed bride of Christ the Bridegroom, Dove of virgins, holy maid,

Starry hosts proclaim thy praises, Praise the crown to virtue paid,

While our joyful hearts and voices Join the anthem unafraid.\\

Wise wert thou to spurn the riches And the crowns that heaven miss,

Wise to choose thy brother's teachings, Change thy way of life for his:

In the savour of thine ointments Didst thou enter heav'nly bliss.\\

Oh, the might of love unbounded! Victory of all most meet!

At thy tears the heavens open'd, Pour'd their floods about thy feet,

While his soul to thine made answer All night long in converse sweet.\\

Crash of tempest, roll of thunder, Lightning flash from pole to pole,

But the storm of love is stronger, Brighter flashes in the soul,

While the peace of Christ the Bridegroom Holds thee still and keeps thee whole.\\

Now a gleaming cloud in heaven, Now a sun with golden rays,

Never darkling, never fading, Shining still through all our days,

Fill the hearts of all the faithful With the joy of heav'nly lays.\\

Glory to the Father sing we, Glory to the only Son;

To the Paraclete in glory, Equal tribute be begun,

At whose pleasure made and govern'd All the ages' course is run. Amen.\\
\end{hangparas}

℣. Who is this that flieth as a cloud? 

℟. And as a dove to her windows?

\antiphon{Mag.}{Let all the multitude {\dag} of the faithful exult in the glory of the gracious virgin Scholastica: and chiefly let the company of virgins be joyful, celebrating her Solemnity; for she besought the Lord, pouring forth her tears, and of him received greater power, because her love was greater.}

\switchcolumn

\begin{inhead}
	Mattins
\end{inhead}

\begin{hangparas}{1.25em}{1}

The shades of night are fled away, The long desired day is born;

The changing wheel of time has brought Scholastica her wedding morn.\\

The winter's weariness is past, Tempests no longer doth she see;

The fields of heaven bloom with stars, The blossoms of eternity.\\

He calls her forth, the Source of love, He gives her wings that she may rise,

And to the kisses of his mouth, A shining dove she swiftly flies.\\

O royal child, how fair thy course! O maid, how beautiful thy way!

Thy brother marks thy upward flight, Then turns to God his thanks to pay.\\

Lock'd in the Bridegroom's close embrace, She takes the crown to virtue ow'd,

Sunk in a glorious stream of bliss, Inebriated with her God.\\

Thou lily of the valley, Christ, To thee our homage meet we pay:

To Father and to Paraclete, While endless ages roll away. Amen.\\
\end{hangparas}

℣. Behold, thou art fair, my love.  

℟. Behold, thou art fair; thou hast dove's eyes.

\antiphon{Ben.}{O how illustrious {\dag} are the merits of blessed Scholastica! O how great the power of her tears! through which the renowned virgin, out of sunny clearness, drew down from the air a mighty flood of rain.}
\end{paracol}

\begin{rubric}
	In II Evensong, the Office Hymn \& Versicle are of I Evensong, with the following Antiphon.
\end{rubric}

\antiphon{Mag.}{To-day {\dag} the holy virgin Scholastica, in the likeness of a dove, went forth with all gladness to the heavenly places: to-day she was found worthy to enjoy for ever the bliss of celestial life beside her brother.}

\collect
\lett{O}{God,} who didst reveal in a vision the soul of blessed Scholastica thy Virgin entering heaven in the likeness of a dove, that thou mightest shew the way of the undefiled: grant us by the aid of her merits and prayers so innocently to live, that we may worthily attain unto joys eternal. Through.


\subby{Pope St. Gregory II}
\fancyhead[RO,LE]{\textit{Gregory II}}
\fancyhead[RE,LO]{11 February}
\begin{inhead}
    {Memorial\\
11 February}
\end{inhead}

\begin{rubric}
	The propers are from the First Common of a Confessor Bishop (p. \pageref{CommonConfessorBishopI}).
\end{rubric}


\subby{St. Valentine}
\fancyhead[RO,LE]{\textit{Valentine}}
\fancyhead[RE,LO]{14 February}
\begin{inhead}
    {Memorial\\
14 February}
\end{inhead}

\begin{rubric}
	The propers are from the First Common of a Martyr not a Bishop (p. \pageref{CommonMartyrNotBishopI}), except for that which followeth.
\end{rubric}

\collect
\lett{G}{rant,} we beseech thee, almighty God: that we who observe the birthday of blessed Valentine thy Martyr may by his intercession be delivered from all evils that beset us. Through.

\secret
\lett{R}{eceive,} O Lord, we beseech thee, the gifts which we duly offer: and by the pleading of the merits of blessed Valentine thy Martyr, grant; that they may avail to set forward our salvation. Through.

\postcommunion
\lett{M}{ay} this heavenly mystery, O Lord, renew us in soul and body: that as we offer unto thee our worship, so by the intercession of blessed Valentine thy Martyr, we may perceive the benefit of the same. Through.


\subby{Sts. Faustinus and Jovita}
\fancyhead[RO,LE]{\textit{Faustinus \& Jovita}}
\fancyhead[RE,LO]{15 February}
\begin{inhead}
    {Memorial\\
15 February}
\end{inhead}

\begin{rubric}
	The propers are from the Third Common of Many Martyrs (p. \pageref{CommonMartyrsIII}), except for that which followeth.
\end{rubric}

\collect
\lett{O}{God,} who makest us glad with the yearly solemnity of thy holy Martyrs Faustinus and Jovita: mercifully grant; that as we rejoice in their merits, so we may be enkindled by their example. Through.

\secret
\lett{A}{ssist} us mercifully, O Lord, in these our supplications which we make before thee in remembrance of thy Saints: that we who trust not in our own rightneousness may be succoured by the merits of them that have found favour in thy sight. Through.

\postcommunion
\lett{O}{Lord,} who hast fulfilled us with saving mysteries, we beseech thee: that we may be aided by the prayers of those whose festival we celebrate. Through.


\subby{St. Simeon of Jerusalem}
\fancyhead[RO,LE]{\textit{Simeon}}
\fancyhead[RE,LO]{18 February}
\begin{inhead}
    {Memorial\\
18 February}
\end{inhead}

\begin{rubric}
	The propers are from the First Common of a Martyr Bishop (p. \pageref{CommonMartyrBishopI}).
\end{rubric}


\supplement{22 February}{Chair}{of St. Peter at Antioch}\label{CathedraAntioch}

\begin{paracol}{2}[]
\sloppy
\begin{inhead}
	I Evensong
\end{inhead}
\begin{hangparas}{1.25em}{1}
    Peter, whatever thou shalt bind on earth,
    
    The same is bound above the starry sky;
    
    What here thy delegated power doth loose,
    
    Is loosed in heaven's supremest Court on high:
    
    To judgement shalt thou come, when the world’s end is nigh.\\
    
    Praise to the Father, through all ages be;
    
    Praise to the consubstantial sovereign Son,
    
    And Holy Ghost, One glorious Trinity;
    
    To whom all majesty and might belong;
    
    So sing we now, and such be our eternal song. Amen.\\
    \end{hangparas}
    
    ℣. Thou art Peter.

	℟. And upon this rock I will build my Church.

\switchcolumn

\begin{inhead}
	Mattins
\end{inhead}
\begin{hangparas}{1.25em}{1}
Peter, good shepherd, may thy ceaseless orisons,

For us prevailing, break the bands of wickedness:

For thou of old time didst receive authority

The gates to open, or to close, of Paradise.\\

Glory eternal to the blessed Trinity,

With laud and honour, virtue and supremacy,

Trinal yet Onely, reigning in his majesty

Both now and ever, through the ages infinite. Amen.\\
\end{hangparas}

    ℣. Thou art Peter.

	℟. And upon this rock I will build my Church.

\fussy
\end{paracol}
\begin{rubric}
	II Evensong as in I Evensong, except for the Antiphon.
\end{rubric}


\subby{Vigil of St. Matthias}
\fancyhead[RO,LE]{\textit{Matthias Vigil}}
\fancyhead[RE,LO]{23 February}
\begin{inhead}
    {Vigil\\
23 February}
\end{inhead}

\begin{rubric}
	The propers are from the Common of Vigils of the Apostles (p. \pageref{CommonVigilApostles}).
\end{rubric}


\supplement{24 February}{St. Matthias}{}

\begin{rubric}
	The Office Hymn and Versicle are from the Common of Apostles (p. \pageref{CommonApostles}).
\end{rubric}

%TRANSLATE:
\subby{St. Walburga of Heidenheim}
\fancyhead[RO,LE]{\textit{Walburga}}
\fancyhead[RE,LO]{25 February}
\begin{inhead}
    {Memorial\\
25 February (26 February in a Leap Year)}
\end{inhead}

\begin{rubric}
	The propers are from the Common of a Virgin (p. \pageref{CommonVirginOnlyI}), except for that which followeth.
\end{rubric}

%From Latin Diurnale Monasticum \& Missale Romanum (https://www.google.com/books/edition/Missale_Romanum/p2HPgCPUBPsC?hl=en&gbpv=1&dq=%22Walb%C3%BArgae%22&pg=RA3-PA126&printsec=frontcover):
\collect
\lett{D}{eus,} qui inter innúmera gráti{\ae} tu{\ae} dona, étiam in sexu frágili tua operáris magnália: {\dag} concéde propítius; ut beát{\ae} Walbúrg{\ae}, Vírginis tu{\ae}, apud misericórdiam tuam patrocínia sentiámus, * cujus non solum castitátis illustrámur exémplis, verum étiam miraculórum glória jucundámur. Per Dóminum.

\begin{rubric}
	In Lent, Commemoration of the Feria.
\end{rubric}

\secret
\lett{S}{acrifícium} oblátum, quǽsumus Dómine, beát{\ae} Walbúrg{\ae} Vírginis tu{\ae} orátio Majestáti tu{\ae} reddat accéptum: pro cujus veneránda commemoratióne tibi devóto offértur obséquio. Per Dóminum.

\begin{rubric}
	In Lent, Commemoration of the Feria.
\end{rubric}

\postcommunion
\lett{B}{enedictiónis,} Dómine, grátiam, intercedénte beáta Walbúrga Vírgine tua, cónsequi mereámur: ut, cujus venerándam glóriam pr{\ae}dicámus; ejus in ómnibus nostris necessitátibus auxílium sentiámus. Per Dóminum.

\begin{rubric}
	In Lent, Commemoration \& Last Gospel of the Feria.
\end{rubric}


\subby{Pope St. Alexander of Alexandria}
\fancyhead[RO,LE]{\textit{Alexander}}
\fancyhead[RE,LO]{26 February}
\begin{inhead}
    {Memorial\\
26 February (27 February in a Leap Year)}
\end{inhead}

\begin{rubric}
	The propers are from the First Common of a Confessor Bishop (p. \pageref{CommonConfessorBishopI}).
\end{rubric}
\begin{rubric}
	In Lent, Commemoration \& Last Gospel of the Feria.
\end{rubric}


\subby{St. Raphael of Brooklyn}
\fancyhead[RO,LE]{\textit{Raphael}}
\fancyhead[RE,LO]{27 February}
\begin{inhead}
    {Greater Double\\
27 February (28 February in a Leap Year)}
\end{inhead}

\begin{rubric}
	The propers are from the First Common of a Confessor Bishop (p. \pageref{CommonConfessorBishopI}).
\end{rubric}
\begin{rubric}
	In Lent, Commemoration \& Last Gospel of the Feria.
\end{rubric}


\subby{St. David of Wales}
\fancyhead[RO,LE]{\textit{David}}
\fancyhead[RE,LO]{1 March}
\begin{inhead}
    {Memorial\\
1 March}
\end{inhead}

\begin{rubric}
	The propers are from the First Common of a Confessor Bishop (p. \pageref{CommonConfessorBishopI}), except for that which followeth.
\end{rubric}

\collect
\lett{G}{rant} to us, almighty God: that the loving intercession of blessed David, thy Confessor and Bishop, may protect us; that while we celebrate his festival we may imitate his steadfastness in the defence of the Catholic faith. Through.

\begin{rubric}
	In Lent, Commemoration of the Feria.
\end{rubric}

\secret
\lett{W}{e} beseech thee, O Lord: that we, remembering with gladness the merits of thy Saints, may in all places feel the succour of their intercession. Through.

\begin{rubric}
	In Lent, Commemoration of the Feria.
\end{rubric}

\postcommunion
\lett{G}{rant,} we beseech thee, almighty God: that we, shewing forth our thankfulness for the gifts which we have received, may, at the intercession of blessed David, thy Confessor and Bishop, obtain yet more abundant mercies. Through.

\begin{rubric}
	In Lent, Commemoration \& Last Gospel of the Feria.
\end{rubric}


\subby{St. Chad}
\fancyhead[RO,LE]{\textit{Chad}}
\fancyhead[RE,LO]{2 March}
\begin{inhead}
    {Memorial\\
2 March}
\end{inhead}

\begin{rubric}
	The propers are from the Second Common of a Confessor Bishop (p. \pageref{CommonConfessorBishopII}), except for that which followeth.
\end{rubric}

\collect
\lett{A}{lmighty} and everlasting God, who on this day dost gladden us by the festival of blessed Chad, thy Confessor and Bishop: we humbly beseech thy mercy; that we, who devoutly observe and venerate his festival, may by his loving advocacy obtain the reward of everlasting life. Through.

\begin{rubric}
	In Lent, Commemoration of the Feria.
\end{rubric}

\secret
\lett{W}{e} beseech thee, O Lord, mercifully to have respect unto our supplications: and at the intercession of blessed Chad thy Confessor and Bishop on our behalf, grant that we who minister thy heavenly sacraments may be free from all sin; that through thy purifying grace we may be cleansed by those same mysteries which we serve. Through.

\begin{rubric}
	In Lent, Commemoration of the Feria.
\end{rubric}

\postcommunion
\lett{G}{rant,} we beseech thee, O Lord our God: that, at the intercession of blessed Chad, thy Confessor and Bishop, we who have tasted of holy things, being cleansed by divine mysteries, may attain to the fulness of this heavenly sacrament. Through.

\begin{rubric}
	In Lent, Commemoration \& Last Gospel of the Feria.
\end{rubric}


\subby{Pope St. Lucius I}
\fancyhead[RO,LE]{\textit{Lucius I}}
\fancyhead[RE,LO]{4 March}
\begin{inhead}
    {Memorial\\
4 March}
\end{inhead}

\begin{rubric}
	The propers are from the Second Common of a Martyr Bishop (p. \pageref{CommonMartyrBishopII}), except for that which followeth.
\end{rubric}

\collect
\lett{O}{God,} who makest us glad with the yearly solemnity of blessed Lucius thy Martyr and Bishop: mercifully grant; that, as we now celebrate his birthday, so we may likewise rejoice in his protection. Through.

\begin{rubric}
	In Lent, Commemoration of the Feria.
\end{rubric}

\secret
\lett{W}{e} beseech thee, O Lord, mercifully to accept this our sacrifice, which we offer unto thee, pleading the merits of blessed Lucius, thy Martyr and Bishop: that the same may avail for our perpetual succour. Through.

\begin{rubric}
	In Lent, Commemoration of the Feria.
\end{rubric}

\postcommunion
\lett{W}{e} beseech thee, O Lord our God, that like as we whom thou hast refreshed by the partaking of thy sacred gift do offer unto thee our worship: so, by the intercession of blessed Lucius thy Martyr and Bishop, we may perceive the benefit of the same. Through.

\begin{rubric}
	In Lent, Commemoration \& Last Gospel of the Feria.
\end{rubric}


\subby{Sts. Perpetua and Felicitas}
\fancyhead[RO,LE]{\textit{Perpetua \& Felicitas}}
\fancyhead[RE,LO]{4 March}
\begin{inhead}
    {Memorial\\
4 March}
\end{inhead}

\begin{rubric}
	The propers are from the Common of a Martyr not a Virgin (p. \pageref{CommonMartyrNotVirgin}), except for that which followeth.
\end{rubric}

%Felicity changed to Felicitas
\collect
\lett{G}{rant,} we beseech thee, O Lord our God, that we may at all times so devoutly honour the triumphs of thy holy Martyrs Perpetua and Felicitas: that, although we cannot worthily shew forth their praises, yet we may continually honour them with lowly service. Through.

\begin{rubric}
	In Lent, Commemoration of the Feria.
\end{rubric}

\secret
\lett{O}{Lord,} we beseech thee, look down upon these gifts, which we offer on thine altars on this festival of thy holy Martyrs Perpetua and Felicitas: that as by these blessed mysteries thou hast bestowed glory upon them, so likewise of thy bounty thou wouldest vouchsafe to us thy pardon. Through.

\begin{rubric}
	In Lent, Commemoration of the Feria.
\end{rubric}

\postcommunion
\lett{O}{Lord,} who hast fulfilled us with mystic gifts and joys: grant, we beseech thee; that by the intercession of thy holy Martyrs, Perpetua and Felicitas, we may spiritually attain to those things which we temporally perform. Through.

\begin{rubric}
	In Lent, Commemoration \& Last Gospel of the Feria.
\end{rubric}


\subby{St. Gregory of Nyssa}
\fancyhead[RO,LE]{\textit{Gregory}}
\fancyhead[RE,LO]{9 March}
\begin{inhead}
    {Memorial\\
9 March}
\end{inhead}

\begin{rubric}
	The propers are from the Common of Doctors (p. \pageref{CommonDoctors}).
\end{rubric}
\begin{rubric}
	In Lent, Commemoration \& Last Gospel of the Feria.
\end{rubric}


\subby{The Forty Holy Martyrs}
\fancyhead[RO,LE]{\textit{Forty Martyrs}}
\fancyhead[RE,LO]{10 March}
\begin{inhead}
    {Memorial\\
10 March}
\end{inhead}

\begin{rubric}
	The Daily Office propers are from the First Common of Many Martyrs (p. \pageref{CommonMartyrsI}).
\end{rubric}

\introit
\lett{T}{he} just cry, and the Lord heareth them: and delivereth them out of all their troubles. \textit{Ps.} I will alway give thanks unto the Lord; his praise shall ever be in my mouth.

\collect
\lett{G}{rant} we beseech thee, almighty God: that, like as we have known thy glorious Martyrs to be constant in their confession, so we may perceive their loving intercession for us with thee. Through.

\begin{rubric}
	In Lent, Commemoration of the Feria.
\end{rubric}

\begin{rubric}
	The Epistle is the fifth optional Epistle of the Third Common of Many Martyrs (p. \pageref{Hebrews1133}).
\end{rubric}

\gradtr{Behold, how good and joyful a thing it is, brethren, to dwell together in unity! ℣. It is like the precious ointment upon the head, that ran down unto the beard, even unto Aaron's beard.}{They that sow in tears, shall reap in joy. ℣. They that now go on their way weeping, and bear forth good seed. ℣. Shall doubtless come again with joy, and bring their sheaves with them.}

\begin{rubric}
	The Gospel is from the Second Common of Many Martyrs (p. \pageref{CommonMartyrsII}).
\end{rubric}

\offertory{Be glad, O ye righteous, and rejoice in the Lord: and be joyful, all ye that are true of heart.}

\secret
\lett{R}{egard,} O Lord, the prayers and oblations of thy faithful people: that they may be acceptable unto thee for the festival of thy Saints, and bestow on us the succour of thy mercy. Through.

\begin{rubric}
	In Lent, Commemoration of the Feria.
\end{rubric}

\communion{Whosoever shall do the will of my Father which is in heaven: the same is my brother, and sister, and mother, saith the Lord.}

\postcommunion
\lett{G}{rant,} O Lord, we beseech thee, that the intercession of thy Saints may make us acceptable unto thee: that those things which we perform in this temporal celebration we may receive unto eternal salvation. Through.

\begin{rubric}
	In Lent, Commemoration \& Last Gospel of the Feria.
\end{rubric}

\supplement{12 March}{St. Gregory}{Pope, Confessor, \& Martyr}

\begin{paracol}{2}[]
\sloppy
\begin{inhead}
	I Evensong
\end{inhead}
\begin{hangparas}{1.25em}{1}
Apostle of the English, thou

Art comrade of the Angels now:

O help the faithful, Gregory,

Who heardest then the peoples' cry.\\

Abundant wealth was naught to thee

Who, seeking princely penury,

Didst spurn the glory of the earth

To follow Jesus in his dearth.\\

Like some poor needy shipwreck'd soul,

His messenger besought a dole;

Twin gifts thou gavest; nothing loth

To offer self and silver both.\\

First in his Church from that time forth

Christ set thee, in thy proven worth;

So didst thou rise to Peter's throne,

Whose life was pattern for thine own.\\

O Priest and Leader of the flock,

The Church's glory, light, and rock,

Instructed by thy wise command,

Let not the sheep in peril stand.\\

Praise to the unbegotten One,

And glory to his only Son;

And thine, O Spirit, Breath of God,

Be equal majesty and laud. Amen.\\
\end{hangparas}

    ℣. The Lord loved him and adorned him.

	℟. He clothed him with a robe of glory.

\switchcolumn

\begin{inhead}
	Mattins
\end{inhead}
\begin{hangparas}{1.25em}{1}
Thy lips, O Gregory, distil

The honey that thy heart doth fill;

In spiced sweetness floweth thence

The savour of thine eloquence.\\

The hidden mysteries that lie

In Holy Scripture, to thine eye

Are plain; the Truth himself draws near

To make their subtle secrets clear.\\

At once the Apostolic throne

And majesty become thine own:

O free us from sin's binding chain

That heav'nly mansions we may gain.\\

O Priest and Leader of the flock,

The Church's glory, light, and rock,

Instructed by thy wise command,

Let not the sheep in peril stand.\\

Praise to the unbegotten One,

And glory to his only Son;

And thine, O Spirit, Breath of God,

Be equal majesty and laud. Amen.\\
\end{hangparas}

    ℣. The Lord guided the righteous in right paths.

	℟. And shewed him the kingdom of God.

\fussy
\end{paracol}

\begin{rubric}
	In II Evensong, the Office Hymn is of I Evensong, with the Versicle from Mattins.
\end{rubric}


\subby{St. Patrick}
\fancyhead[RO,LE]{\textit{Patrick}}
\fancyhead[RE,LO]{17 March}
\begin{inhead}
    {Double\\
17 March}
\end{inhead}

\begin{rubric}
	The propers are from the First Common of a Confessor Bishop (p. \pageref{CommonConfessorBishopI}), except for that which followeth.
\end{rubric}

\collect
\lett{O}{God,} who for the preaching of thy glory unto the Gentiles wast pleased to send forth blessed Patrick, thy Confessor and Bishop: grant by his merits and intercession; that we may through thy mercy be enabled to accomplish those things which thou commandest us to do. Through.

\begin{rubric}
    Commemoration is made of St. Joseph of Arimathea (p. \pageref{ArimatheanCollect}).
\end{rubric}

\begin{rubric}
	In Lent, Commemoration \& Last Gospel of the Feria.
\end{rubric}

\secret
\lett{W}{e} beseech thee, O Lord, that we remembering with gladness the merits of thy Saints, may in all places feel the succour of their intercession. Through.

\begin{rubric}
    Commemoration is made of St. Joseph of Arimathea, from the Second Common of a Confessor not a Bishop (p. \pageref{CommonConfessorNotBishopII})
\end{rubric}

\begin{rubric}
	In Lent, Commemoration \& Last Gospel of the Feria.
\end{rubric}

\postcommunion
\lett{G}{rant,} we beseech thee, almighty God: that we, shewing forth our thankfulness for the gifts which we have received, may at the intercession of blessed Patrick, thy Confessor and Bishop, obtain yet more abundant mercies. Through.

\begin{rubric}
    Commemoration is made of St. Joseph of Arimathea, from the Second Common of a Confessor not a Bishop (p. \pageref{CommonConfessorNotBishopII})
\end{rubric}

\begin{rubric}
	In Lent, Commemoration \& Last Gospel of the Feria.
\end{rubric}


\subby{St. Joseph of Arimathea}
\fancyhead[RO,LE]{\textit{Joseph}}
\fancyhead[RE,LO]{17 March}
\begin{inhead}
    {Memorial\\
17 March}
\end{inhead}

\begin{rubric}
	The propers are from the Second Common of a Confessor not a Bishop (p. \pageref{CommonConfessorNotBishopII}), except for that which followeth.
\end{rubric}
\begin{rubric}
	In Lent, Commemoration \& Last Gospel of the Feria.
\end{rubric}

    ℣. Then took they the body of Jesus.

	℟. And wound it in linen clothes with the spices.

\antiphon{Mag. \& Ben.}{Joseph of Arimathea, {\dag} being a disciple of Jesus, but secretly for fear of the Jews, besought Pilate that he might take away the body of Jesus: and Pilate gave him leave.}
 
\collect\label{ArimatheanCollect}
\lett{O}{God,} who didst give such grace unto thy servant Joseph that he boldly craved the body of Jesus, and with great reverence laid him in the rock-hewn sepulchre: grant, we beseech thee, that we likewise may be so emboldened for thee, as to do works meet for thy Kingdom. Through the same.


\subby{St. Cyril of Jerusalem}
\fancyhead[RO,LE]{\textit{Cyril}}
\fancyhead[RE,LO]{18 March}
\begin{inhead}
    {Double\\
18 March}
\end{inhead}

\begin{rubric}
	The Daily propers are from the Common of Doctors (p. \pageref{CommonDoctors}), except for that which followeth.
\end{rubric}

\introit
\lett{I}{n} the midst of the Church he opened his mouth: and the Lord filled him with the spirit of wisdom and of understanding: he clothed him with a robe of glory. \textit{Ps.} It is a good thing to give thanks unto the Lord: and to sing praises unto thy name, O most Highest.

\collect\label{CyrilCollect}
\lett{G}{rant} to us, we beseech thee, almighty God, at the intercession of the blessed Bishop Cyril: so to know thee, the only true God, and Jesus Christ whom thou hast sent; that we may be found worthy to be numbered for evermore among the sheep who hear his voice. Through the same.

\begin{rubric}
	Commemoration is made of St. Edward (p. \pageref{EdwardCollect}).
\end{rubric}
\begin{rubric}
	In Lent, Commemoration of the Feria.
\end{rubric}

\begin{rubric}
	The Epistle is from the additional Epistle of the Common of Doctors (p. \pageref{CommonDoctors}).
\end{rubric}

\gradtr{The mouth of the righteous is exercised in wisdom, and his tongue will be talking of judgement. ℣. The law of his God is in his heart: and his goings shall not slide.}{Blessed is the man that feareth the Lord: he hath great delight in his commandments. ℣. His seed shall be mighty upon earth: the generation of the faithful shall be blessed. ℣. Riches and plenteousness shall be in his house: and his righteousness endureth for ever.}

\readingcitation{Gospel}{Matthew 10:23}
\lett{A}{t that time:} Jesus said unto his disciples: When they persecute you in this city, flee ye into another: for verily I say unto you, Ye shall not have gone over the cities of Israel, till the Son of man be come. The disciple is not above his master, nor the servant above his lord. It is enough for the disciple that he be as his master, and the servant as his lord. If they have called the master of the house Beelzebub, how much more shall they call them of his household? Fear them not therefore: for there is nothing covered, that shall not be revealed; and hid, that shall not be known. What I tell you in darkness, that speak ye in light: and what ye hear in the ear, that preach ye upon the housetops. And fear not them which kill the body, but are not able to kill the soul: but rather fear him which is able to destroy both soul and body in hell.

\offertory{The righteous shall flourish like a palm-tree: and shall spread abroad like a cedar in Libanus.}

\secret\label{CyrilSecret}
\lett{L}{ook} down, O Lord, upon the spotless victim which we offer unto thee: and grant; that by the merits of thy blessed Bishop and Confessor Cyril we may endeavour ourselves to receive it with clean hearts. Through.

\begin{rubric}
	Commemoration is made of St. Edward (p. \pageref{EdwardSecret}).
\end{rubric}
\begin{rubric}
	In Lent, Commemoration of the Feria.
\end{rubric}

\communion{A faithful and wise servant, whom the Lord hath made ruler over his household: to give them their portion of meat in due season.}

\postcommunion\label{CyrilPostcommunion}
\lett{O}{Lord} Jesu Christ, may the sacraments of thy Body and Blood, which we have received: sanctify our minds and hearts through the prayers of the blessed Bishop Cyril; that we may be worthy to be made partakers of the divine nature. Who livest.

\begin{rubric}
	Commemoration is made of St. Edward (p. \pageref{EdwardPostcommunion}).
\end{rubric}
\begin{rubric}
	In Lent, Commemoration \& Last Gospel of the Feria.
\end{rubric}


\subby{St. Edward}
\fancyhead[RO,LE]{\textit{Edward}}
\fancyhead[RE,LO]{18 March}
\begin{inhead}
    {Memorial\\
18 March}
\end{inhead}

\begin{rubric}
	The Daily Office propers are from the Common of Doctors (p. \pageref{CommonDoctors}), except for that which followeth.
\end{rubric}

\collect\label{EdwardCollect}
\lett{O}{God,} the triumphant ruler of an everlasting kingdom, mercifully behold this thy family who celebrate the memory of blessed Edward, thy King and Martyr; and, by his merits and intercession, vouchsafe; that they, who glory in his triumph, may also attain unto his rewards. Through.

\begin{rubric}
	Commemoration is made of St. Cyril (p. \pageref{CyrilCollect}).
\end{rubric}
\begin{rubric}
	In Lent, Commemoration of the Feria.
\end{rubric}

\secret\label{EdwardSecret}
\lett{G}{rant,} O Lord, that this our bounden service may be acceptable in thy sight: that these our oblations may, by the prayers of him on whose solemnity they are offered, be made profitable unto our salvation. Through.
\begin{rubric}
	Commemoration is made of St. Cyril (p. \pageref{CyrilSecret}).
\end{rubric}
\begin{rubric}
	In Lent, Commemoration of the Feria.
\end{rubric}

\postcommunion\label{EdwardPostcommunion}
\lett{W}{e} beseech thee, O Lord our God, that like as we, whom thou hast refreshed by the partaking of thy sacred gift, do offer unto thee our worship: so by the intercession of blessed Edward thy Martyr, we may perceive the benefit of the same. Through.

\begin{rubric}
	Commemoration is made of St. Cyril (p. \pageref{CyrilPostcommunion}).
\end{rubric}
\begin{rubric}
	In Lent, Commemoration \& Last Gospel of the Feria.
\end{rubric}

\supplement{19 March}{St. Joseph}{Spouse of the B.V.M.}

\begin{paracol}{2}[]
\sloppy
\begin{inhead}
	I Evensong
\end{inhead}
\begin{hangparas}{1.25em}{1}
O Joseph, heav'nly hosts thy worthiness proclaim,

And Christendom conspires to celebrate thy fame,

Thou who in purest bonds wert to the Virgin bound;

How glorious is thy name renown'd.\\

Thou, when thou didst behold thy Spouse about to bear,

Wert sore oppress'd with doubt, wert fill'd with wond'ring care;

At length the Angel's word thy anxious heart reliev'd:

She by the Spirit hath conceiv'd.\\

Thou with thy new-born Lord didst seek far Egypt's land,

As wandering pilgrims, ye fled o'er the desert sand;

That Lord, when lost, by thee is in the temple found,

While tears are shed, and joys abound.\\

Not till death's hour is past do other men obtain

The meed of holiness, and glorious rest attain;

Thou, like to Angels made, in life completely blest,

Dost clasp thy God unto thy breast.\\

O Holy Trinity, thy suppliant servants spare;

Grant us to ride to heaven, for Joseph's sake and pray'r,

And so our grateful hearts to thee shall ever raise

Exulting canticles of praise. Amen.\\
\end{hangparas}
    ℣. He made him lord of his house.

	℟. And ruler of all his substance.
	
\switchcolumn

\begin{inhead}
	Mattins
\end{inhead}
\begin{hangparas}{1.25em}{1}
He, whom the faithful joyously do honour,

Singing his praises with devout affection,

Won on this feast day, in eternal glory

Life everlasting.\\

Blest beyond others, and exceeding blissful,

For, when the moment of his death was nearing,

Jesus and Mary at his side were standing,

Soothing his spirit.\\

Death doth he conquer, laying down his burden,

Calmly he slumbers, rest he gains eternal;

Lo, round his forehead, bright with rays of splendour,

Shineth a garland.\\

Then, as he reigneth, earnestly beseech we

That he may utter fervent intercessions,

Praying that pardon and the peace of heaven

May be our portion.\\

Glory we give thee, hymn of praise and blessing,

One in Three Persons, who above art reigning,

God, who hast honoured with thy crown for ever

This thy true servant. Amen.\\
\end{hangparas}
    ℣. The mouth of the righteous is exercised in wisdom.

	℟. And his tongue will be talking of judgement.

\fussy
\end{paracol}

\begin{rubric}
	In II Evensong, the Office Hymn is of I Evensong, with the following Versicle.
\end{rubric}

    ℣. Riches and plenteousness shall be in his house.

	℟. And his righteousness endureth for ever.


\subby{St. Cuthbert}
\fancyhead[RO,LE]{\textit{Cuthbert}}
\fancyhead[RE,LO]{20 March}
\begin{inhead}
    {Double\\
20 March}
\end{inhead}

\begin{rubric}
	The propers are from the Second Common of a Bishop Confessor (p. \pageref{CommonConfessorBishopII}), except for that which followeth.
\end{rubric}

\collect
\lett{O}{God,} who dost make thy Saints glorious by the inestimable gift of thy grace: grant, we beseech thee; that at the intercession of blessed Cuthbert, thy Confessor and Bishop, we may be found worthy to attain to the perfection of all virtue. Through.

\begin{rubric}
	In Lent, Commemoration of the Feria.
\end{rubric}

\secret
\lett{A}{ccept,} we beseech thee, O Lord, the sacrifice of man's redemption: and at the intercession of blessed Cuthbert, thy Confessor and Bishop, mercifully grant us health of mind and of body. Through.

\begin{rubric}
	In Lent, Commemoration of the Feria.
\end{rubric}

\postcommunion
\lett{W}{e} beseech thee, O Lord, that thy holy things which we have received may protect us by their power: and at the intercession of blessed Cuthbert, thy Confessor and Bishop, whose life shone forth in glory, guard us in peace and holiness. Through.

\begin{rubric}
	In Lent, Commemoration \& Last Gospel of the Feria.
\end{rubric}


\supplement{21 March}{St. Benedict}{}

\begin{paracol}{2}[]
\sloppy
\begin{inhead}
	I Evensong
\end{inhead}
\begin{hangparas}{1.25em}{1}
Shout, all ye people! Let your measur'd praises

Ring through the churches solemnly and sweetly:

On this his feast day, Benedict ascended

Heaven's high summit.\\

He, when his youthful joyous years were blooming,

Yet in his boyhood, left his native dwelling,

Seeking concealment hid within a cavern

Lonely and silent.\\

There amid nettles, rigid thorns, and briars

Won he the battle over youth's enticement,

Nurse of pollution: then he wrote a Holy

Rule of blest living.\\

Thy brazen image, infamous Apollo,

Soon hath he smitten; burnt the grove of Venus;

Then to the Baptist, on the sacred mountain

Stablish'd a chapel.\\

Now doth he witness happily in heaven

Seraphim, leading throngs of shining Angels,

While he refreshes faithful hearts who need him

With living waters.\\

Praise to the Father, to the Sole-begotten,

And to thee, alway with the Twain co-equal,

Fostering Spirit; one and only Godhead

Through all the ages. Amen.\\
\end{hangparas}

    ℣. The Lord loved him and adorned him.

	℟. He clothed him with a robe of glory.
	
	\switchcolumn
	
\begin{inhead}
	Mattins
\end{inhead}
\begin{hangparas}{1.25em}{1}
Gem of the highest, diadem immortal,

Cherish'd for ever after sacred struggle,

Thou, among many, first in lofty merit,

Benedict, shinest.\\

Thee in thy boyhood holy eld adorned;

Surely of nothing hath desire despoil'd thee;

Earth's blossom wither'd, to thy heart uplifted

Unto the highest.\\

Fleeing in secret, thou didst leave thy native

Country and parents: then, a forest-dweller,

Into subjection unto Christ, by torment

Broughtest thy body.\\

Not long securely couldest thou be hidden,

Good deeds and wonders brought thee to men's knowledge;

Through the world widely spread the happy rumour

Speedily flying.\\

Praise to the Father, to the Sole-begotten,

And to thee, alway with the Twain co-equal,

Fostering Spirit; one and only Godhead

Through all the ages. Amen.\\
\end{hangparas}

    ℣. The Lord guided the righteous in right paths.

	℟. And shewed him the kingdom of God.

\fussy
\end{paracol}

\begin{rubric}
	In II Evensong, the Office Hymn is of I Evensong, with the Versicle from Mattins.
\end{rubric}


\supplement{24 March}{St. Gabriel}{}

\begin{paracol}{2}[]
\sloppy
\begin{inhead}
	I Evensong
\end{inhead}
\begin{hangparas}{1.25em}{1}

Christ, the fair glory of the holy Angels,

Thou who hast made us, thou who o'er us rulest,

Grant of thy mercy unto us thy servants

Steps up to heaven.\\

Send thy Archangel, Gabriel, the mighty,

Herald of heaven; may he from us mortals

Spurn the old serpent, watching o'er the temples

Where thou art worshipp'd.\\

May the blest Mother of our God and Saviour,

May the assembly of the Saints in glory,

May the celestial companies of Angels

Ever assist us.\\

This he vouchsafe us, God for ever blessed

Father eternal, Son, and Holy Spirit,

Whose is the glory which through all creation.

Ever resoundeth. Amen.\\
\end{hangparas}

    ℣. An Angel stood at the altar of the temple.

	℟. Having in his hand a golden censer.
	
	\switchcolumn
	
\begin{inhead}
	Mattins
\end{inhead}
\begin{hangparas}{1.25em}{1}
O Christ, Redeemer of us all,

Protect thy servants when they call,

And hear with reconciling care

The blessed Virgin's holy prayer.\\

Be ever present, Angel high

Whose name `God's might' doth signify:

To all the weak new strength impart,

And solace to the sad of heart.\\

And ye, O ever blissful throng

Of heav'nly Spirits, guardians strong,

Our past and present ills dispel,

From future peril shield us well.\\

From lands wherein thy faithful dwell

Drive far away the infidel;

So we to Christ due hymns of praise

Henceforth with eager hearts may raise.\\

To thee, O Father, born of none,

And thee, O sole-begotten Son,

One with the Holy Paraclete,

Be glory ever, as is meet. Amen.\\
\end{hangparas}

    ℣. An Angel stood at the altar of the temple.

	℟. Having in his hand a golden censer.

\fussy
\end{paracol}

\begin{rubric}
	In II Evensong, the Office Hymn is of I Evensong, with the following Versicle.
\end{rubric}

    ℣. In the presence of the Angels I will sing praise unto thee, O my God.

	℟. I will worship toward thy holy temple, and praise thy Name.


\supplement{25 March}{Annunciation}{of the Blessed Virgin Mary}

\begin{rubric}
	The Office Hymn as in the Common of the Blessed Virgin Mary (p. \pageref{CommonBVM}), with the following Versicle.
\end{rubric}

    ℣. Hail Mary, full of grace.

	℟. The Lord is with thee.


\subby{St. John Damascene}
\fancyhead[RO,LE]{\textit{John}}
\fancyhead[RE,LO]{27 March}
\begin{inhead}
    {Double\\
27 March}
\end{inhead}

\begin{rubric}
	The Daily Office propers are from the Common of Doctors (p. \pageref{CommonDoctors}), except for that which followeth.
\end{rubric}

\introit
\lett{T}{hou} hast holden me by my right hand: thou shalt guide me with thy counsel, and after that receive me with glory. \textit{Ps.} Truly God is loving unto Israel, even unto such as are of a clean heart!

\collect
\lett{A}{lmighty} and everlasting God, who, for the defence of the veneration of sacred images, didst endue blessed John with heavenly doctrine and wondrous strength of spirit: grant unto us by his intercession and example; that we may imitate the virtues and perceive the advocacy of those whose images we venerate. Through.

\begin{rubric}
	Commemoration of the Feria.
\end{rubric}

\readingcitation{Epistle}{Wisdom 10:10}
%RV (unedited):
\lett{W}{isdom} guided him in straight paths; She shewed him God’s kingdom, and gave him knowledge of holy things; She prospered him in his toils, and multiplied the fruits of his labour; When in their covetousness men dealt hardly with him, She stood by him and made him rich; She guarded him from enemies, And from those that lay in wait she kept him safe, And over his sore conflict she watched as judge, That he might know that godliness is more powerful than all. When a righteous man was sold, wisdom forsook him not, But from sin she delivered him; She went down with him into a dungeon, And in bonds she left him not, Till she brought him the sceptre of a kingdom, And authority over those that dealt tyrannously with him; She shewed them also to be false that had mockingly accused him, And gave him eternal glory. Wisdom delivered a holy people and a blameless seed from a nation of oppressors. She entered into the soul of a servant of the Lord, And withstood terrible kings in wonders and signs. She rendered unto holy men a reward of their toils.

\gradtr{It is God that girdeth me with strength of war: and maketh my way perfect, ℣. He teacheth mine hands to fight: and mine arms shall break even a bow of steel.}{I will follow upon mine enemies, and overtake them. ℣. I will smite them that they shall not be able to stand: but fall under my feet. ℣. For this cause will I give thanks unto thee, O Lord, among the Gentiles, and sing praises unto thy name.}

\readingcitation{Gospel}{Luke 6:6}
\lett{A}{t that time:} It came to pass on another sabbath, that Jesus entered into the synagogue and taught. And there was a man whose right hand was withered. And the scribes and Pharisees watched him, whether he would heal on the sabbath day; that they might find an accusation against him. But he knew their thoughts, and said to the man which had the withered hand, Rise up, and stand forth in the midst. And he arose and stood forth. Then said Jesus unto them, I will ask you one thing; Is it lawful on the sabbath days to do good, or to do evil? to save life, or to destroy it? And looking round about upon them all, he said unto the man, Stretch forth thy hand. And he did so: and his hand was restored whole as the other. And they were filled with madness; and communed one with another what they might do to Jesus.

\offertory{There is hope of a tree, if it be cut down, that it will sprout again, and that the tender branch thereof will not cease.}

\secret
\lett{O}{Lord,} let the devout intercession of blessed John and of the Saints, who through his labours are set forth in the temples for our veneration, avail to render the gifts which we offer acceptable in thy sight. Through.

\begin{rubric}
	Commemoration of the Feria.
\end{rubric}

\communion{The arms of the ungodly shall be broken, and the Lord upholdeth the righteous.}

\postcommunion
\lett{W}{e} beseech thee, O Lord, that the gifts which we have received may shield us with heavenly armour: and that the advocacy of blessed John, together with the united intercession of the Saints, the veneration of whose images in the Church be victoriously upheld, be our defence. Through.

\begin{rubric}
	Commemoration \& Last Gospel of the Feria.
\end{rubric}


\subby{St. John Climacus}
\fancyhead[RO,LE]{\textit{John}}
\fancyhead[RE,LO]{30 March}
\begin{inhead}
    {Memorial\\
30 March}
\end{inhead}

\begin{rubric}
	The propers are from the Common of Abbots (p. \pageref{CommonAbbots}).
\end{rubric}
\begin{rubric}
	Commemoration \& Last Gospel of the Feria.
\end{rubric}


\subby{St. Innocent of Alaska}
\fancyhead[RO,LE]{\textit{Innocent}}
\fancyhead[RE,LO]{31 March}
\begin{inhead}
    {Memorial\\
31 March}
\end{inhead}

\begin{rubric}
	The propers are from the Second Common of a Confessor Bishop (p. \pageref{CommonConfessorBishopII}).
\end{rubric}
\begin{rubric}
	Commemoration \& Last Gospel of the Feria.
\end{rubric}


%%%%%%%%%%%%%%%


%%%%%%%%%%%%%


%%%%%%%%%%%%%

\subby{Tsar St. Nicholas of Russia}
\fancyhead[RO,LE]{\textit{Nicholas}}
\fancyhead[RE,LO]{17 July}
\begin{inhead}
    {Optional Simple\\
17 July}
\end{inhead}

\textit{Opening Sentence.} To the Lord our God belong mercies and forgivenesses, though we have rebelled against him: neither have we obeyed the voice of the Lord our God, to walk in his laws which he set before us.\vr{Dan 9:9-10}

\textit{Opening Sentence.} Correct us, O Lord, but with judgement: not in thine anger, lest thou bring us to nothing.\vr{Jer 10:24}

\textit{Opening Sentence.} Enter not into judgement with thy servants, O Lord: for in thy sight shall no man living be justified.\vr{Ps 143:2}

\subbysub{I Evensong}

\begin{rubric}
	Proper Psalms: 79 \& 85
\end{rubric}
\begin{rubric}
	Proper Lessons: Jeremiah 12 \& Hebrews 11:32-12:7
\end{rubric}

With thankful hearts thy glory, O King of Saints, we sing:

Shown in the saintly story Of Nich'las Martyr-King;

Who chose to die, obeying The voice of conscience dear,

Not live on earth, betraying All that he counted dear.\\

Shall, then, his mem'ry perish? His name we venerate;

The Faith he lov'd we cherish, His spirit emulate.

That so by Christ-like living, With charity resign'd:

Each others' faults forgiving, We may enrich mankind.\\

%Changed to match change of alone to with kin.
For long his foes assailed him, Till friends were so fallen;

And this world's weapons fail'd him, And he was left with kin.

No whit his foes relented, Successful in the strife,

But to their King presented The choice of death or life.\\

For all lives liv'd sincerely In Christ may God be blest,

To mortals thus most clearly In mortals manifest:

The Father, who forgiveth Man's failures in the strife;

The Son, in whom he liveth; The Spirit, Source of Life! Amen.\\

    ℣. Thou, O Lord, art my defender.

	℟. And the lifter up of my head.\\
	
\antiphon{Mag.}{The true glory of princes {\dag} consisteth in advancing God's glory, in the maintenance of true religion and the Church's good.}

\subbysub{Mattins}

\begin{rubric}
	Proper Psalms: 9-11
\end{rubric}
\begin{rubric}
	Proper Lessons: 2 Samuel 1 \& Matthew 27
\end{rubric}

\begin{rubric}
	Instead of the \emph{Venite} is said the following Hymn.
\end{rubric}

\lett{R}{ighteous} art thou, O Lord: and just are thy judgements.

\secondline{\textit{Thou art just, O Lord, in all that is brought upon us: for thou hast done right, but we have done wickedly.}}

\thirdline{Nevertheless, our feet were almost gone: our treadings
had well-nigh slipped.}

\textit{For why? we were grieved at the wicked: we did also see the ungodly in such prosperity.}

The people stood up, and the rulers took counsel together: against the Lord, and against his Anointed.

\textit{They cast their heads together with one consent: and were confederate against him.}

He heard the blasphemy of the multitude, and fear was on every side: while they conspired together againſt him, to take away his life.

\textit{They spoke against him with false tongues, and compassed him about with words of hatred: and fought against him without a cause.}

Yea, his own familiar friends, whom he trusted: they that eat of his bread laid great wait for him.

\textit{They rewarded him evil for good: to the great discomfort of his
soul.}

They took their counsel together, saying, God hath forsaken him: persecute him, and take him, for there is none to deliver him.

\textit{The breath of our nostrils, the Anointed of the Lord was taken in their pits: of whom we said, Under his shadow we shall be safe.}

The adversary and the enemy entered into the gates of Jerusalem: saying, When shall he die, and his name perish?

\textit{Let the sentence of guiltiness proceed against him: and now that he lieth, let him rise up no more.}

False witnesses also did rise up against him: they laid to his charge things that he knew not.

\textit{For the sins of the people, and the iniquities of the priests: they shed the blood of the just in the midst of Jerusalem.}

O my soul, come not thou into their secret; unto their assembly, mine honour, be not thou united: for in their anger they slew a man.

\textit{Even the man of thy right hand: the Son of man, whom thou hadst made so strong for thine own self.}

In the sight of the unwise he seemed to die: and his departure was taken for misery.

\textit{They fools counted his life madness, and his end to be without honour: but he is in peace.}

For though he was punished in the sight of men: yet was his hope full of immortality.

\textit{How is he numbered with the children of God: and his lot is
among the saints!}

But, O Lord God, to whom vengeance belongeth, thou God, to whom vengeance belongeth: be favourable and gracious unto Sion.

\textit{Be merciful, O Lord, unto thy people, whom thou hast redeemed: and lay not innocent blood to our charge.}

O shut not up our souls with sinners: nor our lives with the bloodthirsty.

\textit{Deliver us from blood-guiltiness, O God, thou that art the God of our
salvation: and our tongues shall sing of thy righteousness.}

For thou art the God that hast no pleasure in wickedness: neither shall any evil dwell with thee.

\textit{Thou wilt destroy them that speak leasing: the Lord abhors both the blood-thirsty and deceitful man.}

O how suddenly do they consume: perish, and come to a fearful end!

\textit{Yea, even like a dream, when one awaketh: so didst thou make their image to vanish out of the city.}

Great and marvellous are thy works, O Lord God Almighty: just and true are thy ways, O King of saints!

\textit{Righteous art thou, O Lord: and just are thy judgements.}

Glory to the Father, and to the Son: and to the Holy Ghost;

\textit{As it was in the beginning, is now, and ever shall be: world without end. Amen.}

\begin{rubric}
	The following Hymn may be said after the preceding Hymn.
\end{rubric}
\begin{multicols}{2}
Nicholas, who chose to die

Rather than the Faith deny,

Forfeiting his kingly pride

For the sake of Jesu's Bride;

Lovingly his praise we sing,

Russia's martyr, Russia's king.\\

Mirror fair of courtesy,

Flow'r of wedded chastity,

Humble follow'r day by day,

Of the Church's holy way;

Lovingly his praise we sing,

Russia's martyr, Russia's king.\\

All the way of death he trod

For the glory of his God,

And his dying dignity

Made a bright Epiphany;

Lovingly his praise we sing,

Russia's martyr, Russia's king.\\

Bless we God the Three in One,

For all faithful 'neath the sun,

For the faithful gone before,

And for those our country bore;

Chiefly him whose praise we sing:

Russia's martyr, Russia's king. Amen.\\
\end{multicols}

\begin{rubric}
	The Office Hymn is here provided.
\end{rubric}
\begin{multicols}{2}
Nicholas' praise, our martyr King,

With heart and voice come let us sing:

Who earthly shame with Christ doth bear,

The Victor's laurel now doth wear.\\

The prison's gloom, the battle sore,

Christ's martyr here with gladness bore,

And mounting from the seated shame

To his sure hope, in heav'n he came.\\

For holy Church so he withstood,

By the bullet his lifeblood flow'd:

And where that kingly seed was sown

New harvest unto Christ hath grown.\\

For Russia's cause to peril brought;

Crown, life, and freedom deem'd he not:

For Russia's Church, for Russia's realm

We pray, lest storms our ark should whelm.\\

Jesu, to thee we glory pay,

Ruling earth's kingdoms neath thy sway;

All glory, as is ever meet,

To Father and to Paraclete. Amen.\\

    ℣. Hold fast that which thou hast.

	℟. That no man take thy crown.\\
\end{multicols}

\antiphon{Ben.}{Let my sufferings {\dag} satiate the malice of mine and the Church's enemies, but let their cruelty never exceed the measure of my charity.}

\begin{rubric}
	In the end of the Litany (which shall always on this Day be used) immediately after the Collect \emph{We humbly beseech thee, O Father}, the three Collects next following are to be read.
\end{rubric}

\lett{O}{Lord,} we beseech thee mercifully hear our prayers, and spare all those who confess their sins unto thee; that they whose consciences by sin are accused, by thy merciful pardon may be absolved; through Christ our Lord. \textit{Amen.}

\lett{O}{most} mighty God, and merciful Father, who hast compassion upon all men, and hatest nothing that thou hast made: who wouldest not the death of a sinner, but that he would rather turn from his sin, and be saved: Mercifully forgive us our trespasses; receive and comfort us, who are grieved and wearied with the burden of our sins. Thy property is always to have mercy; to thee only it appertaineth to forgive sins. Spare us therefore, good Lord, spare thy people, whom thou hast redeemed; enter not into judgement with thy servants, who are vile earth and miserable sinners; but so turn thine anger from us, who meekly acknowledge our vileness, and truly repent us of our faults; and so make haste to help us in this world, that we may ever live with thee in the world to come; through Jesus Christ our Lord. \textit{Amen.}

\lett{T}{urn} thou us, O good Lord, and so shall we be turned. Be favourable, O Lord, be favourable to thy people, Who turn to thee in weeping, fasting, and praying. For thou art a merciful God, full of compassion, long-suffering, and of great pity. Thou sparest, when we deserve punishment, And in thy wrath thinkest upon mercy. Spare thy people, good Lord, spare them, And let not thine heritage be brought to confusion. Hear us, O Lord, for thy mercy is great, And after the multitude of thy mercies look upon us, Through the merits and mediation of thy blessed Son Jesus Christ our Lord. \textit{Amen.}

\introit
\lett{L}{et} us rejoice in the Lord, celebrating a festival day in honour of blessed Nicholas the Martyr: at whose passion the Angles rejoice and praise the Son of God. \textit{Ps.} Rejoice in the Lord, O ye righteous: for it becometh well the just to be thankful.

\collect
%\lett{O}{most} mighty God, terrible in thy judgements, and wonderful in thy doings toward the children of men; who in thy heavy displeasure didst suffer the life of our gracious Sovereign Emperor Nicholas the Second, to be (as on this day) taken away by the hands of cruel and bloody men: We thy sinful creatures here assembled before thee, do, in the behalf of all this Nation, which brought down this heavy judgement upon us. But, O gracious, when thou makest inquisition for blood, lay not the guilt of this innocent blood, (the shedding whereof nothing but the blood of thy Son can expiate,) lay it not to the charge of the people of this land; not let it ever be required of us, or our posterity. Be merciful, O Lord, be merciful unto thy people, whom thou hast redeemed; and be not angry with us for ever: But pardon us for thy mercy's sake. through the merits of thy Son Jesus Christ our Lord. \textit{Amen.}
%
\lett{B}{lessed} Lord, in whose sight the death of thy saints is precious; We magnify thy Name for thine abundant grace bestowed upon the martyred Sovereign, Emperor Nicholas of Russia; by which he was enabled so cheerfully to follow the steps of his blessed Master and Saviour, in a constant meek suffering of all barbarous indignities, and at the last resisting unto blood; and even then according to the same pattern, praying for his murderers. Let his memory, O Lord, be ever blessed among us; that we may follow the example of his courage and constancy, his meekness and patience, and great charity. And grant, by his merits and prayers, that this our land may be ever defended from all disquietude, and thy mercy glorified in the forgiveness of our sins; and all for Jesus Christ his sake, our only Mediator and Advocate. \textit{Amen.}

\readingcitation{Epistle}{1 Peter 2:13}
\lett{B}{rethren:} Submit yourselves to every ordinance of man for the Lord's sake: whether it be to the king, as supreme; Or unto governors, as unto them that are sent by him for the punishment of evildoers, and for the praise of them that do well. For so is the will of God, that with well doing ye may put to silence the ignorance of foolish men: As free, and not using your liberty for a cloke of maliciousness, but as the servants of God. Honour all men. Love the brotherhood. Fear God. Honour the king. Servants, be subject to your masters with all fear; not only to the good and gentle, but also to the froward. For this is thankworthy, if a man for conscience toward God endure grief, suffering wrongfully. For what glory is it, if, when ye be buffeted for your faults, ye shall take it patiently? but if, when ye do well, and suffer for it, ye take it patiently, this is acceptable with God. For even hereunto were ye called: because Christ also suffered for us, leaving us an example, that ye should follow his steps: Who did no sin, neither was guile found in his mouth:

\gradall{I have found David my servant: with my holy oil have I anointed him. My hand shall hold him fast; and mine arm shall strengthen him. ℣. The enemy shall not be able to do him violence: the son of wickedness shall not hurt him.}{Alleluia, alleluia. ℣. Blessed is the man that endureth temptation: for when he is tried, he shall receive a crown of life.}

\begin{rubric}
	In Septuagesimatide \& Lent, the following Tract is said instead of the Alleluia.
\end{rubric}

\tract{Thou hast given him his heart's desire: and hast not denied him the request of his lips. ℣. For thou shalt prevent him with the blessings of goodness. ℣. Thou shalt see a crown of pure gold upon his head.}

\subbysub{Sequence}
\begin{multicols}{2}
Heav'nly King, of Kings the Pastor,

Giv'r of laws, of justice master,

Ruling all by thy behest,

Unto thee to-day we render,

Praise for him, to mem'ry tender,

Nicholas, of kings the best.\\

Traitors shedding blood like water

Fill'd the land with crime and slaughter,

Law was trampl'd in the mud,

Noble churches left forsaken

And the White Rose, overtaken

By the gun, was red with blood.\\

Thus the bardic verse fulfilling

There shall be a time of killing

When the ravens shall be fed,

And a King without pollution

Midst a realm in revolution

Shall be number'd with the dead.\\

Violent men without compassion

Proudly spurned the ancient fashion

Of the sacred right divine;

From his friends by madmen riven

Was our King to slaughter driven

Stain'd with blood his Royal line.\\

Faithful son of Mother holy,

To the Church devoted solely,

He to keep her laws was fain.

He her champion ever glorious,

Was beaten still victorious,

Robb'd of life, but conqueror slain.\\

%`He nothing common did nor mean,

%Upon that memorable scene,'

`With simple and sincere faith blest,

Self-sacrificing until rest'

Against the wicked stood his face
%When on the block he laid his head;

Nor call'd the gods with vulgar spite

To vindicate his helpless right,

But went to death as to his bed.\\

Fair exchange King Nich'las making
%Fair exchange King Charles was making

When, the crown immortal taking

For the earthly crown he wore,

By bullet he follow'd faster

To the realm of Christ his master,

And the Cross behind him bore.\\

%ORIGINAL:
Woe to pretended justiciers,

Why, like assassins, did they fear?

To efface their wicked deed?

From which men, then, were they hiding

But from God no favour finding

And their Tsar ascending high.\\

%Lo, the priest who shares his glory
%(Laud his name and laud his story),
%For his fellow-martyr waits,
%And the white-robed host upraising,
%Heart and voice their Saviour
%Greets him at the heavenly gates.

He by dying brought salvation

To the torn and shatter'd nation,

Life restor'd and liberty;

For the Martyr's blood was sowing

Seed from which the Church is growing

Seed of immortality.\\

Six words said before him they slew,
%Ere his death one word was spoken

%That 'Remember' was the token

He said, `You know not what you do.'

Of his coming victory.

So his blood brought life and healing,

And the Church's triumph sealing,

Never shall forgotten be. Amen.
\end{multicols}
\readingcitation{Gospel}{Matthew 21:33}
\lett{A}{t that time:} Jesus spake a parable: There was a certain householder, which planted a vineyard, and hedged it round about, and digged a winepress in it, and built a tower, and let it out to husbandmen, and went into a far country: And when the time of the fruit drew near, he sent his servants to the husbandmen, that they might receive the fruits of it. And the husbandmen took his servants, and beat one, and killed another, and stoned another. Again, he sent other servants more than the first: and they did unto them likewise. But last of all he sent unto them his son, saying, They will reverence my son. But when the husbandmen saw the son, they said among themselves, This is the heir; come, let us kill him, and let us seize on his inheritance. And they caught him, and cast him out of the vineyard, and slew him. When the lord therefore of the vineyard cometh, what will he do unto those husbandmen? They say unto him, He will miserably destroy those wicked men, and will let out his vineyard unto other husbandmen, which shall render him the fruits in their seasons.

\offertory{In the sight of the unwise he seemed to die and his departure was taken for misery; but he is in peace.}

\secret
\lett{M}{ercifully} regard, we beseech thee, almighty God, this our atoning sacrifice, and give ear to the prayer of the blessed King Nicholas thy Martyr; graciously receive it in propitiation for the sins of this thine household. Through.

\communion{The people stood up, and the rulers took counsel together; against the Lord, and against his Anointed.}

\postcommunion
\lett{M}{ay} the bread of heaven imbue us, O Lord, with the spirit of fortitude; for this was the Bread which gave blessed King Nicholas thy Martyr strength to fight victoriously for thy Church. Through.


\subby{St. Elias the Prophet}
\fancyhead[RO,LE]{\textit{Elias}}
\fancyhead[RE,LO]{20 July}
\begin{inhead}
    {Double\\
20 July}
\end{inhead}\par\noindent
%Elijah changed to Elias \& Elisha to Eliseus.
%From the Carmelite Breviary and American Missal:
%Also from the `Saints of Mount Carmel' (https://archive.org/details/saintsofmountcar0000john/page/252/mode/2up):
%%From the Carmelite Breviary (1930s), translation from the KJV:
\textit{Opening Sentence.} Then stood up Elias the prophet as fire, and his word burned like a lamp, alleluia.\vr{Ecclus. 48:1}\par

\begin{paracol}{2}[]
\sloppy
\begin{inhead}
	I Evensong
\end{inhead}
\begin{hangparas}{1.25em}{1}
The lofty peaks of Carmel

With tuneful praises ring,

The anthems of Elias

'Tis our delight to sing.\\

%Order changed to Churches:
The glory of our Churches,

Our leader, prop, and stay,

From east to west his offspring

Increaseth day by day.\\

When sorely press'd with famine,

A raven serv'd him bread,

With meal and cruse unfailing,

The widow'd hearth was fed.\\

The boy from death delivered

Is to his home restor'd,

And light so much desir'd,

In radiant flood is pour'd.\\

Behold the Heaven closeth,

To open at his voice,

And copious welcome showers

The thirsty lands rejoice.\\

To Father, Son, and Spirit,

Be equal power and praise,

All glory and dominion

Henceforth for endless days. Amen.\\
\end{hangparas}

    ℣. By the word of the Lord he shut up the heaven.

	℟. And he brought down fire from heaven thrice.

\antiphon{Mag.}{Behold, I {\dag}  will send you Elias the prophet before the coming of the great and dreadful day of the Lord: And he shall turn the heart of the fathers to the children, and the heart of the children to their fathers}\par\noindent

\switchcolumn

\begin{inhead}
	Mattins
\end{inhead}
\begin{hangparas}{1.25em}{1}
Come, blest companions, let our joy resounding

Extol to heav'n the leader of our line.

`Tis meet the mem'ry of his deeds abounding

Should waken ceaseless canticles divine.\\

He knows the gentle breathing of the Spirit

Cloth'd in the whistling murmur of the air,

By God's command the chastisements they merit

Proud Jezebel and Ahab justly share.\\

The caverns green of Carmel form his dwelling,

With leathern tunic is he rudely clad,

To impious Ahaziah his foretelling

Gives portent of a dissolution sad.\\

Twice at his pray'r the fire from Heav'n descending

Consumeth trembling soldiers in its flame,

The flowing wat'rs met with his mantle rending,

Dry shod he passeth safely through the same.\\

O Father, let thy help and thy protection

Be o'er thy children as they humbly plead,

Entreat the Spirit, by his sweet election,

To multiply his graces in their need.\\

O unbegotten Father, we adore thee,

O Son begotten, rev'rence be to thee,

O glorious Spirit, bow we low before thee,

Thou simple undivided Trinity. Amen.\\
\end{hangparas}

    ℣. Elias was covered with the whirlwind.

	℟. And his spirit was filled up in Eliseus.

\antiphon{Ben.}{Elias was a man {\dag} subject to like passions as we are, and he prayed earnestly that it might not rain: and it rained not on the earth by the space of three years and six months. And he prayed again, and the heaven gave rain, and the earth brought forth her fruit.}\par\noindent

\fussy
\end{paracol}

\begin{inhead}
	II Evensong
\end{inhead}

%Order changed to Churches
Thou prop of our Churches, thou pride of our race,

Let thy praises resound far and near,

Let sea and the land and the air give them place,

Rehearsing in gladness thy glory and grace,

Till the earth and the heav'ns give ear.\\

O sun of the heavens, how lovely thy rays!

What power thy wonders unfold,

How fruitful in merits the length of thy days,

Commission'd by God in His manifold ways,

For noble endeavours of old!\\

To regions celestial, in power and might,

Triumphant thy chariot speeds;

Uplifted by angels to marvellous height,

While shining in splendour and dazzling with light,

Thou guidest the fiery steeds.\\

As witness to men of his Sonship divine,

With Jesus thy glory we view;

The Father hath call'd thee on Tabor to shine,

Companion to Moses, and with him to sign

A testament faithful and true.\\

Protect us we pray, ‘neath thy powerful shield,

Incline to our aid from above,

Let thy fost'ring guidance be ever reveal'd

To thy children of Carmel, whose bosoms are seal'd

With the strength of thy fath'rly love\\

All power, dominion, all glory and praise,

Be given to Father and Son,

To thee, Holy Spirit, for numberless days

Our homage eternal we equally raise

All glory to God, Three in One. Amen.\\

    ℣. Blessed are they that saw thee.

	℟. And were honoured with thy friendship.
	
%Based on KJV with changes based off the Latin.
\antiphon{Mag.}{And Elias took {\dag} his mantle, and wrapped it together, and smote the waters of the Jordan, and they were divided hither and thither, so that he and Eliseus went over on dry ground. As they still went on, behold, there appeared a chariot of fire, and horses of fire, and parted them both asunder; and Elias went up by a whirlwind into heaven, and Eliseus saw him no more.}

%Tulit Elías {\dag} pállium suum, et invólvit illud, et percússit aquas jordanis, qua divisa sunt in utrámque partem, et transiérunt ipse et Eliséus per siccum. Qui incedentes, ecce, currus igneus et equi igner diviserunt utrúmque : et ascéndit Elías per túrbinem in c{\ae}lum, Eliséus autem non vidit eum ámplius.

\introit
\lett{I}{have} been very zealous for the Lord God of hosts: for the children of Israel have forsaken thy covenant, thrown down thine altars, and slain thy prophets with the sword; and I, even I only, am left; and they seek my life, to take it away. \textit{Ps.} I will magnify thee, O Lord, for thou hast set me up: and not made my foes to triumph over me.
\collect
\lett{G}{rant,} we beech thee, Almighty God: that we who believe that thou didst marvellously lift up thy Prophet Elias in a fiery chariot, while yet in this life; may at his intercession, while still alive, be raised to spiritual heights and rejoice in the resurrection of the just. Through.
\begin{rubric}
	Commemoration of St. Margaret of Antioch, from the Second Common of a Virgin Martyr (p. \pageref{CommonVirginMartyrII}).
\end{rubric}

\readingcitation{Epistle}{Ecclesiasticus 48:1}
%RV:
\lett{T}{here} arose Elijah the prophet as fire, And his word burned like a torch:  Who brought a famine upon them, And by his zeal made them few in number. By the word of the Lord he shut up the heaven: Thrice did he thus bring down fire. How wast thou glorified, O Elijah, in thy wondrous deeds! And who shall glory like unto thee? Who did raise up a dead man from death, And from the place of the dead, by the word of the Most High: Who brought down kings to destruction, And honourable men from their bed: Who heard rebuke in Sinai, And judgements of vengeance in Horeb: Who anointed kings for retribution, And prophets to succeed after him: Who was taken up in a tempest of fire, In a chariot of fiery horses: Who was recorded for reproofs in their seasons, To pacify anger, before it brake forth into wrath; To turn the heart of the father unto the son, And to restore the tribes of Jacob. 

%\lett{T}{hen} stood up Elias the prophet as fire, and his word burned like a lamp. He brought a sore famine upon them, and by his zeal he diminished their number. By the word of the Lord he shut up the heaven, and also three times brought down fire. O Elias, how wast thou honoured in thy wondrous deeds! and who may glory like unto thee! Who didst raise up a dead man from death, and his soul from the place of the dead, by the word of the most High: Who broughtest kings to destruction, and honorable men from their bed: Who heardest the rebuke of the Lord in Sinai, and in Horeb the judgment of vengeance: Who annointedst kings to take revenge, and prophets to succeed after him: Who was taken up in a whirlwind of fire, and in a chariot of fiery horses: Who wast ordained for reproofs in their times, to pacify the wrath of the Lord's judgment, before it brake forth into fury, and to turn the heart of the father unto the son, and to restore the tribes of Jacob.

\gradall{It came to pass, when the Lord would take up Elias into heaven by a whirlwind, that Elias went with Eliseus from Gilgal. ℣. And it came to pass, as they still went on, that, behold, there appeared a chariot of fire, and horses of fire, and parted them both asunder; and Elias went up by a whirlwind into heaven.}{Alleluia, alleluia. ℣. Elias, while he was full of zeal for the law, was taken up into heaven. Alleluia.}

\readingcitation{Gospel}{Luke 9:28}
\lett{A}{t that time:} It came to pass about an eight days after these sayings, Jesus took Peter and John and James, and went up into a mountain to pray. And as he prayed, the fashion of his countenance was altered, and his raiment was white and glistering. And, behold, there talked with him two men, which were Moses and Elias: Who appeared in glory, and spake of his decease which he should accomplish at Jerusalem. But Peter and they that were with him were heavy with sleep: and when they were awake, they saw his glory, and the two men that stood with him. And it came to pass, as they departed from him, Peter said unto Jesus, Master, it is good for us to be here: and let us make three tabernacles; one for thee, and one for Moses, and one for Elias: not knowing what he said. While he thus spake, there came a cloud, and overshadowed them: and they feared as they entered into the cloud. And there came a voice out of the cloud, saying, This is my beloved Son: hear him. And when the voice was past, Jesus was found alone. And they kept it close, and told no man in those days any of those things which they had seen.
\offertory{Elias was a man subject to like passions as we are: and he prayed earnestly that it might not rain, and it rained not on the earth for the space of three years and six months: and he prayed again, and the heaven gave rain, and the earth brought forth her fruit.}

\secret
\lett{W}{e} offer unto thee, O Lord, this sacrifice of praise in honour of thy Prophet Elias: that, as thou didst accept his burnt-offering, so thou wouldest vouchsafe to accept our own sacrifice; that through it we may be made worthy to attain unto everlasting gladness. Through.
\begin{rubric}
	Commemoration of St. Margaret of Antioch, from the Second Common of a Virgin Martyr (p. \pageref{CommonVirginMartyrII}).
\end{rubric}

\subbysub{Preface}
\lett{A}{nd} that we, with glad hearts, should praise, bless, and glorify thee in the Solemnity of blessed Elias thy Prophet: Who at thy word stood up like fire; shut the heavens and raised the dead; smote tyrants and slew the blasphemous; and laid the foundation of the monastic profession: Who, being fed with meat and drink served by Angels, walked in the strength of that food, even to the holy mountain: Who, being raised up from earth in a fiery chariot, will return to us as the Forerunner of the second Coming of Jesus Christ our Lord. Therefore.

\communion{Behold, I will send you Elias the Prophet before the coming of the great and terrible day of the Lord: and he shall turn the hearts of the fathers to the children, and the hearts of the children to their fathers.}

\postcommunion
\lett{O}{God,} who by thy holy Angel didst give meat and drink to blessed Elias thy Prophet: grant, by his intercession, that what we have received of thy heavenly table, we may keep undefiled in purity of mind. Through.
\begin{rubric}
	Commemoration of St. Margaret of Antioch, from the Second Common of a Virgin Martyr (p. \pageref{CommonVirginMartyrII}).
\end{rubric}


\subby{St. Margaret of Antioch}
\fancyhead[RO,LE]{\textit{Margaret}}
\fancyhead[RE,LO]{20 July}
\begin{inhead}
    {Memorial\\
20 July}
\end{inhead}

\begin{rubric}
	The propers are from the Second Common of a Virgin Martyr (p. \pageref{CommonVirginMartyrII}).
\end{rubric}

\supplement{22 July}{St. Mary}{Magdalene}

\begin{paracol}{2}[]
\sloppy
\begin{inhead}
	I Evensong
\end{inhead}
\begin{hangparas}{1.25em}{1}
O Father of celestial rays,
	
When thou on Magdalene dost gaze,

The flame of burning love appears,

Her icy heart dissolves in tears.\\

Wounded by love, she hastens o'er

The feet of Christ her tears to pour,

Anoints them, wipes them with her hair,

And prints adoring kisses there.\\

Fearless, the Cross she will not leave:

And grieving, to the Tomb doth cleave:

No ruthless soldiers cause her dread:

For from her love all fear hath fled.\\

O Christ, true Charity thou art;

Purge thou the foulness of our heart,

Fill every soul with grace and love,

And give us thy rewards above.\\

All laud to God the Father be;

All praise, eternal Son, to thee;

All glory, as is ever meet, To God the Holy Paraclete. Amen.\\
\end{hangparas}

    ℣. Mary hath chosen that good part.

	℟. Which shall not be taken away from her.

\switchcolumn

\begin{inhead}
	Mattins
\end{inhead}
\begin{hangparas}{1.25em}{1}
Thou only Son of God on high,

Regard us with a gracious eye,

Who weeping Magdalene dost own

And call unto thy glorious throne.\\

Lo! in the royal coffers laid

Again the long lost coin display'd;

The noble gem of sparkling sheen,

From mire recover'd, glows serene.\\

Jesu, our refuge sure and sweet,

Thee, hope of penitents, we greet;

Forgive the hearts that fain would break,

For that repentant sinner's sake.\\

And may that Mother kind and meek

Think on our nature frail and weak,

And raise her prayer that we may gain

A passage safe o'er life's rough main.\\

\begin{rubric}
	The following Ending is never changed.
\end{rubric}

To God alone be honour paid

For grace so manifold display'd:

Their guilt he pardons who repent,

And gives reward for punishment. Amen.\\
\end{hangparas}

    ℣. Her sins, which are many, are forgiven.

	℟. For she loved much.

\fussy
\end{paracol}

\begin{rubric}
	In II Evensong, the Office Hymn is of I Evensong with the following Versicle.
\end{rubric}

    ℣. God hath chosen her and preferred her.

	℟. He hath made her to dwell in his tabernacle.


\subby{Pope St. Stephen}
\fancyhead[RO,LE]{\textit{Pope St. Stephen}}
\fancyhead[RE,LO]{2 August}
\begin{inhead}
    {Memorial\\
2 August}
\end{inhead}

\begin{rubric}
	The propers are from the Second Common for Confessor Bishops (p. \pageref{CommonConfessorBishopII}), except for that which followeth.
\end{rubric}

\introit
\lett{I}{will} deck her priests with health, and her saints shall rejoice and sing. \textit{Ps.} Lord, remember David: and all his trouble.

\readingcitation{Epistle}{Acts 20:17}
\lett{I}{n those days:} From Miletus Paul sent to Ephesus, and called the elders of the church. And when they were come to him, he said unto them, Ye know, from the first day that I came into Asia, after what manner I have been with you at all seasons, serving the Lord with all humility of mind, and with many tears, and temptations, which befell me by the lying in wait of the Jews: and how I kept back nothing that was profitable unto you, but have shewed you, and have taught you publickly, and from house to house, testifying both to the Jews, and also to the Greeks, repentance toward God, and faith toward our Lord Jesus Christ.

\gradall{Behold, a great priest, who in his days pleased God. ℣. There was none found like unto him, who kept the law of the Most High.}{Alleluia, alleluia. ℣. Thou art a priest for ever, after the order of Melchisedech. Alleluia.}

\communion{Lord, thou deliveredst unto me five talents: behold, I have gained beside them five talents more. Well done, thou good and faithful servant, thou hast been faithful over a few things. I will make thee ruler over many things, enter thou into the joy of thy Lord.}

\subby{The Invention of St. Stephen}
\fancyhead[RO,LE]{\textit{Invention of St. Stephen}}
\fancyhead[RE,LO]{3 August}
\begin{inhead}
    {Double\\
3 August}
\end{inhead}
\begin{rubric}
	The Office \& Mass as on 26 December, except for the following.\par
	\textsc{Note,} The Creed is not said.
\end{rubric}
\collect
\lett{G}{rant} us, we beseech thee, O Lord, so to imitate that which we revere: that we may learn to love even our enemies; forasmuch as we celebrate the Finding of him, who was able to pray even for his persecutors to our Lord Jesus Christ thy Son. Who liveth.

\readingcitation{Epistle}{Acts 6:8}
\lett{I}{n those days:} Stephen, full of faith and power, did great wonders and miracles among the people. Then there arose certain of the synagogue, which is called the synagogue of the Libertines, and Cyrenians, and Alexandrians, and of them of Cilicia and of Asia, disputing with Stephen. And they were not able to resist the wisdom and the spirit by which he spake. When they heard these things, they were cut to the heart, and they gnashed on him with their teeth. But he, being full of the Holy Ghost, looked up stedfastly into heaven, and saw the glory of God, and Jesus standing on the right hand of God, and said, Behold, I see the heavens opened, and the Son of man standing on the right hand of God. Then they cried out with a loud voice, and stopped their ears, and ran upon him with one accord, and cast him out of the city, and stoned him: and the witnesses laid down their clothes at a young man’s feet, whose name was Saul. And they stoned Stephen, calling upon God, and saying, Lord Jesus, receive my spirit. And he kneeled down, and cried with a loud voice, Lord, lay not this sin to their charge. And when he had said this, he fell asleep in the Lord.


\subby{Dedication of Our Lady of the Snows}
\fancyhead[RO,LE]{\textit{Our Lady of the Snows}}
\fancyhead[RE,LO]{5 August}
\begin{inhead}
    {Greater Double\\
5 August}
\end{inhead}
\begin{rubric}
	The Office and Mass are from the Common of the Blessed Virgin Mary (p. \pageref{CommonBVM}).\par
	\textsc{Note,} The Creed is said and the Preface of the B.V.M. \emph{And that in the Festivity.}
\end{rubric}
\begin{rubric}
	Commemoration is made of St. Oswald, whose propers come from the First Common of a Martyr not a Bishop (p. \pageref{CommonMartyrNotBishopII}), except for that which followeth.
\end{rubric}
\collect
\lett{A}{lmighty} and everlasting God, who by the martyrdom of the blessed King Oswald hast hallowed this day with holy joy and gladness: grant unto our hearts the increase of thy charity; that we, who honour his glorious battle for the faith, may imitate his constancy even unto death. Through.

\readingcitation{Epistle}{Wisdom 4:7}
%RV:
\lett{B}{ut} a righteous man, though he die before his time, shall be at rest. (For honourable old age is not that which standeth in length of time, Nor is its measure given by number of years: But understanding is gray hairs unto men, And an unspotted life is ripe old age.) Being found well-pleasing unto God he was beloved of him, And while living among sinners he was translated: He was caught away, lest †† wickedness should change his understanding, Or guile deceive his soul. (For the bewitching of naughtiness bedimmeth the things which are good, And the giddy whirl of desire perverteth an innocent mind.) Being made perfect in a little while, he fulfilled long years; For his soul was pleasing unto the Lord: Therefore hasted he out of the midst of wickedness. But as for the peoples, seeing and understanding not, Neither laying this to heart, That grace and mercy are with his chosen, And that he visiteth his holy ones.

%\lett{T}{hough} the righteous be prevented with death, yet shall he be in rest. For honourable age is not that which standeth in length of time, nor that is measured by number of years. But wisdom is the gray hair unto men, and an unspotted life is old age. He pleased God, and was beloved of him: so that living among sinners he was translated. Yea speedily was he taken away, lest that wickedness should alter his understanding, or deceit beguile his soul. For the bewitching of naughtiness doth obscure things that are honest; and the wandering of concupiscence doth undermine the simple mind. He, being made perfect in a short time, fulfilled a long time: For his soul pleased the Lord: therefore hasted he to take him away from among the wicked. This the people saw, and understood it not, neither laid they up this in their minds, That his grace and mercy is with his saints, and that he hath respect unto his chosen.

\readingcitation{Gospel}{Matthew 16:24}
\lett{A}{t that time:} Jesus said unto his disciples: If any man will come after me, let him deny himself, and take up his cross, and follow me. For whosoever will save his life shall lose it: and whosoever will lose his life for my sake shall find it. For what is a man profited, if he shall gain the whole world, and lose his own soul? or what shall a man give in exchange for his soul? For the Son of man shall come in the glory of his Father with his angels; and then he shall reward every man according to his works.


\supplement{6 August}{Transfiguration}{of Our Lord Jesus Christ}

\subby{St. Donatus}
\fancyhead[RO,LE]{\textit{Donatus}}
\fancyhead[RE,LO]{7 August}
\begin{inhead}
    {Memorial\\
7 August}
\end{inhead}
\begin{rubric}
	The Office and Mass come from the Second Common of Bishop Martyrs (p. \pageref{CommonMartyrBishopII}), except for that which followeth.
\end{rubric}

\introit
\lett{O}{ye} priest of the Lord, bless ye the Lord: O ye holy and humble men of heart, bless ye the Lord. \textit{Cant.} O all ye works of the Lord, bless ye the Lord: praise him and magnify him for ever.

\collect
\lett{O}{God,} the glory of thy priests: grant, we beseech thee; that we may perceive the succour of thy holy Martyr and Bishop Donatus, whose festival we celebrate. Through.

\gradall{The mouth of the just is exercised in wisdom, and his tongue will be talking of judgment. ℣. The law of his God is in his heart: and his goings shall not slide.}{Alleluia, alleluia. ℣. The just shall not be moved, for the Lord strengtheneth his hand. Alleluia.}

\offertory{I have found David my servant, with my holy oil have I anointed him: my hand shall hold him fast, and my arm shall strengthen him.}

\secret
\lett{G}{rant,} we beseech thee, O Lord: that, as by these gifts which we offer to the praise of thy name, we render honour to thy holy Martyr and Bishop Donatus, so by his intercession we may receive the reward of this our bounden service. Through.

\communion{A faithful and wise servant, whom the lord hath made ruler over his household: to give them their portion of meat in due season.}

\postcommunion
\lett{A}{lmighty} and merciful God, who makest us alike partakers and ministers of thy sacraments: grant, we beseech thee; that at the intercession of blessed Donatus, thy Martyr and Bishop, we, sharing in his faith and worthily serving thee, may be profited by the same. Through.

\subby{Sts. Cyriacus, Largus, \& Smaragdus}
\fancyhead[RO,LE]{\textit{Cyriacus \& Company}}
\fancyhead[RE,LO]{8 August}
\begin{inhead}
    {Memorial\\
8 August}
\end{inhead}
\begin{rubric}
	The Daily Office propers are from the First Common of Many Martyrs (p. \pageref{CommonMartyrsI}).
\end{rubric}

\introit
\lett{O}{fear} the Lord, ye that are his saints, for they that fear him lack nothing: the lions do lack, and suffer hunger: but they who seek the Lord shall want no manner of thing that is good. \textit{Ps.} I will alway give thanks unto the Lord: his praise shall ever be in my mouth.

\lett{O}{God,} who makest us glad with the yearly solemnity of thy holy Martyrs Cyriacus, Largus, and Smaragdus: mercifully grant; that as we now celebrate their birthday, so we may imitate their constancy in suffering. Through.

\begin{rubric}
	If today be Saturday, Commemoration is made of the anticipated Vigil of St. Lawrence, as on the following day; the \nth{3} Collect of St. Mary, and the Last Gospel of the Vigil.
\end{rubric}

\readingcitation{Epistle}{1 Thessalonians 2:13}
\lett{B}{rethren:} We thank God without ceasing, because, when ye received the word of God which ye heard of us, ye received it not as the word of men, but as it is in truth, the word of God, which effectually worketh also in you that believe. For ye, brethren, became followers of the churches of God which in Jud{\ae}a are in Christ Jesus: for ye also have suffered like things of your own countrymen, even as they have of the Jews: who both killed the Lord Jesus, and their own prophets, and have persecuted us; and they please not God, and are contrary to all men: forbidding us to speak to the Gentiles that they might be saved, to fill up their sins alway: for the wrath is come upon them to the uttermost.

\gradall{O fear the Lord, ye that are his saints: for they that fear him lack nothing. ℣. But they who seek the Lord, shall want no manner of thing that is good.}{Alleluia, alleluia. ℣. The righteous shall shine, and run to and fro like sparks among the stubble for ever. Alleluia.}

\readingcitation{Gospel}{Mark 16:15}
\lett{A}{t that time:} Jesus said to his disciples: Go ye into all the world, and preach the gospel to every creature. He that believeth and is baptized shall be saved; but he that believeth not shall be damned. And these signs shall follow them that believe; In my name shall they cast out devils; they shall speak with new tongues; they shall take up serpents; and if they drink any deadly thing, it shall not hurt them; they shall lay hands on the sick, and they shall recover.

\offertory{Be glad, O ye righteous, and rejoice in the Lord: and be joyful, all ye that are true of heart.}

\secret
\lett{G}{rant,} O Lord, that this our bounden service may be acceptable in thy sight: that these our oblations may, by the prayers of those on whose solemnity they are offered, be made profitable unto our salvation. Through.

\communion{These signs shall follow them that believe in me: they shall cast out devils: they shall lay hands on the sick, and they shall recover.}

\postcommunion
\lett{W}{e} beseech thee, O Lord our God, that like as we, whom thou hast refreshed by the partaking of thy sacred gift, do offer unto thee our worship: so by the intercession of thy holy Martyrs, Cyriacus, Largus and Smaragdus, we may perceive the benefit of the same. Through.

\subby{Vigil of St. Lawrence}
\fancyhead[RO,LE]{\textit{Lawrence Vigil}}
\fancyhead[RE,LO]{9 August}
\begin{inhead}
    {Vigil\\
9 August}
\end{inhead}

\introit
\lett{H}{e} hath dispersed abroad, and given to the poor: his righteousness remaineth for ever: his horn shall be exalted with honour. \textit{Ps.} Blessed is the man that feareth the Lord: he hath great delight in his commandments.
\begin{rubric}
	\emph{Gloria in excelsis} is not said.
\end{rubric}

\collect
\lett{A}{ssist} us, O Lord, in these our supplications: and at the intercession of thy blessed Martyr Lawrence, whose festival we prevent, graciously bestow upon us thy perpetual mercy. Through.
\needspace{4\baselineskip}
\lett{G}{rant,} we beseech thee, almighty God: that, at the intercession of blessed Romanus, thy Martyr, we may both be delivered from all adversities which may happen to the body, and from all evil thoughts which may assault and hurt the soul. Through.

\readingcitation{Epistle}{Ecclesiasticus 51:1}
%RV:
\lett{I}{will} give thanks unto thee, O Lord, O King, And will praise thee, O God my Saviour: I do give thanks unto thy name: For thou wast my protector and helper, And didst deliver my body out of destruction, And out of the snare of a slanderous tongue, From lips that forge lies, And wast my helper before them that stood by; And didst deliver me, according to the abundance of thy mercy, and greatness of thy name, From the gnashings of teeth ready to devour, Out of the hand of such as sought my life, Out of the manifold afflictions which I had; From the choking of a fire on every side, And out of the midst of fire which I kindled not; Out of the depth of the belly of the grave, And from an unclean tongue, And from lying words, The slander of an unrighteous tongue unto the king. My soul drew near even unto death, And my life was near to the grave beneath. They compassed me on every side, And there was none to help me. I was looking for the succour of men, And it was not. And I remembered thy mercy, O Lord, And thy working which hath been from everlasting, How thou deliverest them that wait for thee, And savest them out of the hand of the enemies, O Lord our God.

%\lett{I}{will} thank thee, O Lord and King, and praise thee, O God my Saviour. I do give praise unto thy name: for thou art my defender and helper, and hast preserved my body from destruction, and from the snare of the slanderous tongue, and from the lips that forge lies, and hast been mine helper against mine adversaries. And hast delivered me, according to the multitude of thy mercies and greatness of thy name, from the teeth of them that were ready to devour me, and out of the hands of such as sought after my life, and from the manifold afflictions which I had: from the choking of fire on every side, and from the midst of the fire which I kindled not: from the depth of the belly of hell, from an unclean tongue and from lying words. By an accusation to the king from an unrighteous tongue my soul drew near even unto death: but thou deliverest such as wait for thee, and savest them out of the hands of their enemies, O Lord our God.

\gradual{He hath dispersed abroad, and given to the poor: and his righteousness remaineth for ever. ℣. His seed shall be mighty upon earth: the generation of the faithful shall be blessed.}

\readingcitation{Gospel}{Matthew 16:24}
\lett{A}{t that time:} Jesus said unto his disciples: If any man will come after me, let him deny himself, and take up his cross, and follow me. For whosoever will save his life shall lose it: and whosoever will lose his life for my sake shall find it. For what is a man profited, if he shall gain the whole world, and lose his own soul? or what shall a man give in exchange for his soul? For the Son of man shall come in the glory of his Father with his angels; and then he shall reward every man according to his works.

\offertory{My prayer is pure: and therefore I ask that a place be given to my voice in heaven: for my witness is in heaven, and my record is on high: let my prayer ascend to the Lord.}

\secret
\lett{O}{Lord,} mercifully regard the sacrifices which we offer unto thee: and at the intercession of blessed Lawrence, thy Martyr, absolve us from the bonds of our sins. Through.
\needspace{4\baselineskip}
\lett{W}{e} beseech thee, O Lord, to accept our prayers and oblations: and graciously hearken unto us, whom thou dost cleanse by thy heavenly mysteries. Through.

\communion{He that will come after me, let him deny himself, and take up his cross, and follow me.}

\postcommunion
\lett{G}{rant,} we beseech thee, O Lord, our God: that like as we in this life do gladly honour the memory of blessed Lawrence, thy Martyr; so we may rejoice to behold him for ever. Through.
\needspace{4\baselineskip}
\lett{W}{e} beseech thee, almighty God: that we, who have received this heavenly food, may, at the intercession of blessed Romanus, thy Martyr, be thereby defended against all adversities. Through.


\supplement{10 August}{St. Lawrence}{}

\subby{Sts. Tiburtius \& Susanna}
\fancyhead[RO,LE]{\textit{Tiburtius \& Susanna}}
\fancyhead[RE,LO]{11 August}
\begin{inhead}
    {Memorial\\
11 August}
\end{inhead}
\begin{rubric}
	The propers are from the Third Common of Many Martyrs (p. \pageref{CommonMartyrsIII}), except for that which followeth.
\end{rubric}

\collect
\lett{O}{Lord,} let the protection of thy holy Martyrs Tiburtius and Susanna continually defend us: forasmuch as thou failest not to look with mercy on those to whom thou dost grant the succour of their assistance. Through.

\begin{rubric}
	The Epistle is Romans 8:18 from the Additional Epistles.
\end{rubric}

\secret
\lett{A}{ssist,} O Lord, the prayers of thy people, assist their oblations: that those things which are offered in these sacred mysteries may, by the intercession of thy Saints, be acceptable unto thee. Through.

\postcommunion
\lett{O}{Lord,} through whom we have received the pledge of eternal redemption: we beseech thee that at the intercession of thy holy Martyrs, it may avail for our succour both in this life and that which is to come. Through.

\subby{St. Maximus of Constantinople}
\fancyhead[RO,LE]{\textit{Maximus}}
\fancyhead[RE,LO]{13 August}
\begin{inhead}
    {Double\\
13 August}
\end{inhead}

\begin{rubric}
	The propers are from the Second Common of a Confessor not a Bishop (p. \pageref{CommonConfessorNotBishopII}).
\end{rubric}
\begin{rubric}
	Commemoration is made of Sts. Hippolytus \& Cassian with the following prayers.
\end{rubric}
\collect
\lett{G}{rant,} we beseech thee, almighty God: that the venerable solemnity of thy blessed Martyrs, Hippolytus and Cassian, may increase our devotion and set forward our salvation. Through.

\secret
\lett{R}{egard,} O Lord, the gifts of thy people, which we offer on the festival of thy Saints: and let this confession of thy truth be profitable for our salvation. Through.

\postcommunion
\lett{M}{ay} the communion of thy sacraments, O Lord, which we have received, avail for our salvation: stablish us in the light of thy truth. Through.

\begin{rubric}
	If today be Saturday, the anticipated Vigil of the Assumption of the B.V. Mary is kept, as is noted on the following day, but with Commemoration of Sts. Hippolytus and Cassian, MM., instead of St. Eusebius.
\end{rubric}

%The following is from Sarum, to match the BCP. It's mostly similar to the English Missal's text.
\subby{Vigil of the Assumption of the Blessed Virgin Mary}
\fancyhead[RO,LE]{\textit{Assumption Vigil}}
\fancyhead[RE,LO]{14 August}
\begin{inhead}
    {Vigil\\
14 August}
\end{inhead}

\introit
\lett{H}{ail,} Holy Mother, who didst bring forth the King who ruleth over heaven and earth for ever and ever. \textit{Gospel.} Blessed art thou among women, and blessed is the fruit of thy womb.

\collect
\lett{O}{God,} who didst vouchsafe to choose the virgin womb of blessed Mary wherein to make thy dwelling: grant, we beseech thee; that, being defended by her protection, we may by thee be enabled to attain with gladness to her festival. Who livest.
\needspace{4\baselineskip}
\lett{O}{God,} who makest us glad with the yearly solemnity of blessed Eusebius thy Confessor: mercifully grant; that we, who celebrate his birthday, may by his example, be drawn nearer unto thee. (Through.)
\begin{rubric}
	\nth{3} Collect of the Holy Ghost.
\end{rubric}

\readingcitation{Epistle}{Ecclesiasticus 24:9}
%RV:
\lett{H}{e} created me from the beginning before the world; And to the end I shall not fail. In the holy tabernacle I ministered before him; And so was I established in Sion. In the beloved city likewise he gave me rest; And in Jerusalem was my authority. And I took root in a people that was glorified, Even in the portion of the Lord's own inheritance. 

%\lett{H}{e} created me from the beginning before the world, and I shall never fail. In the holy tabernacle I served before him; and so was I established in Sion. Likewise in the beloved city he gave me rest, and in Jerusalem was my power. And I took root in an honourable people, even in the portion of the Lord's inheritance, and my abode is in the full congregation of the saints.

\gradual{Blessed and venerable art thou, O Virgin Mary: who without spot wast found the Mother of the Saviour. ℣. Virgin, Mother of God, he whom the whole world containeth not, being made man lay hid in thy womb.}

\readingcitation{Gospel}{Luke 11:27}
\lett{A}{t that time:} As Jesus spake to the multitudes, a certain woman of the company lifted up her voice, and said unto him, Blessed is the womb that bare thee, and the paps which thou hast sucked. But he said, Yea rather, blessed are they that hear the word of God, and keep it.

\offertory{Happy art thou, O holy Virgin Mary, and most worthy of all praise, for from thee sprang the Sun of Righteousness, Christ our God. ℣. Blessed art thou, Virgin Mary, who didst bear the Lord, didst give birth to the Creator of the world, Who made thee, and ever remainest Virgin.}

\secret
\lett{O}{Lord,} who didst translate the Mother of God from this present life, to the intent that she might faithfully intercede before thee for our sins: grant that her prayers may render these our oblations acceptable in the sight of thy mercy. Through the same.
\needspace{4\baselineskip}
\lett{G}{rant,} we beseech thee, O Lord, that we who, trusting in this our sacrifice of praise, do offer it before thee to the honour of thy Saints: may by the same be delivered from all evils both in this life and in that which is to come (Through.)
\begin{rubric}
	\nth{3} Secret of the Holy Ghost.
\end{rubric}

\communion{Gentle Mother of God, succour all that pray; we, too, with them, humbly entreat that by the aid of thy prayers we may sing praises unto the Trinity.}

%While the English Missal's text and the Sarum's are very similar, the Sarum is used because of its distinctive use of `requiem'.
\postcommunion
\lett{W}{e} beseech thee, O merciful God, to strengthen our frailty, that we who keep the requiem of the Holy Virgin Mother of God, may by her intercession rise again from our iniquities. Through.
\needspace{4\baselineskip}
\lett{O}{Lord,} our God, who hast refreshed us with heavenly meat and drink, we humbly beseech thee: that we may be defended by the prayers of him in whose memory we have received the same. (Through.)
\begin{rubric}
	\nth{3} Postcommunion of the Holy Ghost.
\end{rubric}

\begin{center}
	[15 August. Assumption of the Blessed Virgin Mary]
\end{center}

\begin{center}
	[16 August. St. Joachim]
\end{center}

\subby{St. Helen}
\fancyhead[RO,LE]{\textit{Helen}}
\fancyhead[RE,LO]{18 August}
\begin{inhead}
    {Memorial\\
18 August}
\end{inhead}

\begin{rubric}
	When celebrated within the Octave of the Assumption, this Memorial is merely commemorated through its Collect in the Office and its Collect, Secret, and Postcommunion at Mass.
\end{rubric}
\begin{rubric}
	The propers come from the Common of Neither Virgin nor Martyr (p. \pageref{CommonNeitherVirginMartyr}), except for that which followeth.
\end{rubric}

\introit
\lett{B}{ut} God forbid that I should glory, save in the Cross of our Lord Jesus Christ: by whom the world is crucified unto me, and I unto the world. \textit{Ps.} Thy rod and thy staff comfort me.

\collect
\lett{O}{Lord} Jesu Christ, who didst reveal unto blessed Helen the place where thy Cross lay hid, that through her thou mightest enrich thy Church with this precious treasure: grant unto us at her intercession; that by the ransom of the life-giving tree we may attain unto the rewards of everlasting life. (Who livest.)

\begin{rubric}
	In Septuagesimatide \& Lent, the following Tract replaces the Gradual \& Lesser Alleluia.
\end{rubric}
\tract{The rich among the people shall make their supplication before thee: Kings' daughters were among thine honourable women. ℣. The Virgins that be her fellows shall be brought unto the King: they that bear her company shall be brought unto thee. ℣. With joy and gladness shall they be brought: they shall enter into the palace of the King.}

\begin{rubric}
	In Eastertide, the following Alleluia Verse replaces the Gradual \& Lesser Alleluia.
\end{rubric}
\alleluia{Alleluia, alleluia. ℣. She hath dispersed abroad, and given to the poor: and her righteousness remaineth for ever. Alleluia. ℣. In thy comeliness and in thy beauty go forth, proceed prosperously and reign. Alleluia.}

\offertory{For I determined not to know any thing, save Jesus Christ, and him crucified.}

\secret
\lett{T}{hrough} these sacred mysteries vouchsafe unto us, O Lord: that as in thy mercy thou didst grant unto blessed Helen ever to carry thy Son crucified in her heart; so we may likewise continually hear him in our hearts. (Who liveth and reigneth with thee.)

\communion{I will go up to the palm tree, I will take hold of the boughs thereof.}

\postcommunion
\lett{G}{rant} unto us, O merciful God: that we who have been refreshed by the benefits of thy life-giving Cross on earth; may through the intercession of blessed Helen attain unto the eternal fruition of the same in heaven. (Who livest.)

\subby{St. Agapitus}
\fancyhead[RO,LE]{\textit{Agapitus}}
\fancyhead[RE,LO]{18 August}
\begin{inhead}
    {Memorial\\
18 August}
\end{inhead}
\begin{rubric}
	When celebrated within the Octave of the Assumption, this Memorial is merely commemorated through its Collect in the Office and its Collect, Secret, and Postcommunion at Mass.\par
	When they are celebrated on the same day, the Feast of St. Helen has priority.
\end{rubric}

\begin{rubric}
	The propers come from the Second Common of a Martyr not a Bishop (p. \pageref{CommonMartyrNotBishopII}), except for that which followeth.
\end{rubric}

\collect
\lett{O}{Lord,} let thy Church trust with gladness in the advocacy of thy blessed Martyr Agapitus: that by his glorious prayers she may continue in devotion, and abide in safety. (Through.)

\begin{rubric}
	The Gospel is of the Feast of St. Lawrence (10 August).
\end{rubric}

\secret
\lett{R}{eceive,} O Lord, the gifts which we offer on the solemnity of him, through whose advocacy we trust to be delivered. (Through.)

\postcommunion
\lett{O}{Lord,} who hast satisfied thy family with sacred gifts: we beseech thee; that we may at all times be comforted by the intercession of him whose festival we celebrate. (Through.)

\subby{Octave Day of the Assumption of the Blessed Virgin Mary}
\fancyhead[RO,LE]{\textit{Assumption Octave}}
\fancyhead[RE,LO]{22 August}
\begin{inhead}
    {Greater Double\\
22 August}
\end{inhead}
\begin{rubric}
	Mass as on the Feast with commemoration of Sts. Timothy, Hippolytus, \& Symphorian, as in the following Mass.
\end{rubric}
\begin{rubric}
	If today be Saturday, the anticipated Vigil of St. Bartholomew is kept, as is noted on the following day.
\end{rubric}

\subby{Sts. Timothy, Hippolytus, \& Symphorian}
\fancyhead[RO,LE]{\textit{Timothy \& Company}}
\fancyhead[RE,LO]{22 August}
\begin{inhead}
    {Memorial\\
22 August}
\end{inhead}

\begin{rubric}
	The propers come from the Third Common of Many Martyrs (p. \pageref{CommonMartyrsIII}), except for that which followeth.
\end{rubric}

\collect
\lett{W}{e} beseech thee, O Lord, graciously to impart unto us thy help: and, at the intercession of thy blessed Martyrs Timothy, Hippolytus, and Symphorian, stretch forth upon us the right hand of thy mercy. Through.

\secret
\lett{G}{rant,} O Lord, that like as thy dedicated people do acknowledge that in tribulation they have been succoured by the merits of thy Saints: so this oblation, which they offer unto thee in honour of the same, may be acceptable in thy sight. Through.

\postcommunion
\lett{O}{Lord} our God, who hast fulfilled us with the bounty of thy heavenly gift: we beseech thee, that, at the intercession of thy holy Martyrs, Timothy, Hippolytus, and Symphorian, we may ever live by the partaking of the same. Through.


\subby{Vigil of St. Bartholomew}
\fancyhead[RO,LE]{\textit{Bartholomew Vigil}}
\fancyhead[RE,LO]{23 August}
\begin{inhead}
    {Vigil\\
23 August}
\end{inhead}

\begin{rubric}
	The propers come from the Common of the Vigil of Apostles (p. \pageref{CommonVigilApostles}).\par
	\textsc{Note,} \nth{2} Collect of St. Mary \& \nth{3} against the Persecutors of the Church or for the Chief Bishop.
\end{rubric}


\begin{center}
	[24 August. St. Bartholomew]
\end{center}

\subby{St. Zephyrinus}
\fancyhead[RO,LE]{\textit{Zephyrinus}}
\fancyhead[RE,LO]{26 August}
\begin{inhead}
    {Memorial\\
26 August}
\end{inhead}

\begin{rubric}
	The propers come from the Second Common of a Martyr Bishop out of Eastertide (p. \pageref{CommonMartyrBishopII}), except for that which followeth.
\end{rubric}

\collect
\lett{G}{rant,} we beseech thee, almighty God: that we who rejoice in the merits of blessed Zephyrinus, thy Martyr and Bishop, may be instructed by his example. Through.

\secret
\lett{S}{anctify,} O Lord, the gifts which we dedicate to thee: that at the intercession of blessed Zephyrinus thy Martyr and Bishop they may obtain for us thy gracious favour. Through.

\postcommunion
\lett{M}{ay} this communion, O Lord, cleanse us from guilt: and, at the intercession of blessed Zephyrinus, thy Martyr and Bishop, make us partakers of thy heavenly healing. Through.

\subby{St. Augustine of Hippo}
\fancyhead[RO,LE]{\textit{Augustine}}
\fancyhead[RE,LO]{28 August}
\begin{inhead}
    {Greater Double\\
28 August}
\end{inhead}
\begin{rubric}
	The propers are from the Common of Doctors (p. \pageref{CommonDoctors}), except for that which followeth.
\end{rubric}

\introit
\lett{I}{n} the midst of the Church he opened his mouth: and the Lord filled him with the spirit of wisdom and of understanding: he clothed him with a robe of glory. \textit{Ps.} It is a good thing to give thanks unto the Lord: and to sing praises unto thy name, O most Highest.

\collect
\lett{A}{ssist} us, almighty God, in these our supplications: that as thou dost suffer us to put our trust and confidence in thy mercy, so, at the intercession of blessed Augustine thy Confessor and Bishop, thou wouldest graciously vouchsafe unto us the wonted effects of thy compassion. Through.
\needspace{4\baselineskip}
\lett{O}{God,} who didst strengthen blessed Hermes thy Martyr with the virtue of constancy in his passion: Grant unto us by his example; to despise for love of thee the prosperity of the world, and to fear none of its adversities. Through.

\gradall{The mouth of the righteous is exercised in wisdom, and his tongue will be talking judgment. ℣. The law of his God is in his heart: and his goings shall not slide.}{Alleluia, alleluia. ℣. I have found David my servant, with my holy oil have I anointed him. Alleluia.}

\offertory{The righteous shall flourish like a palm-tree: and shall spread abroad like a cedar in Libanus.}

\secret
\lett{M}{ay} the devout prayers of thy Bishop and Doctor, Saint Augustine, never fail to succour us, O Lord: that they may render our oblations acceptable in thy sight; and may ever obtain for us thy merciful pardon. Through.
\needspace{4\baselineskip}
\lett{W}{e} offer unto thee, O Lord, this sacrifice of praise in commemoration of thy Saints: grant, we beseech thee; that as it hath bestowed glory on them, so may it avail for our salvation. Through.

\communion{A faithful and wise servant, whom the lord hath made ruler over his household: to give them their portion of meat in due season.}

\postcommunion
\lett{W}{e} beseech thee, O Lord, that blessed Augustine, thy Bishop and illustrious Doctor, may stand before thee as our advocate: that these thy sacrifices may avail for our salvation. Through.
\needspace{4\baselineskip}
\lett{O}{Lord,} who hast fulfilled us with thy heavenly benediction, we beseech thy mercy: that, at the intercession of thy blessed Martyr Hermes, this service of our lowliness may avail for the comforting of our souls. Through.

\subby{St. Hermes}
\fancyhead[RO,LE]{\textit{Hermes}}
\fancyhead[RE,LO]{28 August}
\begin{inhead}
    {Memorial\\
28 August}
\end{inhead}

\begin{rubric}
	The propers are from the Second Common of a Martyr not a Bishop (p. \pageref{CommonMartyrNotBishopII}), except for his Prayers above.
\end{rubric}

\subby{St. Sabina}
\fancyhead[RO,LE]{\textit{Sabina}}
\fancyhead[RE,LO]{29 August}
\begin{inhead}
    {Memorial\\
29 August}
\end{inhead}

\begin{rubric}
	The propers are from the Common of a Martyr not a Virgin (p. \pageref{CommonMartyrNotVirgin}), except for her Prayers above.
\end{rubric}

\subby{Sts. Felix \& Adauctus}
\fancyhead[RO,LE]{\textit{Felix \& Adauctus}}
\fancyhead[RE,LO]{30 August}
\begin{inhead}
    {Memorial\\
30 August}
\end{inhead}

\begin{rubric}
	The Daily Office Antiphons are from the First Common of Many Martyrs (p. \pageref{CommonMartyrsI}).
\end{rubric}

\introit
\lett{L}{et} the people tell of the wisdom of the Saints, and let the church shew forth their praise: their names shall live for evermore. \textit{Ps.} Rejoice in the Lord, O ye righteous: for it becometh well the just to be thankful.

\collect
\lett{O}{Lord,} we humbly entreat thy majesty: that, like as thou dost continually gladden us with the commemoration of thy Saints; so thou wouldest evermore defend us with their supplication. Through.

\begin{rubric}
	The Epistle is Wisdom 10:17 from the Third Common of Many Martyrs (p. \pageref{CommonMartyrsIII}).
\end{rubric}

\gradall{The souls of the just are in the hand of God, and there shall no torment of malice touch them. ℣. In the eyes of the unwise they seemed to die: but they are in peace.}{Alleluia, alleluia. ℣. The righteous shall shine forth, and run to and fro like sparks among the stubble for ever. Alleluia.}

\begin{rubric}
	The Gospel is Luke 10:16 from the Third Common of Many Martyrs (p. \pageref{CommonMartyrsIII}).
\end{rubric}

\offertory{Be glad, O ye righteous, and rejoice in the Lord: and be joyful, all ye that are true of heart.}

\secret
\lett{L}{ook} down, O Lord, upon the sacrifices of thy people: that, as with devout hearts they celebrate them to the honour of thy Saints, so they may perceive them to be profitable to their salvation. Through.

\communion{What I tell you in darkness, that speak ye in light, saith the Lord: and what ye hear in the ear, that preach ye upon the housetops.}

\postcommunion
\lett{W}{e} beseech thee, O Lord: that we, being filled with the sacred gifts; may at the intercession of thy Saints ever continue in thanksgiving for the same. Through.

\subby{St. Aidan of Lindisfarne}
\fancyhead[RO,LE]{\textit{Aidan}}
\fancyhead[RE,LO]{31 August}
\begin{inhead}
    {Memorial\\
31 August}
\end{inhead}

\begin{rubric}
	The propers are from the First Common of a Confessor Bishop (p. \pageref{CommonConfessorBishopI}).
\end{rubric}

\subby{St. Giles}
\fancyhead[RO,LE]{\textit{Giles}}
\fancyhead[RE,LO]{1 September}
\begin{inhead}
    {Simple\\
1 September}
\end{inhead}

\begin{rubric}
	The propers are from the First Common of Confessor not Bishop (p. \pageref{CommonConfessorNotBishopI}), except for that which followeth.\par
	Commemoration is made of the Twelve Brothers.
\end{rubric}


\collect
\lett{O}{Lord,} we beseech thee, let the intercession of the blessed Abbot Giles commend us unto thee: that those things which for our own merits we cannot ask, we may through his advocacy obtain. Through.


\secret
\lett{W}{e} beseech thee, O Lord that thy holy Abbot Giles may intercede for us that this sacrifice which we offer and present upon thy holy altar may be profitable unto us for our salvation. Through.

\postcommunion
\lett{L}{et} thy sacrament, O Lord, which we have now received and the prayers of the blessed Abbot Giles effectually defend us: that we may both imitate the example of his conversation, and receive the succour of his intercession. Through.

\subby{St. Stephen of Hungary}
\fancyhead[RO,LE]{\textit{Stephen}}
\fancyhead[RE,LO]{2 September}
\begin{inhead}
	{Simple\\
		2 September}
\end{inhead}

\begin{rubric}
	The Mass propers are from the First Common of Confessor not Bishop (p. \pageref{CommonConfessorNotBishopI}), with the optional Gospel \emph{A certain nobleman}, and with the Prayers which followeth.
\end{rubric}

\collect
\lett{G}{rant,} we beseech thee, almighty God unto thy Church: that as thy blessed Confessor Stephen, while he reigned on earth, did spread abroad her faith, so she may be found worthy to have him for her glorious defender in the heavens. Through.

\secret
\lett{A}{lmighty} God, look upon the sacrifices which we offer: and grant; that we, who celebrate the mysteries of the Passion of the Lord, may imitate that which we perform. Through the same.

\postcommunion
\lett{G}{rant,} we beseech thee, almighty God: that, as thy blessed Confessor Stephen, for the propagation of thy faith, was counted worthy to pass from an earthly kingdom to the glory of the heavenly realm; so we may imitate his faith with due devotion. Through.

\subby{St. Gorazde of Prague}
\fancyhead[RO,LE]{\textit{Gorazde}}
\fancyhead[RE,LO]{4 September}
\begin{inhead}
	{Double\\
		4 September}
\end{inhead}

\begin{rubric}
	The propers are from the Second Common of Bishop \& Martyr (p. \pageref{CommonMartyrBishopII}.
\end{rubric}

\subby{St. Gorgonius}
\fancyhead[RO,LE]{\textit{Gorgonius}}
\fancyhead[RE,LO]{9 September}
\begin{inhead}
	{Simple\\
		9 September}
\end{inhead}

\begin{rubric}
	The Mass propers are from the Second Common of One Martyr not a Bishop out of Eastertide (p. \pageref{CommonMartyrNotBishopII}) with the following Prayers.
\end{rubric}

\collect
\lett{L}{et} thy Saint Gorgonius, O Lord, gladden us by his intercession; and make us to rejoice in this holy solemnity. Through.

\secret
\lett{O}{Lord,} let thy holy Martyr Gorgonius so intercede for us: that this oblation of our service may be acceptable unto thee. Through.

\postcommunion
\lett{G}{rant,} O God, that thy household may be quickened and refreshed by thine eternal goodness: and in thy Martyr Gorgonius be continually nourished with the sweet savour of Christ, thy Son. Who liveth and reigneth with thee.

\subby{Sts. Protus \& Hyacinth}
\fancyhead[RO,LE]{\textit{Protus \& Hyacinth}}
\fancyhead[RE,LO]{11 September}
\begin{inhead}
	{Simple\\
		11 September}
\end{inhead}

\begin{rubric}
	The propers are from the Third Common of Many Martyrs out of Eastertide (p. \pageref{CommonMartyrsIII}) with the following Prayers.
\end{rubric}

\collect
\lett{O}{Lord,} let the meritorious confession of thy blessed Martyrs, Protus and Hyacinth, comfort us; and may their loving intercession ever defend us. Through.

\secret
\lett{G}{rant,} we beseech thee, O Lord, that the oblation of our bounden service which we ofier unto thee for the commemoration of thy holy Martyrs, Protus and Hyacinth; may effectually avail for our healing unto everlasting salvation. Through.

\postcommunion
\lett{W}{e} beseech thee, O Lord, that by the supplication of thy blessed Martyrs, Protus and Hyacinth: thy holy mysteries which we have received may avail for our cleansing. Through.

\subby{Most Holy Name of Mary}
\fancyhead[RO,LE]{\textit{Most Holy Name of Mary}}
\fancyhead[RE,LO]{12 September}
\begin{inhead}
	{Greater Double\\
		12 September}
\end{inhead}

\begin{rubric}
	The Daily Office propers are from the Common of the Blessed Virgin Mary (p. \pageref{CommonBVM}), except for the antiphon for I Evensong, as followeth.
\end{rubric}

%Last clause translated from the Diurnale Monasticum:
\antiphon{Mag.}{O holy Mary, {\dag} help thou the suffering, strengthen the faint-hearted, comfort the sorrowful; pray for the people, entreat for the clergy, intercede for all womankind vowed unto God: may all acknowledge the help of thy prayer, who celebrate the commemoration of thy holy Name.}\par\noindent

\introit
\lett{A}{ll} the rich among the people shall make their supplication before thee: the Virgins that be her fellows shall be brought unto the King: they that bear her company shall be brought unto thee with joy and gladness. \textit{Ps.} My heart is inditing of a good matter: I speak of the things which I have made unto the King.

\collect
\lett{G}{rant,} we beseech thee, almighty God: that thy faithful people, who rejoice in the Name and protection of the most Holy Virgin Mary; may by her loving intercession be delivered from all evils upon earth, and be found worthy to attain unto everlasting joys in heaven. Through.

\readingcitation{Epistle}{Ecclesiasticus 24:17}
%KJV:
%\lett{I}{as} the vine brought forth pleasant savour, and my flowers are the fruit of honour and riches. I am the mother of fair love, and fear, and knowledge, and holy hope: I therefore, being eternal, am given to all my children which are named of him. Come unto me, all ye that be desirous of me, and fill yourselves with my fruits. For my memorial is sweeter than honey, and mine inheritance than the honeycomb. They that eat me shall yet be hungry, and they that drink me shall yet be thirsty. He that obeyeth me shall never be confounded, and they that work by me shall not do amiss.
%RV:
\lett{A}{s} the vine I put forth grace; And my flowers are the fruit of glory and riches. Come unto me, ye that are desirous of me, And be ye filled with my produce. For my memorial is sweeter than honey, And mine inheritance than the honeycomb. They that eat me shall yet be hungry; And they that drink me shall yet be thirsty. He that obeyeth me shall not be ashamed; And they that work in me shall not do amiss.

\gradall{Blessed and venerable art thou, O Virgin Mary: who without spot wast found the Mother of the Saviour. ℣. Virgin, Mother of God, he whom the world containeth not, being made man lay hid in thy womb.}{Alleluia, alleluia. ℣. After child-birth, O Virgin, thou didst remain inviolate: Mother of God, intercede for us. Alleluia.}

\readingcitation{Gospel}{Luke 1:26}
\lett{A}{t that time:} The angel Gabriel was sent from God unto a city of Galilee, named Nazareth, to a virgin espoused to a man whose name was Joseph, of the house of David; and the virgin’s name was Mary. And the angel came in unto her, and said, Hail, thou that art highly favoured, the Lord is with thee: blessed art thou among women. And when she saw him, she was troubled at his saying, and cast in her mind what manner of salutation this should be. And the angel said unto her, Fear not, Mary: for thou hast found favour with God. And, behold, thou shalt conceive in thy womb, and bring forth a son, and shalt call his name \divineName{Jesus}. He shall be great, and shall be called the Son of the Highest: and the Lord God shall give unto him the throne of his father David: and he shall reign over the house of Jacob for ever; and of his kingdom there shall be no end. Then said Mary unto the angel, How shall this be, seeing I know not a man? And the angel answered and said unto her, The Holy Ghost shall come upon thee, and the power of the Highest shall overshadow thee: therefore also that holy thing which shall be born of thee shall be called the Son of God. And, behold, thy cousin Elisabeth, she hath also conceived a son in her old age: and this is the sixth month with her, who was called barren. For with God nothing shall be impossible. And Mary said, Behold the handmaid of the Lord; be it unto me according to thy word.

\offertory{Hail, Mary, full of grace; the Lord is with thee: blessed art thou among women, and blessed is the fruit of thy womb.}

\secret
\lett{T}{hrough} thy mercy, O Lord, and the intercession of blessed Mary ever Virgin, may this oblation avail for our prosperity and peace, both now and for ever. Through.

\communion{Blessed is the womb of the Virgin Mary, that bore the Son of the everlasting Father.}

\postcommunion
\lett{G}{rant,} we beseech thee, O Lord: that we, who have received these aids to our salvation, may at all times and in all places be protected through the advocacy of blessed Maty ever Virgin; in whose honour we have made these offerings to thy majesty. Through.



\subby{Sts. Cornelius \& Cyprian}
\fancyhead[RO,LE]{\textit{Cornelius \& Cyprian}}
\fancyhead[RE,LO]{16 September}
\begin{inhead}
	{Double\\
		16 September}
\end{inhead}

\begin{rubric}
	The propers are from the First Common of Many Martyrs out of Eastertide (p. \pageref{CommonMartyrsI}) with the following Prayers.
\end{rubric}

\collect
\lett{P}{rotect} us, O Lord, we beseech thee, who observe the feast of thy blessed Martyrs and Bishops Cornelius and Cyprian: and grant that by their meritorious supplication we may find favour in thy sight. Through.

\lett{G}{rant,} O Lord, that our prayers in this time of our rejoicing may be brought to good effect: that as with yearly service we recall the day of the passion of thy holy Martyrs, Euphemia, Lucy, and Geminian, so we may imitate the steadfastness of their faith. Through.

\secret
\lett{A}{ssist} us mercifully, O Lord, in these our supplications which we make before thee in remembrance of thy Saints: that we who trust not in our own righteousness may be succoured by the merits of them that have found favour in thy sight. Through.

\lett{G}{raciously} hearken, we beseech thee, O Lord, unto the prayers of thy people: and make us to rejoice in the intercession of those, whose festival thou dost suffer us to celebrate. Through.

\postcommunion
\lett{O}{Lord,} who hast fulfilled us with saving mysteries, we beseech thee: that we may be aided by the prayers of those whose festival we celebrate. Through.

\lett{O}{Lord,} graciously hear our prayers: that we, who solemnly observe the feast of thy holy Martyrs, Euphemia, Lucy, and Geminian, may be succoured by their continual help. Through.

\begin{rubric}
	The Mass for Sts. Euphemia, Lucy, \& Geminian is said as above with the relevant Prayers, but with the Gospel of the Second Common of Many Martyrs (p. \pageref{CommonMartyrsII}).
\end{rubric}

\subby{St. Januarius \& Companions}
\fancyhead[RO,LE]{\textit{Januarius}}
\fancyhead[RE,LO]{19 September}
\begin{inhead}
	{Double\\
		19 September}
\end{inhead}
%%%%%
\begin{rubric}
	The propers are from the First Common of Many Martyrs out of Eastertide (p. \pageref{CommonMartyrsI}) with the following Prayers.
\end{rubric}


%%%%%%%%%%%%%%


\subby{St. Jerome}
\fancyhead[RO,LE]{\textit{Jerome}}
\fancyhead[RE,LO]{30 September}
\begin{inhead}
	{Double\\
		30 September}
\end{inhead}
%%%%%
\begin{rubric}
	The propers are from the First Common of Many Martyrs out of Eastertide (p. \pageref{CommonMartyrsI}) with the following Prayers.\par
	\textsc{Note,} The Creed is said.
\end{rubric}

\collect
\lett{O}{God,} who for the exposition of the sacred Scriptures didst bestow upon thy Church blessed Jerome, thy Confessor and most illustrious Doctor: grant, we beseech thee; that, by the intercession of his merits, we may through thine assistance be enabled to perform those things which he taught both in word and deed. Through.

\secret
\lett{G}{rant} us, we beseech thee, O Lord, through these heavenly gifts to serve thee in freedorn of spirit: that the gifts which we offer may, through the mediation of blessed Jerome, thy Confessor, work in us both healing and glory. Through.

\postcommunion
\lett{W}{e} beseech thee, O Lord, that we whom thou hast fulfilled with heavenly nourishment: may through the mediation of blessed Jerome, thy Confessor, be found worthy to obtain the grace of thy loving-kindness. Through.

\subby{St. Remigius}
\fancyhead[RO,LE]{\textit{Remigius}}
\fancyhead[RE,LO]{1 October}
\begin{inhead}
	{Simple\\
		1 October}
\end{inhead}
%%%%%CHECK:
\begin{rubric}
	The propers are from the Common of a Confessor Bishop (p. \pageref{CommonConfessorBishopI}).
\end{rubric}



%%%%%%%%


\subby{St. Mark of Rome}
\fancyhead[RO,LE]{\textit{Mark of Rome}}
\fancyhead[RE,LO]{7 October}
\begin{inhead}
	{Simple\\
		7 October}
\end{inhead}

\begin{rubric}
	The propers are from the Second Common of a Confessor Bishop (p. \pageref{CommonConfessorBishopII}) with the following Prayers.
\end{rubric}

\collect
\lett{G}{raciously} hear our prayers, O Lord: and at the intercession of blessed Mark, thy Confessor and Bishop, mercifully grant us pardon and peace. Through.

\lett{M}{ay} the blessed merits of thy holy Martyrs Sergius, Bacchus, Marcellus, and Apuleius uphold us, O Lord: and ever make us fervent in thy love. Through.

\secret
\lett{G}{rant,} O Lord, that like as thy dedicated people do acknowledge that in tribulation they have been succoured by the merits of thy Saints: so this oblation, which they offer unto thee in honour of the same, may be acceptable in thy sight. Through.

\lett{W}{e} beseech thee, O Lord, that, through the meritorious supplication of thy Saints: this sacrifice, which we offer, may obtain for us the favour of thy majesty. Through.

\postcommunion
\lett{G}{rant,} we beseech thee, O Lord, that thy faithful people may ever rejoice in the veneration of thy Saints: and be defended by their perpetual supplication. Through.

\lett{G}{rant,} O Lord, that, being strengthened by the sacraments which we have received: we may, at the intercession of thy holy Martyrs, Sergius, Bacchus, Marcellus, and Apuleius, fight against all iniquity, and be defended by thy heavenly armour. Through.

\begin{rubric}
	The Mass for Sts. Sergius, Bacchus, Marcellus, and Apuleius is said as in the Second Common of Many Martyrs (p. \pageref{CommonMartyrsII}), with the relevant Prayers as above.
\end{rubric}


%%%%%%%%%%%%%



\subby{St. Clement}
\fancyhead[RO,LE]{\textit{Clement}}
\fancyhead[RE,LO]{23 November}
\begin{inhead}
	{Double\\
		23 November}
\end{inhead}

\antiphon{Ben.}{When he had started going to the sea, {\dag} the people cried out with loud voices, O Lord Jesus Christ, save him: and Clement responded, weeping, Heavenly Father, receive my spirit.}

\begin{rubric}
	The Magnificat antiphon is of the Common of Many Martyrs (p. \pageref{CommonMartyrsI}).
\end{rubric}

\introit
\lett{T}{he} Lord saith: My words which I have put in thy mouth, shall not depart out of thy mouth: and thy gifts shall be accepted upon mine altar. \textit{Ps.} Blessed is the man that feareth the Lord: he hath great delight in his commandments.

\collect
\lett{O}{God,} who makest us glad with the yearly solemnity of blessed Clement thy Martyr and Bishop: mercifully grant; that as we now celebrate his birthday, so we may imitate his constancy in suffering. Through.

\lett{G}{rant,} we beseech thee, almighty God: that we, celebrating the festival of blessed Felicity, thy Martyr, may be protected by her merits and prayers. Through.

\readingcitation{Epistle}{Philippians 3:17}
\lett{B}{rethren:} Be followers together of me, and mark them which walk so as ye have us for an ensample. (For many walk, of whom I have told you often, and now tell you even weeping, that they are the enemies of the cross of Christ: Whose end is destruction, whose God is their belly, and whose glory is in their shame, who mind earthly things.) For our conversation is in heaven; from whence also we look for the Saviour, the Lord Jesus Christ: Who shall change our vile body, that it may be fashioned like unto his glorious body, according to the working whereby he is able even to subdue all things unto himself. Therefore, my brethren dearly beloved and longed for, my joy and crown, so stand fast in the Lord, my dearly beloved. I beseech Euodias, and beseech Syntyche, that they be of the same mind in the Lord. And I intreat thee also, true yokefellow, help those women which laboured with me in the gospel, with Clement also, and with other my fellowlabourers, whose names are in the book of life.

\gradall{The Lord sware, and will not repent: Thou art a priest for ever, after the order of Melchisedech. ℣. The Lord said unto my Lord: Sit thou on my right hand.}{Alleluia, alleluia. ℣. This is a priest whom the Lord hath crowned. Alleluia.}

\begin{rubric}
	The Gospel is of the Second Common of a Confessor Bishop (\pageref{CommonConfessorBishopII}).
\end{rubric}

\offertory{My truth and my mercy shall be with him: and in my name shall his horn be exalted.}

\secret
\lett{S}{anctify,} O Lord, the gifts which we offer unto thee: and, at the intercession of blessed Clement, thy Martyr and Bishop, cleanse us thereby from the defilement of our sins. Through.

\lett{G}{raciously} hearken, O Lord, to the prayers of thy people: and make us to rejoice in the intercession of her whose festival thou dost grant unto us to celebrate. Through.

\communion{Blessed is the servant, whom the Lord when he cometh shall find watching: verily I say unto you, that he shall make him ruler over all his goods.}

\postcommunion
\lett{O}{Lord} our God, who hast fulfilled us with the partaking of the sacred Body, and the precious Blood, we beseech thee: that those things which we perform with godly devotion, we may at the intercession of blessed Clement, thy Martyr and Bishop, attain in the assurance of our redemption. Through.

\lett{W}{e} humbly beseech thee, almighty God: that by the intercession of thy Saints thou wouldest both multiply in us thy gifts, and likewise dispose our times according to thy will. Through.

\subby{St. Felicity}
\fancyhead[RO,LE]{\textit{Felicity}}
\fancyhead[RE,LO]{23 November}
\begin{inhead}
    {Memorial\\
23 November}
\end{inhead}

\begin{rubric}
	The propers are from the Common of a Martyr not a Virgin (p. \pageref{CommonMartyrNotVirgin}), except for her Prayers above.
\end{rubric}