\phantomsection
\addcontentsline{toc}{section}{Proper of Season}
\fancyhead[C]{\LARGE Proper of Season}

\begin{secrubric}
	Except during Advent \& Lent, the preceding Sunday's propers may be used for a weekday Feria.
\end{secrubric}



\subby{Magnificat Antiphons before Christmas}
\feastday{Advent Antiphons}
\begin{rubric}
    The Magnificat Antiphon is always taken from here on these days.
\end{rubric}
\begin{multicols}{2}
\subbysub{16 December: O Sapientia}
\lett{O}{Wisdom,} {\dag} which camest out of the mouth of the Most High, and reachest from one end to another, mightily and sweetly ordering all things: Come and teach us the way of prudence.
\subbysub{17 December: O Adonai}
\lett{O}{Adonai,} {\dag} and Leader of the house of Israel, who appearedst in the bush to Moses in a flame of fire, and gavest him the law in Sinai: Come and redeem us with an outstretched arm.
\subbysub{18 December: O Radix Jesse}
\lett{O}{Root of Jesse,} {\dag} which standest for an ensign of the people, at whom kings shall shut their mouths, unto whom the Gentiles shall seek: Come and deliver us, and tarry not.
\subbysub{19 December: O Clavis David}
\lett{O}{Key of David,} {\dag} and Sceptre of the house of Israel; that openest and no man shutteth, and shuttest and no man openeth: Come, and bring the prisoners out of the prison-house, them that sit in darkness and the shadow of death.
\subbysub{20 December: O Oriens}
\lett{O}{Day-spring,} {\dag} Brightness of the Light everlasting, and Sun of righteousness: Come and enlighten them that sit in darkness and the shadow of death.
\subbysub{21 December: O Rex gentium}
\lett{O}{King of Nations,} {\dag} and their Desire; the Cornerstone, who makest both one: Come and save mankind, whom thou formedst of clay.
\subbysub{22 December: O Emmanuel}
\lett{O}{Emmanuel,} {\dag} our King and Lawgiver, the Desire of all nations and their Salvation: Come and save us, O Lord our God.
\subbysub{23 December: O Virgo virginum}
\lett{O}{Virgin of Virgins,} {\dag} how shall this be? for neither before thee was any seen like thee, nor shall there be after. Daughters of Jerusalem, why marvel ye at me? The thing which ye behold is a divine mystery.
\end{multicols}

\supplement{25 December}{Nativity}{of Our Lord}

\begin{paracol}{2}[]
\sloppy
\begin{inhead}
	I Evensong
\end{inhead}
\begin{hangparas}{1.25em}{1}
	
Jesu, the Father's only Son,

Whose death for all redemption won,

Before the worlds, of God Most High,

Begotten all ineffably.\\

The Father's Light and Splendour thou,

Their endless Hope to thee that bow:

Accept the pray'rs and praise to-day

That through the world thy servants pay.\\

Salvation's Author, call to mind

Thou took'st the form of humankind,

When of the Virgin undefil'd

Thou in man's flesh becam'st a Child.\\

Thus testifies the present day

Through every year in long array,

That thou, salvation's source alone,

Proceededst from the Father's Throne.\\

Whence sky, and stars, and sea's abyss,

And earth, and all that therein is,

Shall still, with laud and carol meet,

The Author of thine Advent greet.\\

And we who, by thy precious Blood

From sin redeem'd, are mark'd for God,

On this, the day that saw thy Birth,

Sing the new song of ransom'd earth.\\

All honour, laud, and glory be,

O Jesu, Virgin-born, to thee;

All glory, as is ever meet,

To Father and to Paraclete. Amen.\\
\end{hangparas}

℣. To-morrow the iniquity of the earth shall be blotted out.

℟. And the Saviour of the world shall reign over us.

\properantiphon{Mag.}{When the sun hath risen {\dag} in the heavens, ye shall see the King of kings proceeding from the Father, as a bridegroom out of his chamber.}

\switchcolumn

\begin{inhead}
	Mattins
\end{inhead}

\begin{hangparas}{1.25em}{1}

From lands that see the sun arise.

To earth's remotest boundaries,

The Virgin-born to-day we sing,

The Son of Mary, Christ the King.\\

Blest Author of this earthly frame,

To take a servant's form he came,

That, liberating flesh by flesh,

Whom he had made might live afresh.\\

In that chaste parent's holy womb,

Celestial grace hath found its home:

And she, as earthly bride unknown,

Yet calls that Offspring blest her own.\\

The mansion of the modest breast

Becomes a shrine where God shall rest:

The pure and undefiled one

Conceived in her womb the Son.\\

That Son, that royal Son she bore,

Whom Gabr'el's voice had told afore;

Whom, in his Mother yet conceal'd,

The Infant Baptist had reveal'd.\\

The manger and the straw he bore,

The cradle did he not abhor:

A little milk his infant fare

Who feedeth ev'n each fowl of air.\\

The heav'nly chorus filled the sky,

The Angels sang to God on high,

What time to shepherds, watching lone,

They made creation's Shepherd known.\\

All honour, laud, and glory be,

O Jesu, Virgin-born, to thee;

All glory, as is ever meet,

To Father and to Paraclete. Amen.\\
\end{hangparas}

℣. The Lord declared, alleluia.

℟. His salvation, alleluia.

\properantiphon{Ben.}{Glory {\dag} to God in the highest, and on earth peace to men of good will, alleluia, alleluia.}

\fussy
\end{paracol}

\begin{rubric}
	The Office Hymn is from I Evensong with the following Versicle and Antiphon.
\end{rubric}

℣. The Lord declared, alleluia.

℟. His salvation, alleluia.

\properantiphon{Mag.}{To-day {\dag} the Christ is born; to-day hath a Saviour appeared: to-day on earth Angels are singing, Archangels rejoicing: today the righteous exult and say. Glory to God in the highest, alleluia.}

\supplement{26 December}{St. Stephen}{}

\begin{paracol}{2}[]
\sloppy
\begin{inhead}
	Mattins
\end{inhead}
\begin{hangparas}{1.25em}{1}

Thou foll'west, Martyr of God,

The path the only Son hath trod,

Thy conquer'd foes thou treadest down,

And gloriest in a victor's crown.\\

O may thy prayer for us obtain

The cleansing of each guilty stain,

Shield us from sin's contagious blight,

Put life's long weariness to flight.\\

The cruel chains are now unwound

That once thy sacred body bound,

So may God's Son earth's fetters break

From us, for his own love's dear sake.\\

All honour, laud, and glory be,

O Jesu, Virgin-born, to thee;

All glory, as is ever meet,

To Father and to Paraclete. Amen.\\
\end{hangparas}

℣. Devout men carried Stephen to his burial.

℟. And made great lamentation over him.

\properantiphon{Ben.}{And Stephen, {\dag} full of faith and power, did great wonders and miracles among the people.}

\switchcolumn

\begin{inhead}
	II Evensong
\end{inhead}

\begin{hangparas}{1.25em}{1}

Of all thy warrior Saints, O Lord,

The portion, crown, and great reward:

From all transgressions set us free

Who sing thy Martyr's victory.\\

The pleasures of the world he spurn'd.

From sin's pernicious lures he turn'd;

Accounting them as transient all,

He reach'd at length thy heav'nly hall.\\

For thee through many a woe he ran,

In many a fight he play'd the man;

For thee his blood he dar'd to pour,

And thence hath joy for evermore.\\

We therefore pray thee, full of love,

Regard us from thy throne above:

On this thy Martyr's triumph day,

Wash ev'ry stain of sin away.\\

All honour, laud, and glory be,

O Jesu, Virgin-born, to thee;

All glory, as is ever meet,

To Father and to Paraclete. Amen.\\
\end{hangparas}

℣. Stephen saw the heavens opened.

℟. He saw, and entered in: how blessed is the man unto whom heaven stood open!

\properantiphon{Mag.}{Devout men {\dag} carried Stephen to his burial, and made great lamentation over him.}

\fussy
\end{paracol}

\supplement{27 December}{St. John}{}

\begin{inhead}
	Mattins
\end{inhead}

\begin{multicols}{2}
Let heav'n's exultant praises ring,

And earth with joy responsive sing:

Th' Apostle's deeds and high estate

This festal-tide we celebrate.\\

O ye who, thron'd in glory dread,

Shall judge the living and the dead,

True lights, the world illumining,

Regard the suppliant prayer we bring.\\

The gates of heaven, at your command,

To all or clos'd or open stand:

May we, at your august decree,

Be loos'd from our iniquity.\\

The power, of old to you convey'd,

Sickness and health alike obey'd:

May ye our ailing souls once more

To strength and holiness restore;\\

That Christ, th' unerring Judge of doom,

When he at time's last end shall come,

May grant us, for his mercy's sake,

Of joys eternal to partake.\\

All honour, laud, and glory be,

O Jesu, Virgin-born, to thee;

All glory, as is ever meet,

To Father and to Paraclete. Amen.\\

℣. This is that disciple which testifieth of these things.

℟. And we know that his testimony is true.

\properantiphon{Ben.}{This is the same John, {\dag} who leaned on the Lord's bosom at the last supper: the blessed Apostle, unto whom were revealed the secret things of heaven.}

\end{multicols}

\begin{rubric}
	In II Evensong, the Office Hymn is of Mattins with the following Versicle.
\end{rubric}

℣. Right worthy of honour is blessed John the Apostle.

℟. Who leaned on the Lord's bosom at the last supper.

\properantiphon{Mag.}{There went {\dag} this saying abroad among the brethren, that that disciple should not die: yet Jesus said not unto him, He shall not die; but, If I will that he tarry till I come.}


\supplement{28 December}{Holy Innocents}{}

\begin{inhead}
	Mattins
\end{inhead}

\begin{multicols}{2}
All hail! ye infant martyr flowers.

Cut off in life's first dawning hours:

As rosebuds snapt in tempest strife

When Herod sought your Saviour's life.\\

You, tender flock of Christ, we sing,

First victims slain for Christ your King:

Beneath the Altar's heav'nly ray

With Martyr palms and crowns ye play.\\

All honour, laud, and glory be,

O Jesu, Virgin-born, to thee;

All glory, as is ever meet,

To Father and to Paraclete. Amen.\\

℣. Herod was exceeding wroth, and slew many children.

℟. In Bethlehem Judah, the city of David.

\properantiphon{Ben.}{These are they {\dag} which were not defiled with women; for they are virgins: and they follow the Lamb whithersoever he goeth.}

\end{multicols}

\begin{rubric}
	In II Evensong, the Office Hymn is of Mattins with the following Versicle.
\end{rubric}

℣. Beneath the altar all the Saints do cry aloud.

℟. Shall not our blood be avenged, O our God.

\properantiphon{Mag.}{Then were innocent children {\dag} slain instead of Christ by a wicked ruler; the very sucklings were put to death: spotless, they follow the Lamb himself, and say for ever: Glory be to thee, O Lord.}


\supplement{1 January}{Circumcision}{of Our Lord}

\begin{paracol}{2}[]
\sloppy
\begin{inhead}
	I Evensong
\end{inhead}
\begin{hangparas}{1.25em}{1}

Jesu, the Father's only Son,

Whose death for all redemption won,

Before the worlds, of God Most High

Begotten all ineffably.\\

The Father's Light and Splendour thou,

Their endless Hope to thee that bow:

Accept the prayers and praise to-day

That through the world thy servants pay.\\

Salvation's Author, call to mind

Thou took'st the form of humankind,

When of the Virgin undefil'd

Thou in man's flesh becam'st a Child.\\

Thus testifies the present day

Through every year in long array,

That thou, salvation's source alone,

Proceededst from the Father's Throne.\\

Whence sky, and stars, and sea's abyss,

And earth, and all that therein is,

Shall still, with laud and carol meet,

The Author of thine Advent greet.\\

And we who, by thy precious Blood

From sin redeem'd, are mark'd for God,

On this, the day that saw thy birth,

Sing the new song of ransom'd earth.\\

All honour, laud, and glory be,

O Jesu, Virgin-born, to thee;

All glory, as is ever meet,

To Father and to Paraclete. Amen.\\
\end{hangparas}

℣. The Lord declared, alleluia.

℟. His salvation, alleluia.

\properantiphon{Mag.}{God, {\dag} for his great love wherewith he loved us, hath sent his own Son in the likeness of sinful flesh, alleluia.}

\switchcolumn

\begin{inhead}
	Mattins
\end{inhead}

\begin{hangparas}{1.25em}{1}
From lands that see the sun arise,

To earth's remotest boundaries,

The Virgin-born to-day we sing,

The Son of Mary, Christ the King.\\

Blest Author of this earthly frame,

To take a servant's form he came,

That, liberating flesh by flesh,

Whom he had made might live afresh.\\

In that chaste parent's holy womb,

Celestial grace hath found its home:

And she, as earthly bride unknown,

Yet calls that Offspring blest her own.\\

The mansion of the modest breast

Becomes a shrine where God shall rest:

The pure and undefiled one

Conceived in her womb the Son.\\

That Son, that royal Son she bore

Whom Gabr'el's voice had told afore;

Whom, in his Mother yet conceal'd,

The infant Baptist had reveal'd.\\

The manger and the straw he bore.

The cradle did he not abhor:

A little milk his infant fare

Who feedeth even each fowl of air.\\

The heav'nly chorus filled the sky,

The Angels sang to God on high,

What time to shepherds, watching lone,

They made creation's Shepherd known.\\

All honour, laud, and glory be,

O Jesu, Virgin-born, to thee;

All glory, as is ever meet,

To Father and to Paraclete. Amen.\\
\end{hangparas}

℣. The Word was made flesh, alleluia.

℟. And dwelt among us, alleluia.

\properantiphon{Ben.}{A great and wondrous mystery {\dag} is made known to us this day; a new thing is wrought in both natures: God is made man; that which was, remained, and that which was not, he assumed; suffering no confusion, nor yet division.}

\fussy
\end{paracol}

\begin{rubric}
	II Evensong as in I Evensong, with the following Antiphon.
\end{rubric}

\properantiphon{Mag.}{Great {\dag} is the mystery of the inheritance: the womb of her that knew not man is become the temple of the Godhead: by taking flesh of her, he was no way defiled: all the nations shall gather, saying: Glory be to thee, O Lord.}

\supplement{2 January}{Holy Name of Jesus}{}

\properantiphon{Mag.}{For he that is mighty {\dag} hath magnified me, and holy is his Name, alleluia.}

\properantiphon{Ben.}{He gave himself {\dag} that he might deliver his people, and get him a perpetual Name, alleluia.}

\properantiphon{Mag.}{Thou shalt call {\dag} his Name Jesus, for he shall save his people from their sins, alleluia.}

\subby{Octave Day of St. Stephen}
\fancyhead[RO,LE]{\textit{Stephen Octave Day}}
\fancyhead[RE,LO]{2 January}
\begin{inhead}
    {Simple\\
2 January}
\end{inhead}

\begin{rubric}
	The propers are from the Feast Day, except for that which followeth.
\end{rubric}

\collect
\lett{A}{lmighty} and everlasting God, who, in the blood of the blessed Levite Stephen, didst consecrate the first-fruits of the Martyrs: grant, we beseech thee; that he may ever intercede for us, who prayed even for his persecutors to our Lord Jesus Christ, thy Son. Who with thee liveth.

\subby{Octave Day of St. John}
\fancyhead[RO,LE]{\textit{John Octave Day}}
\fancyhead[RE,LO]{3 January}
\begin{inhead}
    {Simple\\
3 January}
\end{inhead}

\begin{rubric}
	The propers are from the Feast Day, except for that which followeth.
\end{rubric}

\antiphon{Mag.}{This is the same John {\dag} who learned on the Lord's bosom at the last supper: the blessed Apostle, unto whom were revealed the secret things of heaven.}

\collect
\lett{M}{erciful} Lord, we beseech thee to cast thy bright beams of light upon thy Church: that it being enlightened by the doctrine of thy blessed Apostle and Evangelist Saint John may so walk in the light of thy truth, that it may at length attain to the light of everlasting life. Through.

\secret
\lett{R}{eceive,} O Lord, the gifts which we offer unto thee on the solemnity of him, in whose advocacy we trust for deliverance. Through.

\postcommunion
\lett{O}{God,} who hast refreshed us with heavenly meat and drink, we humbly beseech thee: that we may be defended by the prayers of him, in whose memory we have received the same. Through.

\begin{rubric}
	If the Postcommunion of this Mass shall have been already said for some other Saint, the Postcommunion for St. John shall be that of his Feast before the Latin Gate, 6 May (p. BCP 504).
\end{rubric}


\subby{Octave Day of Holy Innocents}
\fancyhead[RO,LE]{\textit{Innocents Octave Day}}
\fancyhead[RE,LO]{4 January}
\begin{inhead}
    {Simple\\
4 January}
\end{inhead}

\begin{rubric}
	The propers are from the Feast Day, except for that which followeth.\par
	\textsc{Note,} The \emph{Gloria in excelsis} is said.
\end{rubric}

℣. Herod was exceeding wroth, and slew many children.

℟. In Bethlehem Judah, the city of David.\\

\antiphon{Mag.}{These are they {\dag} which were not defiled with women; for they are virgins: and they follow the Lamb whithersoever he goeth.}

\begin{rubric}
	If a Commemoration only is made of the Octave, the Prayers (of the Feast) will be as followeth.
\end{rubric}

\collect
\lett{O}{almighty} God, who out of the mouths of babes and sucklings hast ordained strength, and madest infants to glorify thee by their deaths: mortify and kill all vices in us; and so strengthen us by thy grace, that by the innocency of our lives, and constancy of our faith even unto death, we may glorify thy holy name. Through.

\begin{rubric}
	Commemoration is made of St. Titus (p. \pageref{TitusCollect}).
\end{rubric}

\secret
\lett{M}{ay} the devout prayers of thy Saints never fail us, O Lord: that they may render our gifts acceptable unto thee, and ever obtain for us thy pardon. Through.

\begin{rubric}
	Commemoration is made of St. Titus (p. \pageref{TitusSecret}).
\end{rubric}

\postcommunion
\lett{W}{e} beseech thee, O Lord: that the gifts which we have offered and received may, through the prayers of the Saints, effectually avail for our succour both in this life, and in that which is to come. Through.

\begin{rubric}
	Commemoration is made of St. Titus (p. \pageref{TitusPostcommunion}).
\end{rubric}

\begin{rubric}
	\textsc{Note,} In Lent, either the Ferial Day's propers or Ash Wednesday's propers may be said.
\end{rubric}

\begin{rubric}
	The Office Hymn \& Versicles of the Feast of the Epiphany of Our Lord shall be used for the foregoing Sundays and Ferial Days until I Evensong of Septuagesima, exclusive.
\end{rubric}


\subby{St. Titus}\label{Titus}
\fancyhead[RO,LE]{\textit{Titus}}
\fancyhead[RE,LO]{4 January}
\begin{inhead}
    {Simple\\
4 January}
\end{inhead}

\begin{rubric}
	The propers are from the First Common of a Confessor Bishop (p. \pageref{CommonConfessorBishopI}), except for the Gospel \& Prayers as followeth.
\end{rubric}

\collect\label{TitusCollect}
\lett{O}{God,} who didst adorn blessed Titus, thy Confessor and Bishop, with apostolic virtues: grant, through his merits and intercession; that, living justly and godly in this world, we may be found worthy to attain unto the heavenly country. Through.

\readingcitation{Gospel}{Luke 10:1}
\lett{A}{t that time:} The Lord appointed seventy also: and sent them two and two before his face into every city and place, whither he himself would come. Therefore said he unto them, The harvest truly is great, but the labourers are few: pray ye therefore the Lord of the harvest, that he would send forth labourers into his harvest. Go your ways: behold, I send you forth as lambs among wolves. Carry neither purse, nor scrip, nor shoes: and salute no man by the way. And into whatsoever house ye enter, first say, Peace be to this house. And if the son of peace be there, your peace shall rest upon it: if not, it shall turn to you again. And in the same house remain, eating and drinking such things as they give: for the labourer is worthy of his hire. Go not from house to house. And into whatsoever city ye enter, and they receive you, eat such things as are set before you: And heal the sick that are therein, and say unto them, The kingdom of God is come nigh unto you.

\secret\label{TitusSecret}
\lett{W}{e} beseech thee, O Lord, that we remembering with gladness the merits of thy Saints, may in all places feel the succour of their intercession. Through.

\postcommunion\label{TitusPostcommunion}
\lett{G}{rant,} we beseech thee, almighty God: that we, shewing forth our thankfulness for the gifts which we have received, may, at the intercession of blessed Titus, thy Confessor and Bishop, obtain yet more abundant mercies. Through.


\subby{Vigil of Epiphany}
\fancyhead[RO,LE]{\textit{Epiphany Vigil}}
\fancyhead[RE,LO]{5 January}
\begin{inhead}
    {Second Class Vigil\\
5 January}
\end{inhead}

\begin{rubric}
	The Daily Office propers are as on the Feast of the Circumcision, except that which followeth.
\end{rubric}
\begin{rubric}
	\textsc{Note,} Commemoration is made of St. Telesphorus with the Prayers of the next Mass.
\end{rubric}

\antiphon{Mag.}{The Child Jesus {\dag} increased in wisdom and stature in the sight of God and man.}

\antiphon{Ben.}{While all things {\dag} were in quiet silence, and that night was in the midst of her swift course, thine Almighty Word, O Lord, leaped down out of thy royal throne, alleluia.}

\introit
\lett{W}{hile} all things were in in quiet silence, and night was in the midst of her swift course, thine almighty Word, O Lord, leaped down from heaven out of thy royal throne. \textit{Ps.} The Lord is King, and hath put on put glorious apparel: the Lord hath put on his apparel, and girded himself with strength.

\collect
\lett{A}{lmighty} and everlasting God, direct our actions according to thy good pleasure: that in the name of thy well-beloved Son we may be made worthy to abound in good works. Who liveth and reigneth with thee.

\begin{rubric}
	Commemoration of St. Telesphorus (p. \pageref{TelesphorusCollect}) and for St. Mary after Christmas (p. BCP 594) is said.
\end{rubric}

\readingcitation{Epistle}{Galatians 4:1}
\lett{B}{rethren:} The heir, as long as he is a child, differeth nothing from a servant, though he be lord of all; But is under tutors and governors until the time appointed of the father. Even so we, when we were children, were in bondage under the elements of the world: But when the fulness of the time was come, God sent forth his Son, made of a woman, made under the law, To redeem them that were under the law, that we might receive the adoption of sons. And because ye are sons, God hath sent forth the Spirit of his Son into your hearts, crying, Abba, Father. Wherefore thou art no more a servant, but a son; and if a son, then an heir of God through Christ.

\gradall{Thou art fairer than the children of men: full of grace are thy lips. ℣. My heart is inditing of a good matter, I speak of the things which I have made unto the King: my tongue is the pen of a ready writer.}{Alleluia, alleluia. ℣. The Lord is King, and hath put on glorious apparel: the Lord hath put on his apparel, and girded himself with strength. Alleluia.}

\readingcitation{Gospel}{Matthew 2:19}
\lett{A}{t that time:} When Herod was dead, behold, an angel of the Lord appeareth in a dream to Joseph in Egypt, Saying, Arise, and take the young child and his mother, and go into the land of Israel: for they are dead which sought the young child's life. And he arose, and took the young child and his mother, and came into the land of Israel. But when he heard that Archelaus did reign in Judaea in the room of his father Herod, he was afraid to go thither: notwithstanding, being warned of God in a dream, he turned aside into the parts of Galilee: And he came and dwelt in a city called Nazareth: that it might be fulfilled which was spoken by the prophets, He shall be called a Nazarene.

\offertory{God hath made the round world so sure: that it cannot be moved: ever since the world began, hath thy seat, O God, been prepared, thou art from everlasting.}

\secret
\lett{G}{rant,} we beseech thee, almighty God: that the gift which we offer in the sight of thy majesty, may obtain for us grace to serve thee with all godliness, and bring us in the end to everlasting felicity. Through.

\begin{rubric}
	Commemoration of St. Telesphorus (p. \pageref{TelesphorusSecret}) and for St. Mary after Christmas (p. BCP 594) is said.
\end{rubric}

\communion{Take the young Child and his Mother and go into the land of Israel: for they are ead which sought the young Child's life.}

\postcommunion
\lett{M}{ay} the operation of this mystery, Lord, avail for the cleansing of our sins, and for the fulfilment of our godly desires. Through.

\begin{rubric}
	Commemoration of St. Telesphorus (p. \pageref{TelesphorusPostcommunion}) and for St. Mary after Christmas (p. BCP 594) is said.
\end{rubric}


\subby{St. Telesphorus}
\fancyhead[RO,LE]{\textit{Telesphorus}}
\fancyhead[RE,LO]{5 January}
\begin{inhead}
    {Memorial\\
5 January}
\end{inhead}

\begin{rubric}
	The propers are from the Second Common of a Martyr Bishop (p. \pageref{CommonMartyrBishopII}), with the Prayers below.
\end{rubric}

\collect\label{TelesphorusCollect}
\lett{O}{God,} who makest us glad with the yearly solemnity of blessed Telesphorus, thy Martyr and Bishop: mercifully grant; that, as we now celebrate his birthday so we may likewise rejoice in his protection. (Through.)

\secret\label{TelesphorusSecret}
\lett{S}{anctify,} O Lord, the gifts which we dedicate to thee: that at the intercession of blessed Telesphorus, thy Martyr and Bishop, they may obtain for us thy gracious favour. (Through.)

\postcommunion\label{TelesphorusPostcommunion}
\lett{W}{e} beseech thee, O Lord our God, that like as we, whom thou hast refreshed by the partaking of thy sacred gift, do offer unto thee our worship: so by the intercession of blessed Telesphorus thy Martyr and Bishop, we may perceive the benefit of the same. (Through.)\\

\supplement{6 January}{Epiphany}{of Our Lord}

\begin{paracol}{2}[]
\sloppy
\begin{inhead}
	I Evensong
\end{inhead}
\begin{hangparas}{1.25em}{1}
Why, impious Herod, vainly fear

That Christ the Saviour cometh here?

He takes not earthly realms away

Who gives the crown that lasts for aye.\\

To greet his birth the wise men went,

Led by the star before them sent;

Call'd on by light, towards Light they press'd,

And by their gifts their God confess'd.\\

In holy Jordan's purest wave

The heav'nly Lamb vouchsaf'd to lave;

That he, to whom was sin unknown,

Might cleanse his people from their own.\\

New miracle of power divine!

The water reddens into wine:

He spake the word, and pour'd the wave

In other streams than nature gave.\\

All glory, Lord, to thee we pay

For thine Epiphany today:

All glory, as is ever meet,

To Father and to Paraclete. Amen.
\end{hangparas}

\begin{rubric}
	All Hymns of the same metre throughout the Octave end thus.
\end{rubric}

℣. The kings of Tharsis and of the isles shall give presents.

℟. The kings of Arabia and Saba shall bring gifts.

\properantiphon{Mag.}{The wise men, {\dag} beholding the star, said one to another, This is the sign of a mighty King; forth fare we, and let us seek him: and let us offer him gifts, gold, incense, and myrrh, alleluia.}

\switchcolumn

\begin{inhead}
	Mattins
\end{inhead}
\begin{hangparas}{1.25em}{1}
O more than mighty cities known,

Dear Bethlehem, in thee alone

Salvation’s Lord from heav'n took birth

In human form upon the earth.\\

And from a star that far outshone

The radiant circle of the sun

In beauty, swift the tidings ran

Of God on earth in flesh of man.\\

The wise men, seeing him, so fair,

Bow low before him, and with prayer

Their treasur'd orient gifts unfold

Of incense, myrrh, and royal gold.\\

The fragrant incense which they bring,

The gold, proclaim him God and King;

The bitter spicy dust of myrrh

Foreshadows his new sepulchre.\\

All glory, Lord, to thee we pay

For thine Epiphany today;

All glory, as is ever meet,

To Father and to Paraclete. Amen.\\
\end{hangparas}

	℣. O worship the Lord, alleluia.

	℟. All ye Angels of his, alleluia.
	
	\properantiphon{Ben.}{To-day {\dag} the Church is joined to her heavenly Bridegroom; because in Jordan Christ hath washed away her offences: the wise men with their offerings hasten to the royal marriage, and the guests are regaled with water made wine, alleluia.}
	
	\fussy
\end{paracol}

\begin{rubric}
	In II Evensong, the Office Hymn \& Versicle are of I Evensong, with the following Antiphon.
\end{rubric}

\properantiphon{Mag.}{Now do we celebrate {\dag} a holy day adorned by three miracles: to-day a star led the wise men to the manger; to-day water was made wine at the wedding feast; to-day Christ vouchsafed to be baptized of John in Jordan that he might save us, alleluia.}


\begin{multicols}{2}

\subby{Septuagesima Sunday}
\fancyhead[RO,LE]{\textit{Septuagesima}}
\fancyhead[RE,LO]{Sunday}
\begin{inhead}
    {Second Class Semidouble}
\end{inhead}

\begin{rubric}
	In the Daily Office, the Office Hymns \& Versicles are from the Daily Hymns (p. \pageref{DailyHymns}).
\end{rubric}

\begin{rubric}
On Ferias from Septuagesima Sunday until Ash Wednesday, exclusive, when the Office is not of the Feria, it is always commemorated at Mattins and Evensong on Double Feasts, even of the I Class, and on days within Octaves.
\end{rubric}
\begin{rubric}
	The \emph{Benedictus} Antiphons are said on Ferias as set forth in the Psalter, through Tuesday after Quinquagesima.
\end{rubric}
\begin{rubric}
	\textsc{Note,} No notice is taken of an occurrent Vigil, either in the Office of a Feast, or of a day within an Octave, or of a Feria. The Office of St. Mary on Saturday is not said.
\end{rubric}


\subby{Septuagesima Monday}
\fancyhead[RO,LE]{\textit{Septuagesima}}
\fancyhead[RE,LO]{Monday}
\begin{inhead}
    {Feria}
\end{inhead}

\antiphon{Mag.}{These last {\dag} have wrought but one hour, and thou hast made them equal unto us, which have borne the burden and heat of the day.}


\subby{Septuagesima Tuesday}
\fancyhead[RO,LE]{\textit{Septuagesima}}
\fancyhead[RE,LO]{Tuesday}
\begin{inhead}
    {Feria}
\end{inhead}

\antiphon{Mag.}{The householder answered and said, {\dag} Friend, I do thee no wrong: didst not thou agree with me for a penny? Take that thine is, and go thy way.}


\subby{Septuagesima Wednesday}
\fancyhead[RO,LE]{\textit{Septuagesima}}
\fancyhead[RE,LO]{Wednesday}
\begin{inhead}
    {Feria}
\end{inhead}

\antiphon{Mag.}{Take that thine is {\dag} and go thy way: because I am good, saith the Lord.}


\subby{Septuagesima Thursday}
\fancyhead[RO,LE]{\textit{Septuagesima}}
\fancyhead[RE,LO]{Thursday}
\begin{inhead}
    {Feria}
\end{inhead}

\antiphon{Mag.}{Is it not lawful {\dag} for me to do what I will? Is thine eye evil, because I am good? saith the Lord.}


\subby{Septuagesima Friday}
\fancyhead[RO,LE]{\textit{Septuagesima}}
\fancyhead[RE,LO]{Friday}
\begin{inhead}
    {Feria}
\end{inhead}

\antiphon{Mag.}{So the last {\dag} shall be first, and the first shall be last: for many are called, but few are chosen.}


\subby{Sexagesima Monday}
\fancyhead[RO,LE]{\textit{Sexagesima}}
\fancyhead[RE,LO]{Monday}
\begin{inhead}
    {Feria}
\end{inhead}

\antiphon{Mag.}{If ye seek {\dag} the summit of true honour, hasten to yon heavenly country with what speed ye may.}


\subby{Sexagesima Tuesday}
\fancyhead[RO,LE]{\textit{Sexagesima}}
\fancyhead[RE,LO]{Tuesday}
\begin{inhead}
    {Feria}
\end{inhead}

\antiphon{Mag.}{The seed {\dag} is the word of God, but Christ is the Sower: every one that heareth him shall abide for ever.}


\subby{Sexagesima Wednesday}
\fancyhead[RO,LE]{\textit{Sexagesima}}
\fancyhead[RE,LO]{Wednesday}
\begin{inhead}
    {Feria}
\end{inhead}

\antiphon{Mag.}{But that {\dag} which fell on the good ground are they which in an honest and good heart receive the word; and bring forth fruit with patience.}


\subby{Sexagesima Thursday}
\fancyhead[RO,LE]{\textit{Sexagesima}}
\fancyhead[RE,LO]{Thursday}
\begin{inhead}
    {Feria}
\end{inhead}

\antiphon{Mag.}{Some seed {\dag} fell on good ground, and brought forth fruit; some an hundred-fold, and some sixtyfold.}


\subby{Sexagesima Friday}
\fancyhead[RO,LE]{\textit{Sexagesima}}
\fancyhead[RE,LO]{Friday}
\begin{inhead}
    {Feria}
\end{inhead}

\antiphon{Mag.}{They who keep the word of God {\dag} with an honest and perfect heart bring forth fruit with patience.}


\subby{Quinquagesima Monday}
\fancyhead[RO,LE]{\textit{Quinquagesima}}
\fancyhead[RE,LO]{Monday}
\begin{inhead}
    {Feria}
\end{inhead}

\antiphon{Mag.}{And they which went before {\dag} rebuked him that he should hold his peace: but he cried so much the more, Have mercy on me, thou Son of David.}


\subby{Quinquagesima Tuesday}
\fancyhead[RO,LE]{\textit{Quinquagesima}}
\fancyhead[RE,LO]{Tuesday}
\begin{inhead}
    {Feria}
\end{inhead}

\antiphon{Mag.}{Have mercy on me, {\dag} thou Son of David. What wilt thou that I shall do unto thee? Lord, that I may receive my sight.}


\begin{rubric}
	If on the following Ash Wednesday there occur a Double Feast of the I or II Class, it is transferred to the first unhindered day. Greater and lesser Doubles and Memorials are only commemorated on the other days of Lent.
\end{rubric}

\end{multicols}

\subby{Ash Wednesday}
\fancyhead[RO,LE]{\textit{Ash Wednesday}}
\fancyhead[RE,LO]{}
\begin{inhead}
    {Greater Privileged Feria}
\end{inhead}

\begin{rubric}
	On this day all Octaves cease until Holy Sabbath. On this and other Ferias through None of the following Saturday all is said as in the Psalter throughout the Year except that which is appointed as proper.
\end{rubric}


\subby{Thursday after Ash Wednesday}
\fancyhead[RO,LE]{\textit{Ash Wednesday}}
\fancyhead[RE,LO]{Thursday}
\begin{inhead}
    {Greater Feria}
\end{inhead}

\begin{paracol}{2}
\antiphon{Ben.}{Lord, my servant {\dag} lieth at home sick of the palsy, grievously tormented. Verily I say unto thee, I will come and heal him.}

\lett{O}{God,} who art offended by sin, and reconciled by penitence: mercifully regard the prayers of thy suppliant people, and turn away the scourge of thy wrath, which for our sins we have justly deserved. Through.

\switchcolumn

\antiphon{Mag.}{Lord, I am not worthy {\dag} that thou shouldest enter under my roof: but speak the word only and my servant shall be healed.}

\lett{S}{pare} us, O Lord, spare thy people: that they who are justly chastised by thy scourges, may be relieved by thy tender mercy. Through.

\end{paracol}

\introit
\lett{W}{hen} I called upon the Lord, he heard my voice from the battle that was against me, yea, even God, that endureth for ever, shall bring them down: O cast thy burden upon the Lord, and he shall nourish thee. \textit{Ps.} Hear my prayer, O God, and hide not thyself from my petition: take heed unto me and hear me.

\collect
\lett{O}{God,} who art wroth with them that sin against thee, and sparest them that are penitent: mercifully look upon the prayers of thy people which call upon thee; and turn away the scourges of thy wrath which for our sins we justly deserve. Through.

\begin{rubric}
    \nth{2} Collect is \emph{Of Saints} \& \nth{3} \emph{Of the Living and Departed}.
\end{rubric}

\readingcitation{Epistle}{Isaiah 38:1}
\lett{I}{n those days:} Was Hezekiah sick unto death. And Isaiah the prophet the son of Amoz came unto him, and said unto him, Thus saith the \divineName{Lord}, Set thine house in order: for thou shalt die, and not live. Then Hezekiah turned his face toward the wall, and prayed unto the \divineName{Lord}, And said, Remember now, O \divineName{Lord}, I beseech thee, how I have walked before thee in truth and with a perfect heart, and have done that which is good in thy sight. And Hezekiah wept sore. Then came the word of the \divineName{Lord} to Isaiah, saying, Go, and say to Hezekiah, Thus saith the \divineName{Lord}, the God of David thy father, I have heard thy prayer, I have seen thy tears: behold, I will add unto thy days fifteen years. And I will deliver thee and this city out of the hand of the king of Assyria: and I will defend this city, saith the Lord almighty.

\gradual{O cast thy burden upon the Lord, and he shall nourish thee. ℣. When I called upon the Lord, he heard my voice from the battle that was against me.}

\readingcitation{Gospel}{Matthew 8:5}
\lett{A}{t that time:} When Jesus was entered into Capernaum, there came unto him a centurion, beseeching him, And saying, Lord, my servant lieth at home sick of the palsy, grievously tormented. And Jesus saith unto him, I will come and heal him. The centurion answered and said, Lord, I am not worthy that thou shouldest come under my roof: but speak the word only, and my servant shall be healed. For I am a man under authority, having soldiers under me: and I say to this man, Go, and he goeth; and to another, Come, and he cometh; and to my servant, Do this, and he doeth it. When Jesus heard it, he marvelled, and said to them that followed, Verily I say unto you, I have not found so great faith, no, not in Israel. And I say unto you, That many shall come from the east and west, and shall sit down with Abraham, and Isaac, and Jacob, in the kingdom of heaven. But the children of the kingdom shall be cast out into outer darkness: there shall be weeping and gnashing of teeth. And Jesus said unto the centurion, Go thy way; and as thou hast believed, so be it done unto thee. And his servant was healed in the selfsame hour.

\offertory{Unto thee, O Lord, will I lift up my soul: my God, I have put my trust in thee, O let me not be confounded: neither let mine enemies triumph over me: for all they that hope in thee shall not be ashamed.}

\secret
\lett{W}{e} beseech thee, O Lord, favourably to regard these our sacrifices: that they may be profitable for our devotion and set forward our salvation. Through.

\begin{rubric}
    \nth{2} Secret is \emph{Of Saints} \& \nth{3} \emph{Of the Living and Departed}.
\end{rubric}

\communion{Thou shalt be pleased with the sacrifice of righteousness, with the burnt-offerings and oblations upon thine altar, O Lord.}

\postcommunion
\lett{W}{e} humbly beseech thee, almighty God: that, as we have received the blessing of this heavenly gift; so it may be made to us thy sacrament, and avail to our salvation. Through.

\begin{rubric}
    \nth{2} Postcommunion is \emph{Of Saints} \& \nth{3} \emph{Of the Living and Departed}.
\end{rubric}

\textsc{Priest.} Let us pray.\par
\textsc{Deacon.} Humble your heads before God.\par
\begin{rubric}
    The Priest then prays the following:
\end{rubric}
\lett{S}{pare,} O Lord, spare thy people: that they who are justly chastised by thy scourges, may by thy mercy be relieved. Through.


\subby{Friday after Ash Wednesday}
\fancyhead[RO,LE]{\textit{Ash Wednesday}}
\fancyhead[RE,LO]{Friday}
\begin{inhead}
    {Greater Feria}
\end{inhead}

\begin{paracol}{2}
\antiphon{Ben.}{When thou doest thine alms, {\dag} let not thy left hand know what thy right. hand doeth.}

\lett{W}{e} beseech thee, O Lord, to accompany with thy bounteous favour the fast upon which we have entered: that the observance which we shew forth in our bodies, we may be able also to practise with sincerity of heart. Through.

\switchcolumn

\antiphon{Mag.}{But thou, when thou prayest, {\dag} enter into thy closet, and when thou hast shut thy door, pray to thy Father.}

\lett{D}{efend} thy people, O Lord, and mercifully cleanse them from all their sins: for no adversity will harm them over whom iniquity hath no dominion. Through.

\end{paracol}

\introit
\lett{T}{he} Lord heard, and had mercy upon me: the Lord became my helper. \textit{Ps.} I will magnify thee, O Lord, for thou hast set me up: and not made my foes to triumph over me.

\collect
\lett{W}{e} beseech thee, O Lord, to further with thy gracious favour the fasts which we have begun: that as we keep this observance in the flesh, so we may have strength to perform the same in singleness of heart. Through.

\begin{rubric}
    \nth{2} Collect is \emph{Of Saints} \& \nth{3} \emph{Of the Living and Departed}.
\end{rubric}

\readingcitation{Epistle}{Isaiah 58:1}
\lett{T}{hus saith the Lord God:} Cry aloud, spare not, lift up thy voice like a trumpet, and shew my people their transgression, and the house of Jacob their sins. Yet they seek me daily, and delight to know my ways, as a nation that did righteousness, and forsook not the ordinance of their God: they ask of me the ordinances of justice; they take delight in approaching to God. Wherefore have we fasted, say they, and thou seest not? wherefore have we afflicted our soul, and thou takest no knowledge? Behold, in the day of your fast ye find pleasure, and exact all your labours. Behold, ye fast for strife and debate, and to smite with the fist of wickedness: ye shall not fast as ye do this day, to make your voice to be heard on high. Is it such a fast that I have chosen? a day for a man to afflict his soul? is it to bow down his head as a bulrush, and to spread sackcloth and ashes under him? wilt thou call this a fast, and an acceptable day to the \divineName{Lord}? Is not this the fast that I have chosen? to loose the bands of wickedness, to undo the heavy burdens, and to let the oppressed go free, and that ye break every yoke? Is it not to deal thy bread to the hungry, and that thou bring the poor that are cast out to thy house? when thou seest the naked, that thou cover him; and that thou hide not thyself from thine own flesh? Then shall thy light break forth as the morning, and thine health shall spring forth speedily: and thy righteousness shall go before thee; the glory of the \divineName{Lord} shall be thy reward. Then shalt thou call, and the \divineName{Lord} shall answer; thou shalt cry, and he shall say, Here I am. For I the \divineName{Lord} thy God am merciful.

\gradual{One thing have I desired of the Lord, which I will require, even that I may dwell in the house of the Lord. ℣. To behold the fair beauty of the Lord, and to hide me in his holy temple.}

\tract{O Lord, deal not with us after our sins: nor reward us according to our wickednesses. ℣. O Lord, remember not our old sins: but have mercy upon us, and that soon, for we are come to great misery. \inrub{Here genuflect.} ℣. Help us, O God of our salvation: and for the glory of thy name, O Lord, deliver us: and be merciful unto our sins, for thy name's sake.}

\readingcitation{Gospel}{Matthew 5:43}
\lett{A}{t that time:} Jesus said unto his disciples: Ye have heard that it hath been said, Thou shalt love thy neighbour, and hate thine enemy. But I say unto you, Love your enemies, bless them that curse you, do good to them that hate you, and pray for them which despitefully use you, and persecute you; That ye may be the children of your Father which is in heaven: for he maketh his sun to rise on the evil and on the good, and sendeth rain on the just and on the unjust. For if ye love them which love you, what reward have ye? do not even the publicans the same? And if ye salute your brethren only, what do ye more than others? do not even the publicans so? Be ye therefore perfect, even as your Father which is in heaven is perfect. Take heed that ye do not your alms before men, to be seen of them: otherwise ye have no reward of your Father which is in heaven. Therefore when thou doest thine alms, do not sound a trumpet before thee, as the hypocrites do in the synagogues and in the streets, that they may have glory of men. Verily I say unto you, They have their reward. But when thou doest alms, let not thy left hand know what thy right hand doeth: That thine alms may be in secret: and thy Father which seeth in secret himself shall reward thee openly.

\offertory{Quicken me, O Lord, according to thy word: that I may know thy testimonies.}

\secret
\lett{G}{rant,} we beseech thee, O Lord, that this sacrifice of Lenten observance which we offer: may both render our souls acceptable unto thee, and make us more readily to serve thee in continence. Through.

\begin{rubric}
    \nth{2} Secret is \emph{Of Saints} \& \nth{3} \emph{Of the Living and Departed}.
\end{rubric}

\communion{Serve the Lord in fear, and rejoice unto him with reverence: lay hold on discipline, lest ye perish from the right way.}

\postcommunion
\lett{P}{our} forth upon us, O Lord, the Spirit of thy charity: that as thou hast fulfilled us with one heavenly bread, so of thy goodness thou wouldest make us to be of one heart and mind. Through . . . in the unity of the same.

\begin{rubric}
    \nth{2} Postcommunion is \emph{Of Saints} \& \nth{3} \emph{Of the Living and Departed}.
\end{rubric}

\textsc{Priest.} Let us pray.\par
\textsc{Deacon.} Humble your heads before God.\par
\begin{rubric}
    The Priest then prays the following:
\end{rubric}
\lett{D}{efend,} O Lord, thy people, and mercifully cleanse them from all their sins: that no adversity may harm them, over whom iniquity hath no dominion. Through.


\subby{Saturday after Ash Wednesday}
\fancyhead[RO,LE]{\textit{Ash Wednesday}}
\fancyhead[RE,LO]{Saturday}
\begin{inhead}
    {Greater Feria}
\end{inhead}

\antiphon{Ben.}{Yet they seek me daily, {\dag} and delight to know my ways.}

\lett{A}{ssist} us, Lord, in these our supplications: and grant that we may keep with devout service this solemn fast, ordained for the healing of both soul and body. Through.

\introit
\lett{T}{he} Lord heard, and had mercy upon me: the Lord became my helper. \textit{Ps.} I will magnify thee, O Lord, for thou hast set me up: and not made my foes to triumph over me.

\collect
\lett{A}{ssist} us, O Lord, in these our supplications: and grant; that like as this solemn fast hath been ordained for the safety and healing of our bodies and our souls, so we may with devout observance celebrate the same. Through.

\begin{rubric}
    \nth{2} Collect is \emph{Of Saints} \& \nth{3} \emph{Of the Living and Departed}.
\end{rubric}

\readingcitation{Epistle}{Isaiah 58:9}
\lett{T}{hus saith the Lord God:} If thou take away from the midst of thee the yoke, the putting forth of the finger, and speaking vanity; And if thou draw out thy soul to the hungry, and satisfy the afflicted soul; then shall thy light rise in obscurity, and thy darkness be as the noonday: And the \divineName{Lord} shall guide thee continually, and satisfy thy soul in drought, and make fat thy bones: and thou shalt be like a watered garden, and like a spring of water, whose waters fail not. And they that shall be of thee shall build the old waste places: thou shalt raise up the foundations of many generations; and thou shalt be called, The repairer of the breach, The restorer of paths to dwell in. If thou turn away thy foot from the sabbath, from doing thy pleasure on my holy day; and call the sabbath a delight, the holy of the \divineName{Lord}, honourable; and shalt honour him, not doing thine own ways, nor finding thine own pleasure, nor speaking thine own words: Then shalt thou delight thyself in the \divineName{Lord}; and I will cause thee to ride upon the high places of the earth, and feed thee with the heritage of Jacob thy father: for the mouth of the \divineName{Lord} hath spoken it.

\gradual{One thing have I desired of the Lord, which I will require, even that I may dwell in the house of the Lord. ℣. To behold the fair beauty of the Lord, and to visit his temple.}


\readingcitation{Gospel}{Mark 6:47}
\lett{A}{t that time:} When even was come, the ship was in the midst of the sea, and Jesus alone on the land. And he saw them toiling in rowing; for the wind was contrary unto them: and about the fourth watch of the night he cometh unto them, walking upon the sea, and would have passed by them. But when they saw him walking upon the sea, they supposed it had been a spirit, and cried out: For they all saw him, and were troubled. And immediately he talked with them, and saith unto them, Be of good cheer: it is I; be not afraid. And he went up unto them into the ship; and the wind ceased: and they were sore amazed in themselves beyond measure, and wondered. For they considered not the miracle of the loaves: for their heart was hardened. And when they had passed over, they came into the land of Gennesaret, and drew to the shore. And when they were come out of the ship, straightway they knew him, And ran through that whole region round about, and began to carry about in beds those that were sick, where they heard he was. And whithersoever he entered, into villages, or cities, or country, they laid the sick in the streets, and besought him that they might touch if it were but the border of his garment: and as many as touched him were made whole.

\offertory{Quicken me, O Lord, according to thy word: that I may know thy testimonies.}

\secret
\lett{A}{ccept,} O Lord, the sacrifice which thou hast ordained to be a worthy propitiation unto thee: and grant, we beseech thee, that we being cleansed by the operation of the same, may offer unto thee the acceptable devotion of our hearts. Through.

\begin{rubric}
    \nth{2} Secret is \emph{Of Saints} \& \nth{3} \emph{Of the Living and Departed}.
\end{rubric}

\communion{Serve the Lord in fear, and rejoice unto him with reverence: lay hold on discipline, lest ye perish from the right wav.}

\postcommunion
\lett{O}{Lord,} who hast quickened us with the gift of heavenly life: we beseech thee, that those things which in this present life are to us a mystery, may be our succour unto life eternal. Through.

\begin{rubric}
    \nth{2} Postcommunion is \emph{Of Saints} \& \nth{3} \emph{Of the Living and Departed}.
\end{rubric}

\textsc{Priest.} Let us pray.\par
\textsc{Deacon.} Humble your heads before God.\par
\begin{rubric}
    The Priest then prays the following:
\end{rubric}
\lett{M}{ay} thy faithful people, O God, be strengthened by thy gifts: that they receiving the same may seek them the more, and seeking them may obtain them everlastingly. Through.


\subby{First Sunday of Lent}
\fancyhead[RO,LE]{\textit{Lent I}}
\fancyhead[RE,LO]{}
\begin{inhead}
    {First Class Semidouble}
\end{inhead}

\begin{paracol}{2}[]
\sloppy
\begin{inhead}
	I Evensong
\end{inhead}
\begin{hangparas}{1.25em}{1}
O maker of the world, give ear;

Accept the pray'r and own the tear

Towards thy seat of mercy sent

In this most holy fast of Lent.\\

Each heart is manifest to thee;

Thou knowest our infirmity;

Forgive thou then each soul that fain

Would seek to thee, and turn again.\\

Our sins are manifold and sore,

But pardon them that sin deplore:

And, for thy Name's sake, make each soul

That feels and owns its languor, whole.\\

So mortify we every sense

By grace of outward abstinence,

That from each stain and spot of sin

The soul may keep her fast within.\\

Grant, O thou blessed Trinity,

Grant, O essential Unity,

That this our fast of forty days

May work our profit and thy praise. Amen.\\

\end{hangparas}

℣. God shall give his Angels charge over thee.

℟. To keep thee in all thy ways.


\switchcolumn

\begin{inhead}
	Mattins
\end{inhead}
\begin{hangparas}{1.25em}{1}
Now Christ, thou Sun of righteousness,

Let dawn our darken'd spirits bless:

The light of grace to us restore

While day to earth returns once more.\\

Thou who dost give th' accepted time,

Give, too, a heart that mourns for crime,

Let those by mercy now be cur'd

Whom loving-kindness long endur'd.\\

Spare not, we pray, to send us here

Some penance kindly but severe,

So let thy gift of pard'ning grace

Our grievous sinfulness efface.\\

Soon will that day, thy day, appear

And all things with its brightness cheer:

We will rejoice in it, as we

Return thereby to grace, and thee.\\

Let all the world from shore to shore

Thee, gracious Trinity, adore;

Right soon thy loving pardon grant,

That we our new-made song may chant. Amen.\\

\end{hangparas}

℣. God shall give his Angels charge over thee.

℟. To keep thee in all thy ways.

\fussy
\end{paracol}

\begin{rubric}
	In II Evensong, the Office Hymn \& Versicle are of I Evensong.
\end{rubric}

\subby{Second Sunday of Lent}
\fancyhead[RO,LE]{\textit{Lent II}}
\fancyhead[RE,LO]{}
\begin{inhead}
    {First Class Semidouble}
\end{inhead}

\begin{rubric}
	The Office Hymn \& Versicle are of the First Sunday of Lent.
\end{rubric}


\subby{Third Sunday of Lent}
\fancyhead[RO,LE]{\textit{Lent III}}
\fancyhead[RE,LO]{}
\begin{inhead}
    {First Class Semidouble}
\end{inhead}

\begin{rubric}
	The Office Hymn \& Versicle are of the First Sunday of Lent.
\end{rubric}


\subby{Fourth Sunday of Lent}
\fancyhead[RO,LE]{\textit{Lent IV}}
\fancyhead[RE,LO]{}
\begin{inhead}
    {First Class Semidouble}
\end{inhead}

\begin{rubric}
	The Office Hymn \& Versicle are of the First Sunday of Lent.
\end{rubric}