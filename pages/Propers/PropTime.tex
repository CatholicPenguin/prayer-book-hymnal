\phantomsection
\addcontentsline{toc}{section}{Proper of Saints}
\fancyhead[C]{\LARGE Proper of Saints}


\subby{Vigil of St. Andrew}
\feastday{St. Andrew Vigil}
\fancyhead[RE,LO]{29 November}
\begin{inhead}
    {Vigil\\
29 November}
\end{inhead}

\begin{rubric}
	If to-day be Saturday, the anticipated Vigil of St. Andrew is kept, but the commemoration of St. Saturninus is omitted, in which case the \nth{2} is of St. Mary (p. \SPMaryAdvent) and the \nth{3} against the persecutors of the Church (p. \SPAgainst) or for the Chief Bishop (p. \SPChiefBishop).
\end{rubric}

\introit
\lett{T}{he} Lord, walking by the sea of Galilee, saw two brethren, Peter and Andrew, and he called them, saying: Follow me: I will make you fishers of men. \textit{Ps.} The heavens declare the glory of God: and the firmament sheweth his handy-work.

\collect
\lett{W}{e} beseech thee, almighty God: that the blessed Apostle Andrew, whose festival we prevent, may implore thy help for us; that we, being absolved from our offences, may likewise be delivered from all dangers. Through.

\begin{rubric}
	In Advent, \nth{2} Collect of the Feria, \nth{3} of St. Saturninus (p. \pageref{SaturninusCollect}).
\end{rubric}
\begin{rubric}
	Outside of Advent, \nth{2} Collect of St. Saturninus (p. \pageref{SaturninusCollect}), \nth{3} of St. Mary  in Eastertide (p. \SPMaryEaster).
\end{rubric}

\readingcitation{Epistle}{Ecclesiasticus 44:22}
%RV:
\lett{I}{n} Isaac did the Lord establish, for Abraham his father's sake, The blessing of all men, and the covenant: And he made it rest upon the head of Jacob; He acknowledged him in his blessings, And gave to him by inheritance, And divided his portions; Among twelve tribes did he part them. And he brought out of him a man of mercy, Which found favour in the sight of all flesh; A man beloved of God and men, even Moses, Whose memorial is blessed. He made him like to the glory of the saints, And magnified him in the fears of his enemies. By his words he caused the wonders to cease; He glorified him in the sight of kings; He gave him commandment for his people, And shewed him part of his glory. He sanctified him in his faithfulness and meekness; He chose him out of all flesh. He made him to hear his voice, And led him into the thick darkness, And gave him commandments face to face, Even the law of life and knowledge, That he might teach Jacob the covenant, And Israel his judgments. He exalted Aaron, a holy man like unto him, Even his brother, of the tribe of Levi. He established for him an everlasting covenant, And gave him the priesthood of the people; He beautified him with comely ornaments, And girded him about with a robe of glory.

%Traditional Text:
%\lett{T}{he} blessing of the Lord was upon the head of the righteous. Therefore did the Lord give him an heritage, and divided his portions among the twelve tribes: and he found favour in the sight of all flesh. And he magnified him so that his enemies stood in fear of him, and by his words he caused the wonders to cease. He made him glorious in the sight of kings, and gave him a commandment for his people, and shewed him his glory. He sanctified him in his faithfulness and meekness, and chose him out of all men. And he gave him commandments before his face, even the law of life and knowledge, and made him to be exalted. An everlasting covenant he made with him, and girded him about with the girdle of righteousness: and the Lord crowned him with the crown of glory.

\gradual{Right honourable are thy friends, O God: right well is their princedom established. ℣. If I tell them: they are more in number than the sand.}

\readingcitation{Gospel}{John 1:35}
\lett{A}{t that time:} John stood, and two of his disciples; And looking upon Jesus as he walked, he saith, Behold the Lamb of God! And the two disciples heard him speak, and they followed Jesus. Then Jesus turned, and saw them following, and saith unto them, What seek ye? They said unto him, Rabbi, (which is to say, being interpreted, Master,) where dwellest thou? He saith unto them, Come and see. They came and saw where he dwelt, and abode with him that day: for it was about the tenth hour. One of the two which heard John speak, and followed him, was Andrew, Simon Peter's brother. He first findeth his own brother Simon, and saith unto him, We have found the Messias, which is, being interpreted, the Christ. And he brought him to Jesus. And when Jesus beheld him, he said, Thou art Simon the son of Jona: thou shalt be called Cephas, which is by interpretation, A stone. The day following Jesus would go forth into Galilee, and findeth Philip, and saith unto him, Follow me. Now Philip was of Bethsaida, the city of Andrew and Peter. Philip findeth Nathanael, and saith unto him, We have found him, of whom Moses in the law, and the prophets, did write, Jesus of Nazareth, the son of Joseph. And Nathanael said unto him, Can there any good thing come out of Nazareth? Philip saith unto him, Come and see. Jesus saw Nathanael coming to him, and saith of him, Behold an Israelite indeed, in whom is no guile! Nathanael saith unto him, Whence knowest thou me? Jesus answered and said unto him, Before that Philip called thee, when thou wast under the fig tree, I saw thee. Nathanael answered and saith unto him, Rabbi, thou art the Son of God; thou art the King of Israel. Jesus answered and said unto him, Because I said unto thee, I saw thee under the fig tree, believest thou? thou shalt see greater things than these. And he saith unto him, Verily, verily, I say unto you, Hereafter ye shall see heaven open, and the angels of God ascending and descending upon the Son of man.

\offertory{Thou hast crowned him with glory and worship: thou hast made him to have dominion of the works of thy hands, O Lord.}

\secret
\lett{W}{e} offer, O Lord, this gift to be hallowed unto thee: whereby, recalling the festival of thy blessed Apostle Andrew, we likewise implore the purification of our souls. Through.

\begin{rubric}
	In Advent, \nth{2} Collect of the Feria, \nth{3} of St. Saturninus (p. \pageref{SaturninusSecret}).
\end{rubric}
\begin{rubric}
	Outside of Advent, \nth{2} Collect of St. Saturninus (p. \pageref{SaturninusSecret}), \nth{3} of St. Mary  in Eastertide (p. \SPMaryEaster).
\end{rubric}

\communion{Andrew saith to Simon his brother: We have found the Messias, which is called the Christ: and he brought him to Jesus.}

\postcommunion
\lett{O}{Lord,} who hast bestowed on us these sacraments, we humbly beseech thee: that, at the intercession of thy blessed Apostle Andrew, the mysteries which we offer for his venerable passion may be profitable for our healing. Through.

\begin{rubric}
	In Advent, \nth{2} Collect of the Feria, \nth{3} of St. Saturninus (p. \pageref{SaturninusPostcommunion}).
\end{rubric}
\begin{rubric}
	Outside of Advent, \nth{2} Collect of St. Saturninus (p. \pageref{SaturninusPostcommunion}), \nth{3} of St. Mary  in Eastertide (p. \SPMaryEaster).
\end{rubric}


\subby{St. Saturninus}
\feastday{{St. Saturninus}}
\fancyhead[RE,LO]{29 November}
\begin{inhead}
    {Memorial\\
29 November}
\end{inhead}

\begin{rubric}
	The propers are from the Second Common of a Martyr not a Bishop (p. \pageref{CommonMartyrNotBishopII}), except the following Prayers.
\end{rubric}

\collect\label{SaturninusCollect}
\lett{O}{God,} who vouchsafest unto us to rejoice in the birthday of thy blessed Martyr Saturninus: grant, we pray thee; that we may be succoured by his merits. Through.

\secret\label{SaturninusSecret}
\lett{S}{anctify,} O Lord, we beseech thee, the offerings which we dedicate unto thee: and at the intercession of blessed Saturninus, thy Martyr, for their sake graciously regard us. Through.

\postcommunion\label{SaturninusPostcommunion}
\lett{W}{e} beseech thee, O Lord, that we, being sanctified by the receiving of thy sacrament: may, at the intercession of thy Saints, be thereby rendered acceptable unto thee. Through.

%MANUAL ADJUSTMENT:
\clearpage
\bcpfeast{30 November. St. Andrew}{St. Andrew}{30 November}
%\supplement{30 November}{St. Andrew}{}

\begin{secrubric}
	The Hymns and Versicles are of the Common of Apostles (p. \pageref{CommonApostles}), with the Antiphons as followeth.
\end{secrubric}

\properantiphon{Mag.}{One of the two {\dag} which followed the Lord was Andrew, Simon Peter's brother, alleluia.}

\properantiphon{Ben.}{Yield up to us {\dag} a man so righteous, restore to us a man so holy: destroy not a man so dear to God, righteous, dutiful, and gentle.}

\properantiphon{Mag.}{When blessed Andrew {\dag} came to the place where the Cross had been prepared, he cried out and said: O goodly Cross, so long desired, and now made ready for my eager spirit; fearless and joyful do I come to thee: therefore do thou also receive me gladly, as his disciple, who did hang upon thee.}

%MANUAL ADJUSTMENT:
\vspace{-1.75ex}

\subby{St. Peter Chrysologus}
\feastday{{St. Peter Chrysologus}}
\fancyhead[RE,LO]{2 December}
\begin{inhead}
    {Double\\
2 December}
\end{inhead}

\begin{rubric}
	The propers are from the Common of Doctors (p. \pageref{CommonDoctors}), except for the following.
\end{rubric}

\begin{rubric}
	\textsc{Note,} Commemoration is made of St. Bibiana and of the Advent Feria.
\end{rubric}

%MANUAL ADJUSTMENT:
\vspace{-1.5ex}

\collect
\lett{O}{God,} who by divine foreshewing wast pleased to choose blessed Peter Chrysologus thy illustrious Doctor to rule and instruct thy Church: grant, we beseech thee; that, as we have had him for a Doctor of life on earth, so we may be found worthy to have him for an intercessor in heaven. Through.

\gradual{Behold a great priest, who in his days pleased God. ℣. There was none found like unto him, who kept the law of the Most High.}

\communion{Lord. thou deliveredst unto me five talents: behold, I have gained beside them five talents more. Well done, thou good and faithful servant, thou hast been faithful over a few things, I will make thee ruler over many things, enter thou into the joy of thy lord.}

%MANUAL ADJUSTMENT:
\vspace{-1.75ex}

\subby{St. Bibiana}
\feastday{{St. Bibiana}}
\fancyhead[RE,LO]{2 December}
\begin{inhead}
    {Memorial\\
2 December}
\end{inhead}

\begin{rubric}
	The propers are from the Second Common of a Virgin Martyr (p. \pageref{CommonVirginMartyrII}), except for the following.
\end{rubric}

%MANUAL ADJUSTMENT:
\vspace{-1ex}

%MANUAL ADJUSTMENT:
\subsubsec{Collect}
\lett{O}{God,} the giver of all good gifts, who in thy handmaid Bibiana didst unite the palm of martyrdom with the flower of virginity: unite by her intercession our hearts in charity with thee; that all perils being done away, we may attain unto everlasting rewards. Through.

\secret
\lett{G}{raciously} receive, O Lord, through the merits of blessed Bibiana, thy Virgin and Martyr, the sacrifices which we offer unto thee: and grant that they may avail for our continual help. Through.

\postcommunion
\lett{O}{Lord} our God, who hast fulfilled us with the bounty of thy heavenly gift: we beseech thee, at the intercession of blessed Bibiana, thy Virgin and Martyr, we may ever live by the partaking of the same. Through.


\subby{St. Barbara}
\feastday{{St. Barbara}}
\fancyhead[RE,LO]{4 December}
\begin{inhead}
    {Memorial\\
4 December}
\end{inhead}

\begin{rubric}
	The propers are from the First Common of a Virgin Martyr (p. \pageref{CommonVirginMartyrI}).
\end{rubric}


\subby{St. Sabbas of Jud{\ae}a}
\feastday{{St. Sabbas}}
\fancyhead[RE,LO]{5 December}
\begin{inhead}
    {Memorial\\
5 December}
\end{inhead}

\begin{rubric}
	The propers are from the Common of an Abbot (p. \pageref{CommonAbbots}).
\end{rubric}

\begin{rubric}
	If today be Saturday, the Mass is of St. Mary on Saturday with \nth{2} Collect of the Feria \& \nth{3} of St. Sabbas.
\end{rubric}


\subby{St. Nicholas}
\feastday{{St. Nicholas}}
\fancyhead[RE,LO]{6 December}
\begin{inhead}
    {Double\\
6 December}
\end{inhead}

\begin{rubric}
	The Daily Office propers are from the First Common of a Confessor Bishop (p. \pageref{CommonConfessorBishopI}).
\end{rubric}

\introit
\lett{T}{he} Lord hath established a covenant of peace with him, and made him a prince: that he should have the dignity of the priesthood for ever. \textit{Ps.} Lord, remember David: and all his trouble.

\collect
\lett{O}{God,} who didst adorn thy blessed Bishop Nicholas with innumerable miracles: grant, we beseech thee; that by his merits and prayers we may be delivered from the fires of hell. Through.

\begin{rubric}
	Commemoration is of the Feria, unless it be Saturday.
\end{rubric}

\readingcitation{Epistle}{Hebrews 13:7}
\lett{B}{rethren:} Remember them which have the rule over you, who have spoken unto you the word of God: whose faith follow, considering the end of their conversation. Jesus Christ the same yesterday, and to day, and for ever. Be not carried about with divers and strange doctrines. For it is a good thing that the heart be established with grace; not with meats, which have not profited them that have been occupied therein. We have an altar, whereof they have no right to eat which serve the tabernacle. For the bodies of those beasts, whose blood is brought into the sanctuary by the high priest for sin, are burned without the camp. Wherefore Jesus also, that he might sanctify the people with his own blood, suffered without the gate. Let us go forth therefore unto him without the camp, bearing his reproach. For here have we no continuing city, but we seek one to come. By him therefore let us offer the sacrifice of praise to God continually, that is, the fruit of our lips giving thanks to his name. But to do good and to communicate forget not: for with such sacrifices God is well pleased. Obey them that have the rule over you, and submit yourselves: for they watch for your souls, as they that must give account.

\gradall{I have found David my servant, with my holy oil have I anointed him: my hand shall hold him fast, and my arm shall strengthen him. ℣. The enemy shall not be able to do him violence, the son of wickedness shall not hurt him.}{Alleluia, alleluia. ℣. The righteous shall flourish like a palm-tree: and shall spread abroad like a cedar in Libanus. Alleluia.}

\readingcitation{Gospel}{Matthew 25:14}
\lett{A}{t that time:} Jesus spake this parable to his disciples: A man travelling into a far country, who called his own servants, and delivered unto them his goods. And unto one he gave five talents, to another two, and to another one; to every man according to his several ability; and straightway took his journey. Then he that had received the five talents went and traded with the same, and made them other five talents. And likewise he that had received two, he also gained other two. But he that had received one went and digged in the earth, and hid his lord's money. After a long time the lord of those servants cometh, and reckoneth with them. And so he that had received five talents came and brought other five talents, saying, Lord, thou deliveredst unto me five talents: behold, I have gained beside them five talents more. His lord said unto him, Well done, thou good and faithful servant: thou hast been faithful over a few things, I will make thee ruler over many things: enter thou into the joy of thy lord. He also that had received two talents came and said, Lord, thou deliveredst unto me two talents: behold, I have gained two other talents beside them. His lord said unto him, Well done, good and faithful servant; thou hast been faithful over a few things, I will make thee ruler over many things: enter thou into the joy of thy lord.

%MANUAL ADJUSTMENT:
\clearpage
\offertory{My truth and my mercy shall be with him: and in my name shall his horn be exalted.}

\secret
\lett{S}{anctify,} we bescech thee, O Lord God, these gifts which we offer on the solemnity of thy holy Bishop Nicholas: that our life may ever thereby be directed both in prosperity and in adversity. Through.

\begin{rubric}
	Commemoration is of the Feria, unless it be Saturday.
\end{rubric}

\communion{I have sworn once by my holiness: His seed shall endure for ever, and his seat is like as the sun before me, he shall stand fast for evermore as the moon, and as the faithful witness in heaven.}

\postcommunion
\lett{M}{ay} the sacrifices which we have received, O Lord, for the solemnity of thy holy Bishop Nicholas, preserve us by their everlasting protection. Through.

\begin{rubric}
	Commemoration is of the Feria, unless it be Saturday.
\end{rubric}


\subby{St. Ambrose}
\feastday{{St. Ambrose}}
\fancyhead[RE,LO]{7 December}
\begin{inhead}
    {Greater Double\\
7 December}
\end{inhead}

\begin{rubric}
	The Daily Office propers are from the Common of a Confessor Bishop (p. \pageref{CommonConfessorBishopI}).
\end{rubric}

\introit
\lett{I}{n} the midst of the Church he opened his mouth: and the Lord filled him with the spirit of wisdom and of understanding: he clothed him with a robe of glory. \textit{Ps.} It is a good thing to give thanks unto the Lord: and to sing praises unto thy name, O most Highest.

\collect
\lett{O}{God,} who didst give blessed Ambrose unto thy people to be a minister of everlasting salvation: grant, we beseech thee; that as we have learned of him the doctrine of life on earth, so we may be found worthy to have him for our advocate in heaven. Through.

\begin{rubric}
	Commemoration is of the Feria.
\end{rubric}

\begin{rubric}
	The Epistle is from the Common of Doctors (p. \pageref{CommonDoctors}).
\end{rubric}

\gradall{Behold a great priest, who in his days pleased God. ℣. There was none foun like unto him, who kept the law of the Most High.}{Alleluia, alleluia. ℣. The Lord sware, and will not repent: Thou art a priest for ever after the order of Melchisedech.}

\begin{rubric}
{In Septuagesimatide or Lent, replacing the Alleluia:}
\end{rubric}\par\noindent
\tract{Blessed is the man that feareth the Lord: he hath great delight in his commandments. ℣. His seed shall be mighty upon earth: the generation of the faithful shall be blessed. ℣. Riches and plenteousness shall be in his house: and his righteousness endureth for ever.}

\begin{rubric}
{In Eastertide, replacing the Lesser Alleluia:}
\end{rubric}\par\noindent
\alleluia{Alleluia, alleluia. ℣. The Lord loved him, and adorned him: and clothed him with a robe of glory. Alleluia. ℣. The righteous shall grow as the lily and flourish for ever before the Lord. Alleluia.}

\begin{rubric}
	The Gospel is from the Common of Doctors (p. \pageref{CommonDoctors}).
\end{rubric}

\offertory{My truth and my mercy shall be with him: and in my name shall his horn be exalted.}

\secret
\lett{A}{lmighty} and everlasting God, grant, that the gifts which we present unto thy majesty, may through the intercession of blessed Ambrose, thy Confessor and Bishop, be profitable unto us for everlasting salvation. Through.

\begin{rubric}
	Commemoration is of the Feria.
\end{rubric}

\communion{I have sworn once by my holiness: His seed shall endure for ever, and his seat is like as the sun before me, he shall stand fast for evermore as the moon, and as the faithful witness in heaven.}

\postcommunion
\lett{G}{rant,} we beseech thee, almighty God: that we, receiving the sacraments of our salvation, may ever be aided by the prayer of blessed Ambrose thy Confessor and Bishop; in whose honour we have made these offerings unto thy majesty. Through.

\begin{rubric}
	Commemoration is of the Feria. In Lent, the Last Gospel is of the Feria.
\end{rubric}

%\begin{rubric}
%	If today be Saturday, the Mass is of the anticipated Vigil of the Conception of the Blessed Virgin Mary.
%\end{rubric}


\bcpfeast{8 December. Conception of the Blessed Virgin Mary}{Conception B.V.M.}{8 December}
%\supplement{8 December}{Conception}{of the Blessed Virgin Mary}

\begin{secrubric}
	The Hymns are from the Common of the Blessed Virgin Mary (p. \pageref{CommonBVM}) with the following Versicle and Antiphons.
\end{secrubric}

℣. To-day is the Conception of the holy Virgin Mary.

℟. Whose glorious life illumineth all the churches.

\properantiphon{Mag.}{All generations {\dag} shall call me blessed: for he that is mighty hath magnified me, alleluia.}\\

\properantiphon{Ben.}{The Lord God said {\dag} unto the serpent, I will put enmity between thee and the woman, and between thy seed and her seed; it shall bruise thy head, alleluia.}\\

\properantiphon{Mag.}{Let us celebrate {\dag} the worshipful Conception of the blessed and glorious Virgin Mary, whose lowliness the Lord regarded when at the word of an Angel she conceived the world's Redeemer, alleluia.}


\subby{St. Melchiades}
\feastday{{St. Melchiades}}
\fancyhead[RE,LO]{10 December}
\begin{inhead}
    {Memorial\\
10 December}
\end{inhead}

\begin{rubric}
	The propers are from the First Common of a Martyr Bishop (p. \pageref{CommonMartyrBishopI}).
\end{rubric}


\subby{St. Damasus}
\feastday{{St. Damasus}}
\fancyhead[RE,LO]{11 December}
\begin{inhead}
    {Memorial\\
11 December}
\end{inhead}

\introit
\lett{L}{et} thy priests, O Lord, be clothed with righteousness, and let thy Saints sing with joyfulness: for thy servant David's sake turn not away the presence of thine Anointed. \textit{Ps.} Lord, remember David: and all his trouble.

\collect
\lett{G}{raciously} hear our prayers, O Lord: and at the intercession of blessed Damasus, thy Confessor and Bishop, mercifully grant us pardon and peace. Through.

\begin{rubric}
	Commemoration is of the Octave \& Feria.
\end{rubric}

\begin{rubric}
	The Epistle is from the Second Common of a Confessor Bishop (p. \pageref{CommonConfessorBishopII}).
\end{rubric}

\gradall{Behold a great priest who in his days pleased God. ℣. There was none found like unto him, who kept the law of the Most High.}{Alleluia, alleluia. ℣. Thou art a priest for ever after the order of Melchisedech. Alleluia.}

\begin{rubric}
	The Gospel is from the Second Common of a Confessor Bishop (p. \pageref{CommonConfessorBishopII}).
\end{rubric}

\communion{I have found David my servant, with my holy oil have I anointed him: my hand shall hold him fast, and my arm shall strengthen him.}

\secret
\lett{G}{rant,} O Lord, that like as thy dedicated people do acknowledge that in tribulation they have been succoured by the merits of thy Saints: so this oblation, which they offer unto thee in honour of the same, may be acceptable in thy sight. Through.

\begin{rubric}
	Commemoration is of the Octave \& Feria.
\end{rubric}

\communion{Lord, thou deliveredst unto me five talents, behold I have gained beside them five talents more. Well done, thou good and faithful servant, thou hast been faithful over a few things, I will make thee ruler over many things, enter thou into the joy of thy Lord.}

\postcommunion
\lett{G}{rant,} we beseech thee, O Lord, that thy faithful people may ever rejoice in the veneration of thy saints: and be defended by their perpetual supplication. Through.

\begin{rubric}
	Commemoration is of the Octave \& Feria.
\end{rubric}


\subby{St. Lucy}
\feastday{{St. Lucy}}
\fancyhead[RE,LO]{13 December}
\begin{inhead}
    {Greater Double\\
13 December}
\end{inhead}

\begin{rubric}
	The Daily Office propers are from the Common of Virgins (p. \pageref{CommonVirginOnlyI}), except that which followeth.
\end{rubric}

\antiphon{Mag.}{In thy patience {\dag}} thou didst possess thy soul, O Lucy, spouse of Christ: thou didst hate the things which are in the world, and thou shinest among the Angels: resisting unto blood, thou didst vanquish the enemy.

\antiphon{Ben.}{A pillar art thou {\dag} that may not be moved, O Lucy, spouse of Christ: and all the people are waiting until thou receive the crown of life, alleluia.}\\

℣. Full of grace are thy lips.

℟. Because God hath blessed thee for ever.

\antiphon{Mag.}{With such gravity {\dag} was she endued by the Holy Spirit, that the virgin of the Lord remained unmoved.}

\introit
\lett{T}{hou} hast loved righteousness, and hated iniquity: wherefore God, even thy God, hath anointed thee with the oil of gladness above thy fellows. \textit{Ps.} My heart is inditing of a good matter: I speak of the things which I have made unto the King.

\collect
\lett{G}{raciously} hear us, O God of our salvation: that, like as we do rejoice in the festival of blessed Lucy thy Virgin; so we may be instructed in all godly and devout affection. Through.

\begin{rubric}
	Commemoration is of the Octave \& Feria.
\end{rubric}

\begin{rubric}
	The Epistle is from the First Common of a Virgin (p. \pageref{CommonVirginOnlyI}).
\end{rubric}

\gradall{Thou hast loved righteousness, and hated iniquity. ℣. Wherefore God, even thy God, hath anointed thee with the oil of gladness.}{Alleluia, alleluia. ℣. Full of grace are thy lips: because God hath blessed thee for ever. Alleluia.}

\begin{rubric}
	In Votive Masses after Septuagesima, the Tract, and in Eastertide, the Alleluia, is from the First Common of a Virgin (p. \pageref{CommonVirginOnlyI}).
\end{rubric}

\readingcitation{Gospel}{Matthew 13:44}
\lett{A}{t that time:} Jesus spake this parable unto his disciples: The kingdom of heaven is like unto treasure hid in a field; the which when a man hath found, he hideth, and for joy thereof goeth and selleth all that he hath, and buyeth that field. Again, the kingdom of heaven is like unto a merchant man, seeking goodly pearls: Who, when he had found one pearl of great price, went and sold all that he had, and bought it. Again, the kingdom of heaven is like unto a net, that was cast into the sea, and gathered of every kind: Which, when it was full, they drew to shore, and sat down, and gathered the good into vessels, but cast the bad away. So shall it be at the end of the world: the angels shall come forth, and sever the wicked from among the just, And shall cast them into the furnace of fire: there shall be wailing and gnashing of teeth. Jesus saith unto them, Have ye understood all these things? They say unto him, Yea, Lord. Then said he unto them, Therefore every scribe which is instructed unto the kingdom of heaven is like unto a man that is an householder, which bringeth forth out of his treasure things new and old.

\offertory{The Virgins that be her fellows shall be brought unto the King: they that bear her company shall be brought unto thee with joy and gladness: and shall enter into the palace of the Lord the King.}

\secret
\lett{G}{rant,} O Lord, that like as thy dedicated people do acknowledge that in tribulation they have been succoured by the merits of thy Saints: so this oblation, which they offer unto thee in honour of the same, may be acceptable in thy sight. Through.

\begin{rubric}
	Commemoration is of the Octave \& Feria.
\end{rubric}

\communion{Princes have persecuted me without a cause, but my heart standeth in awe of thy word: I am as glad of thy word, as one that findeth great spoils.}

\postcommunion
\lett{O}{Lord,} who hast satisfied thy family with sacred gifts: we beseech thee; that we may at all times be comforted by the intercession of her whose festival we celebrate. Through.

\begin{rubric}
	Commemoration is of the Octave \& Feria.
\end{rubric}

\subby{St. Herman of Alaska}
\feastday{{St. Herman}}
\fancyhead[RE,LO]{13 December}
\begin{inhead}
    {Memorial\\
13 December}
\end{inhead}

\begin{rubric}
	The propers are from the First Common of a Confessor not Bishop (p. \pageref{CommonConfessorNotBishopI}).
\end{rubric}


\subby{Day VII within the Octave of the Conception}
\feastday{{Conception Octave}}
\fancyhead[RE,LO]{14 December}
\begin{inhead}
    {Semidouble\\
14 December}
\end{inhead}

\begin{rubric}
	Of the Octave, as on December 9. But if Ember Wednesday fall on this day, of the Ember Day, with a Commemoration of the Octave. The Mass is with the \nth{3} Prayer of the Holy Ghost.
\end{rubric}

\begin{rubric}
	If an Ember Day occur on any of the following Feasts, the Commemoration of the Feria is omitted.
\end{rubric}


\subby{Octave Day of the Conception of the Blessed Virgin Mary}
\feastday{{Conception Octave Day}}
\fancyhead[RE,LO]{15 December}
\begin{inhead}
    {Greater Double\\
15 December}
\end{inhead}

\begin{rubric}
	Hymn, Versicle, Antiphon, \& Mass as on the Feast, with Commemoration of the Feria.
\end{rubric}


\subby{St. Eusebius of Vercelli}
\feastday{{St. Eusebius}}
\fancyhead[RE,LO]{16 December}
\begin{inhead}
    {Memorial\\
16 December}
\end{inhead}

\begin{rubric}
	The propers are from the Second Common of a Martyr Bishop (p. \pageref{CommonMartyrBishopII}), with commemoration of the Feria (and the Collect of St. Mary in Mass).
\end{rubric}


\subby{Expectation of the Blessed Virgin Mary}
\feastday{{Expectation}}
\fancyhead[RE,LO]{18 December}
\begin{inhead}
    {Double\\
18 December}
\end{inhead}

%Is this the same as the Mass of the BVM from Advent to Christmas except for the Gradual?
%\subsubsec{Daily Office Propers}\par\noindent
%%From the Dominican Breviary (https://archive.org/details/BrevariumIuxtaRitumSacriOrdinisPraedicatorum/black-and-white%20version/Breviarium_Cormier_1909_prior_cropped%20%285.18x6.5%29%20pg%201-1577%20%28vol.%201%29/page/n826/mode/1up):
\par\noindent
\textit{Opening Sentence.} Send ye the lamb to the ruler of the land from Sela to the wilderness, unto the mount of the daughter of Zion.\vr{Is. 16:1}\par

\begin{rubric}
	\textsc{Note,} The Magnificat Antiphons are of the `O' Antiphons (p. \pageref{OAntiphons}).
\end{rubric}

\begin{paracol}{2}[]
\sloppy
\begin{inhead}
	I Evensong
\end{inhead}

℣. Hail Mary, full of grace.

℟. The Lord is with thee.

\begin{rubric}
	The Office Hymns are from the First Sunday of Advent (p. \pageref{AdventI}).
\end{rubric}

\switchcolumn
	
\begin{inhead}
	Mattins
\end{inhead}

℣. The Holy Ghost shall come upon thee.

℟. And the power of the Highest shall overshadow thee.\\

\antiphon{Ben.}{He shall sit upon the throne of David, {\dag} and of his kingdom, for ever.}

\fussy

\end{paracol}

\begin{rubric}
	II Evensong as in I Evensong.
\end{rubric}


%%From Bute's Roman Breviary translation (https://archive.org/details/theromanbreviary01unknuoft/page/665/mode/1up):
%O Antiphons will always take precedence.
%\antiphon{Mag.}{The Holy Ghost shall come upon thee, O Mary, * fear not; thou shalt bear in thy womb the Son of God. Alleluia.}\par

%\antiphon{Mag.}{O maiden of maidens, how shall this be, since neither before nor henceforth hath there been, nor shall be such another? Daughters of Jerusalem, why look ye curiously upon me? What ye see is a mystery of God.}\par

\begin{rubric}
	The Mass propers are from the Common of the Blessed Virgin Mary (p. \pageref{CommonBVM}), except for the Collect as followeth.
\end{rubric}

\collect
\lett{O}{God,} who wast pleased that thy Word should take flesh of the womb of the Blessed Virgin Mary at the message of an Angel: grant to thy humble servants; that we who believe her to be truly the Mother of God may be aided by her intercession in thy sight. Through the same.

%MANUAL ADJUSTMENT:
\clearpage

\subby{Vigil of St. Thomas}
\feastday{{St. Thomas Vigil}}
\fancyhead[RE,LO]{20 December}
\begin{inhead}
    {Vigil\\
20 December}
\end{inhead}

\begin{rubric}
	If today be Sunday, in the Ember Saturday Mass, Commemoration is made of the anticipated Vigil of St. Thomas and the last Gospel of the Vigil is read at the end, the \nth{3} Collect is of St. Mary.
\end{rubric}

\begin{rubric}
	The Mass propers are from the Vigil of Apostles (p. \pageref{CommonVigilApostles}), with Commemoration of the Feria of Advent and the \nth{3} Collect of St. Mary. But if an Ember Day occur, Commemoration is made of the Vigil in the Mass of the Feria.
\end{rubric}

\bcpfeast{21 December. St. Thomas}{St. Thomas}{21 December}
%\supplement{21 December}{St. Thomas}{}

\begin{secrubric}
	The Office Hymn and Versicle are from the Common of Apostles (p. \pageref{CommonApostles}), with the following Antiphon.
\end{secrubric}

\properantiphon{Mag. \& Ben.}{Because thou hast seen me, {\dag} Thomas, thou hast believed: blessed are they that have not seen, and yet have believed, alleluia.}


\subby{St. Paul the First Hermit}
\feastday{{St. Paul Hermit}}
\fancyhead[RE,LO]{10 January}
\begin{inhead}
    {Memorial\\
10 January}
\end{inhead}

\begin{rubric}
	The propers are from the Common of Abbots (p. \pageref{CommonAbbots}), except that which followeth.
\end{rubric}

\introit
\lett{T}{he} just shall flourish like a palm-tree: and shall spread abroad like a cedar in Libanus: planted in the house of the Lord: in the courts of the house of our God. \textit{Ps.} It is a good thing to give thanks unto the Lord: and to sing praises unto thy name, O Most Highest.

\collect
\lett{O}{God,} who makest us glad with the yearly solemnity of blessed Paul, thy Confessor: mercifully grant; that, as we now celebrate his birthday, so we may follow the example of his life. Through.

\gradall{The just shall flourish like a palm-tree: and shall spread abroad like a cedar in Libanus in the house of the Lord. ℣. To tell of thy loving-kindness early in the morning, and of thy truth in the night-season.}{Alleluia, alleluia. ℣. The just shall grow as the lily: and flourish for ever before the Lord. Alleluia.}

\readingcitation{Gospel}{Matthew 11:25}
\lett{A}{t that time:} Jesus answered and said: I thank thee, O Father, Lord of heaven and earth, because thou hast hid these things from the wise and prudent, and hast revealed them unto babes. Even so, Father: for so it seemed good in thy sight. All things are delivered unto me of my Father: and no man knoweth the Son, but the Father; neither knoweth any man the Father, save the Son, and he to whomsoever the Son will reveal him. Come unto me, all ye that labour and are heavy laden, and I will give you rest. Take my yoke upon you, and learn of me; for I am meek and lowly in heart: and ye shall find rest unto your souls. For my yoke is easy, and my burden is light.

\offertory{The just shall rejoice in thy strength, O Lord: exceeding glad shall he be of thy salvation: thou hast given him his heart's desire.}

\secret
\lett{G}{rant,} we beseech thee, O Lord, that we who, trusting in this our sacrifice of praise, do offer it before thee to the honour of thy Saints: may by the same be delivered from all evils both in this life and that which is to come. Through.

\communion{The just shall rejoice in the Lord, and put his trust in him: and all they that are true of heart shall be glad.}

\postcommunion
\lett{O}{Lord,} our God, who hast refreshed us with heavenly meat and drink, we humbly beseech thee: that we may be defended by the prayers of him in whose memory we have received the same. Through.


\subby{St. Hyginus of Rome}
\feastday{{St. Hyginus}}
\fancyhead[RE,LO]{11 January}
\begin{inhead}
    {Memorial\\
11 January}
\end{inhead}

\begin{rubric}
	The propers are from the First Common of a Martyr not a Bishop (p. \pageref{CommonMartyrNotBishopI}).
\end{rubric}


%MANUAL ADJUSTMENT:
\clearpage
\subby{St. Benedict Biscop}
\feastday{{St. Benedict Biscop}}
\fancyhead[RE,LO]{12 January}
\begin{inhead}
    {Memorial\\
12 January}
\end{inhead}

\begin{rubric}
	The propers are from the Common of Abbots (p. \pageref{CommonAbbots}), except for that which followeth.
\end{rubric}

\collect
\lett{O}{God,} by whose gift the blessed Abbot Benedict left all things that he might be made perfect: grant unto all those who have entered upon the path of evangelical perfection, that they may neither look back nor linger in the way; but hastening to thee without stumbling, may lay hold upon eternal life. Through.

\secret
\lett{W}{e} beseech thee, O Lord, that thy holy Abbot Benedict, may may intercede for us: that this sacrifice which we offer and present upon thy holy altar may be profitable unto us for our salvation. Through.

\postcommunion
\lett{L}{et} thy sacrament, O Lord, which we have now received and the prayers of the blessed Abbot Benedict, effectually defend us: that we may both imitate the example of his conversion, and receive the succour of his intercession. Through.


\subby{St. Hilary}
\feastday{{St. Hilary}}
\fancyhead[RE,LO]{14 January}
\begin{inhead}
    {Double\\
14 January}
\end{inhead}


\begin{rubric}
	The propers are from the Common of Doctors (p. \pageref{CommonDoctors}).\par
	\textsc{Note,} Commemoration is made of St. Felix, with the Prayers from the Mass below.
\end{rubric}


\subby{St. Felix}
\feastday{{St. Felix}}
\fancyhead[RE,LO]{14 January}
\begin{inhead}
    {Memorial\\
14 January}
\end{inhead}

\begin{rubric}
	The propers are from the Second Common of a Martyr not a Bishop (p. \pageref{CommonMartyrNotBishopII}), except that which followeth.\par
\end{rubric}

\collect
\lett{G}{rant,} we beseech thee, almighty God, that the examples of thy Saints may provoke us to a better life: that as we celebrate their festival so we may imitate their actions. Through.

\secret
\lett{W}{e} beseech thee, O Lord, mercifully to accept this our sacrifice, which we offer unto thee, pleading the merits of blessed Felix, thy Martyr: that the same may avail for our perpetual succour. Through.

\postcommunion
\lett{O}{Lord,} who hast fulfilled us with saving mysteries: we beseech thee that we may be aided by the prayers of blessed Felix thy Martyr, whose festival we celebrate. Through.


%https://archive.org/details/missale-monasticum-propers/mode/2up?view=theater

\subby{St. Maurus}
\feastday{{St. Maurus}}
\fancyhead[RE,LO]{15 January}
\begin{inhead}
    {Greater Double\\
15 January}
\end{inhead}

\begin{paracol}{2}[]
\sloppy
\begin{inhead}
	I Evensong\label{MaurusEvensong}
\end{inhead}
\begin{hangparas}{1.25em}{1}
Defender, leader true, thine own companions deem'd

Thy splendour half divine, since all were less esteem'd:

For thy most worthy deeds, Maurus, accept the lays

Wherewith we celebrate thy praise.\\

Born of a noble stock, great honour was his due,

But palaces he spurn'd, and from the world withdrew;

Delights he trampl'd down, estates and robes unpric'd,

To undergo the yoke of Christ.\\

The holy Abbot's grace, before his eyes display'd,

By deeds of equal worth he eagerly portray'd;

The pattern of the life monastic shone in truth

From every action of the youth.\\

Sternly, with sackcloth rough, self-mastery he wrought,

And by the curb of law, unbroken silence sought;

The ever-watchful nights in fervent prayer he spent;

Whole days of fasting underwent\\

Right speedily he flew to do the father's hest,

Dry-shod, the waters deep with fearless feet he press'd;

And safely he return'd with Placidus, set free

Like Peter walking on the sea.\\

To thee, O Trinity, high praise and honour be,

Whose countenance desir'd the heaven-dwellers see:

Grant that the Holy Rule may be our pathway plain

The prize of Maurus to attain.  Amen.\\
\end{hangparas}

	℣. The Lord loved him and adorned him.

	℟. He clothed him with a robe of glory.
	
	\antiphon{Mag.}{O most blessed of men! {\dag} who, rejecting this world, bore the yoke of Holy Rule from tender years so lovingly; and being made obedient even unto death, he denied himself, that he might wholly cling to Christ his Master, alleluia.}
	
	\switchcolumn
	
\begin{inhead}
	Mattins Invitatory\label{MaurusInvitatory}
\end{inhead}
\begin{hangparas}{1.25em}{1}
In childhood Placidus was by his father giv'n;

Thus offer'd, he himself did freely yield to God.

Since first he came, he ever shone with wondrous grace,

A pattern to all zealous souls.\\

His Abbot sent him forth to fill an earthen crock,

He dips it in the lake, unwary, slips and falls,

A wave then carries him a bow-shot from the shore,

Out in the deep drowning is nigh.\\

But Maurus, what is this, that hast'ning to the lake,

Thou runnest o'er the waves as though thou wast on land?

Wont to obey, 'tis thus thy holy father's voice

Doth lead thee to a miracle.\\

Straightway the lake restores Placidus safe again,

But whose the merit? Did his Nursian father draw

Him from the swirling depths, or was it Maurus' act?

The child resolves their questioning.\\

O Holy Trinity, through prayers of Placidus,

Grant to thy monks that by the narrow path of Rule

They may at length attain unto the courts of heav'n,

And mingle with celestial choirs. Amen.\\
\end{hangparas}

\fussy
\end{paracol}

\subbysub{Mattins}

\begin{rubric}
	The Office Hymn \& Versicle are from the First Common of a Confessor not a Bishop (p. \pageref{CommonConfessorNotBishopI}), with the Antiphon from I Evensong.
\end{rubric}

\subbysub{II Evensong}

\begin{rubric}
	The Office Hymn is of I Evensong, with the Versicle \& Antiphon as followeth.
\end{rubric}

℣. The Lord guided the righteous in right paths.

℟. And shewed him the kingdom of God.\\

\antiphon{Mag.}{To-day holy Maurus, {\dag} lying upon a goat-skin, died happily before the altar; to-day the first-begotten disciple of blessed Benedict, through the guiding of the Holy Rule, came up to Christ, rising untroubled, accompanied by choirs of Angels; today the obedient man, telling his victories, was worthy to be crowned by the Lord, alleluia.}

\introit
\lett{T}{hy} way is in the sea, and thy paths in the great waters: and thy footsteps are not known. Thou leddest thy people like sheep. \textit{Ps.} The waters saw thee, O God, the waters saw thee, and were afraid: the depths also were troubled.

\collect
\lett{O}{God,} who for a pattern of obedience didst cause blessed Maurus to walk dry-shod upon the waters: grant that we may both follow perfectly the example of his virtues, and also be worthy to share in his reward. Through.

\readingcitation{Epistle}{Ecclesiasticus 51:13}
%RV:
\lett{W}{hen} I was yet young, Or ever I went abroad, I sought wisdom openly in my prayer. Before the temple I asked for her, And I will seek her out even to the end. From her flower as from the ripening grape my heart delighted in her: My foot trod in uprightness, From my youth I tracked her out. I bowed down mine ear a little, and received her, And found for myself much instruction. I profited in her: Unto him that giveth me wisdom I will give glory. For I purposed to practice her, And I was zealous for that which is good; And I shall never be put to shame. My soul hath wrestled in her, And in my doing I was exact: I spread forth my hands to the heaven above, And bewailed my ignorances of her. I set my soul aright unto her, And in pureness I found her. I gat me a heart joined with her from the beginning: Therefore shall I not be forsaken. My inward part also was troubled to seek her: Therefore have I gotten a good possession. The Lord gave me a tongue for my reward; And I will praise him therewith.
\par
Draw near unto me, ye unlearned, And lodge in the house of instruction. Say, wherefore are ye lacking in these things, And your souls are very thirsty? I opened my mouth, and spake, Get her for yourselves without money. Put your neck under the yoke, And let your soul receive instruction: She is hard at hand to find. Behold with your eyes, How that I laboured but a little, And found for myself much rest. Get you instruction with a great sum of silver, And gain much gold by her. May your soul rejoice in his mercy, And may ye not be put to shame in praising him. Work your work before the time cometh, And in his time he will give you your reward. 

\gradall{I will bring thy seed from the east, and gather thee from the west. ℣. When thou passest through the waters, I will be with thee; and through the rivers, they shall not overflow thee.}{Alleluia, alleluia. ℣. An obedient man shall speak of victory; Wealth and riches shall be in his house. Alleluia.}

\begin{rubric}
	In Septuagesimatide \& Lent, the Alleluia is omitted and the Tract is said instead.
\end{rubric}

\tract{Blessed is the man that feareth the Lord: he hath great delight in his commandments. ℣. His seed shall be mighty upon earth: the generation of the faithful shall be blessed. ℣. Riches and plenteousness shall be in his house: and his righteousness endureth for ever.}

\readingcitation{Gospel}{Matthew 14:28}
\lett{A}{t that time:} Peter answered Jesus and said, Lord, if it be thou, bid me come unto thee on the water. And he said, Come. And when Peter was come down out of the ship, he walked on the water, to go to Jesus. But when he saw the wind boisterous, he was afraid; and beginning to sink, he cried, saying, Lord, save me. And immediately Jesus stretched forth his hand, and caught him, and said unto him, O thou of little faith, wherefore didst thou doubt? And when they were come into the ship, the wind ceased. Then they that were in the ship came and worshipped him, saying, Of a truth thou art the Son of God.

\offertory{And the places that have been desolate for ages shall be built in thee: thou shalt raise up the foundations of generation and generation: turning the paths into rest. And I will feed thee with the inheritance of thy Father.}

%CHECK TRANSLATION:
\secret
\lett{M}{ay} the sacrifices we offer ascend unto thee, O Lord, as an odour of sweetness; and may the intercession of blessed Maurus, Abbot, intervene for us, that thy propitious power may descend upon us. Through.

\communion{I have chosen you, and ordained you, that ye should go and bring forth fruit, and that your fruit should remain: that whatsoever ye shall ask of the Father in my name, he may give it you.}

%CHECK TRANSLATION:
\postcommunion
\lett{W}{e} implore thy mercy, O Lord our God, that having received the pledges of our salvation and giving thanks for thy help; thou may accept our offerings unto our salvation; sending upon us thy grace to support us; that the celestial blessing, which thou hast brought us by the patronage of blessed Maurus, Abbot, may be perfected by continuing in imitation of him. Through.


\subby{St. Marcellus}
\feastday{{St. Marcellus}}
\fancyhead[RE,LO]{16 January}
\begin{inhead}
    {Memorial\\
16 January}
\end{inhead}

\begin{rubric}
	The propers are from the Second Common of a Martyr Bishop (p. \pageref{CommonMartyrBishopII}), except for that which followeth.
\end{rubric}

\introit
\lett{T}{he} Lord hath established a covenant of peace with him, and made him a prince: that he should have the dignity of the priesthood for ever. \textit{Ps.} Lord, remember David: and all his trouble.

\collect
\lett{O}{Lord,} we beseech thee favourably to hear the prayers of thy people: that, as we do rejoice in the passion of blessed Marcellus thy Martyr and Bishop, so we may be succoured by his merits. Through.

\gradall{I have found David my servant, with my holy oil have I anointed him: my hand shall hold him fast, and my arm shall strengthen him. ℣. The enemy shall not be able to do him violence, the son of wickedness shall not hurt him.}{Alleluia, alleluia. ℣. Thou art a priest for ever, after the order of Melchisedech. Alleluia.}

\offertory{My truth and my mercy shall be with him: and in my name shall his horn be exalted.}

\secret
\lett{R}{eceive,} O Lord, we beseech thee, the gifts which we duly offer: and by the pleading of the merits of blessed Marcellus thy Martyr and Bishop, grant that they may avail to set forward our salvation. Through.

\communion{Lord, thou deliveredst unto me five talents: behold, I have gained beside them five talents more. Well done, thou good and faithful servant, thou hast been faithful over a few things, I will make thee ruler over many things, enter thou into the joy of thy lord.}

\postcommunion
\lett{O}{Lord,} who hast satisfied thy family with sacred gifts: we beseech thee; that we may at all times be comforted by the intercession of him whose festival we celebrate. Through.


\subby{St. Anthony}
\feastday{{St. Anthony}}
\fancyhead[RE,LO]{17 January}
\begin{inhead}
    {Double\\
17 January}
\end{inhead}

\begin{rubric}
	The propers are from the Common of Abbots (p. \pageref{CommonAbbots}), except for the Gospel which is from the First Common of a Confessor not a Bishop (p. \pageref{CommonConfessorNotBishopI}).
\end{rubric}


\subby{Chair of St. Peter at Rome}
\feastday{{Chair St. Peter Rome}}
\fancyhead[RE,LO]{18 January}
\begin{inhead}
    {Greater Double\\
18 January}
\end{inhead}

\begin{rubric}
	The Daily Office propers are from the Chair of St. Peter at Antioch (p. \pageref{CathedraAntioch}).
\end{rubric}

\introit
\lett{T}{he} Lord hath established a covenant of peace with him, and made him a prince: that he should have the dignity of the priesthood for ever. \textit{Ps.} Lord, remember David: and all his trouble.

\collect
\lett{O}{God,} who didst bestow upon thy blessed Apostle Peter the keys of the kingdom of heaven, and didst appoint unto him the high priesthood of binding and loosing: vouchsafe; that by the help of his intercession we may be delivered from the bonds of our iniquities. Who livest and reignest.

\lett{O}{God,} who by the preaching of the blessed Apostle Paul didst teach the multitude of the Gentiles: grant to us, we beseech thee; that we who celebrate his commemoration may know him to be our advocate with thee. (Through.)

\begin{rubric}
    Commemoration is made of St. Prisca of Rome (p. \pageref{PriscaCollect}).
\end{rubric}

\readingcitation{Epistle}{1 Peter 1:1}
\lett{P}{eter} Peter, an apostle of Jesus Christ, to the strangers scattered throughout Pontus, Galatia, Cappadocia, Asia, and Bithynia, elect according to the foreknowledge of God the Father, through sanctification of the Spirit, unto obedience and sprinkling of the blood of Jesus Christ: Grace unto you, and peace, be multiplied. Blessed be the God and Father of our Lord Jesus Christ, which according to his abundant mercy hath begotten us again unto a lively hope by the resurrection of Jesus Christ from the dead, to an inheritance incorruptible, and undefiled, and that fadeth not away, reserved in heaven for you, who are kept by the power of God through faith unto salvation ready to be revealed in the last time. Wherein ye greatly rejoice, though now for a season, if need be, ye are in heaviness through manifold temptations: that the trial of your faith, being much more precious than of gold that perisheth, though it be tried with fire, might be found unto praise and honour and glory at the appearing of Jesus Christ our Lord.

\gradall{Let them exalt him in the congregation of the people: and praise him in the seat of the elders. ℣. O that men would praise the Lord for his goodness, and declare the wonders that he doeth for the children of men.}{Alleluia, alleluia. ℣. Thou art Peter, and upon this rock I will build my Church. Alleluia.}

\begin{rubric}
	In Septuagesimatide \& Lent, the Alleluia is replaced with the following.
\end{rubric}

\tract{Thou art Peter, and upon this rock I will build my Church. ℣. And the gates of hell shall not prevail against it: and I will give unto thee the keys of the kingdom of heaven. ℣. Whatsoever thou shalt bind on earth shall be bound in heaven. ℣. And whatsoever thou shalt loose on earth shall be loosed in heaven.}

\begin{rubric}
	In Eastertide, the Alleluia is replaced with the following.
\end{rubric}

\alleluia{Alleluia, alleluia. ℣. O that men would praise the Lord for his goodness, and declare the wonders that he doeth for the children of men. Alleluia. ℣. Thou art Peter, and upon this rock I will build my Church. Alleluia.}

\readingcitation{Gospel}{Matthew 16:13}
\lett{A}{t that time:} When Jesus came into the coasts of C{\ae}sarea Philippi, he asked his disciples, saying, Whom do men say that I the Son of man am? And they said, Some say that thou art John the Baptist: some, Elias; and others, Jeremias, or one of the prophets. He saith unto them, But whom say ye that I am? And Simon Peter answered and said, Thou art the Christ, the Son of the living God. And Jesus answered and said unto him, Blessed art thou, Simon Bar-jona: for flesh and blood hath not revealed it unto thee, but my Father which is in heaven. And I say also unto thee, That thou art Peter, and upon this rock I will build my church; and the gates of hell shall not prevail against it. And I will give unto thee the keys of the kingdom of heaven: and whatsoever thou shalt bind on earth shall be bound in heaven: and whatsoever thou shalt loose on earth shall be loosed in heaven.

\offertory{Thou art Peter, and upon this rock I will build my Church: and the gates of hell shall not prevail against it: and I will give unto thee the keys of the kingdom of heaven.}

\secret
\lett{W}{e} beseech thee, O Lord, that the intercession of blessed Peter the Apostle may commend unto thee the prayers and sacrifices of thy Church: that those things which we celebrate for his glory may avail for our pardon. Through.
\needspace{4\baselineskip}
\lett{S}{anctify,} O Lord, through the prayers of thine Apostle Paul, the gifts of thy people: that those things, which by thine institution are pleasing unto thee, may be made more pleasing by his prayer and advocacy. (Through.)

\begin{rubric}
    Commemoration is made of St. Prisca of Rome (p. \pageref{PriscaSecret}).
\end{rubric}

%MANUAL ADJUSTMENT:
\vspace{-1.5ex}

\communion{Thou art Peter, and upon this rock I will build my Church.}

%MANUAL ADJUSTMENT:
\vspace{-1ex}

\postcommunion
\lett{M}{ay} the gift, O Lord, which we have offered, make us to rejoice: that as we proclaim thy wonders in thine Apostle Peter; so through him we may receive the abundance of thy loving-kindness. Through.
\needspace{4\baselineskip}
\lett{O}{Lord,} who hast sanctified us with this saving mystery: we beseech thee; that he whom thou hast given to be our advocate and guide may never fail in prayer for us. (Through.)

\begin{rubric}
    Commemoration is made of St. Prisca of Rome (p. \pageref{PriscaPostcommunion}).
\end{rubric}

%MANUAL ADJUSTMENT:
\vspace{-1ex}

\subby{St. Prisca of Rome}
\feastday{{St. Prisca Rome}}
\fancyhead[RE,LO]{18 January}
\begin{inhead}
    {Memorial\\
18 January}
\end{inhead}

\begin{rubric}
	The propers are from the Second Common of a Virgin Martyr (p. \pageref{CommonVirginMartyrII}), except for that which followeth.
\end{rubric}

%MANUAL ADJUSTMENT:
\vspace{-1ex}

\collect\label{PriscaCollect}
\lett{G}{rant,} we beseech thee, almighty God: that we who celebrate the heavenly birthday of blessed Prisca, thy Virgin and Martyr; may both rejoice in her yearly solemnity, and profit by the example of so great a faith. Through.

%MANUAL ADJUSTMENT:
\vspace{-1ex}

\secret\label{PriscaSecret}
\lett{W}{e} beseech thee, O Lord, that this sacrifice which we offer in remembrance of the birthday of thy Saints may both loose us from the bonds of our iniquity, and obtain for us the gifts of thy mercy. Through.

%MANUAL ADJUSTMENT:
\subsubsec{Postcommunion}\label{PriscaPostcommunion}
%\postcommunion\label{PriscaPostcommunion}
\lett{O}{Lord,} who hast fulfilled us with saving mysteries: we beseech thee that we may be aided by the prayers of her whose festival we celebrate. Through.

%MANUAL ADJUSTMENT:
\clearpage
\subby{Sts. Marius, Martha, Audifax, \& Abachum}
\feastday{{Sts. Marius \&c.}}
\fancyhead[RE,LO]{19 January}
\begin{inhead}
    {Memorial\\
19 January}
\end{inhead}

\introit
\lett{L}{et} the righteous be glad and rejoice before God: let them also be merry and joyful. \textit{Ps.} Let God arise, and let his enemies be scattered: let them also that hate him flee before him.

\collect
\lett{G}{raciously} hear thy people, O Lord, who call upon thee with the advocacy of thy Saints: and grant us both to rejoice in peace in this temporal life; and to find succour unto life everlasting. Through.

\begin{rubric}
    Commemoration is made of St. Mark of Ephesus (p. \pageref{EphesianCollect}) and St. Mary in Epiphanytide (p. \SPMaryChristmas).
\end{rubric}

\begin{rubric}
	The Epistle is from the Third Common of Many Martyrs (p. \pageref{CommonMartyrsIII}).
\end{rubric}

\gradall{The souls of the just are in the hand of God: and there shall no torment of malice touch them. ℣. In the sight of the unwise they seemed to die: but they are in peace.}{Alleluia, alleluia. ℣. Our God is wonderful in his Saints. Alleluia.}

\begin{rubric}
	In Septuagesimatide \& Lent, the Alleluia is replaced with the Tract from the Third Common of Many Martyrs (p. \pageref{CommonMartyrsIII}).
\end{rubric}

\begin{rubric}
	The Gospel is from the first additional Gospel of the Third Common of Many Martyrs (p. \pageref{Matthew243}).
\end{rubric}

\offertory{Our soul is escaped, even as a bird out of the snare of the fowler: the snare is broken, and we are delivered.}

\secret
\lett{R}{egard,} O Lord, the prayers and oblations of thy faithful people: that they may be acceptable unto thee for the festival of thy Saints, and bestow on us the succour of thy mercy. Through.

\begin{rubric}
    Commemoration is made of St. Mark of Ephesus (p. \pageref{EphesianCollect}) and St. Mary in Epiphanytide (p. \SPMaryChristmas).
\end{rubric}

\communion{I say unto you, my friends: Be not afraid of them that persecute you.}

\postcommunion
\lett{G}{rant,} we beseech thee, that the intercession of thy Saints may make us acceptable unto thee: that those things which we perform in this temporal celebration we may receive unto eternal salvation. Through.

\begin{rubric}
    Commemoration is made of St. Mark of Ephesus (p. \pageref{EphesianCollect}) and St. Mary in Epiphanytide (p. \SPMaryChristmas).
\end{rubric}

%MANUAL ADJUSTMENT:
\vspace{-2ex}

%PROVISIONAL:
\subby{St. Mark of Ephesus}
\feastday{{St. Mark Ephesus}}
\fancyhead[RE,LO]{19 January}
\begin{inhead}
    {Memorial\\
19 January}
\end{inhead}

\begin{rubric}
	The propers are from the Second Common of a Confessor Bishop (p. \pageref{CommonConfessorBishopII}), except for that which followeth.
\end{rubric}

%Through thee shone uncreated light

%Of the life given to the least,

%And like a giant of great might

%Thou didst advance from the East.\\

%Thou didst illumine the whole world

%With the rays of thy true words,

%O divine Mark, Bishop of Christ,

%Pray for us for that same light.\\

%Thine el'quent lips and honey'd tongue

%Became the mouthpiece of his grace;

%Thy sacred tongue was shown to be

%A pen of wisdom, grace, and peace.\\

%All glory, Lord, to thee we pay

%For the Ephesian which thou gav'st,

%All glory, as is ever meet,

%To Father and to Paraclete. Amen.

\par\noindent
\textit{Yet to be approved.}

\vspace{-1ex}

\begin{tcolorbox}[
  colback=white,           % background color
  colframe=black,          % frame color
  fonttitle=\itshape,     % optional font for title
  title style={left=2mm},  % optional title left alignment
  attach title to upper={},
  enhanced,
  boxed title style={size=small,colframe=black!50!white,colback=white},
]

\antiphon{Mag.}{O blessed Mark {\dag} clothed with invincible armour, thou didst cast down rebellious pride.}

\antiphon{Ben.}{Thou didst serve {\dag} as the instrument of the Paraclete, and shone forth as the champion of Orthodoxy.}

\antiphon{Mag.}{Rejoice, Mark, {\dag} the boast of the Orthodox and joy to all true Catholics!}

\collect
\label{EphesianCollect}
\lett{O}{Lord} Jesu Christ, the only Shepherd and Bishop of our souls; we beseech thee to keep us in thy truth, preserve us by thy grace, and govern us according to thy loving-kindness. That as we imitate the good example of blessed Mark thy Confessor and Bishop, we may grow in love for thy Word and service of thy Church. Who with.

\secret
\label{EphesianSecret}
\lett{R}{eceive,} O most holy Father, these gifts now offered unto thee in memory of Saint Mark. That as he was a bulwark and defender of the Catholic Faith, so we may, by these gifts, be given the same firm love of thee. Through.

\postcommunion
\label{EphesianPostcommunion}
\lett{A}{bide} in us, O divine Comforter, who have here received thine holy Sacraments. That just as thy faithful servant Mark did contend for the unity of the Body of Christ, so we may be cleansed and knit together by the same. Who with the Father, from whom alone thou dost proceed, and the Son, who sendeth thy power upon us, liveth and reigneth, God, world without end. Amen.
\end{tcolorbox}

%MANUAL ADJUSTMENT:
\clearpage
\subby{Sts. Fabian \& Sebastian}
\feastday{{Sts. Fabian \& Sebastian}}
\fancyhead[RE,LO]{20 January}
\begin{inhead}
    {Double\\
20 January}
\end{inhead}

\begin{rubric}
	The Daily Office propers are from the First Common of Many Martyrs (p. \pageref{CommonMartyrsI}).
\end{rubric}

\introit
\lett{L}{et} the sorrowful sighing of the prisoners, O Lord, come before thee, reward thou our neighbours seven-fold into their bosom: avenge thou the blood of thy Saints that is shed. \textit{Ps.} O God, the heathen are come into thine inheritance: thy holy temple have they defiled: and made Jerusalem an heap of stones.

\collect
\lett{A}{lmighty} God, mercifully look upon our infirmities: that whereas we are oppressed by the burden of our sins, the glorious intercession of thy blessed Martyrs Fabian and Sebastian may be our succour and defence. Through.

\begin{rubric}
	The Epistle is the fifth additional Epistle of the Third Common of Many Martyrs (p. \pageref{Hebrews1133}).
\end{rubric}

\gradall{God is glorious in his holy ones: fearful in praises, doing wonders. ℣. Thy right hand, O Lord, is become glorious in power: thy right hand hath dashed in pieces the enemy.}{Alleluia, alleluia. ℣. Thy Saints give thanks unto thee, O Lord: they shew the glory of thy kingdom. Alleluia.}

\begin{rubric}
	In Septuagesimatide \& Lent, the Alleluia is replaced with the following.
\end{rubric}

\tract{They that sow in tears, shall reap in joy. ℣. He that now goeth on his way weeping, and beareth forth good seed. ℣. Shall doubtless come again with joy, and bring his sheaves with him.}

\readingcitation{Gospel}{Luke 6:17}
\lett{A}{t that time:} Jesus came down from the mountain, and stood in the plain, and the company of his disciples, and a great multitude of people out of all Jud{\ae}a and Jerusalem, and from the sea coast of Tyre and Sidon, which came to hear him, and to be healed of their diseases; and they that were vexed with unclean spirits: and they were healed. And the whole multitude sought to touch him: for there went virtue out of him, and healed them all. And he lifted up his eyes on his disciples, and said, Blessed be ye poor: for yours is the kingdom of God. Blessed are ye that hunger now: for ye shall be filled. Blessed are ye that weep now: for ye shall laugh. Blessed are ye, when men shall hate you, and when they shall separate you from their company, and shall reproach you, and cast out your name as evil, for the Son of man’s sake. Rejoice ye in that day, and leap for joy: for, behold, your reward is great in heaven.

\offertory{Be glad, O ye righteous, and rejoice in the Lord: and be joyful, all ye that are true of heart.}

\secret
\lett{W}{e} beseech thee, O Lord, mercifully to accept this our sacrifice which we offer unto thee, pleading the merits of thy blessed Martyrs Fabian and Sebastian: that the same may avail for our perpetual succour. Through.

\communion{A multitude of sick folk, and they that were vexed with unclean spirits came to him: for there went virtue out of him, and healed them all.}

\postcommunion
\lett{W}{e} beseech thee, O Lord our God, that like as we whom thou hast refreshed by the partaking of thy sacred gift do offer unto thee our worship: so, by the intercession of thy holy Martyrs Fabian and Sebastian, we may perceive the benefit of the same. Through.


\subby{St. Agnes of Rome}
\feastday{{St. Agnes Rome}}
\fancyhead[RE,LO]{21 January}
\begin{inhead}
    {Greater Double\\
21 January}
\end{inhead}

\begin{rubric}
	The Daily Office propers are from the Common of a Virgin (p. \pageref{CommonVirginOnlyI}), except for that which followeth.
\end{rubric}

\antiphon{Mag.}{Blessed Agnes, {\dag} in the midst of the flames, stretched out her hands and prayed: I call on thee, O Father transcendent, august, and dread; for by thy holy Son's protection I have escaped the threats of an impious tyrant, and passed unscathed through the foulness of fleshly pollution: and behold, I come to thee, whom I have loved, whom I have sought, whom I have alway desired.}

\antiphon{Ben.}{Lo, that which I desired, {\dag} now I see; that for which I hoped, I now possess; I am united in heaven unto him, whom on earth I loved with a perfect devotion.}\\

℣. Full of grace are thy lips.

℟. Because God hath blessed thee for ever.

\antiphon{Mag.}{While the blessed Agnes {\dag} was standing in the midst of the flames, she stretched out her hands, and prayed unto God: Almighty Lord, worthy of all adoration, fear, and worship: I bless thy holy Name, and I glorify thee for ever and ever.}

\introit
\lett{T}{he} ungodly wait for me to destroy me: O Lord, I will consider thy testimonies: I see that all things come to an end: but thy commandment is exceeding broad. \textit{Ps.} Blessed are those that are undefiled in the way: and walk in the law of the Lord.

\collect
\lett{A}{lmighty} and everlasting God, who dost choose the weak things of the world to confound those things that are strong: mercifully grant; that we who keep the feast of blessed Agnes thy Virgin and Martyr may feel the succour of her intercession in thy sight. Through.

\readingcitation{Epistle}{Ecclesiasticus 51:1}
%RV:
\lett{I}{will} give thanks unto thee, O Lord, O King, And will praise thee, O God my Saviour: I do give thanks unto thy name: For thou wast my protector and helper, And didst deliver my body out of destruction, And out of the snare of a slanderous tongue, From lips that forge lies, And wast my helper before them that stood by; And didst deliver me, according to the abundance of thy mercy, and greatness of thy name, From the gnashings of teeth ready to devour, Out of the hand of such as sought my life, Out of the manifold afflictions which I had; From the choking of a fire on every side, And out of the midst of fire which I kindled not; Out of the depth of the belly of the grave, And from an unclean tongue, And from lying words, The slander of an unrighteous tongue unto the king. My soul drew near even unto death, And my life was near to the grave beneath. They compassed me on every side, And there was none to help me. I was looking for the succour of men, And it was not. And I remembered thy mercy, O Lord, And thy working which hath been from everlasting, How thou deliverest them that wait for thee, And savest them out of the hand of the enemies, O Lord our God.

%\lett{I}{will} give thanks unto thee, O Lord, O King, And will praise thee, O God my Saviour: I do give thanks unto thy name: For thou wast my protector and helper, And didst deliver my body out of destruction, And out of the snare of a slanderous tongue, From lips that forge lies, And wast my helper before them that stood by;  And didst deliver me, according to the abundance of thy mercy, and greatness of thy name, From the gnashings of teeth ready to devour, Out of the hand of such as sought my life, Out of the manifold afflictions which I had; From the choking of a fire on every side, And out of the midst of fire which I kindled not; Out of the depth of the belly of the grave, And from an unclean tongue, And from lying words, The slander of an unrighteous tongue unto the king. My soul drew near even unto death, And my life was near to the grave beneath. And I remembered thy mercy, O Lord, And thy working which hath been from everlasting, How thou deliverest them that wait for thee, And savest them out of the hand of the enemies, O Lord our God.

\gradall{Full of grace are thy lips: because God hath blessed thee for ever. ℣. Because of the word of truth, of meekness, and righteousness: and thy right hand shall teach thee terrible things.}{Alleluia, alleluia. ℣. The five wise virgins took oil in their vessels with their lamps: and at midnight there was a cry made: Behold, the bridegroom cometh: go ye out to meet Christ the Lord. Alleluia.}

\begin{rubric}
{In Septuagesimatide or Lent, replacing the Alleluia:}
\end{rubric}\par\noindent
\tract{Come, Spouse of Christ, receive the crown which the Lord hath prepared for thee for ever: for love of whom thou didst shed thy blood. ℣. Thou hast loved righteousness and hated iniquity: wherefore God, even thy God, hath anointed thee with the oil of gladness above thy fellows. ℣. In thy comeliness and in thy beauty go forth, ride prosperously, and reign.}

\readingcitation{Gospel}{Matthew 25:1}
\lett{A}{t that time:} Jesus spake this parable unto his disciples: The kingdom of heaven shall be likened unto ten virgins, which took their lamps, and went forth to meet the bridegroom. And five of them were wise, and five were foolish. They that were foolish took their lamps, and took no oil with them:  But the wise took oil in their vessels with their lamps. While the bridegroom tarried, they all slumbered and slept. And at midnight there was a cry made, Behold, the bridegroom cometh; go ye out to meet him. Then all those virgins arose, and trimmed their lamps. And the foolish said unto the wise, Give us of your oil; for our lamps are gone out. But the wise answered, saying, Not so; lest there be not enough for us and you: but go ye rather to them that sell, and buy for yourselves. And while they went to buy, the bridegroom came; and they that were ready went in with him to the marriage: and the door was shut. Afterward came also the other virgins, saying, Lord, Lord, open to us. But he answered and said, Verily I say unto you, I know you not. Watch therefore, for ye know neither the day nor the hour wherein the Son of man cometh.

\offertory{The Virgins that be her fellows shall be brought unto the King: they that bear her company shall be brought unto thee with joy and gladness: and shall enter into the palace of the Lord the King.}

\secret
\lett{O}{Lord,} mercifully regard the sacrifices which we offer unto thee: and at the intercession of blessed Agnes, thy Virgin and Martyr, absolve us from the bonds of our sins. Through.

\communion{The five wise virgins took oil in their vessels with their lamps: and at midnight there was a cry made: Behold, the bridegroom cometh: go ye out to meet Christ the Lord.}

\postcommunion
\lett{O}{Lord,} our God, who hast refreshed us with heavenly meat and drink, we humbly beseech thee: that we may be defended by the prayers of her in whose memory we have received the same. Through.


\subby{Sts. Vincent \& Anastasius}
\feastday{{St. Vincent \& Anastasius}}
\fancyhead[RE,LO]{22 January}
\begin{inhead}
    {Memorial\\
22 January}
\end{inhead}

%New propers needed for addition of Anastasius

\begin{rubric}
	The propers are from the First Common of Many Martyrs (p. \pageref{CommonMartyrsI}).
\end{rubric}

%\begin{rubric}
%	The propers are from the Second Common of a Martyr not a Bishop (p. \pageref{CommonMartyrNotBishopII}), except for that which followeth.
%\end{rubric}

%\collect
%\lett{A}{ssist} us, O Lord, in our supplications: that we who perceive the guilt of our iniquities, may be delivered through the intercession of thy blessed Martyr Vincent. Through.

%CHECK TRANSLATION:
%\secret
%\lett{W}{e} beseech thee, O Lord, to accept our service and prayers, cleanse us by thy heavenly mysteries, and graciously hear us. Through.

%CHECK TRANSLATION:
%\postcommunion
%\lett{W}{e} beseech thee, almighty God; that, by the intercession of thy blessed Martyr Vincent, we who have received heavenly food may be strengthened against all adversity by the same. Through.


\subby{St. Emerentiana}
\feastday{{St. Emerentiana}}
\fancyhead[RE,LO]{23 January}
\begin{inhead}
    {Memorial\\
23 January}
\end{inhead}

\begin{rubric}
	The propers are from the Second Common of a Virgin and Martyr (p. \pageref{CommonVirginMartyrII}).
\end{rubric}


\subby{St. Timothy}
\feastday{{St. Timothy}}
\fancyhead[RE,LO]{24 January}
\begin{inhead}
    {Double\\
24 January}
\end{inhead}

\begin{rubric}
	The propers come from the First Common of a Bishop and Martyr (p. \pageref{CommonMartyrBishopI}), except for the Epistle, as below.
\end{rubric}

\readingcitation{Epistle}{1 Timothy 6:11}
\lett{D}{early beloved:} Follow after righteousness, godliness, faith, love, patience, meekness. Fight the good fight of faith, lay hold on eternal life, whereunto thou art also called, and hast professed a good profession before many witnesses. I give thee charge in the sight of God, who quickeneth all things, and before Christ Jesus, who before Pontius Pilate witnessed a good confession; That thou keep this commandment without spot, unrebukeable, until the appearing of our Lord Jesus Christ: Which in his times he shall shew, who is the blessed and only Potentate, the King of kings, and Lord of lords; Who only hath immortality, dwelling in the light which no man can approach unto; whom no man hath seen, nor can see: to whom be honour and power everlasting. Amen.

\bcpfeast{25 January. Conversion of St. Paul the Apostle}{St. Paul's Conversion}{25 January}
%\supplement{25 January}{Conversion}{of St. Paul the Apostle}

\subbysub{I Evensong}\label{PaulEvensong}

\gregorioscore{resources/gabc/ProperTime/PaulEvensong.gabc}

℣. Thou art a chosen vessel, holy Apostle Paul.

℟. A preacher of the truth throughout all the world.

\properantiphon{Mag.}{Go forth, Ananias, {\dag} and enquire for Saul, for behold, he prayeth: for he is a chosen vessel unto me, to bear my Name before the Gentiles, and Kings, and the children of Israel.}

\subbysub{Mattins}

\begin{rubric}
	The Invitatory Hymn is as in I Evensong.
\end{rubric}

\begin{rubric}
The Office Hymn is from the Common of the Apostles (p. \pageref{CommonApostles}), with the following Versicle \& Antiphon.
\end{rubric}

℣. Thou art a chosen vessel, holy Apostle Paul.

℟. A preacher of the truth throughout all the world.

\properantiphon{Ben.}{Ye which have followed me {\dag} shall sit upon twelve thrones, judging the twelve tribes of Israel, saith the Lord.}

\subbysub{II Evensong}

\begin{rubric}
	The Office Hymn \& Versicle are as in I Evensong, with the following Antiphon.
\end{rubric}

\properantiphon{Mag.}{O holy Apostle Paul, {\dag} thou preacher of the truth and Doctor of the Gentiles, intercede for us unto God, who hath chosen thee.}


\subby{St. Polycarp of Smyrna}
\feastday{{St. Polycarp Smyrna}}
\fancyhead[RE,LO]{26 January}
\begin{inhead}
    {Memorial\\
26 January}
\end{inhead}

\introit
\lett{O}{ye} priests of the Lord, bless ye the Lord: O ye holy and humble men of heart, bless ye the Lord. \textit{Ps.} O all ye works of the Lord, bless ye the Lord: praise him, and magnify him for ever.

\collect
\lett{O}{God,} who makest us glad with the yearly solemnity of blessed Polycarp, thy Martyr and Bishop: mercifully grant; that, as we now cebrate his birthday, so we may likewise rejoice in his protection. Through.

\readingcitation{Epistle}{1 John 3:10}
\lett{D}{early beloved:} Whosoever doeth not righteousness is not of God, neither he that loveth not his brother. For this is the message that ye heard from the beginning, that we should love one another. Not as Cain, who was of that wicked one, and slew his brother. And wherefore slew he him? Because his own works were evil, and his brother's righteous. Marvel not, my brethren, if the world hate you. We know that we have passed from death unto life, because we love the brethren. He that loveth not his brother abideth in death. Whosoever hateth his brother is a murderer: and ye know that no murderer hath eternal life abiding in him. Hereby perceive we the love of God, because he laid down his life for us: and we ought to lay down our lives for the brethren.

\gradall{Thou hast crowned him with glory and worship. ℣. Thou hast made him to have dominion of the works of thy hands, O Lord.}{Alleluia, alleluia. ℣. This is a priest whom the Lord hath crowned. Alleluia.}

\begin{rubric}
{In Septuagesimatide or Lent, replacing the Alleluia:}
\end{rubric}\par\noindent
\tract{Blessed is the man that feareth the Lord: he hath great delight in his commandments. ℣. His seed shall be mighty upon earth: the generation of the faithful shall be blessed. ℣. Riches and plenteousness shall be in his house: and his righteousness endureth for ever.}

\readingcitation{Gospel}{Matthew 10:26}
\lett{A}{t that time:} Jesus said to his disciples: There is nothing covered, that shall not be revealed; and hid, that shall not be known. What I tell you in darkness, that speak ye in light: and what ye hear in the ear, that preach ye upon the housetops. And fear not them which kill the body, but are not able to kill the soul: but rather fear him which is able to destroy both soul and body in hell. Are not two sparrows sold for a farthing? and one of them shall not fall on the ground without your Father. But the very hairs of your head are all numbered. Fear ye not therefore, ye are of more value than many sparrows. Whosoever therefore shall confess me before men, him will I confess also before my Father which is in heaven.

\offertory{I have found David my servant, with my holy oil have I anointed him: my hand shall hold him fast, and my arm shall strengthen him.}

\secret
\lett{S}{anctify,} O Lord, the gifts which we dedicate to thee: that at the intercession of blessed Polycarp, thy Martyr and Bishop, they may obtain for us thy gracious favour. Through.

\communion{Thou hast set, O Lord, a crown of pure gold upon his head.}

\postcommunion
\lett{W}{e} beseech thee, O Lord our God, that like as we, whom thou hast refreshed by the partaking of thy sacred gift, do offer unto thee our worship: so, by the intercession of blessed Polycarp, thy Martyr and Bishop, we may perceive the benefit of the same. Through.


\subby{St. John Chrysostom}
\feastday{{St. John Chrysostom}}
\fancyhead[RE,LO]{27 January}
\begin{inhead}
    {Double\\
27 January}
\end{inhead}

\begin{rubric}
	The propers are from the Common of Doctors (p. \pageref{CommonDoctors}), except for that which followeth.
\end{rubric}

\collect
\lett{M}{ultiply,} we beseech thee, O Lord, thy Church with thy heavenly grace: even as thou didst vouchsafe to enlighten her with the glorious merits and doctrine of blessed John Chrysostom, thy Confessor and Bishop. Through.

\gradall{Behold a great priest, who in his days pleased God. ℣. There was none found like unto him, who kept the law of the Most High.}{Alleluia, alleluia. ℣. Blessed is the man that endureth temptation: for when he is tried, he shall receive the crown of life, alleluia.}

\begin{rubric}
{In Septuagesimatide or Lent, replacing the Alleluia:}
\end{rubric}\par\noindent
\tract{Blessed is the man that feareth the Lord: he hath great delight in his commandments. ℣. His seed shall be mighty upon earth: the generation of the faithful shall be blessed. ℣. Riches and plenteousness shall be in his house: and his righteousness remaineth for ever.}

\secret
\lett{M}{ay} the devout prayers of Saint John Chrysostom thy Bishop and Doctor, never fail to succour us, O Lord: that they may render our oblations acceptable in thy sight; and may ever obtain for us thy merciful pardon. Through.

\postcommunion
\lett{W}{e} beseech thee, O Lord, that blessed John Chrysostom, thy Bishop and illustrious Doctor; may stand before thee as our advocate: that these thy sacrifices may avail for our salvation. Through.


\subby{St. Cyril of Alexandria}
\feastday{{St. Cyril Alexandria}}
\fancyhead[RE,LO]{28 January}
\begin{inhead}
    {Double\\
28 January}
\end{inhead}

\begin{rubric}
	The propers are from the Common of Doctors (p. \pageref{CommonDoctors}), except for that which followeth.
\end{rubric}

\collect
\lett{O}{God,} who didst make blessed Cyril, thy Confessor and Bishop, an invincible defender of the divine Motherhood of the most blessed Virgin Mary: grant, by his intercession; that we, who believe her to be indeed the Mother of God, may be saved through her maternal protection. Through the same.

\begin{rubric}
    Commemoration is made of St. Agnes (p. \pageref{AgnesCollectII}).
\end{rubric}

\secret
\lett{A}{lmighty} God, graciously look upon our gifts: and at the intercession of blessed Cyril vouchsafe; that we may be found meet worthily to receive in our hearts thine only-begotten Son, Jesus Christ our Lord, co-eternal with thee in thy glory. Who liveth and reigneth with thee.

\begin{rubric}
    Commemoration is made of St. Agnes (p. \pageref{AgnesSecretII}).
\end{rubric}

\postcommunion
\lett{O}{Lord,} who hast refreshed us with divine mysteries, we humbly beseech thee: that, being aided by the example and merits of the blessed Bishop Cyril, we may be enabled worthily to serve the most holy Mother of thine only-begotten Son. Who liveth and reigneth with thee.

\begin{rubric}
    Commemoration is made of St. Agnes (p. \pageref{AgnesPostcommunionII}).
\end{rubric}


%MANUAL ADJUSTMENT:
\clearpage
\subby{The Second Feast of St. Agnes}
\feastday{{Second St. Agnes}}
\fancyhead[RE,LO]{28 January}
\begin{inhead}
    {Memorial\\
28 January}
\end{inhead}

\begin{rubric}
	The propers are from the First Common of a Virgin (p. \pageref{CommonVirginOnlyI}), except for that which followeth.
\end{rubric}

\introit
\lett{A}{ll} the rich among the people shall make their supplication before thee: the Virgins that be her fellows shall be brought unto the King: they that bear her company shall be brought unto thee with joy and gladness. \textit{Ps.} My heart is inditing of a good matter: I speak of the things which I have made unto the King.

\collect\label{AgnesCollectII}
\lett{O}{God,} who makest us glad with the yearly solemnity of blessed Agnes, thy Virgin and Martyr: grant, we beseech thee; that as we venerate her in our service, so we may follow the example of her godly conversation. Through.

\readingcitation{Gospel}{Matthew 13:44}
\lett{A}{t that time:} Jesus spake this parable unto his disciples: The kingdom of heaven is like unto treasure hid in a field; the which when a man hath found, he hideth, and for joy thereof goeth and selleth all that he hath, and buyeth that field. Again, the kingdom of heaven is like unto a merchant man, seeking goodly pearls: Who, when he had found one pearl of great price, went and sold all that he had, and bought it. Again, the kingdom of heaven is like unto a net, that was cast into the sea, and gathered of every kind: Which, when it was full, they drew to shore, and sat down, and gathered the good into vessels, but cast the bad away. So shall it be at the end of the world: the angels shall come forth, and sever the wicked from among the just, And shall cast them into the furnace of fire: there shall be wailing and gnashing of teeth. Jesus saith unto them, Have ye understood all these things? They say unto him, Yea, Lord. Then said he unto them, Therefore every scribe which is instructed unto the kingdom of heaven is like unto a man that is an householder, which bringeth forth out of his treasure things new and old.

\offertory{Full of grace are thy lips, because God hath blessed thee for ever and ever.}

\secret\label{AgnesSecretII}
\lett{L}{et} thy plenteous benediction, we beseech thee, O Lord, come down upon these sacrifices: that it may mercifully work out our sanctification, and make us to rejoice in the solemnity of thy Martyrs. Through.

\communion{The kingdom of heaven is like unto a merchant man, seeking goodly pearls: who when he had found one pearl of great price, gave all that he had, and bought it.}

\postcommunion\label{AgnesPostcommunionII}
\lett{G}{rant,} we beseech thee, O Lord, that the sacrament which we have received in our observance of this yearly festival: may bestow on us thy healing; both in this temporal life and unto life eternal. Through.


\subby{St. Martina}
\feastday{{St. Martina}}
\fancyhead[RE,LO]{30 January}
\begin{inhead}
    {Memorial\\
30 January}
\end{inhead}

\begin{rubric}
	The propers are from the First Common of a Virgin and Martyr (p.\pageref{CommonVirginMartyrI}).
\end{rubric}


\subby{St. Ignatius of Antioch}
\feastday{{St. Ignatius Antioch}}
\fancyhead[RE,LO]{1 February}
\begin{inhead}
    {Double\\
1 February}
\end{inhead}

\begin{rubric}
	The Daily Office propers are from the First Common of Martyr Bishops (p. \pageref{CommonMartyrBishopI}).
\end{rubric}

\introit
\lett{B}{ut} God forbid that I should glory, save in the Cross of our Lord Jesus Christ: by whom the world is crucified unto me, and I unto the world. \textit{Ps.} Lord, remember David: and all his trouble.

\collect
\lett{A}{lmighty} God, mercifully look upon our infirmities; that whereas we are oppressed by the burden of our sins, the glorious intercession of blessed Ignatius thy Martyr and Bishop may be our succour and defence. Through.

\begin{rubric}
    Commemoration is made of St. Bridget of Ireland, from the First Common of a Virgin (p. \pageref{CommonVirginOnlyI}).
\end{rubric}

\readingcitation{Epistle}{Romans 8:35}
\lett{B}{rethren:} Who shall separate us from the love of Christ? shall tribulation, or distress, or persecution, or famine, or nakedness, or peril, or sword? As it is written, For thy sake we are killed all the day long; we are accounted as sheep for the slaughter. Nay, in all these things we are more than conquerors through him that loved us. For I am persuaded, that neither death, nor life, nor angels, nor principalities, nor powers, nor things present, nor things to come, Nor height, nor depth, nor any other creature, shall be able to separate us from the love of God, which is in Christ Jesus our Lord.

\gradall{Behold a great priest, who in his days pleased God. ℣. There was none found like unto him, who kept the law of the Most High.}{Alleluia, alleluia. ℣. I am crucifed with Christ: I live, yet not I, but Christ liveth in me. Alleluia.}

\begin{rubric}
{In Septuagesimatide or Lent, replacing the Alleluia:}
\end{rubric}\par\noindent
\tract{Thou hast given him his heart's desire: and hast not denied him the request of his lips. ℣. For thou hast prevented him with the blessings of goodness. ℣. Thou hast set a crown of pure gold upon his head.}

\readingcitation{Gospel}{John 12:24}
\lett{A}{t that time:} Jesus said unto his disciples: Verily, verily, I say unto you, Except a corn of wheat fall into the ground and die, it abideth alone: but if it die, it bringeth forth much fruit. He that loveth his life shall lose it; and he that hateth his life in this world shall keep it unto life eternal. If any man serve me, let him follow me; and where I am, there shall also my servant be: if any man serve me, him will my Father honour.

\offertory{Thou hast crowned him with glory and worship: and hast made him to have dominion of the works of thy hands, O Lord.}

\secret
\lett{W}{e} beseech thee, O Lord, mercifully to accept this our sacrifice which we offer unto thee, pleading the merits of blessed Ignatius thy Martyr and Bishop: that the same may avail for our perpetual succour. Through.

\begin{rubric}
    Commemoration is made of St. Bridget of Ireland, from the First Common of a Virgin (p. \pageref{CommonVirginOnlyI}).
\end{rubric}

\communion{I am the wheat of Christ: let me be ground by the teeth of beasts, that I may be found pure bread.}

\postcommunion
\lett{W}{e} beseech thee, O Lord, our God, that as we whom thou hast refreshed by the partaking of thy sacred gift offer unto thee our worship: so by the intercession of blessed Ignatius thy Martyr and Bishop, we may perceive the benefit of the same. Through.

\begin{rubric}
    Commemoration is made of St. Bridget of Ireland, from the First Common of a Virgin (p. \pageref{CommonVirginOnlyI}).
\end{rubric}


%%CHECK MISSAL:
\subby{St. Bridget of Ireland}
\feastday{{St. Bridget}}
\fancyhead[RE,LO]{1 February}
\begin{inhead}
    {Memorial\\
1 February}
\end{inhead}

\begin{rubric}
	The propers are from the First Common of a Virgin (p. \pageref{CommonVirginOnlyI}).
\end{rubric}

%From Common(?):
%\collect\label{BridgetCollect}
%\lett{H}{ear} us, O God our Saviour: and grant that as we rejoice in the festival of blessed Bridget, thy Virgin, so we may be brought to perfect holiness by the love of piety and devotion. Through.

\bcpfeast{2 February. Purification of the Blessed Virgin Mary}{Purification B.V.M.}{2 February}
%\supplement{2 February}{Purification}{of the Blessed Virgin Mary}

\begin{secrubric}
	The Office Hymn and Versicle are from the Common of the Blessed Virgin Mary (p. \pageref{CommonBVM}), except for the Evensong Versicles and the Antiphons as followeth.
\end{secrubric}

℣. It was revealed unto Simeon by the Holy Ghost. 

℟. That he should not see death before he had seen the Lord's Christ.

\properantiphon{Mag.}{The ancient {\dag} carried the Infant, but the Infant governed the ancient: he whom a Virgin bare, and after bearing, remained virgin, the same was worshipped by her who bare him.}\\

\properantiphon{Ben.}{And when the parents {\dag} brought in the Child Jesus, then Simeon took him up in his arms, and blessed God, saying: Lord, now lettest thou thy servant depart in peace.}\\

\properantiphon{Mag.}{To-day {\dag} the blessed Virgin Mary presented the Child Jesus in the temple; and Simeon, filled with the Holy Spirit, received him into his arms, and blessed God for ever.}


\subby{St. Blaise}
\feastday{{St. Blaise}}
\fancyhead[RE,LO]{3 February}
\begin{inhead}
    {Memorial\\
3 February}
\end{inhead}

\begin{rubric}
	The propers are from the Second Common of a Martyr Bishop (p. \pageref{CommonMartyrBishopII}).
\end{rubric}


\subby{St. Joseph of Aleppo}
\feastday{{St. Joseph Aleppo}}
\fancyhead[RE,LO]{4 February}
\begin{inhead}
    {Memorial\\
4 February}
\end{inhead}

\begin{rubric}
	The propers are from the First Common of a Martyr not a Bishop (p. \pageref{CommonMartyrNotBishopI}).
\end{rubric}

\begin{rubric}
	Commemoration is made of the New Martyrs of Russia, from the Prayers in the First Common of Many Martyrs (p. \pageref{CommonMartyrsI}).
\end{rubric}

\subby{The New Martyrs of Russia}
\feastday{{New Russian Martyrs}}
\fancyhead[RE,LO]{4 February}
\begin{inhead}
    {Memorial\\
4 February}
\end{inhead}

\begin{rubric}
	The propers are from the First Common of Many Martyrs (p. \pageref{CommonMartyrsI}).
\end{rubric}


\subby{St. Agatha}
\feastday{{St. Agatha}}
\fancyhead[RE,LO]{5 February}
\begin{inhead}
    {Greater Double\\
5 February}
\end{inhead}

\begin{rubric}
	The Office Hymns and Versicles are from the First Common of a Virgin Martyr (p. \pageref{CommonVirginMartyrI}), except for the II Evensong Versicle as below.
\end{rubric}

\antiphon{Mag.}{The blessed Agatha, {\dag} standing in the midst of the prison, with outstretched hands entreated the Lord: O Lord Jesus Christ, my gracious Master, I give thanks unto thee, who hast enabled me to overcome the torments of the executioner: bid me now, O Lord, joyfully to enter into thine unfading glory.}\\

\antiphon{Ben.}{The multitude {\dag} of the heathen, fleeing to the tomb of the virgin, took thence her veil to defend them from the fire: that the Lord might shew himself a deliverer from the burning, for the merits of Agatha his blessed Martyr.}\\

℣. Full of grace are thy lips.

℟. Because God hath blessed thee for ever.

\antiphon{Mag.}{The blessed Agatha, {\dag} standing in the midst of the prison, with outstretched hands entreated the Lord: O Lord Jesus Christ, my gracious Master, I give thanks unto thee, who hast enabled me to overcome the torments of the executioner: bid me now, O Lord, joyfully to enter into thine unfading glory.}

\introit
\lett{R}{ejoice} we all in the Lord, keeping feast day in honour of blessed Agatha, the Virgin and Martyr: in whose passion the Angels rejoice, and glorify the Son of God. \textit{Ps.} My heart is inditing of a good matter: I speak of the things which I have made unto the King.

\collect
\lett{O}{God,} who among the manifold works of thy power hast bestowed even upon the weakness of women the victory of martyrdom: mercifully grant; that we, who celebrate the birthday of blessed Agatha, thy Virgin and Martyr, may by her example be drawn nearer unto thee. Through.

\readingcitation{Epistle}{1 Corinthians 1:26}
\lett{B}{rethren:} Ye see your calling: how that not many wise men after the flesh, not many mighty, not many noble, are called: But God hath chosen the foolish things of the world to confound the wise; and God hath chosen the weak things of the world to confound the things which are mighty; And base things of the world, and things which are despised, hath God chosen, yea, and things which are not, to bring to nought things that are: That no flesh should glory in his presence. But of him are ye in Christ Jesus, who of God is made unto us wisdom, and righteousness, and sanctification, and redemption: That, according as it is written, He that glorieth, let him glory in the Lord.

\gradall{God shall help her with his countenance: God is in the midst of her, therefore shall she not be removed. ℣. The rivers of the flood thereof shall make glad the city of God: the holy place of the tabernacle of the Most Highest.}{Alleluia, alleluia. ℣. I will speak of thy testimonies even before kings: and will not be ashamed. Alleluia.}

\vspace{1ex}

\begin{rubric}
{In Septuagesimatide or Lent, replacing the Alleluia:}
\end{rubric}\par\noindent
\vspace{-3ex}
\tract{They that sow in tears, shall reap in joy. ℣. They that now go on their way weeping, and bear forth good seed. ℣. Shall doubtless come again with joy, and bring their sheaves with them.}

\readingcitation{Gospel}{Matthew 19:3}
\lett{A}{t that time:} The Pharisees came unto Jesus, tempting him, and saying unto him: Is it lawful for a man to put away his wife for every cause? And he answered and said unto them, Have ye not read, that he which made them at the beginning made them male and female, And said, For this cause shall a man leave father and mother, and shall cleave to his wife: and they twain shall be one flesh? Wherefore they are no more twain, but one flesh. What therefore God hath joined together, let not man put asunder. They say unto him, Why did Moses then command to give a writing of divorcement, and to put her away? He saith unto them, Moses because of the hardness of your hearts suffered you to put away your wives: but from the beginning it was not so. And I say unto you, Whosoever shall put away his wife, except it be for fornication, and shall marry another, committeth adultery: and whoso marrieth her which is put away doth commit adultery. His disciples say unto him, If the case of the man be so with his wife, it is not good to marry. But he said unto them, All men cannot receive this saying, save they to whom it is given. For there are some eunuchs, which were so born from their mother's womb: and there are some eunuchs, which were made eunuchs of men: and there be eunuchs, which have made themselves eunuchs for the kingdom of heaven's sake. He that is able to receive it, let him receive it.

\offertory{The Virgins that be her fellows shall be brought unto the King: they that bear her company shall be brought unto thee.}

%MANUAL ADJUSTMENT:
\vspace{-1ex}

\secret
\lett{R}{eceive,} O Lord, the gifts which we offer on the solemnity of blessed Agatha, thy Virgin and Martyr, through whose advocacy we trust to be delivered. Through.

\communion{He who deigned to heal my every wound, and to restore my breast unto my body, on him do I call, the living God.}

%MANUAL ADJUSTMENT:
\vspace{-1ex}

\postcommunion
\lett{M}{ay} the mysteries which we have received be for our succour, O Lord: and at the intercession of blessed Agatha, thy Virgin and Martyr, cause us to rejoice in thy continual protection. Through.


\subby{St. Dorothea}
\feastday{{St. Dorothea}}
\fancyhead[RE,LO]{6 February}
\begin{inhead}
    {Memorial\\
6 February}
\end{inhead}

\begin{rubric}
	The propers are from the Second Common of a Virgin Martyr (p. \pageref{CommonVirginMartyrII}), except for that which followeth.
\end{rubric}

%MANUAL ADJUSTMENT:
\vspace{-1.5ex}

\collect
\lett{W}{e} beseech thee, O Lord, that as blessed Dorothea, thy Virgin and Martyr, was ever found pleasing unto thee, both by the merit of her chastity, and by her confession of thy power: so she may implore for us thy pardon. Through.

%MANUAL ADJUSTMENT:
\vspace{-1.5ex}

\secret
\lett{G}{raciously} receive, O Lord, through the merits of bressed Dorotnea thy Virgin and Martyr, the sacrifices which we offer unto thee and grant that they may avail for our continual help. Through.

\postcommunion
\lett{O}{Lord} our God, who hast fulfilled us with the bounty of thy heavenly gift: we beseech thee that, at the intercession of blessed Dorothea thy Virgin and Martyr, we may ever live by the partaking of the same. Through.

%MANUAL ADJUSTMENT:
\clearpage
\subby{St. Romuald}
\feastday{{St. Romuald}}
\fancyhead[RE,LO]{7 February}
\begin{inhead}
    {Double\\
7 February}
\end{inhead}

\begin{rubric}
	The propers are from the Common of Abbots (p. \pageref{CommonAbbots}).
\end{rubric}


\subby{St. Apollonia of Alexandria}
\feastday{{St. Apollonia}}
\fancyhead[RE,LO]{9 February}
\begin{inhead}
    {Memorial\\
9 February}
\end{inhead}

\begin{rubric}
	The propers are from the First Common of a Virgin Martyr (p. \pageref{CommonVirginMartyrI}), except for that which followeth.
\end{rubric}

\collect
\lett{O}{God} who among the manifold works of thy power hast bestowed even upon the weakness of women the victory of martyrdom mercifully grant; that we, who celebrate the birthday of blessed Apollonia thy Virgin and Martyr, may by her example be drawn nearer unto thee. Through.

\secret
\lett{R}{eceive,} O Lord, the gifts which we offer on the solemnity of blessed Apollonia thy Virgin and Martyr: through whose advocacy we trust to be delivered. Through.

\postcommunion
\lett{M}{ay} the mysteries which we have received be for our succour, O Lord: and at the intercession of blessed Apollonia thy Virgin and Martyr, cause us to rejoice in thy continual protection. Through.


\bcpfeast{10 February. St. Scholastica}{St. Scholastica}{10 February}

\subbysub{I Evensong}\label{ScholasticaEvensong}

\gregorioscore{resources/gabc/ProperTime/ScholasticaEvensong.gabc}

℣. Who is this that flieth as a cloud? 

℟. And as a dove to her windows?

\antiphon{Mag.}{Let all the multitude {\dag} of the faithful exult in the glory of the gracious virgin Scholastica: and chiefly let the company of virgins be joyful, celebrating her Solemnity; for she besought the Lord, pouring forth her tears, and of him received greater power, because her love was greater.}

\subbysub{Mattins}

\invitatoryhymn\label{ScholasticaInvitatory}

\gregorioscore{resources/gabc/ProperTime/ScholasticaInvitatory.gabc}

\officehymn\label{ScholasticaMattins}

\gregorioscore{resources/gabc/ProperTime/ScholasticaMattins.gabc}

℣. Behold, thou art fair, my love.  

℟. Behold, thou art fair; thou hast dove's eyes.

\antiphon{Ben.}{O how illustrious {\dag} are the merits of blessed Scholastica! O how great the power of her tears! through which the renowned virgin, out of sunny clearness, drew down from the air a mighty flood of rain.}

\subbysub{II Evensong}

\begin{rubric}
	The Office Hymn \& Versicle are of I Evensong, with the following Antiphon.
\end{rubric}

\antiphon{Mag.}{To-day {\dag} the holy virgin Scholastica, in the likeness of a dove, went forth with all gladness to the heavenly places: to-day she was found worthy to enjoy for ever the bliss of celestial life beside her brother.}


\subby{Pope St. Gregory II}
\feastday{{Pope St. Gregory II}}
\fancyhead[RE,LO]{11 February}
\begin{inhead}
    {Memorial\\
11 February}
\end{inhead}

\begin{rubric}
	The propers are from the First Common of a Confessor Bishop (p. \pageref{CommonConfessorBishopI}).
\end{rubric}


\subby{St. Valentine}
\feastday{{St. Valentine}}
\fancyhead[RE,LO]{14 February}
\begin{inhead}
    {Memorial\\
14 February}
\end{inhead}

\begin{rubric}
	The propers are from the First Common of a Martyr not a Bishop (p. \pageref{CommonMartyrNotBishopI}), except for that which followeth.
\end{rubric}

\collect
\lett{G}{rant,} we beseech thee, almighty God: that we who observe the birthday of blessed Valentine thy Martyr may by his intercession be delivered from all evils that beset us. Through.

\secret
\lett{R}{eceive,} O Lord, we beseech thee, the gifts which we duly offer: and by the pleading of the merits of blessed Valentine thy Martyr, grant; that they may avail to set forward our salvation. Through.

\postcommunion
\lett{M}{ay} this heavenly mystery, O Lord, renew us in soul and body: that as we offer unto thee our worship, so by the intercession of blessed Valentine thy Martyr, we may perceive the benefit of the same. Through.


\subby{Sts. Faustinus and Jovita}
\feastday{{Sts. Faustinus \& Jovita}}
\fancyhead[RE,LO]{15 February}
\begin{inhead}
    {Memorial\\
15 February}
\end{inhead}

\begin{rubric}
	The propers are from the Third Common of Many Martyrs (p. \pageref{CommonMartyrsIII}), except for that which followeth.
\end{rubric}

\collect
\lett{O}{God,} who makest us glad with the yearly solemnity of thy holy Martyrs Faustinus and Jovita: mercifully grant; that as we rejoice in their merits, so we may be enkindled by their example. Through.

\secret
\lett{A}{ssist} us mercifully, O Lord, in these our supplications which we make before thee in remembrance of thy Saints: that we who trust not in our own rightneousness may be succoured by the merits of them that have found favour in thy sight. Through.

\postcommunion
\lett{O}{Lord,} who hast fulfilled us with saving mysteries, we beseech thee: that we may be aided by the prayers of those whose festival we celebrate. Through.


\subby{St. Simeon of Jerusalem}
\feastday{{St. Simeon}}
\fancyhead[RE,LO]{18 February}
\begin{inhead}
    {Memorial\\
18 February}
\end{inhead}

\begin{rubric}
	The propers are from the First Common of a Martyr Bishop (p. \pageref{CommonMartyrBishopI}).
\end{rubric}

\bcpfeast{22 February. Chair of St. Peter at Antioch}{Chair St. Peter Antioch}{22 February}\label{CathedraAntioch}
%\supplement{22 February}{Chair}{of St. Peter at Antioch}

\subbysub{I Evensong}

\gregorioscore{resources/gabc/ProperTime/PeterAntiochEvensong.gabc}\label{PeterAntiochEvensong}
    
    ℣. Thou art Peter.

	℟. And upon this rock I will build my Church.
	
\properantiphon{Mag.}{Thou art the shepherd of the sheep, {\dag} Prince of the Apostles, unto thee were given the keys of the kingdom of heaven.}

\subbysub{Mattins}

\begin{rubric}
	The Invitatory Hymn is as in I Evensong.
\end{rubric}

\officehymn\label{PeterAntiochMattins}

\gregorioscore{resources/gabc/ProperTime/PeterAntiochMattins.gabc}

    ℣. Thou art Peter.

	℟. And upon this rock I will build my Church.
	
	\properantiphon{Ben.}{Whatsoever {\dag} thou shalt bind on earth shall be bound in heaven: and whatsoever thou shalt loose on earth shall be loosed in heaven, saith the Lord unto Simon Peter.}

\subbysub{II Evensong}

\begin{rubric}
	II Evensong as in I Evensong, except for the Antiphon, as followeth.
\end{rubric}

\properantiphon{Mag.}{Whilst he was chief Bishop, he feared nothing on earth, but ascended gloriously to the heavenly kingdoms.}


\subby{Vigil of St. Matthias}
\feastday{{St. Matthias Vigil}}
\fancyhead[RE,LO]{23 February}
\begin{inhead}
    {Vigil\\
23 February}
\end{inhead}

\begin{rubric}
	The propers are from the Common of Vigils of the Apostles (p. \pageref{CommonVigilApostles}).
\end{rubric}


\bcpfeast{24 February. St. Matthias}{St. Matthias}{24 February}
%\supplement{24 February}{St. Matthias}{}

\begin{secrubric}
	The Daily Office propers are of the Common of Apostles (p. \pageref{CommonApostles}).
\end{secrubric}


\subby{St. Walburga of Heidenheim}
\feastday{{St. Walburga}}
\fancyhead[RE,LO]{25 February}
\begin{inhead}
    {Memorial\\
25 February (26 February in a Leap Year)}
\end{inhead}

\begin{rubric}
	The propers are from the Common of a Virgin (p. \pageref{CommonVirginOnlyI}), except for that which followeth.
\end{rubric}

%From Latin Diurnale Monasticum \& Missale Romanum (https://www.google.com/books/edition/Missale_Romanum/p2HPgCPUBPsC?hl=en&gbpv=1&dq=%22Walb%C3%BArgae%22&pg=RA3-PA126&printsec=frontcover):
\collect
\lett{O}{God,} who among the manifold gifts of thy grace dost also work great wonders even in the weaker sex: mercifully grant that we may feel the help of the intercession of blessed Walburga, thy Virgin, whose example of chastity doth enlighten us, and whose glory in miracles doth gladden us. Through.
%\lett{D}{eus,} qui inter innúmera gráti{\ae} tu{\ae} dona, étiam in sexu frágili tua operáris magnália: {\dag} concéde propítius; ut beát{\ae} Walbúrg{\ae}, Vírginis tu{\ae}, apud misericórdiam tuam patrocínia sentiámus, * cujus non solum castitátis illustrámur exémplis, verum étiam miraculórum glória jucundámur. Per Dóminum.

\begin{rubric}
	In Lent, Commemoration of the Feria.
\end{rubric}

\secret
\lett{W}{e} beseech thee, O Lord, that the prayer of blessed Walburga thy Virgin may render this sacrifice, which we devoutly offer unto thee in her venerable commemoration, acceptable unto thy Majesty. Through.
%\lett{S}{acrifícium} oblátum, quǽsumus Dómine, beát{\ae} Walbúrg{\ae} Vírginis tu{\ae} orátio Majestáti tu{\ae} reddat accéptum: pro cujus veneránda commemoratióne tibi devóto offértur obséquio. Per Dóminum.

\begin{rubric}
	In Lent, Commemoration of the Feria.
\end{rubric}

\postcommunion
\lett{G}{rant,} O Lord, that by the intercession of blessed Walburga, thy Virgin, we may obtain the grace of thy blessing: that as we proclaim her venerable glory, she may aid us in all our necessities. Through.
%\lett{B}{enedictiónis,} Dómine, grátiam, intercedénte beáta Walbúrga Vírgine tua, cónsequi mereámur: ut, cujus venerándam glóriam pr{\ae}dicámus; ejus in ómnibus nostris necessitátibus auxílium sentiámus. Per Dóminum.

\begin{rubric}
	In Lent, Commemoration \& Last Gospel of the Feria.
\end{rubric}


\subby{Pope St. Alexander of Alexandria}
\feastday{{Pope St. Alexander Alexandria}}
\fancyhead[RE,LO]{26 February}
\begin{inhead}
    {Memorial\\
26 February (27 February in a Leap Year)}
\end{inhead}

\begin{rubric}
	The propers are from the First Common of a Confessor Bishop (p. \pageref{CommonConfessorBishopI}).
\end{rubric}
\begin{rubric}
	In Lent, Commemoration \& Last Gospel of the Feria.
\end{rubric}


\subby{St. Raphael of Brooklyn}
\feastday{{St. Raphael Brooklyn}}
\fancyhead[RE,LO]{27 February}
\begin{inhead}
    {Greater Double\\
27 February (28 February in a Leap Year)}
\end{inhead}

\begin{rubric}
	In Lent, Commemoration only.
\end{rubric}

\begin{rubric}
	The propers are from the First Common of a Confessor Bishop (p. \pageref{CommonConfessorBishopI}).
\end{rubric}
\begin{rubric}
	In Lent, Commemoration \& Last Gospel of the Feria.
\end{rubric}


\subby{St. David of Wales}
\feastday{{St. David Wales}}
\fancyhead[RE,LO]{1 March}
\begin{inhead}
    {Memorial\\
1 March}
\end{inhead}

\begin{rubric}
	The propers are from the First Common of a Confessor Bishop (p. \pageref{CommonConfessorBishopI}), except for that which followeth.
\end{rubric}

\vspace{-1ex}

\collect
\lett{G}{rant} to us, almighty God: that the loving intercession of blessed David, thy Confessor and Bishop, may protect us; that while we celebrate his festival we may imitate his steadfastness in the defence of the Catholic faith. Through.

\begin{rubric}
	In Lent, Commemoration of the Feria.
\end{rubric}

%MANUAL ADJUSTMENT:
\subsubsec{Secret}
%\secret
\lett{W}{e} beseech thee, O Lord: that we, remembering with gladness the merits of thy Saints, may in all places feel the succour of their intercession. Through.

\begin{rubric}
	In Lent, Commemoration of the Feria.
\end{rubric}

\postcommunion
\lett{G}{rant,} we beseech thee, almighty God: that we, shewing forth our thankfulness for the gifts which we have received, may, at the intercession of blessed David, thy Confessor and Bishop, obtain yet more abundant mercies. Through.

\begin{rubric}
	In Lent, Commemoration \& Last Gospel of the Feria.
\end{rubric}


\subby{St. Chad}
\feastday{{St. Chad}}
\fancyhead[RE,LO]{2 March}
\begin{inhead}
    {Memorial\\
2 March}
\end{inhead}

\begin{rubric}
	The propers are from the Second Common of a Confessor Bishop (p. \pageref{CommonConfessorBishopII}), except for that which followeth.
\end{rubric}

\collect
\lett{A}{lmighty} and everlasting God, who on this day dost gladden us by the festival of blessed Chad, thy Confessor and Bishop: we humbly beseech thy mercy; that we, who devoutly observe and venerate his festival, may by his loving advocacy obtain the reward of everlasting life. Through.

\begin{rubric}
	In Lent, Commemoration of the Feria.
\end{rubric}

\secret
\lett{W}{e} beseech thee, O Lord, mercifully to have respect unto our supplications: and at the intercession of blessed Chad thy Confessor and Bishop on our behalf, grant that we who minister thy heavenly sacraments may be free from all sin; that through thy purifying grace we may be cleansed by those same mysteries which we serve. Through.

\begin{rubric}
	In Lent, Commemoration of the Feria.
\end{rubric}

\postcommunion
\lett{G}{rant,} we beseech thee, O Lord our God: that, at the intercession of blessed Chad, thy Confessor and Bishop, we who have tasted of holy things, being cleansed by divine mysteries, may attain to the fulness of this heavenly sacrament. Through.

\begin{rubric}
	In Lent, Commemoration \& Last Gospel of the Feria.
\end{rubric}


\subby{Pope St. Lucius I}
\feastday{{Pope St. Lucius I}}
\fancyhead[RE,LO]{4 March}
\begin{inhead}
    {Memorial\\
4 March}
\end{inhead}

\begin{rubric}
	The propers are from the Second Common of a Martyr Bishop (p. \pageref{CommonMartyrBishopII}), except for that which followeth.
\end{rubric}

\collect
\lett{O}{God,} who makest us glad with the yearly solemnity of blessed Lucius thy Martyr and Bishop: mercifully grant; that, as we now celebrate his birthday, so we may likewise rejoice in his protection. Through.

\begin{rubric}
	In Lent, Commemoration of the Feria.
\end{rubric}

\secret
\lett{W}{e} beseech thee, O Lord, mercifully to accept this our sacrifice, which we offer unto thee, pleading the merits of blessed Lucius, thy Martyr and Bishop: that the same may avail for our perpetual succour. Through.

\begin{rubric}
	In Lent, Commemoration of the Feria.
\end{rubric}

\postcommunion
\lett{W}{e} beseech thee, O Lord our God, that like as we whom thou hast refreshed by the partaking of thy sacred gift do offer unto thee our worship: so, by the intercession of blessed Lucius thy Martyr and Bishop, we may perceive the benefit of the same. Through.

\begin{rubric}
	In Lent, Commemoration \& Last Gospel of the Feria.
\end{rubric}


\subby{Sts. Perpetua and Felicitas}
\feastday{{Sts. Perpetua \& Felicitas}}
\fancyhead[RE,LO]{6 March}
\begin{inhead}
    {Double\\
6 March}
\end{inhead}

\begin{rubric}
	In Lent, Commemoration only.
\end{rubric}

\begin{rubric}
	The propers are from the Common of a Martyr not a Virgin (p. \pageref{CommonMartyrNotVirgin}), except for that which followeth.
\end{rubric}

%Felicity changed to Felicitas
\collect
\lett{G}{rant,} we beseech thee, O Lord our God, that we may at all times so devoutly honour the triumphs of thy holy Martyrs Perpetua and Felicitas: that, although we cannot worthily shew forth their praises, yet we may continually honour them with lowly service. Through.

\begin{rubric}
	In Lent, Commemoration of the Feria.
\end{rubric}

\secret
\lett{O}{Lord,} we beseech thee, look down upon these gifts, which we offer on thine altars on this festival of thy holy Martyrs Perpetua and Felicitas: that as by these blessed mysteries thou hast bestowed glory upon them, so likewise of thy bounty thou wouldest vouchsafe to us thy pardon. Through.

\begin{rubric}
	In Lent, Commemoration of the Feria.
\end{rubric}

\postcommunion
\lett{O}{Lord,} who hast fulfilled us with mystic gifts and joys: grant, we beseech thee; that by the intercession of thy holy Martyrs, Perpetua and Felicitas, we may spiritually attain to those things which we temporally perform. Through.

\begin{rubric}
	In Lent, Commemoration \& Last Gospel of the Feria.
\end{rubric}


\subby{St. Gregory of Nyssa}
\feastday{{St. Gregory Nyssa}}
\fancyhead[RE,LO]{9 March}
\begin{inhead}
    {Memorial\\
9 March}
\end{inhead}

\begin{rubric}
	The propers are from the Common of Doctors (p. \pageref{CommonDoctors}).
\end{rubric}
\begin{rubric}
	In Lent, Commemoration \& Last Gospel of the Feria.
\end{rubric}


\subby{The Forty Holy Martyrs}
\feastday{{40 Holy Martyrs}}
\fancyhead[RE,LO]{10 March}
\begin{inhead}
    {Memorial\\
10 March}
\end{inhead}

\introit
\lett{T}{he} just cry, and the Lord heareth them: and delivereth them out of all their troubles. \textit{Ps.} I will alway give thanks unto the Lord; his praise shall ever be in my mouth.

\collect
\lett{G}{rant} we beseech thee, almighty God: that, like as we have known thy glorious Martyrs to be constant in their confession, so we may perceive their loving intercession for us with thee. Through.

\begin{rubric}
	In Lent, Commemoration of the Feria.
\end{rubric}

\begin{rubric}
	The Epistle is the fifth additional Epistle of the Third Common of Many Martyrs (p. \pageref{Hebrews1133}).
\end{rubric}

\gradtr{Behold, how good and joyful a thing it is, brethren, to dwell together in unity! ℣. It is like the precious ointment upon the head, that ran down unto the beard, even unto Aaron's beard.}{They that sow in tears, shall reap in joy. ℣. They that now go on their way weeping, and bear forth good seed. ℣. Shall doubtless come again with joy, and bring their sheaves with them.}

\begin{rubric}
	The Gospel is from the Second Common of Many Martyrs (p. \pageref{CommonMartyrsII}).
\end{rubric}

\offertory{Be glad, O ye righteous, and rejoice in the Lord: and be joyful, all ye that are true of heart.}

\secret
\lett{R}{egard,} O Lord, the prayers and oblations of thy faithful people: that they may be acceptable unto thee for the festival of thy Saints, and bestow on us the succour of thy mercy. Through.

\begin{rubric}
	In Lent, Commemoration of the Feria.
\end{rubric}

\communion{Whosoever shall do the will of my Father which is in heaven: the same is my brother, and sister, and mother, saith the Lord.}

\postcommunion
\lett{G}{rant,} O Lord, we beseech thee, that the intercession of thy Saints may make us acceptable unto thee: that those things which we perform in this temporal celebration we may receive unto eternal salvation. Through.

\begin{rubric}
	In Lent, Commemoration \& Last Gospel of the Feria.
\end{rubric}


\bcpfeast{12 March. Pope St. Gregory the Great}{Pope St. Gregory}{12 March}
%\supplement{12 March}{St. Gregory}{Pope, Confessor, \& Martyr}

\subbysub{I Evensong}\label{GregoryEvensong}

\gregorioscore{resources/gabc/ProperTime/GregoryEvensong.gabc}

    ℣. The Lord loved him and adorned him.

	℟. He clothed him with a robe of glory.

\properantiphon{Mag.}{O Teacher right excellent, {\dag} O light of Holy Church, O blessed Gregory, lover of the divine law: intercede for us unto the Son of God.}

\subbysub{Mattins}\label{GregoryMattins}

\begin{rubric}
	The Invitatory Hymn is as in I Evensong.
\end{rubric}

\gregorioscore{resources/gabc/ProperTime/GregoryMattins.gabc}

    ℣. The Lord guided the righteous in right paths.

	℟. And shewed him the kingdom of God.

\properantiphon{Ben.}{Gregory, {\dag} when he looked upon the youthful Angles, said: They have the countenance of Angels, and such as these should be of the fellowship of the Angels in heaven.}

\subbysub{II Evensong}

\begin{rubric}
	The Office Hymn \& Antiphon are of I Evensong, with the Versicle from Mattins.
\end{rubric}


\subby{St. Patrick}
\feastday{{St. Patrick}}
\fancyhead[RE,LO]{17 March}
\begin{inhead}
    {Double\\
17 March}
\end{inhead}

\begin{rubric}
	In Lent, Commemoration only.
\end{rubric}

\begin{rubric}
	The propers are from the First Common of a Confessor Bishop (p. \pageref{CommonConfessorBishopI}), except for that which followeth.
\end{rubric}

\collect
\lett{O}{God,} who for the preaching of thy glory unto the Gentiles wast pleased to send forth blessed Patrick, thy Confessor and Bishop: grant by his merits and intercession; that we may through thy mercy be enabled to accomplish those things which thou commandest us to do. Through.

\begin{rubric}
    Commemoration is made of St. Joseph of Arimathea (p. \pageref{ArimatheanCollect}).
\end{rubric}

\begin{rubric}
	In Lent, Commemoration \& Last Gospel of the Feria.
\end{rubric}

\secret
\lett{W}{e} beseech thee, O Lord, that we remembering with gladness the merits of thy Saints, may in all places feel the succour of their intercession. Through.

\begin{rubric}
    Commemoration is made of St. Joseph of Arimathea, from the Second Common of a Confessor not a Bishop (p. \pageref{CommonConfessorNotBishopII})
\end{rubric}

\begin{rubric}
	In Lent, Commemoration \& Last Gospel of the Feria.
\end{rubric}

\postcommunion
\lett{G}{rant,} we beseech thee, almighty God: that we, shewing forth our thankfulness for the gifts which we have received, may at the intercession of blessed Patrick, thy Confessor and Bishop, obtain yet more abundant mercies. Through.

\begin{rubric}
    Commemoration is made of St. Joseph of Arimathea, from the Second Common of a Confessor not a Bishop (p. \pageref{CommonConfessorNotBishopII})
\end{rubric}

\begin{rubric}
	In Lent, Commemoration \& Last Gospel of the Feria.
\end{rubric}


\subby{St. Joseph of Arimathea}
\feastday{{St. Joseph Arimathea}}
\fancyhead[RE,LO]{17 March}
\begin{inhead}
    {Memorial\\
17 March}
\end{inhead}

\begin{rubric}
	The propers are from the Second Common of a Confessor not a Bishop (p. \pageref{CommonConfessorNotBishopII}), except for that which followeth.
\end{rubric}
\begin{rubric}
	In Lent, Commemoration \& Last Gospel of the Feria.
\end{rubric}

    ℣. Then took they the body of Jesus.

	℟. And wound it in linen clothes with the spices.

\antiphon{Mag. \& Ben.}{Joseph of Arimathea, {\dag} being a disciple of Jesus, but secretly for fear of the Jews, besought Pilate that he might take away the body of Jesus: and Pilate gave him leave.}
 
\collect\label{ArimatheanCollect}
\lett{O}{God,} who didst give such grace unto thy servant Joseph that he boldly craved the body of Jesus, and with great reverence laid him in the rock-hewn sepulchre: grant, we beseech thee, that we likewise may be so emboldened for thee, as to do works meet for thy Kingdom. Through the same.


\subby{St. Cyril of Jerusalem}
\feastday{{St. Cyril Jerusalem}}
\fancyhead[RE,LO]{18 March}
\begin{inhead}
    {Double\\
18 March}
\end{inhead}

\begin{rubric}
	In Lent, Commemoration only.
\end{rubric}

\begin{rubric}
	The Daily propers are from the Common of Doctors (p. \pageref{CommonDoctors}), except for that which followeth.
\end{rubric}

\introit
\lett{I}{n} the midst of the Church he opened his mouth: and the Lord filled him with the spirit of wisdom and of understanding: he clothed him with a robe of glory. \textit{Ps.} It is a good thing to give thanks unto the Lord: and to sing praises unto thy name, O most Highest.

\collect\label{CyrilCollect}
\lett{G}{rant} to us, we beseech thee, almighty God, at the intercession of the blessed Bishop Cyril: so to know thee, the only true God, and Jesus Christ whom thou hast sent; that we may be found worthy to be numbered for evermore among the sheep who hear his voice. Through the same.

\begin{rubric}
	Commemoration is made of St. Edward (p. \pageref{EdwardCollect}).
\end{rubric}
\begin{rubric}
	In Lent, Commemoration of the Feria.
\end{rubric}

\begin{rubric}
	The Epistle is from the additional Epistle of the Common of Doctors (p. \pageref{CommonDoctors}).
\end{rubric}

\gradtr{The mouth of the righteous is exercised in wisdom, and his tongue will be talking of judgement. ℣. The law of his God is in his heart: and his goings shall not slide.}{Blessed is the man that feareth the Lord: he hath great delight in his commandments. ℣. His seed shall be mighty upon earth: the generation of the faithful shall be blessed. ℣. Riches and plenteousness shall be in his house: and his righteousness endureth for ever.}

\readingcitation{Gospel}{Matthew 10:23}
\lett{A}{t that time:} Jesus said unto his disciples: When they persecute you in this city, flee ye into another: for verily I say unto you, Ye shall not have gone over the cities of Israel, till the Son of man be come. The disciple is not above his master, nor the servant above his lord. It is enough for the disciple that he be as his master, and the servant as his lord. If they have called the master of the house Beelzebub, how much more shall they call them of his household? Fear them not therefore: for there is nothing covered, that shall not be revealed; and hid, that shall not be known. What I tell you in darkness, that speak ye in light: and what ye hear in the ear, that preach ye upon the housetops. And fear not them which kill the body, but are not able to kill the soul: but rather fear him which is able to destroy both soul and body in hell.

\offertory{The righteous shall flourish like a palm-tree: and shall spread abroad like a cedar in Libanus.}

\secret\label{CyrilSecret}
\lett{L}{ook} down, O Lord, upon the spotless victim which we offer unto thee: and grant; that by the merits of thy blessed Bishop and Confessor Cyril we may endeavour ourselves to receive it with clean hearts. Through.

\begin{rubric}
	Commemoration is made of St. Edward (p. \pageref{EdwardSecret}).
\end{rubric}
\begin{rubric}
	In Lent, Commemoration of the Feria.
\end{rubric}

\communion{A faithful and wise servant, whom the Lord hath made ruler over his household: to give them their portion of meat in due season.}

\postcommunion\label{CyrilPostcommunion}
\lett{O}{Lord} Jesu Christ, may the sacraments of thy Body and Blood, which we have received: sanctify our minds and hearts through the prayers of the blessed Bishop Cyril; that we may be worthy to be made partakers of the divine nature. Who livest.

\begin{rubric}
	Commemoration is made of St. Edward (p. \pageref{EdwardPostcommunion}).
\end{rubric}
\begin{rubric}
	In Lent, Commemoration \& Last Gospel of the Feria.
\end{rubric}


\subby{St. Edward}
\feastday{{St. Edward}}
\fancyhead[RE,LO]{18 March}
\begin{inhead}
    {Memorial\\
18 March}
\end{inhead}

\begin{rubric}
	The propers are from the First Common of a Martyr not a Bishop (p. \pageref{CommonMartyrNotBishopI}), except for the following.
\end{rubric}

\collect\label{EdwardCollect}
\lett{O}{God,} the triumphant ruler of an everlasting kingdom, mercifully behold this thy family who celebrate the memory of blessed Edward, thy King and Martyr; and, by his merits and intercession, vouchsafe; that they, who glory in his triumph, may also attain unto his rewards. Through.

\begin{rubric}
	Commemoration is made of St. Cyril (p. \pageref{CyrilCollect}).
\end{rubric}
\begin{rubric}
	In Lent, Commemoration of the Feria.
\end{rubric}

\secret\label{EdwardSecret}
\lett{G}{rant,} O Lord, that this our bounden service may be acceptable in thy sight: that these our oblations may, by the prayers of him on whose solemnity they are offered, be made profitable unto our salvation. Through.
\begin{rubric}
	Commemoration is made of St. Cyril (p. \pageref{CyrilSecret}).
\end{rubric}
\begin{rubric}
	In Lent, Commemoration of the Feria.
\end{rubric}

\postcommunion\label{EdwardPostcommunion}
\lett{W}{e} beseech thee, O Lord our God, that like as we, whom thou hast refreshed by the partaking of thy sacred gift, do offer unto thee our worship: so by the intercession of blessed Edward thy Martyr, we may perceive the benefit of the same. Through.

\begin{rubric}
	Commemoration is made of St. Cyril (p. \pageref{CyrilPostcommunion}).
\end{rubric}
\begin{rubric}
	In Lent, Commemoration \& Last Gospel of the Feria.
\end{rubric}


\bcpfeast{19 March. St. Joseph, Spouse of the Blessed Virgin Mary}{St. Joseph Spouse}{19 March}
%\supplement{19 March}{St. Joseph}{Spouse of the B.V.M.}

\subbysub{I Evensong}\label{JosephEvensong}

\gregorioscore{resources/gabc/ProperTime/JosephEvensong.gabc}

    ℣. He made him lord of his house.

	℟. And ruler of all his substance.
	
	\properantiphon{Mag.}{Then Joseph, {\dag} being raised from sleep, did as the Angel of the Lord had bidden him, and took unto him his wife.}
	
\subbysub{Mattins}
	
	\invitatoryhymn\label{JosephInvitatory}
	
\gregorioscore{resources/gabc/ProperTime/JosephInvitatory.gabc}
	
	\officehymn\label{JosephMattins}
	
\gregorioscore{resources/gabc/ProperTime/JosephMattins.gabc}

    ℣. The mouth of the righteous is exercised in wisdom.

	℟. And his tongue will be talking of judgement.

	\properantiphon{Ben.}{Jesus himself {\dag} began to be about thirty years of age, being, as was supposed, the son of Joseph.}

\subbysub{II Evensong}

\begin{rubric}
	The Office Hymn is of I Evensong, with the following Versicle \& Antiphon.
\end{rubric}

    ℣. Riches and plenteousness shall be in his house.

	℟. And his righteousness endureth for ever.
	
	\properantiphon{Mag.}{Behold a faithful {\dag} and wise servant, whom his Lord hath made ruler over his household.}
	

\subby{St. Cuthbert}
\feastday{{St. Cuthbert}}
\fancyhead[RE,LO]{20 March}
\begin{inhead}
    {Double\\
20 March}
\end{inhead}

\begin{rubric}
	In Lent, Commemoration only.
\end{rubric}

\begin{rubric}
	The propers are from the Second Common of a Bishop Confessor (p. \pageref{CommonConfessorBishopII}), except for that which followeth.
\end{rubric}

\collect
\lett{O}{God,} who dost make thy Saints glorious by the inestimable gift of thy grace: grant, we beseech thee; that at the intercession of blessed Cuthbert, thy Confessor and Bishop, we may be found worthy to attain to the perfection of all virtue. Through.

\begin{rubric}
	In Lent, Commemoration of the Feria.
\end{rubric}

\secret
\lett{A}{ccept,} we beseech thee, O Lord, the sacrifice of man's redemption: and at the intercession of blessed Cuthbert, thy Confessor and Bishop, mercifully grant us health of mind and of body. Through.

\begin{rubric}
	In Lent, Commemoration of the Feria.
\end{rubric}

\postcommunion
\lett{W}{e} beseech thee, O Lord, that thy holy things which we have received may protect us by their power: and at the intercession of blessed Cuthbert, thy Confessor and Bishop, whose life shone forth in glory, guard us in peace and holiness. Through.

\begin{rubric}
	In Lent, Commemoration \& Last Gospel of the Feria.
\end{rubric}

\bcpfeast{21 March. St. Benedict}{St. Benedict}{21 March}
%\supplement{21 March}{St. Benedict}{}

\subbysub{I Evensong}

	\gregorioscore{resources/gabc/ProperTime/BenedictEvensong.gabc}\label{BenedictEvensong}

    ℣. The Lord loved him and adorned him.

	℟. He clothed him with a robe of glory.
	
	\properantiphon{Mag.}{Let all the multitude {\dag} of the faithful exult in the glory of the gracious Father Benedict: and chiefly let the company of monks be joyful, celebrating his festival on earth, in whose goodly fellowship the Saints rejoice in heaven.}

\subbysub{Mattins}
	
	\invitatoryhymn
	
	\gregorioscore{resources/gabc/ProperTime/BenedictMattins.gabc}\label{BenedictInvitatory}
	
	\officehymn

	\gregorioscore{resources/gabc/ProperTime/BenedictMattins.gabc}\label{BenedictMattins}

    ℣. The Lord guided the righteous in right paths.

	℟. And shewed him the kingdom of God.
	
	\properantiphon{Ben.}{Benedict, {\dag} thou father and guide of monks, thou most holy Confessor of the Lord, intercede for us all and for our salvation.}

\subbysub{II Evensong}

\begin{rubric}
	The Office Hymn is of I Evensong, with the Versicle from Mattins and the following Antiphon.
\end{rubric}

\properantiphon{Mag.}{To-day {\dag} holy Benedict, while his disciples beheld it, ascending by way of the East on a straight pathway to heaven: to-day with hands uplifted, he died between the words of his supplication: to-day he was received into glory by the Angels.}


\subby{St. Gabriel}
\feastday{St. Gabriel Archangel}
\fancyhead[RE,LO]{24 March}
\begin{inhead}
    {Greater Double\\
24 March}
\end{inhead}
\par\noindent

\begin{rubric}
	Commemoration only is made, unless a Votive Mass of St. Gabriel is said outside of Lent or as a patronal Feast Day.
\end{rubric}

\begin{rubric}
	I Evensong: Psalms 9 \& 11,15. Daniel 8:15-19. Revelation 8:1-5.\par
	Mattins: Psalms 30, 34, \& 77. Daniel 9:20-26. Luke 1:5-17\par
	II Evensong: Psalms 97, 99, \& 103. Isaiah 7:10-14. Luke 1:26-38.
\end{rubric}
\par\noindent
%AOB:
\textit{Opening Sentence.} I saw another angel come down from heaven, having great power; and the earth was lightened with his glory.

\begin{paracol}{2}[]
\sloppy
\begin{inhead}
	I Evensong
\end{inhead}
\begin{hangparas}{1.25em}{1}

Christ, the fair glory of the holy Angels,

Thou who hast made us, thou who o'er us rulest,

Grant of thy mercy unto us thy servants

Steps up to heaven.\\

Send thy Archangel, Gabriel, the mighty,

Herald of heaven; may he from us mortals

Spurn the old serpent, watching o'er the temples

Where thou art worshipp'd.\\

May the blest Mother of our God and Saviour,

May the assembly of the Saints in glory,

May the celestial companies of Angels

Ever assist us.\\

This he vouchsafe us, God for ever blessed

Father eternal, Son, and Holy Spirit,

Whose is the glory which through all creation.

Ever resoundeth. Amen.\\
\end{hangparas}

    ℣. An Angel stood at the altar of the temple.

	℟. Having in his hand a golden censer.
	
	\properantiphon{Mag.}{The Angel Gabriel {\dag} came in unto Mary and said, Hail, thou that art full of grace, the Lord is with thee; blessed art thou among women.}
	
	\switchcolumn
	
\begin{inhead}
	Mattins
\end{inhead}
\begin{hangparas}{1.25em}{1}
O Christ, Redeemer of us all,

Protect thy servants when they call,

And hear with reconciling care

The blessed Virgin's holy prayer.\\

Be ever present, Angel high

Whose name `God's might' doth signify:

To all the weak new strength impart,

And solace to the sad of heart.\\

And ye, O ever blissful throng

Of heav'nly Spirits, guardians strong,

Our past and present ills dispel,

From future peril shield us well.\\

From lands wherein thy faithful dwell

Drive far away the infidel;

So we to Christ due hymns of praise

Henceforth with eager hearts may raise.\\

To thee, O Father, born of none,

And thee, O sole-begotten Son,

One with the Holy Paraclete,

Be glory ever, as is meet. Amen.\\
\end{hangparas}

    ℣. An Angel stood at the altar of the temple.

	℟. Having in his hand a golden censer.
	
	\properantiphon{Ben.}{The Angel Gabriel {\dag} descended to Zacharias, and said unto him: Thy wife shall bear thee a son, and thou shalt call his name John; and many shall rejoice at his birth; for he shall go before the face of the Lord, to prepare his ways.}

\fussy
\end{paracol}

\begin{rubric}
	In II Evensong, the Office Hymn is of I Evensong, with the following Versicle \& Antiphon.
\end{rubric}

    ℣. In the presence of the Angels I will sing praise unto thee, O my God.

	℟. I will worship toward thy holy temple, and praise thy Name.
	
	\properantiphon{Mag.}{The Archangel Gabriel {\dag} said unto Mary: With God nothing shall be impossible. And Mary said, Behold the handmaid of the Lord: be it unto me according to thy word. And the Angel departed from her.}

\introit
\lett{O}{praise} the Lord, ye Angels of his, ye that excel in strength: ye that fulfil his commandment, and hearken unto the voice of his words. \textit{Ps.} Praise the Lord, O my soul: and all that is within me praise his holy name.

\collect
\lett{O}{God,} who from the company of Angels didst choose the Archangel Gabriel to proclaim the mystery of thine Incarnation: mercifully grant; that we who celebrate his festival (commemoration) on earth, may perceive his advocacy in heaven. Who livest.

\begin{rubric}
	Commemoration is made of the Feria.
\end{rubric}

\readingcitation{Epistle}{Daniel 9:21}
\lett{I}{n those days:} Behold the man Gabriel, whom I had seen in the vision at the beginning, being caused to fly swiftly, touched me about the time of the evening oblation. And he informed me, and talked with me, and said, O Daniel, I am now come forth to give thee skill and understanding. At the beginning of thy supplications the commandment came forth, and I am come to shew thee; for thou art greatly beloved: therefore understand the matter, and consider the vision. Seventy weeks are determined upon thy people and upon thy holy city, to finish the transgression, and to make an end of sins, and to make reconciliation for iniquity, and to bring in everlasting righteousness, and to seal up the vision and prophecy, and to anoint the most Holy. Know therefore and understand, that from the going forth of the commandment to restore and to build Jerusalem unto the Messiah the Prince shall be seven weeks, and threescore and two weeks: the street shall be built again, and the wall, even in troublous times. And after threescore and two weeks shall Messiah be cut off, but not for himself: and the people of the prince that shall come shall destroy the city and the sanctuary; and the end thereof shall be with a flood, and unto the end of the war desolations are determined.

\gradtr{O praise the Lord, ye Angels of his, ye that excel in strength, ye that fulfil his commandment. ℣. Praise the Lord, O my soul, and all that is within me praise his holy name.}{Hail, Mary, full of grace; the Lord is with thee. ℣. Blessed art thou among women: and blessed is the fruit of thy womb. ℣. Behold, thou shalt conceive, and bring forth a Son, and shalt call his name Emmanuel. ℣. The Holy Ghost shall come upon thee, and the power of the Highest shall overshadow thee. ℣. Therefore also that Holy Thing which shall be born of thee, shall be called the Son of God.}

\begin{rubric}
	In Votive Masses before Septuagesima or after Pentecost, the Gradual as above, but in place of the Tract is said:
\end{rubric}

\alleluia{Alleluia, alleluia. ℣. O praise the Lord, all ye his hosts: ye servants of his that do his pleasure. Alleluia.}

\begin{rubric}
	In Votive Masses in Eastertide, replacing the Gradual \& Tract:
\end{rubric}

\alleluia{Alleluia, alleluia. ℣. Who maketh his Angels spirits: and his ministers a flaming fire. Alleluia. ℣. Hail, Mary, full of grace; the Lord is with thee: blessed art thou among women. Alleluia.}

\readingcitation{Gospel}{Luke 1:26}
\lett{A}{t that time:} The Angel Gabriel was sent from God unto a city of Galilee named Nazareth, to a virgin espoused to a man whose name was Joseph, of the house of David; and the virgin's name was Mary. And the angel came in unto her, and said, Hail, thou that art highly favoured, the Lord is with thee: blessed art thou among women. And when she saw him, she was troubled at his saying, and cast in her mind what manner of salutation this should be. And the angel said unto her, Fear not, Mary: for thou hast found favour with God. And, behold, thou shalt conceive in thy womb, and bring forth a son, and shalt call his name \divineName{Jesus}. He shall be great, and shall be called the Son of the Highest: and the Lord God shall give unto him the throne of his father David: and he shall reign over the house of Jacob for ever; and of his kingdom there shall be no end. Then said Mary unto the angel, How shall this be, seeing I know not a man? And the angel answered and said unto her, The Holy Ghost shall come upon thee, and the power of the Highest shall overshadow thee: therefore also that holy thing which shall be born of thee shall be called the Son of God. And, behold, thy cousin Elisabeth, she hath also conceived a son in her old age: and this is the sixth month with her, who was called barren. For with God nothing shall be impossible. And Mary said, Behold the handmaid of the Lord; be it unto me according to thy word.

\offertory{An Angel stood at the altar of the temple, having a golden censer in his hand; and there was given unto him much incense: and the smoke of the incense ascended up before God.}

\secret
\lett{M}{ay} the offering of our service, and the prayer of blessed Gabriel the Archangel, be accepted in thy sight, O Lord: that he, who is venerated by us on earth, may be an advocate for us with thee in heaven. Through.

\begin{rubric}
	Commemoration of the Feria.
\end{rubric}

\communion{O all ye Angels of the Lord, bless ye the Lord: sing ye praises, and magnify him above all for ever.}

\postcommunion
\lett{H}{aving} received the mysteries of thy Body and Blood, we entreat thy loving kindness, O Lord our God: that, as we have known thine Incarnation by the message of Gabriel, so we may through his help obtain the benefits of that Incarnation. Who livest.

\begin{rubric}
	Commemoration \& Last Gospel of the Feria.
\end{rubric}


\bcpfeast{25 March. Annunciation of the Blessed Virgin Mary}{Annunciation}{25 March}
%\supplement{25 March}{Annunciation}{of the Blessed Virgin Mary}

\begin{secrubric}
	The Hymns are from the Common of the Blessed Virgin Mary (p. \pageref{CommonBVM}), with the following Versicles \& Antiphons.
\end{secrubric}

    ℣. Hail Mary, full of grace.

	℟. The Lord is with thee.
	
	\properantiphon{Mag.}{The Holy Ghost {\dag} shall come upon thee, Mary: and the power of the Highest shall overshadow thee.}\\
	
	℣. Hail Mary, full of grace.

	℟. The Lord is with thee.

	\properantiphon{Ben.}{How shall this be, {\dag} O Angel of God, seeing I know not a man? Hearken, O Virgin Mary: The Holy Ghost shall come upon thee, and the power of the Highest shall overshadow thee.}\\
	
	℣. Hail Mary, full of grace.

	℟. The Lord is with thee.

	\properantiphon{Mag.}{The Angel Gabriel {\dag} spake unto Mary, saying: Hail, thou that art full of grace, the Lord is with thee; blessed art thou among women.}

%MANUAL ADJUSTMENT:
\clearpage
\subby{St. John Damascene}
\feastday{{St. John Damascene}}
\fancyhead[RE,LO]{27 March}
\begin{inhead}
    {Double\\
27 March}
\end{inhead}

\begin{rubric}
	Commemoration only.
\end{rubric}

\begin{rubric}
	The Daily Office propers are from the Common of Doctors (p. \pageref{CommonDoctors}), except for that which followeth.
\end{rubric}

\introit
\lett{T}{hou} hast holden me by my right hand: thou shalt guide me with thy counsel, and after that receive me with glory. \textit{Ps.} Truly God is loving unto Israel, even unto such as are of a clean heart!

\collect
\lett{A}{lmighty} and everlasting God, who, for the defence of the veneration of sacred images, didst endue blessed John with heavenly doctrine and wondrous strength of spirit: grant unto us by his intercession and example; that we may imitate the virtues and perceive the advocacy of those whose images we venerate. Through.

\begin{rubric}
	Commemoration of the Feria.
\end{rubric}

\readingcitation{Epistle}{Wisdom 10:10}
%RV (unedited):
\lett{W}{isdom} guided him in straight paths; She shewed him God’s kingdom, and gave him knowledge of holy things; She prospered him in his toils, and multiplied the fruits of his labour; When in their covetousness men dealt hardly with him, She stood by him and made him rich; She guarded him from enemies, And from those that lay in wait she kept him safe, And over his sore conflict she watched as judge, That he might know that godliness is more powerful than all. When a righteous man was sold, wisdom forsook him not, But from sin she delivered him; She went down with him into a dungeon, And in bonds she left him not, Till she brought him the sceptre of a kingdom, And authority over those that dealt tyrannously with him; She shewed them also to be false that had mockingly accused him, And gave him eternal glory. Wisdom delivered a holy people and a blameless seed from a nation of oppressors. She entered into the soul of a servant of the Lord, And withstood terrible kings in wonders and signs. She rendered unto holy men a reward of their toils.

\gradtr{It is God that girdeth me with strength of war: and maketh my way perfect, ℣. He teacheth mine hands to fight: and mine arms shall break even a bow of steel.}{I will follow upon mine enemies, and overtake them. ℣. I will smite them that they shall not be able to stand: but fall under my feet. ℣. For this cause will I give thanks unto thee, O Lord, among the Gentiles, and sing praises unto thy name.}

\readingcitation{Gospel}{Luke 6:6}
\lett{A}{t that time:} It came to pass on another sabbath, that Jesus entered into the synagogue and taught. And there was a man whose right hand was withered. And the scribes and Pharisees watched him, whether he would heal on the sabbath day; that they might find an accusation against him. But he knew their thoughts, and said to the man which had the withered hand, Rise up, and stand forth in the midst. And he arose and stood forth. Then said Jesus unto them, I will ask you one thing; Is it lawful on the sabbath days to do good, or to do evil? to save life, or to destroy it? And looking round about upon them all, he said unto the man, Stretch forth thy hand. And he did so: and his hand was restored whole as the other. And they were filled with madness; and communed one with another what they might do to Jesus.

\offertory{There is hope of a tree, if it be cut down, that it will sprout again, and that the tender branch thereof will not cease.}

\secret
\lett{O}{Lord,} let the devout intercession of blessed John and of the Saints, who through his labours are set forth in the temples for our veneration, avail to render the gifts which we offer acceptable in thy sight. Through.

\begin{rubric}
	Commemoration of the Feria.
\end{rubric}

\communion{The arms of the ungodly shall be broken, and the Lord upholdeth the righteous.}

\postcommunion
\lett{W}{e} beseech thee, O Lord, that the gifts which we have received may shield us with heavenly armour: and that the advocacy of blessed John, together with the united intercession of the Saints, the veneration of whose images in the Church be victoriously upheld, be our defence. Through.

\begin{rubric}
	Commemoration \& Last Gospel of the Feria.
\end{rubric}


\subby{St. John Climacus}
\feastday{{St. John Climacus}}
\fancyhead[RE,LO]{30 March}
\begin{inhead}
    {Memorial\\
30 March}
\end{inhead}

\begin{rubric}
	The propers are from the Common of Abbots (p. \pageref{CommonAbbots}).
\end{rubric}
\begin{rubric}
	Commemoration \& Last Gospel of the Feria.
\end{rubric}


\subby{St. Innocent of Alaska}
\feastday{{St. Innocent Alaska}}
\fancyhead[RE,LO]{31 March}
\begin{inhead}
    {Memorial\\
31 March}
\end{inhead}

\begin{rubric}
	The propers are from the Second Common of a Confessor Bishop (p. \pageref{CommonConfessorBishopII}).
\end{rubric}
\begin{rubric}
	Commemoration \& Last Gospel of the Feria.
\end{rubric}


\subby{St. Isidore of Seville}
\feastday{{St. Isidore Seville}}
\fancyhead[RE,LO]{4 April}
\begin{inhead}
    {Double\\
4 April}
\end{inhead}

\begin{rubric}
	In Lent or Easter Week, Commemoration only.
\end{rubric}

\begin{rubric}
	The propers are from the Common of Doctors (p. \pageref{CommonDoctors}).
\end{rubric}
\begin{rubric}
	In Lent, Commemoration \& Last Gospel of the Feria.
\end{rubric}


\subby{St. Tikhon of Moscow}
\feastday{{St. Tikhon Moscow}}
\fancyhead[RE,LO]{7 April}
\begin{inhead}
    {\nth{1} Double, Common Octave\\
7 April}
\end{inhead}

\begin{rubric}
	The propers are from the Second Common of a Bishop Confessor (p. \pageref{CommonConfessorBishopII}).
\end{rubric}
\begin{rubric}
	In Lent, Commemoration \& Last Gospel of the Feria.
\end{rubric}


\subby{St. Leo the Great}
\feastday{{Pope St. Leo the Great}}
\fancyhead[RE,LO]{11 April}
\begin{inhead}
    {Double\\
11 April}
\end{inhead}

\begin{rubric}
	In Lent or Easter Week, Commemoration only.
\end{rubric}

\begin{rubric}
	The Daily Office propers are from the Common of Doctors (p. \pageref{CommonDoctors}).
\end{rubric}

\introit
\lett{I}{n} the midst of the Church he opened his mouth: and the Lord filled him with the spirit of wisdom and of understanding: he clothed him with a robe of glory. (Alleluia, alleluia.) \textit{Ps.} It is  a good thing to give thanks unto the Lord: and to sing praises unto thy name, O Most Highest.

\collect
\lett{W}{e} beseech thee, O Lord, graciously to hear the prayers which we offer unto thee on the solemnity of blessed Leo, thy Confessor and Bishop: that, like as he was found worthy to do thee faithful service, so by his merits and intercession we may be absolved from all our sins. Through.

\begin{rubric}
	In Lent, Commemoration of the Feria.
\end{rubric}

\begin{rubric}
	The Epistle is the additional Epistle of the Common of Doctors (p. \pageref{Ecclesiasticus395}).
\end{rubric}

\gradtr{The mouth of the righteous is exercised in wisdom, and his tongue will be talking of judgement. ℣. The law of his God is in his heart: and his goings shall not slide.}{Blessed is the man that feareth the Lord: he hath great delight in his commandments. ℣. His seed shall be mighty upon earth: the generation of the faithful shall be blessed. ℣. Riches and plenteousness shall be in his house: and his righteousness endureth for ever.}

\begin{rubric}
	In Eastertide, the following Alleluia Verse replaces the Gradual \& Tract.
\end{rubric}

\alleluia{Alleluia, alleluia. ℣. The Lord loved him, and adorned him: he clothed him with a robe of glory. Alleluia. ℣. The righteous shall grow as the lily: and flourish for ever before the Lord. Alleluia.}

\readingcitation{Gospel}{Matthew 16:13}
\lett{A}{t that time:} When Jesus came into the coasts of Caesarea Philippi, he asked his disciples, saying, Whom do men say that I the Son of man am? And they said, Some say that thou art John the Baptist: some, Elias; and others, Jeremias, or one of the prophets. He saith unto them, But whom say ye that I am? And Simon Peter answered and said, Thou art the Christ, the Son of the living God. And Jesus answered and said unto him, Blessed art thou, Simon Barjona: for flesh and blood hath not revealed it unto thee, but my Father which is in heaven. And I say also unto thee, That thou art Peter, and upon this rock I will build my church; and the gates of hell shall not prevail against it. And I will give unto thee the keys of the kingdom of heaven: and whatsoever thou shalt bind on earth shall be bound in heaven: and whatsoever thou shalt loose on earth shall be loosed in heaven.

\offertory{I have found David my servant, with my holy oil have I anointed him: my hand shall hold him fast, and my arm shall strengthen him. (Alleluia.)}


\secret
\lett{W}{e} beseech thee, O Lord, that our devout observance of the yearly solemnity of Saint Leo, thy Confessor and Bishop, may render us acceptable unto thy loving kindness: that this service of propitiation, which we duly offer, may be profitable unto him for the reward of blessed-ness, and obtain for us the gifts of thy grace. Through.

\begin{rubric}
	In Lent, Commemoration of the Feria.
\end{rubric}

\communion{Blessed is the servant, whom the lord when he cometh shall find watching: verily I say unto you, that he shall make him ruler over all his goods. (Alleluia.)}

\postcommunion
\lett{O}{God,} who rewardest the souls of them that put their trust in thee: vouchsafe; that we who keep the solemn festival of blessed Leo, thy Confessor and Bishop, may by his prayers obtain thy merciful pardon. Through.

\begin{rubric}
	In Lent, Commemoration \& Last Gospel of the Feria.
\end{rubric}


\subby{St. Hermengild}
\feastday{{St. Hermengild}}
\fancyhead[RE,LO]{13 April}
\begin{inhead}
    {Memorial\\
13 April}
\end{inhead}

\begin{rubric}
	In Lent, the propers from the First Common of a Martyr not a Bishop (p. \pageref{CommonMartyrNotBishopI}), with the following. And Commemoration \& Last Gospel of the Feria.
\end{rubric}

\begin{rubric}
	In Eastertide, the propers from the Common of Martyrs in Eastertide (p. \pageref{CommonMartyrsEaster}), with the following.
\end{rubric}

\collect
\lett{O}{God,} who didst teach blessed Hermengild, thy Martyr, to lay down an earthly for a heavenly kingdom: grant to us, we beseech thee; by his example to despise things temporal, and to seek after things eternal. Through.

\readingcitation{Gospel}{Luke 14:26}
\lett{A}{t that time:} Jesus said unto the multitudes: If any man come to me, and hate not his father, and mother, and wife, and children, and brethren, and sisters, yea, and his own life also, he cannot be my disciple. And whosoever doth not bear his cross, and come after me, cannot be my disciple. For which of you, intending to build a tower, sitteth not down first, and counteth the cost, whether he have sufficient to finish it? Lest haply, after he hath laid the foundation, and is not able to finish it, all that behold it begin to mock him, Saying, This man began to build, and was not able to finish. Or what king, going to make war against another king, sitteth not down first, and consulteth whether he be able with ten thousand to meet him that cometh against him with twenty thousand? Or else, while the other is yet a great way off, he sendeth an ambassage, and desireth conditions of peace. So likewise, whosoever he be of you that forsaketh not all that he hath, he cannot be my disciple.

\secret
\lett{W}{e} beseech thee, O Lord, to accept our prayers and oblations: and graciously hearken unto us, whom thou dost cleanse by thy heavenly mysteries. Through.

\postcommunion
\lett{G}{rant,} we beseech thee, O Lord our God: that like as we in this life do gladly honour the memory of thy Saints; so we may rejoice to behold them for ever. Through.


\subby{St. Justin}
\feastday{{St. Justin}}
\fancyhead[RE,LO]{14 April}
\begin{inhead}
    {Memorial\\
14 April}
\end{inhead}

\introit
\lett{T}{he} proud have digged pits for me, which are not after thy law: I will speak of thy testimonies also, even before kings, and will not be ashamed. (Alleluia, alleluia.) \textit{Ps.} Blessed are those that are undefiled in the way: and walk in the law of the Lord.

\collect
\lett{O}{God,} who through the foolishness of the Cross didst wondrously teach blessed Justin Martyr the excellent knowledge of Jesus Christ: grant to us through his intercession; that we, driving away the errors that beset us, may attain unto steadfastness of faith. Through the same.

\begin{rubric}
	Commemoration is made of St. Tiburtius, Valerian, \& Maximus (p. \pageref{TiburtiusCollect}).
\end{rubric}
\begin{rubric}
	In Lent, Commemoration of the Feria.
\end{rubric}

\readingcitation{Epistle}{1 Corinthians 1:18}
\lett{B}{rethren:} The preaching of the cross is to them that perish foolishness; but unto us which are saved it is the power of God. For it is written, I will destroy the wisdom of the wise, and will bring to nothing the understanding of the prudent. Where is the wise? where is the scribe? where is the disputer of this world? hath not God made foolish the wisdom of this world? For after that in the wisdom of God the world by wisdom knew not God, it pleased God by the foolishness of preaching to save them that believe. For the Jews require a sign, and the Greeks seek after wisdom: But we preach Christ crucified, unto the Jews a stumblingblock, and unto the Greeks foolishness; But unto them which are called, both Jews and Greeks, Christ the power of God, and the wisdom of God. Because the foolishness of God is wiser than men; and the weakness of God is stronger than men. For ye see your calling, brethren, how that not many wise men after the flesh, not many mighty, not many noble, are called: But God hath chosen the foolish things of the world to confound the wise; and God hath chosen the weak things of the world to confound the things which are mighty; And base things of the world, and things which are despised, hath God chosen, yea, and things which are not, to bring to nought things that are: That no flesh should glory in his presence. But of him are ye in Christ Jesus, who of God is made unto us wisdom, and righteousness, and sanctification, and redemption.

\gradtr{The wisdom of this world is foolishness with God, for it is written: The Lord knoweth the thoughts of the wise, that they are vain. ℣. I will destroy the wisdom of the wise, and will bring to nothing the understanding of the prudent.}{I determined not to know any thing among you, save Jesus Christ, and him crucified. ℣. We speak the wisdom of God in a mystery, even the hidden wisdom, which God ordained before the world unto our glory. ℣. Which none of the princes of this world knew. For had they known it, they would not have crucified the Lord of glory.}

\alleluia{Alleluia, alleluia. ℣. The wisdom of this world is foolishness with God: for it is written: The Lord knoweth the thoughts of the wise, that they are vain. Alleluia. ℣. Yea doubtless, and I count all things but loss for the excellency of the knowledge of Christ Jesus my Lord. Alleluia.}

\begin{rubric}
	In Votive Masses before Septuagesima or after Pentecost, the Gradual is said, as above, but, the Tract being omitted, there is added
\end{rubric}

\alleluia{Alleluia, alleluia. ℣. Yea doubtless, and I count all things but loss for the excellency of the knowledge of Christ Jesus my Lord. Alleluia.}

\readingcitation{Gospel}{Luke 12:2}
\lett{A}{t that time:} Jesus said to his disciples: There is nothing covered, that shall not be revealed; neither hid, that shall not be known. Therefore whatsoever ye have spoken in darkness shall be heard in the light; and that which ye have spoken in the ear in closets shall be proclaimed upon the housetops. And I say unto you my friends, Be not afraid of them that kill the body, and after that have no more that they can do. But I will forewarn you whom ye shall fear: Fear him, which after he hath killed hath power to cast into hell; yea, I say unto you, Fear him. Are not five sparrows sold for two farthings, and not one of them is forgotten before God? But even the very hairs of your head are all numbered. Fear not therefore: ye are of more value than many sparrows. Also I say unto you, Whosoever shall confess me before men, him shall the Son of man also confess before the angels of God.

%MANUAL ADJUSTMENT:
\clearpage
\secret
\lett{O}{Lord} God, graciously receive our gifts: the wondrous mystery whereof the holy Martyr Justin did manfully defend against the slanders of the ungodly. Through.

\begin{rubric}
	Commemoration is made of St. Tiburtius, Valerian, \& Maximus (p. \pageref{TiburtiusSecret}).
\end{rubric}
\begin{rubric}
	In Lent, Commemoration of the Feria.
\end{rubric}

\communion{There is laid up for me a crown of righteousness, which the Lord, the righteous judge, shall give me at that day. (Alleluia.)}

\postcommunion
\lett{O}{Lord,} who hast refreshed us with heavenly food, we humbly beseech thee: that following the counsels of blessed Justin, thy Martyr; we may ever continue in thanksgiving for the gifts which we have received. Through.

\begin{rubric}
	Commemoration is made of St. Tiburtius, Valerian, \& Maximus (p. \pageref{TiburtiusPostcommunion}).
\end{rubric}
\begin{rubric}
	In Lent, Commemoration \& Last Gospel of the Feria.
\end{rubric}


\subby{St. Tiburtius, Valerian, and Maximus}
\feastday{{Sts. Tiburtius \&c.}}
\fancyhead[RE,LO]{14 April}
\begin{inhead}
    {Memorial\\
14 April}
\end{inhead}

\begin{rubric}
	Outside of Eastertide, the propers are from the Second Common of Many Martyrs (p. \pageref{CommonMartyrsII}), with the Prayers following.
\end{rubric}

\begin{rubric}
	In Eastertide, the propers are from the Common of Many Martyrs in Eastertide (p. \pageref{CommonMartyrsEaster}), with the Epistle and Gospel from the Common of a Martyr in Eastertide (p. \pageref{CommonMartyrEaster}), with the Prayers following.
\end{rubric}

\collect\label{TiburtiusCollect}
\lett{G}{rant,} we beseech thee, almighty God: that we, almighty God: that we, who celebrate the festival of thy holy Martyrs, Tiburtius, Valerian, and Maximus; may also imitate their virtues. Through.

\secret\label{TiburtiusSecret}
\lett{W}{e} beseech thee, O Lord, that this oblation, which we offer unto thee in commemoration of the birthday of thy holy Martyrs: may both loose the bonds of our iniquity, and obtain for us the gifts of thy mercy. Through.

\postcommunion\label{TiburtiusPostcommunion}
\lett{O}{Lord,} who hast satisfied us with this sacred gift, we humbly beseech thee: that the mysteries, which we celebrate in this service of our bounden duty, we may know to be an increase of thy salvation. Though.


\subby{St. Anicetus}
\feastday{{St. Anicetus}}
\fancyhead[RE,LO]{17 April}
\begin{inhead}
    {Memorial\\
17 April}
\end{inhead}


\begin{rubric}
	Outside of Eastertide, the propers are from the Second Common of a Martyr Bishop outside of Eastertide (p. \pageref{CommonMartyrBishopII}).
\end{rubric}

\begin{rubric}
	In Eastertide, the propers are from the Common of a Martyr in Eastertide (p. \pageref{CommonMartyrEaster}), using the Second Prayers for a Martyr Bishop, with the additional Gospel from the Common of Many Martyrs in Eastertide (p. \pageref{John1620}).
\end{rubric}


\subby{Sts. Soter and Caius}
\feastday{{Sts. Soter \& Caius}}
\fancyhead[RE,LO]{22 April}
\begin{inhead}
    {Memorial\\
22 April}
\end{inhead}

\begin{rubric}
	Outside of Eastertide, the propers are from the First Common of Many Martyrs outside of Eastertide (p. \pageref{CommonMartyrsI}).
\end{rubric}

\begin{rubric}
	In Eastertide, the propers are from the Common of Many Martyrs in Eastertide (p. \pageref{CommonMartyrsEaster}), with the additional Epistle (p. \pageref{Revelation191}).
\end{rubric}

\bcpfeast{23 April. St. George}{St. George}{23 April}
%\supplement{23 April}{St. George}{}

\begin{secrubric}
	Outside of Eastertide, the Daily Office propers are from the First Common of a Martyr not a Bishop out of Eastertide (p. \pageref{CommonMartyrNotBishopI}).
\end{secrubric}

\begin{rubric}
	In Eastertide, the Daily Office propers are from the Common of a Martyr in Eastertide (p. \pageref{CommonMartyrEaster}).
\end{rubric}


\bcpfeast{25 April. St. Mark}{St. Mark}{25 April}
%\supplement{25 April}{St. Mark}{}

\begin{secrubric}
	Outside of Eastertide, the Daily Office propers are from the Common of Apostles (p. \pageref{CommonApostles}).
\end{secrubric}

\begin{rubric}
	In Eastertide, the Daily Office propers are from the Common of Apostles in Eastertide (p. \pageref{CommonApostlesEaster}).
\end{rubric}


%MANUAL ADJUSTMENT:
\clearpage
\subby{Sts. Cletus and Marcellinus}
\feastday{{Sts. Cletus \& Marcellinus}}
\fancyhead[RE,LO]{26 April}
\begin{inhead}
    {Memorial\\
26 April}
\end{inhead}

\begin{rubric}
	Outside of Eastertide, propers are from the First Common of Many Martyrs out of Eastertide (p. \pageref{CommonMartyrsI}), with the following Prayers.
\end{rubric}

\begin{rubric}
	In Eastertide, the propers are from the Common of Many Martyrs in Eastertide (p. \pageref{CommonMartyrsEaster}), with the following Prayers.
\end{rubric}

\collect
\lett{O}{Lord,} let the precious confession of thy blessed Martyrs and Bishops Cletus and Marcellinus be our defence: and let their loving intercession be a continual defence. Through.

\secret
\lett{A}{ssist} us mercifully, O Lord, in these our supplications, which we make before thee in remembrance of thy Saints: that we who trust not in our own righteousness may be succoured by the merits of them that have found favour in thy sight. Through.

\postcommunion
\lett{O}{Lord,} who hast fulfilled us with saving mysteries: we beseech thee that we may be aided by the prayers of those whose festival we celebrate. Through.


\subby{St. Vitalis}
\feastday{{St. Vitalis}}
\fancyhead[RE,LO]{28 April}
\begin{inhead}
    {Memorial\\
28 April}
\end{inhead}

\begin{rubric}
	Outside of Eastertide, propers are from the First Common of a Martyr not a Bishop out of Eastertide (p. \pageref{CommonMartyrNotBishopI}).
\end{rubric}

\begin{rubric}
	In Eastertide, the propers are from the Common of a Martyr in Eastertide (p. \pageref{CommonMartyrEaster}), using the first Prayers for a Martyr not a Bishop
\end{rubric}

\bcpfeast{1 May. Sts. Philip \& James}{Sts. Philip \& James}{1 May}
%\supplement{1 May}{Sts. Philip \& James}{}

\begin{secrubric}
	The Daily Office propers are from the Common of Apostles in Eastertide (p. \pageref{CommonApostlesEaster}), except for the following.
\end{secrubric}

\properantiphon{Mag.}{Let not your heart {\dag} be troubled, neither let it be afraid; ye believe in God, believe also in me: in my Father's house are many mansions, alleluia, alleluia.}\\

\properantiphon{Ben.}{I am the way, {\dag} the truth, and the life: no man cometh unto the Father, but by me, alleluia.}\\

	℣. Right dear in the sight of the Lord, alleluia.

	℟. Is the death of his Saints, alleluia.

\properantiphon{Mag.}{If ye abide in me, {\dag} and my words abide in you, ye shall ask what ye will, and it shall be done unto you, alleluia, alleluia, alleluia.}


\subby{St. Athanasius}
\feastday{{St. Athanasius}}
\fancyhead[RE,LO]{2 May}
\begin{inhead}
    {Double\\
2 May}
\end{inhead}

\begin{rubric}
	The Daily Office propers are from the Common of Doctors (p. \pageref{CommonDoctors}).
\end{rubric}

\introit
\lett{I}{n} the midst of the Church he opened his mouth: and the Lord filled him with the spirit of wisdom and of understanding: he clothed him with a robe of glory, alleluia, alleluia. \textit{Ps.} It is a good thing to give thanks unto the Lord: and to sing praise unto thy name, O Most Highest.

\collect
\lett{W}{e} beseech thee, O Lord, graciously to hear the prayers which we offer unto thee on the solemnity of blessed Athanasius, thy Confessor and Bishop: that, like as he was found worthy to do thee faithful service, so by his merits and intercession we may be absolved from all our sins. Through.

\readingcitation{Epistle}{2 Corinthians 4:5}
\lett{B}{rethren:} We preach not ourselves, but Christ Jesus the Lord; and ourselves your servants for Jesus' sake. For God, who commanded the light to shine out of darkness, hath shined in our hearts, to give the light of the knowledge of the glory of God in the face of Jesus Christ. But we have this treasure in earthen vessels, that the excellency of the power may be of God, and not of us. We are troubled on every side, yet not distressed; we are perplexed, but not in despair; Persecuted, but not forsaken; cast down, but not destroyed; Always bearing about in the body the dying of the Lord Jesus, that the life also of Jesus might be made manifest in our body. For we which live are alway delivered unto death for Jesus' sake, that the life also of Jesus might be made manifest in our mortal flesh. So then death worketh in us, but life in you. We having the same spirit of faith, according as it is written, I believed, and therefore have I spoken; we also believe, and therefore speak; Knowing that he which raised up the Lord Jesus shall raise up us also by Jesus, and shall present us with you.

\alleluia{Alleluia, alleluia. ℣. Thou art a priest for ever, after the order of Melchisedech. Alleluia. ℣. Blessed is the man that endureth temptation: for when he is tried, he shall receive the crown of life. Alleluia.}

\begin{rubric}
	Outside of Eastertide, the following Gradual \& Lesser Alleluia replaces the Greater Alleluia.
\end{rubric}

\gradall{Behold a great priest who in his days pleased God. ℣. There was none found like unto him who kept the law of the Most High.}{Alleluia, alleluia. ℣. Blessed is the man that endureth temptation: for when he is tried, he shall receive the crown of life. Alleluia.}

\readingcitation{Gospel}{Matthew 10:23}
\lett{A}{t that time:} Jesus said unto his disciples: When they persecute you in this city, flee ye into another: for verily I say unto you, Ye shall not have gone over the cities of Israel, till the Son of man be come. The disciple is not above his master, nor the servant above his lord. It is enough for the disciple that he be as his master, and the servant as his lord. If they have called the master of the house Beelzebub, how much more shall they call them of his household? Fear them not therefore: for there is nothing covered, that shall not be revealed; and hid, that shall not be known. What I tell you in darkness, that speak ye in light: and what ye hear in the ear, that preach ye upon the housetops. And fear not them which kill the body, but are not able to kill the soul: but rather fear him which is able to destroy both soul and body in hell.

\offertory{I have found David my servant, with my holy oil have I anointed him: my hand shall hold him fast, and my arm shall strengthen him, alleluia.}

\secret
\lett{W}{e} beseech thee, O Lord, that our devout observance of the yearly solemnity of Saint Athanasius, thy Confessor and Bishop, may render us acceptable unto thy loving kindness: that this service of propitiation, which we duly offer, may be profitable unto him for the reward of blessed-ness, and obtain for us the gifts of thy grace. Through.

\communion{What I tell you in darkness that speak ye in light, saith the Lord: and what ye hear in the ear, that preach ye upon the housetops, alleluia.}

\postcommunion
\lett{O}{God,} who rewardest the souls of them that put their trust in thee: vouchsafe; that we who keep the solemn festival of blessed Athanasius, thy Confessor and Bishop, may by his prayers obtain thy merciful pardon. Through.

\bcpfeast{3 May. Invention of the Holy Cross}{Invention of the Cross}{3 May}\label{InventionoftheCross}
%\supplement{3 May}{Invention}{of the Holy Cross}

\subbysub{I Evensong}\label{InventionEvensong}

\gregorioscore{resources/gabc/ProperTime/InventionEvensong1.gabc}

\begin{rubric}
	During the following stanza all kneel, and the last stanza is never changed.
\end{rubric}

\gregorioscore{resources/gabc/ProperTime/InventionEvensong2.gabc}

	℣. This sign of the Cross shall be in heaven, alleluia.
	
	℟. When the Lord shall come to judgement, alleluia.

\properantiphon{Mag.}{O Cross, {\dag} surpassing all the stars in splendour, world renowned, exceeding dear unto the hearts of men, holier than all things: thou only wert counted worthy to uphold the world's random. Sweet the wood, sweet the iron, bearing so sweet a burden: bring aid to this congregation, who are here assembled to celebrate thy praises, alleluia, alleluia.}

\subbysub{Mattins}

\invitatoryhymn\label{InventionInvitatory}

\gregorioscore{resources/gabc/ProperTime/InventionInvitatory.gabc}

\officehymn\label{InventionMattins}

\gregorioscore{resources/gabc/ProperTime/InventionMattins.gabc}

	℣. We adore thee, O Christ, and we bless thee, alleluia.
	
	℟. Because by thy Cross thou hast redeemed the world, alleluia.

\properantiphon{Ben.}{Thou alone {\dag} excellest in stature all the cedars of Lebanon: for on thee the Life of the world was hanged, on thee was Christ victorious, and death over death did for ever triumph, alleluia.}

\subbysub{II Evensong}

\begin{rubric}
	II Evensong as in I Evensong, with the following Versicle \& Antiphon.
\end{rubric}

	℣. This sign of the Cross shall be in heaven, alleluia.
	
	℟. When the Lord shall come to judgement, alleluia.

\properantiphon{Mag.}{He the holy Cross endured, {\dag} Burst the gates of hell in twain: Begirt with might and majesty, On Easter morn he rose again, alleluia.}


\subby{Sts. Alexander, Eventius, Theodulus, and Juvenalis}
\feastday{{Sts. Alexander \&c.}}
\fancyhead[RE,LO]{3 May}
\begin{inhead}
    {Memorial\\
3 May}
\end{inhead}

\begin{rubric}
	The propers are from the Common of Many Martyrs in Eastertide (p. \pageref{CommonMartyrsEaster}).
\end{rubric}


\subby{St. Monica}
\feastday{{St. Monica}}
\fancyhead[RE,LO]{4 May}
\begin{inhead}
    {Memorial\\
4 May}
\end{inhead}

\begin{rubric}
	The propers are from the Common of Neither Virgin nor Martyr (p. \pageref{CommonNeitherVirginMartyr}), except for the following.
\end{rubric}

\collect
\lett{O}{God,} the comforter of them that mourn, and the salvation of them that hope in thee, who didst mercifully receive the loving tears of blessed Monica for the conversion of Augustine her son: grant to us by the intercession of them both; that we may bewail our sins, and obtain the pardon of thy grace. Through.

\begin{rubric}
	The Epistle is the additional Epistle from the same Common (p. \pageref{1Timothy53}).
\end{rubric}

\readingcitation{Gospel}{Luke 7:11}
\lett{A}{t that time:} Jesus went into a city called Nain: and many of his disciples went with him, and much people. Now when he came nigh to the gate of the city, behold, there was a dead man carried out, the only son of his mother, and she was a widow: and much people of the city was with her. And when the Lord saw her, he had compassion on her, and said unto her, Weep not. And he came and touched the bier: and they that bare him stood still. And he said, Young man, I say unto thee, Arise. And he that was dead sat up, and began to speak. And he delivered him to his mother. And there came a fear on all: and they glorified God, saying, That a great prophet is risen up among us; and, That God hath visited his people.

\secret
\lett{G}{rant,} O Lord, that like as thy faithful people do acknowledge that in tribulation they have been succoured by the merits of thy Saints: so this oblation, which they here do offer unto thee in honour of the same, may be acceptable in thy sight. Through.

\postcommunion
\lett{O}{Lord,} who satisfied thy family with sacred gifts: we beseech thee that we may at all times be comforted by the intercession of her whose festival we celebrate. Through.


\bcpfeast{6 May. St. John before the Latin Gate}{St. John Latin Gate}{6 May}
%\supplement{6 May}{St. John}{before the Latin Gate}

\begin{secrubric}
	The Daily Office propers are from the Common of Apostles in Easter (p. \pageref{CommonApostlesEaster}), with the following Antiphon for both Evensongs.
\end{secrubric}

\properantiphon{Mag.}{John the Apostle, {\dag} being cast into  a caldron of boiling oil, by virtue of protecting grace came forth unharmed, alleluia.}


\subby{St. Alexis Toth}
\feastday{{St. Alexis}}
\fancyhead[RE,LO]{7 May}
\begin{inhead}
    {Memorial\\
7 May}
\end{inhead}

%CHECK:
\begin{rubric}
	The propers are from the First Common of a Confessor not a Bishop (p. \pageref{CommonConfessorNotBishopI}).
\end{rubric}


\subby{The Apparition of St. Michael the Archangel}
\feastday{{St. Michael}}
\fancyhead[RE,LO]{8 May}
\begin{inhead}
    {Greater Double\\
8 May}
\end{inhead}

\introit
\lett{O}{praise} the Lord, all ye Angels of his: ye that excel in strength ye that fulfil his commandment, and hearken unto the voice of his words. \textit{Ps.} Praise the Lord, O my soul; and all that is within me praise his holy name

\collect
\lett{O}{everlasting} God, who hast ordained and constituted the services of Angels and men in a wonderful order; Mercifully grant that, as thy holy Angels always do thee service in heaven, so, by thy appointment, they may succour and defend us on earth. Through.

\readingcitation{Epistle}{Revelation 1:1}
\lett{I}{n those days:} God shewed the things which must shortly come to pass, and he sent and. signified it by his Angel unto his servant John, Who bare record of the word of God, and of the testimony of Jesus Christ, and of all things that he saw. Blessed is he that readeth, and they that hear the words of this prophecy, and keep those things which are written therein: for the time is at hand. John to the seven churches which are in Asia: Grace be unto you, and peace, from him which is, and which was, and which is to come; and from the seven Spirits which are before his throne; And from Jesus Christ, who is the faithful witness, and the first begotten of the dead, and the prince of the kings of the earth. Unto him that loved us, and washed us from our sins in his own blood.

\alleluia{Alleluia, alleluia. ℣. Holy Archangel Michael, defend us in the battle; that we perish not in the dreadful judgement. Alleluia. ℣. The sea was shaken, and the earth trembled, when the Archangel Michael came down from heaven. Alleluia.}

\readingcitation{Gospel}{Matthew 18:1}
\lett{A}{t that time:} The disciples came unto Jesus, saying, Who is the greatest in the kingdom of heaven? And Jesus called a little child unto him, and set him in the midst of them, And said, Verily I say unto you, Except ye be converted, and become as little children, ye shall not enter into the kingdom of heaven. Whosoever therefore shall humble himself as this little child, the same is greatest in the kingdom of heaven. And whoso shall receive one such little child in my name receiveth me. But whoso shall offend one of these little ones which believe in me, it were better for him that a millstone were hanged about his neck, and that he were drowned in the depth of the sea. Woe unto the world because of offences! for it must needs be that offences come; but woe to that man by whom the offence cometh! Wherefore if thy hand or thy foot offend thee, cut them off, and cast them from thee: it is better for thee to enter into life halt or maimed, rather than having two hands or two feet to be cast into everlasting fire. And if thine eye offend thee, pluck it out, and cast it from thee: it is better for thee to enter into life with one eye, rather than having two eyes to be cast into hell fire. Take heed that ye despise not one of these little ones; for I say unto you, That in heaven their angels do always behold the face of my Father which is in heaven.

\secret
\lett{W}{e} offer thee, O Lord, sacrifices of praise, humbly beseeching thee: that by the prayers of the Angels interceding for us, thou wouldest both graciously accept the same, and grant that they may avail to our salvation. Through.

\communion{O all ye Angels of the Lord, bless ye the Lord: sing ye praises, and magnify him above all for ever.}

\postcommunion
\lett{O}{Lord,} forasmuch as we put our trust in the intercession of thy blessed Archangel Michael; we humbly beseech thee; that those things which we touch with our lips we may likewise receive into our hearts. Through.

\begin{rubric}
	If it be a Sunday, the Gospel of this Feast is read at the end.
\end{rubric}


\subby{St. Boniface IV}
\feastday{{St. Boniface IV}}
\fancyhead[RE,LO]{8 May}
\begin{inhead}
    {Memorial\\
8 May}
\end{inhead}

%CHECK:
\begin{rubric}
	The propers are from the First Common of a Confessor Bishop (p. \pageref{CommonConfessorBishopI}).
\end{rubric}


\subby{St. Gregory Nazianzen}
\feastday{{St. Gregory Nazianzen}}
\fancyhead[RE,LO]{9 May}
\begin{inhead}
    {Double\\
9 May}
\end{inhead}

\begin{rubric}
	The propers are from the Common of Doctors (p. \pageref{CommonDoctors}), using the additional Epistle.
\end{rubric}


\subby{Sts. Gordian \& Epimachus}
\feastday{{St. Gordian \& Epimachus}}
\fancyhead[RE,LO]{10 May}
\begin{inhead}
    {Memorial\\
10 May}
\end{inhead}

\begin{rubric}
	The propers are from the Common of Many Martyrs in Eastertide (p. \pageref{CommonMartyrsEaster}), with the additional Epistle (p. \pageref{Revelation191}) and the Prayers as followeth.
\end{rubric}

\collect
\lett{G}{rant,} we beseech thee, almighty God: that we, who celebrate the festival of thy blessed Martyrs Gordian and Epimachus, may be aided by their intercession with thee. Through.

\secret
\lett{W}{e} beseech thee, O Lord, mercifully to accept this our sacrifice, which we offer unto thee, pleading the merits of thy blessed Martyrs Gordian and Epimachus: that the same may effectually avail for our perpetual succour. Through.

\postcommunion
\lett{W}{e} beseech thee, almighty God: that we, who have received this heavenly food, may at the intercession of thy blessed Martyrs, Gordian and Epimachus, be thereby defended against all adversities. Through.


\subby{Sts. Nereus, Achilles, Domitilla, and Pancras}
\feastday{{Sts. Nereus \&c.}}
\fancyhead[RE,LO]{12 May}
\begin{inhead}
    {Memorial\\
12 May}
\end{inhead}

\introit
\lett{B}{ehold,} the eye of the Lord is upon them that fear him, and put their trust in his mercy, alleluia: to deliver their soul from death: for he is our help and our shield, alleluia, alleluia. \textit{Ps.} Rejoice in the Lord, O ye righteous: for it becometh well the just to be thankful.

\collect
\lett{W}{e} beseech thee, O Lord, that the blessed solemnity of thy Martyrs, Nereus, Achilles, Domitilla, and Pancras, may ever protect us: and render us worthy of thy service. Through.

\begin{rubric}
	The Epistle is from the Common of a Martyr in Eastertide (p. \pageref{CommonMartyrEaster}).
\end{rubric}

\alleluia{Alleluia, alleluia. ℣. This is the true brotherhood which overcame the wickedness of the world: which followed Christ, gaining heaven's glorious realms. Alleluia. ℣. The noble army of Martyrs praise thee, O Lord. Alleluia.}

\readingcitation{Gospel}{John 4:46}
\lett{A}{t that time:} There was a certain nobleman, whose son was sick at Capernaum. When he heard that Jesus was come out of Judaea into Galilee, he went unto him, and besought him that he would come down, and heal his son: for he was at the point of death. Then said Jesus unto him, Except ye see signs and wonders, ye will not believe. The nobleman saith unto him, Sir, come down ere my child die. Jesus saith unto him, Go thy way; thy son liveth. And the man believed the word that Jesus had spoken unto him, and he went his way. And as he was now going down, his servants met him, and told him, saying, Thy son liveth. Then enquired he of them the hour when he began to amend. And they said unto him, Yesterday at the seventh hour the fever left him. So the father knew that it was at the same hour, in the which Jesus said unto him, Thy son liveth: and himself believed, and his whole house.

\offertory{O Lord, the very heavens shall praise thy wondrous works: and thy truth in the congregation of the saints, alleluia, alleluia.}

\secret
\lett{W}{e} beseech thee, O Lord, that the confession of thy holy Martyrs Nereus, Achilles, Domitilla and Pancras may be pleasing unto thee: that it may both commend our gifts, and ever implore for us thy pardon. Through.

\communion{Rejoice in the Lord, O ye righteous, alleluia: for it becometh well the just to be thankful, alleluia.}

\postcommunion
\lett{W}{e} beseech thee, O Lord: that by the prayers of thy blessed Martyrs Nereus, Achilles, Domitilla and Pancras, the holy sacraments which we have received may profit us to the increase of thy merciful pardon. Though.


\subby{St. Boniface of Tarsus}
\feastday{{St. Boniface Tarsus}}
\fancyhead[RE,LO]{14 May}
\begin{inhead}
    {Memorial\\
14 May}
\end{inhead}

\begin{rubric}
	The propers are from the Common of a Martyr in Eastertide (p. \pageref{CommonMartyrEaster}), with the Prayers as followeth..
\end{rubric}

\collect
\lett{G}{rant,} we beseech thee, almighty God: that we, who celebrate the festival of blessed Boniface thy Martyr, may be aided by his intercession with thee. Through.

\secret
\lett{W}{e} beseech thee, O Lord, to accept our prayers and oblations: and graciously hearken unto us whom thou dost cleanse by thy heavenly mysteries. Through.

\postcommunion
\lett{W}{e} beseech thee, O Lord our God, that like as we, whom thou hast refreshed by the partaking of thy sacred gift, do offer unto thee our worship: so by the intercession of blessed Boniface, thy Martyr, we may perceive the benefits of the same. Through.


\subby{St. Venantius}
\feastday{{St. Venantius}}
\fancyhead[RE,LO]{18 May}
\begin{inhead}
    {Double\\
18 May}
\end{inhead}

\begin{rubric}
	The propers are from the Common of a Martyr in Eastertide (p. \pageref{CommonMartyrEaster}), with the Prayers as followeth.
\end{rubric}

\collect
\lett{O}{God,} who hast hallowed this day by the triumph of blessed Venantius, thy Martyr: graciously hear the prayers of thy people, and grant; that we who venerate his merits may imitate the constancy of his faith. Through.

\secret
\lett{A}{lmighty} God, let the merits of blessed Venantius render this oblation acceptable unto thee: that we, being succoured by his assistance, may be made partakers of his glory. Through.

\postcommunion
\lett{W}{e} have received, O Lord, the sacraments of everlasting life, humbly beseeching thee: that through the prayers of blessed Venantius thy Martyr on our behalf, they may obtain for us both pardon and grace. Through.


\subby{St. Pudentiana}
\feastday{{St. Pudentiana}}
\fancyhead[RE,LO]{19 May}
\begin{inhead}
    {Memorial\\
19 May}
\end{inhead}

\begin{rubric}
	The propers are from the First Common of a Virgin (p. \pageref{CommonVirginOnlyI}).
\end{rubric}


\subby{St. Romanus}
\feastday{{St. Romanus}}
\fancyhead[RE,LO]{22 May}
\begin{inhead}
    {Memorial\\
22 May}
\end{inhead}

\begin{rubric}
	The propers are from the Common of Abbots (p. \pageref{CommonAbbots}).
\end{rubric}


\subby{St. Vincent of Lerins}
\feastday{{St. Vincent Lerins}}
\fancyhead[RE,LO]{24 May}
\begin{inhead}
    {Memorial\\
24 May}
\end{inhead}

\begin{rubric}
	The propers are from the First Common of a Confessor not a Bishop (p. \pageref{CommonConfessorNotBishopI}).
\end{rubric}


\subby{Pope St. Urban I}
\feastday{{Pope St. Urban I}}
\fancyhead[RE,LO]{25 May}
\begin{inhead}
    {Memorial\\
25 May}
\end{inhead}

\begin{rubric}
	The propers are from the First Common of a Martyr Bishop (p. \pageref{CommonMartyrBishopI}).
\end{rubric}


%MANUAL ADJUSTMENT:
\clearpage
\subby{St. Augustine of Canterbury}
\feastday{{St. Augustine Canterbury}}
\fancyhead[RE,LO]{26 May}
\begin{inhead}
    {Double\\
26 May}
\end{inhead}

\begin{rubric}
	The Daily Office propers are from the First Common of a Confessor Bishop (p. \pageref{CommonConfessorBishopI}).
\end{rubric}

\introit
\lett{L}{et} thy priests, O Lord, be clothed with righteousness: and let thy saints sing with joyfulness: for thy servant David's sake, turn not away the presence of thine Anointed. Alleluia, alleluia. \textit{Ps.} Lord, remember David: and all his trouble.

\collect
\lett{O}{God,} who didst give the blessed Bishop Augustine to be the first Teacher of the English people: grant us, we beseech thee; that we who proclaim his merits on earth may perceive his intercession in heaven. Through.

\begin{rubric}
	Commemoration of St. Eleutherius (p. \pageref{EleutheriusCollect}).
\end{rubric}

\readingcitation{Epistle}{Hebrews 7:23}
\lett{B}{rethren:} They were many priests, because they were not suffered to continue by reason of death: But this man, because he continueth ever, hath an unchangeable priesthood. Wherefore he is able also to save them to the uttermost that come unto God by him, seeing he ever liveth to make intercession for them. For such an high priest became us, who is holy, harmless, undefiled, separate from sinners, and made higher than the heavens; Who needeth not daily, as those high priests, to offer up sacrifice, first for his own sins, and then for the people's: for this he did once, when he offered up himself, Jesus Christ our Lord.

\alleluia{Alleluia, alleluia. ℣. The Lord sware, and will not repent: Thou art a priest for ever after the order of Melchisedech. Alleluia. ℣. The Lord loved him, and adorned him: he clothed him with a robe of glory. Alleluia.}

\readingcitation{Gospel}{Luke 10:1}
\lett{A}{t that time:} The Lord appointed other seventy also, and sent them two and two before his face into every city and place, whither he himself would come. Therefore said he unto them, The harvest truly is great, but the labourers are few: pray ye therefore the Lord of the harvest, that he would send forth labourers into his harvest. Go your ways: behold, I send you forth as lambs among wolves. Carry neither purse, nor scrip, nor shoes: and salute no man by the way. And into whatsoever house ye enter, first say, Peace be to this house. And if the son of peace be there, your peace shall rest upon it: if not, it shall turn to you again. And in the same house remain, eating and drinking such things as they give: for the labourer is worthy of his hire. Go not from house to house. And into whatsoever city ye enter, and they receive you, eat such things as are set before you: And heal the sick that are therein, and say unto them, The kingdom of God is come nigh unto you.

\offertory{My truth and my mercy shall be with him: and in my name shall his horn be exalted. Alleluia.}

\secret
\lett{W}{e} beseech thee, O Lord, that the gifts which we offer may be acceptable unto thee: whereby we venerate the merits of blessed Augustine, thy Confessor and Bishop, and likewise call to remembrance the pledges of our life and freedom. Through.

\begin{rubric}
	Commemoration of St. Eleutherius (p. \pageref{EleutheriusSecret}).
\end{rubric}

\communion{Blessed is the servant, whom the lord when he cometh shall find watching: verily I say unto you, He shall make him ruler over all his goods. Alleluia.}

\postcommunion
\lett{W}{e} beseech thee, O Lord, that we may be nourished by thy holy things, which we have received for the solemnity of blessed Augustine, thy Confessor and Bishop: wherewith we may continually be satisfied, and evermore desire to be filled. Through.

\begin{rubric}
	Commemoration of St. Eleutherius (p. \pageref{EleutheriusPostcommunion}).
\end{rubric}


\subby{Pope St. Eleutherius}
\feastday{{Pope St. Eleutherius}}
\fancyhead[RE,LO]{26 May}
\begin{inhead}
    {Memorial\\
26 May}
\end{inhead}

\begin{rubric}
	The propers are from the Common of a Martyr in Eastertide (p. \pageref{CommonMartyrEaster}), with the following Prayers.
\end{rubric}

\collect\label{EleutheriusCollect}
\lett{A}{lmighty} God, mercifully look upon our infirmities: that, whereas we are oppressed by the burden of our sins, the glorious intercession of blessed Eleutherius, thy Martyr and Bishop, may be our succour and defence. Through.

\secret\label{EleutheriusSecret}
\lett{S}{anctify,} O Lord, the gifts which we dedicate to thee: that at the intercession of blessed Eleutherius, thy Martyr and Bishop, they may obtain for us thy gracious favour. Through.

\postcommunion\label{EleutheriusPostcommunion}
\lett{M}{ay} this communion, O Lord, cleanse us from guilt: and, at the intercession of blessed Eleutherius, thy Martyr and Bishop, make us partakers of thy heavenly healing. Through.


\subby{St. Bede the Venerable}
\feastday{{St. Bede the Venerable}}
\fancyhead[RE,LO]{27 May}
\begin{inhead}
    {Double\\
27 May}
\end{inhead}

\begin{rubric}
	The propers are from the Common of Doctors (p. \pageref{CommonDoctors}), with the following Prayers.
\end{rubric}

\collect
\lett{O}{God,} who dost illumine thy Church with the learning of blessed Bede, thy Confessor and Doctor: mercifully grant unto thy servants; that they may ever be enlightened by his wisdom, and succoured by his merits. Through.

\begin{rubric}
	Commemoration of Pope St. John I from the Second Collect of the Common of a Martyr in Eastertide (p. \pageref{CommonMartyrEaster}).
\end{rubric}

\secret
\lett{M}{ay} the devout prayers of thy Confessor and Doctor, Saint Bede, never fail to succour us, O Lord: that they may render our oblations acceptable in thy sight; and may ever obtain for us thy merciful pardon. Through.

\begin{rubric}
	Commemoration of Pope St. John I from the Second Secret of the Common of a Martyr in Eastertide (p. \pageref{CommonMartyrEaster}).
\end{rubric}

\postcommunion
\lett{W}{e} beseech thee, O Lord, that blessed Bede, thy Confessor and illustrious Doctor, may stand before thee as our advocate: that these thy sacrifices may avail for our salvation. Through.

\begin{rubric}
	Commemoration of Pope St. John I from the Second Postcommunion of the Common of a Martyr in Eastertide (p. \pageref{CommonMartyrEaster}).
\end{rubric}


\subby{Pope St. John I}
\feastday{{Pope St. John I}}
\fancyhead[RE,LO]{27 May}
\begin{inhead}
    {Memorial\\
27 May}
\end{inhead}

\begin{rubric}
	The propers are from the Common of a Martyr in Eastertide (p. \pageref{CommonMartyrEaster}), using the Second Prayers.
\end{rubric}


\subby{Pope St. Felix I}
\feastday{{Pope St. Felix I}}
\fancyhead[RE,LO]{30 May}
\begin{inhead}
    {Memorial\\
30 May}
\end{inhead}

\begin{rubric}
	In Eastertide, the propers are from Common of a Martyr in Eastertide (p. \pageref{CommonMartyrEaster}), with the First Collect and the Second Secret and Postcommunion.
\end{rubric}

\begin{rubric}
	Outside of Eastertide, the propers are from the First Common of a Martyr Bishop (p. \pageref{CommonMartyrBishopI}).
\end{rubric}


\subby{St. Petronilla}
\feastday{{St. Petronilla}}
\fancyhead[RE,LO]{31 May}
\begin{inhead}
    {Memorial\\
31 May}
\end{inhead}

\begin{rubric}
	The propers are from the Second Common of a Virgin (p. \pageref{CommonVirginOnlyII}).
\end{rubric}


\subby{Sts. Marcellinus, Peter, \& Erasmus}
\feastday{{Sts. Marcellinus \&c.}}
\fancyhead[RE,LO]{2 June}
\begin{inhead}
    {Memorial\\
2 June}
\end{inhead}

\begin{rubric}
	Outside of Eastertide, the propers are as followeth.
\end{rubric}

\begin{rubric}
	Within Eastertide, the propers are from the Common of Many Martyrs in Eastertide (p. \pageref{CommonMartyrsEaster}) with the Prayers and Epistle as followeth.
\end{rubric}

\introit
\lett{T}{he} righteous cry, and the Lord heareth them: and delivereth them out of all their troubles. \textit{Ps.} I will alway give thanks unto the Lord: his praise shall ever be in my mouth.

\collect
\lett{O}{God,} who makest us glad with the yearly solemnity of thy blessed Martyrs, Marcellinus, Peter, and Erasmus: grant, we beseech thee; that, as we rejoice in their merits, so we may be enkindled by their example. Through.

\begin{rubric}
	The Epistle is from the third additional Epistle of the Third Common of Many Martyrs (p. \pageref{Romans818}).
\end{rubric}

\gradall{The righteous cry, and the Lord heareth them: and delivereth them out of all their troubles. ℣. The Lord is nigh unto them that are of a contrite heart: and will save such as be of an humble spirit.}{Alleluia, alleluia. ℣. I have chosen you out of the world that ye should go and bring forth fruit; and that your fruit should remain. Alleluia.}

\begin{rubric}
	In Eastertide, the following Alleluia Verse replaces the Gradual \& Lesser Alleluia.
\end{rubric}

\alleluia{Alleluia, alleluia. ℣. I have chosen you out of the world, that ye should go and bring forth fruit; and that your fruit should remain. Alleluia. ℣. Right dear in the sight of the Lord is the death of his saints. Alleluia.}

\begin{rubric}
	The Gospel is from the First Common of Many Martyrs (p. \pageref{CommonMartyrsI}).
\end{rubric}

\offertory{Be glad, ye righteous, and rejoice in the Lord: and be joyful, all ye that are true of heart.}

\secret
\lett{W}{e} beseech thee, O Lord, that this sacrifice which we offer in remembrance of the birthday of thy holy Martyrs: may both loose the bonds of our iniquity, and obtain for us the gifts of thy mercy. Through.

\communion{The souls of the just are in the hand of God, and there shall no torment of malice touch them: in the sight of the unwise they seemed to die, but they are in peace.}

\postcommunion
\lett{O}{Lord,} who hast satisfied us with this sacred gift, we humbly beseech thee: that in the mysteries which we celebrate in this service of our bounden duty, we may perceive the increase of thy saving grace. Through.


\subby{St. Boniface}
\feastday{{St. Boniface}}
\fancyhead[RE,LO]{5 June}
\begin{inhead}
    {Double\\
5 June}
\end{inhead}

\begin{rubric}
	The Daily Office propers are from the First Common of a Martyr Bishop (p. \pageref{CommonMartyrBishopI}).
\end{rubric}

\introit
\lett{I}{will} rejoice in Jerusalem and joy in my people: and the voice of weeping shall be no more heard in her, nor the voice of crying. Mine elect shall not labour in vain, nor bring forth for trouble: for they are the seed of the blessed of the Lord, and their offspring with them. (Alleluia, alleluia.) \textit{Ps.} We have heard with our ears, O God: our fathers have told us, what thou hast done in their time of old.

\collect
\lett{O}{God,} who by the zeal of blessed Boniface, thy Martyr and Bishop, didst vouchsafe to call a multitude of peoples to the knowledge of thy name: mercifully grant; that we who celebrate his festival may likewise perceive his advocacy. Through.

\readingcitation{Epistle}{Ecclesiasticus 44:1}
\lett{L}{et} us now praise famous men, And our fathers that begat us. The Lord manifested in them great glory, Even his mighty power from the beginning. Such as did bear rule in their kingdoms, And were men renowned for their power, Giving counsel by their understanding, Such as have brought tidings in prophecies: Leaders of the people by their counsels, And by their understanding men of learning for the people; Wise were their words in their instruction: Such as sought out musical tunes, And set forth verses in writing: Rich men furnished with ability, Living peaceably in their habitations: All these were honoured in their generations, And were a glory in their days. There be of them, that have left a name behind them, To declare their praises. And some there be, which have no memorial; Who are perished as though they had not been, And are become as though they had not been born; And their children after them. But these were men of mercy, Whose righteous deeds have not been forgotten. With their seed shall remain continually a good inheritance; Their children are within the covenants. Their seed standeth fast, And their children for their sakes. Their seed shall remain for ever, And their glory shall not be blotted out. Their bodies were buried in peace, And their name liveth to all generations. Peoples will declare their wisdom, And the congregation telleth out their praise.

\gradall{Rejoice, inasmuch as ye are partakers of Christ's sufferings, that, when his glory shall be revealed, ye may be glad also with exceeding joy. ℣. If ye be reproached for the name of Christ, happy are ye: for the spirit of glory and of God resteth upon you.}{Alleluia, alleluia. ℣. I will extend peace to her like a river, and glory like a flowing stream. Alleluia.}

\begin{rubric}
	In Eastertide, the following Alleluia Verse replaces the Gradual \& Lesser Alleluia.
\end{rubric}

\alleluia{Alleluia, alleluia. ℣. Rejoice ye with Jerusalem, and be glad with her, all ye that love the Lord. Alleluia. ℣. When ye see this, your heart shall rejoice: and the hand of the Lord shall be known toward his servants. Alleluia.}

\readingcitation{Gospel}{Matthew 5:1}
\lett{A}{t that time:} Jesus, seeing the multitudes, went up into a mountain: and when he was set, his disciples came unto him: And he opened his mouth, and taught them, saying, Blessed are the poor in spirit: for theirs is the kingdom of heaven. Blessed are they that mourn: for they shall be comforted. Blessed are the meek: for they shall inherit the earth. Blessed are they which do hunger and thirst after righteousness: for they shall be filled. Blessed are the merciful: for they shall obtain mercy. Blessed are the pure in heart: for they shall see God. Blessed are the peacemakers: for they shall be called the children of God. Blessed are they which are persecuted for righteousness' sake: for theirs is the kingdom of heaven. Blessed are ye, when men shall revile you, and persecute you, and shall say all manner of evil against you falsely, for my sake. Rejoice, and be exceeding glad: for great is your reward in heaven.

\offertory{I will thank the Lord for giving me warning: I have set God always before me, for he is on my right hand, therefore I shall not fall. (Alleluia.)}

\secret
\lett{L}{et} thy plenteous benediction, we beseech thee, O Lord, come down upon these sacrifices: that it may mercifully work out our sanctification; and make us to rejoice in the solemnity of holy Boniface, thy Martyr and Bishop. Through.

\communion{To him that overcometh will I grant to sit with me in my throne: even as I also overcame, and am set down with my Father in his throne. (Alleluia.)}

\postcommunion
\lett{O}{Lord,} who hast sanctified us with this saving mystery: we beseech thee; that holy Boniface thy Martyr and Bishop, whom thou hast given to be our advocate and guide, may never fail devoutly to pray for us. Through.


\subby{St. Columba of Iona}
\feastday{{St. Columba}}
\fancyhead[RE,LO]{9 June}
\begin{inhead}
    {Double\\
9 June}
\end{inhead}

\begin{rubric}
	The propers are from the Common of Abbots (p. \pageref{CommonAbbots}), with Commemoration of Sts. Primus \& Felician.
\end{rubric}


\subby{Sts. Primus \& Felician}\label{PrimusFelician}
\feastday{{Sts. Primus \& Felician}}
\fancyhead[RE,LO]{9 June}
\begin{inhead}
    {Memorial\\
9 June}
\end{inhead}

\begin{rubric}
	In Eastertide, the propers are from the Common of Many Martyrs in Eastertide (p. \pageref{CommonMartyrsEaster}, with the Prayers, Greater Alleluia, and Gospel as followeth.
\end{rubric}

\begin{rubric}
	Outside of Eastertide, the propers are as followeth.
\end{rubric}

\introit
\lett{L}{et} the people tell of the wisdom of the Saints, and let the church shew forth their praise: their names shall live for evermore. \textit{Ps.} Rejoice in the Lord, O ye righteous: for it becometh well the just to be thankful.

\collect
\lett{O}{Lord,} we beseech thee, make us ever to rejoice in the festival of thy holy Martyrs Primus and Felician: that by their prayers we may perceive the gifts of thy protection. Through.

\begin{rubric}
	The Epistle is from the Second Common of Many Martyrs (p. \pageref{CommonMartyrsII}).
\end{rubric}

\gradall{O Lord, the very heavens shall praise thy wondrous works: and thy truth in the congregation of the saints. ℣. My song shall be alway of the loving-kindness of the Lord: from one generation to another.}{Alleluia, alleluia. ℣. This is the true brotherhood, which overcame the wickedness of the world: which followed Christ, gaining heaven's glorious realms. Alleluia.}

\begin{rubric}
	In Eastertide, the following Alleluia Verse replaces the Gradual \& Lesser Alleluia.
\end{rubric}

\alleluia{Alleluia, alleluia. ℣. This is the true brotherhood which overcame the wickedness of the world: which followed Christ, gaining heaven's glorious realms. Alleluia. ℣. The noble army of Martyrs praise thee, O Lord. Alleluia.}

\readingcitation{Gospel}{Matthew 11:25}
\lett{A}{t that time:} Jesus answered and said: I thank thee, O Father, Lord of heaven and earth, because thou hast hid these things from the wise and prudent, and hast revealed them unto babes. Even so, Father: for so it seemed good in thy sight. All things are delivered unto me of my Father: and no man knoweth the Son, but the Father; neither knoweth any man the Father, save the Son, and he to whomsoever the Son will reveal him. Come unto me, all ye that labour and are heavy laden, and I will give you rest. Take my yoke upon you, and learn of me; for I am meek and lowly in heart: and ye shall find rest unto your souls. For my yoke is easy, and my burden is light.

\offertory{God is wonderful in his holy ones: even the God of Israel, he will give strength and power unto his people: blessed be God, alleluia.}

\secret
\lett{W}{e} beseech thee, O Lord, that the oblation to be consecrated on the day of the precious death of thy Martyrs may be acceptable unto thee: for the cleansing of our sins, and for the commending to thee of the prayers of thy servants. Through.

\communion{I have chosen you out of the world, that ye should go and bring forth fruit, and that your fruit should remain.}

\postcommunion
\lett{W}{e} beseech thee, almighty God: that the solemnity of thy holy Martyrs Primus and Felician, which we have celebrated with heavenly mysteries, may obtain for us thy merciful pardon. Through.


\subby{St. Margaret of Scotland}
\feastday{{St. Margaret Scotland}}
\fancyhead[RE,LO]{10 June}
\begin{inhead}
    {Memorial\\
10 June}
\end{inhead}

\begin{rubric}
	The propers are from the Common of a Woman Neither Virgin Nor Martyr (p. \pageref{CommonNeitherVirginMartyr}), with the Prayers as followeth.
\end{rubric}

\collect
\lett{O}{God,} who didst render the blessed Queen Margaret wondrous by reason of her eminent charity towards the poor: grant; that, by her intercession and example, thy charity may continually increase in our hearts. Through.

\secret
\lett{G}{rant,} O Lord, that like as thy faithful people do acknowledge that in tribulation they have been succoured by the merits of thy Saints: so this oblation, which they here do offer unto thee in honour of the same, may be acceptable in thy sight. Through.

\postcommunion
\lett{O}{Lord,} who hast satisfied thy family with sacred gifts: we beseech thee that we may at all times be comforted by the intercession of her whose festival we celebrate. Through.


\bcpfeast{11 June. St. Barnabas}{St. Barnabas}{11 June}
%\supplement{11 June}{St. Barnabas}{}

\begin{secrubric}
	The Daily Office propers are from the Common of Apostles (p. \pageref{CommonApostles}).
\end{secrubric}

\begin{secrubric}
	In Eastertide, the Daily Office propers are from the Common of Apostles in Eastertide (p. \pageref{CommonApostlesEaster}).
\end{secrubric}


%MANUAL ADJUSTMENT:
\clearpage
\subby{Sts. Basilides, Cyrinus, Nabor, \& Nazarius}
\feastday{{Sts. Basilides \&c.}}
\fancyhead[RE,LO]{12 June}
\begin{inhead}
    {Memorial\\
12 June}
\end{inhead}

\begin{rubric}
	In Eastertide, the propers are from the Common of Many Martyrs in Eastertide (p. \pageref{CommonMartyrsEaster}), with the following Prayers.
\end{rubric}

\begin{rubric}
	Outside of Eastertide, the propers are as followeth.
\end{rubric}

\introit
\lett{L}{et} the sorrowful sighing of the prisoners, O Lord, come before thee: reward thou our neighbours sevenfold into their bosom: avenge thou the blood of thy Saints, that is shed. \textit{Ps.} O God, the heathen are come into thine inheritance: thy holy temple have they defiled: and made Jerusalem an heap of stones.

\collect
\lett{W}{e} beseech thee, O Lord, that the observance of the birthday of thy holy Martyrs Basilides, Cyrinus, Nabor, and Nazarius may shine brightly upon us: that the everlasting excellence which they have obtained may increase by the fruits of our devotion. Through.

\begin{rubric}
	The Epistle is from the Third Common of Many Martyrs (p. \pageref{Hebrews1032}).
\end{rubric}

\gradall{Avenge thou, O Lord, the blood of thy Saints that is shed. ℣. The dead bodies of thy servants have they given to be meat unto the fowls of the air: and the flesh of thy saints unto the beasts of the land}{Alleluia, alleluia. ℣. The bodies of the Saints are buried in peace: but their name liveth for evermore. Alleluia.}

\begin{rubric}
	The Gospel is from the first additional Gospel from the Third Common of Many Martyrs (p. \pageref{Matthew243}).
\end{rubric}

\offertory{Let the Saints be joyful with glory, let them rejoice in their beds: let the praises of God be in their mouth.}

\secret
\lett{W}{e} solemnly offer unto thee, O Lord, these sacrifices, rehearsing thy wondrous works, in honour of the blood of thy Saints Basilides, Cyrinus, Nabor, and Nazarius: whereby their glorious victory was made perfect. Through.

\communion{The dead bodies of thy servants, O Lord, have they given to be meat unto the fowls of the air, and the flesh of thy saints unto the beasts of the land: according to the greatness of thy power, preserve thou those that are appointed to die.}

\postcommunion
\lett{G}{rant,} we beseech thee, O Lord: that we, ever celebrating the festival of thy holy Martyrs Basilides, Cyrinus, Nabor, and Nazarius; may continually perceive their advocacy. Through.


\subby{St. Basil the Great}
\feastday{{St. Basil the Great}}
\fancyhead[RE,LO]{14 June}
\begin{inhead}
    {Greater Double\\
14 June}
\end{inhead}

\begin{rubric}
	The Daily Office propers are from the Common of Doctors (p. \pageref{CommonDoctors}).
\end{rubric}

\introit
\lett{I}{n} the midst of the Church he opened his mouth: and the Lord filled him with the spirit of wisdom and of understanding: he clothed him with a robe of glory. \textit{Ps.} It is a good thing to give thanks unto the Lord: and to sing praises unto thy name, O Most Highest.

\collect
\lett{W}{e} beseech thee, O Lord, graciously to hear the prayers which we offer unto thee on the solemnity of blessed Basil, thy Confessor and Bishop: that, like as he was found worthy to do thee faithful service, so by his merits and intercession we may be absolved from all our sins. Through.

\begin{rubric}
	The Epistle is from the Common of Doctors (p. \pageref{2Timothy41}).
\end{rubric}

\gradall{The mouth of the righteous is exercised in wisdom, and his tongue will be talking of judgment. ℣. The law of his God is in his heart: and his goings shall not slide.}{Alleluia, alleluia. ℣. I have found David my servant, with my holy oil have I anointed him. Alleluia.}

\readingcitation{Gospel}{Luke 14:26}
\lett{A}{t that time:} Jesus said unto the multitudes: If any man come to me, and hate not his father, and mother, and wife, and children, and brethren, and sisters, yea, and his own life also, he cannot be my disciple. And whosoever doth not bear his cross, and come after me, cannot be my disciple. For which of you, intending to build a tower, sitteth not down first, and counteth the cost, whether he have sufficient to finish it? Lest haply, after he hath laid the foundation, and is not able to finish it, all that behold it begin to mock him, Saying, This man began to build, and was not able to finish. Or what king, going to make war against another king, sitteth not down first, and consulteth whether he be able with ten thousand to meet him that cometh against him with twenty thousand? Or else, while the other is yet a great way off, he sendeth an ambassage, and desireth conditions of peace. So likewise, whosoever he be of you that forsaketh not all that he hath, he cannot be my disciple. Salt is good: but if the salt have lost his savour, wherewith shall it be seasoned? It is neither fit for the land, nor yet for the dunghill; but men cast it out. He that hath ears to hear, let him hear.

\offertory{My truth and my mercy shall be with him: and in my name shall his horn be exalted.}

\secret
\lett{W}{e} beseech thee, O Lord, that our devout observance of the yearly solemnity of Saint Basil, thy Confessor and Bishop, may render us acceptable unto thy loving kindness: that this service of propitiation, which we duly offer, may be profitable unto him for the reward of blessedness, and obtain for us the gifts of thy grace. Through.

\communion{A faithful and wise servant, whom the lord hath made ruler over his household: to give them their portion of meat in due season.}

\postcommunion
\lett{O}{God,} who rewardest the souls of them that put their trust in thee: vouchsafe; that we who keep the solemn festival of blessed Basil, thy Confessor and Bishop, may by his prayers obtain thy merciful pardon. Through.


\subby{Sts. Vitus, Modestus, \& Crescentia}
\feastday{{Sts. Vitus \&c.}}
\fancyhead[RE,LO]{15 June}
\begin{inhead}
    {Memorial\\
15 June}
\end{inhead}

\begin{rubric}
	In Eastertide, the propers are from the Common of Many Martyrs in Eastertide (p. \pageref{CommonMartyrsEaster}), with Prayers and Gospel as followeth.
\end{rubric}

\begin{rubric}
	Outside of Eastertide, the propers are as followeth.
\end{rubric}

\introit
\lett{G}{reat} are the troubles of the righteous, but the Lord delivereth him out of all: the Lord keepeth all his bones: so that not one of them is broken. \textit{Ps.} I will alway give thanks unto the Lord: his praise shall ever be in my mouth.

\collect
\lett{G}{rant,} O Lord, we beseech thee, that, at the intercession of thy holy Martyrs, Vitus, Modestus, and Crescentia, thy Church may learn not to be highminded, but to grow in such lowliness as is acceptable unto thee: that, despising things evil, she may with bounteous charity perform those things that be right. Through.

\begin{rubric}
	The Epistle is from First Common of Many Martyrs (p. \pageref{Wisdom31}).
\end{rubric}

\gradall{Let the Saints be joyful with glory: let them rejoice in their beds. ℣. O sing unto the Lord a new song: let the congregation of saints praise him.}{Alleluia, alleluia. ℣. Thy Saints shall give thanks unto thee, O Lord: they shew the glory of thy kingdom. Alleluia.}

\readingcitation{Gospel}{Luke 10:16}
\lett{A}{t that time:} Jesus said unto his disciples: He that heareth you heareth me; and he that despiseth you despiseth me; and he that despiseth me despiseth him that sent me. And the seventy returned again with joy, saying, Lord, even the devils are subject unto us through thy name. And he said unto them, I beheld Satan as lightning fall from heaven. Behold, I give unto you power to tread on serpents and scorpions, and over all the power of the enemy: and nothing shall by any means hurt you. Notwithstanding in this rejoice not, that the spirits are subject unto you; but rather rejoice, because your names are written in heaven.

\secret
\lett{O}{Lord,} forasmuch as the gifts which we offer for thy Saints do shew forth the glory of thy divine power: so may they achieve in us the effects of thy salvation. Through.

\communion{The souls of the just are in the hand of God, and there shall no torment of malice touch them: in the sight of the unwise they seemed to die: but they are in peace.}

\postcommunion
\lett{O}{Lord,} who hast fulfilled us with thy solemn benediction: we beseech thee; that through the intercession of thy holy Martyrs Vitus, Modestus, and Crescentia, the healing of this sacrament may be profitable both for our bodies and our souls. Through.


\subby{St. Ephrem the Syrian}
\feastday{{St. Ephrem Syrian}}
\fancyhead[RE,LO]{18 June}
\begin{inhead}
    {Double\\
18 June}
\end{inhead}

\begin{rubric}
	The propers are from the Common of Doctors (p. \pageref{CommonDoctors}), with the Prayers as followeth.
\end{rubric}

\begin{rubric}
	\textsc{Note,} The Creed is said.
\end{rubric}

\collect
\lett{O}{God,} who wast pleased to adorn thy Church with the wondrous learning and glorious merits of the life of blessed Ephram, thy Confessor and Doctor: we humbly entreat thee; that at his intercession thou wouldest defend her with thy continual power against the snares of error and wickedness. Through.

\begin{rubric}
	Commemoration of Sts. Marcus \& Marcellianus (p. \pageref{MarcusMarcellianusCollect}).
\end{rubric}

\secret
\lett{M}{ay} the devout prayers of Saint Ephrem, thy Confessor and Doctor, never fail to succour us, O Lord: that they may render our oblations acceptable in thy sight; and may ever obtain for us thy merciful pardon. Through.

\begin{rubric}
	Commemoration of Sts. Marcus \& Marcellianus (p. \pageref{MarcusMarcellianusSecret}).
\end{rubric}

\postcommunion
\lett{W}{e} beseech thee, O Lord, that blessed Ephrem, thy Confessor and illustrious Doctor, may stand before thee as our advocate: that these thy sacrifices may avail for our salvation. Through.

\begin{rubric}
	Commemoration of Sts. Marcus \& Marcellianus (p. \pageref{MarcusMarcellianusPostcommunion}).
\end{rubric}


%MANUAL ADJUSTMENT:
\clearpage
\subby{Sts. Marcus \& Marcellianus}
\feastday{{Sts. Marcus \&c.}}
\fancyhead[RE,LO]{18 June}
\begin{inhead}
    {Memorial\\
18 June}
\end{inhead}

\begin{rubric}
	In Eastertide, the propers are from the Common of Many Martyrs in Eastertide (p. \pageref{CommonMartyrsEaster}), with the Prayers and Gospel as followeth.
\end{rubric}

\begin{rubric}
	Outside Eastertide, the propers are as followeth.
\end{rubric}

\introit
\lett{B}{ut} the salvation of the righteous cometh of the Lord: who is also their strength in the time of trouble. \textit{Ps.} Fret not thyself because of the ungodly: neither be thou envious against the evil-doers.

\collect\label{MarcusMarcellianusCollect}
\lett{G}{rant,} we beseech thee, almighty God: that we who devoutly celebrate the birthday of thy holy Martyrs Marcus and Marcellianus; may, by their intercession, be delivered from all evils that beset us. Through.

\begin{rubric}
	The Epistle is from the second additional Epistle from the Third Common of Many Martyrs (p. \pageref{Romans51}).
\end{rubric}

\gradall{The souls of the just are in the hand of God: and there shall no torment of malice touch them. ℣. In the eyes of the unwise they seemed to die: but they are in peace.}{Alleluia, alleluia. ℣. This is the true brotherhood which no conflict could sunder: they who, shedding their blood, followed the Lord. Alleluía.}

\begin{rubric}
	In Eastertide, the following Alleluia Verse replaces the Gradual \& Lesser Alleluia.
\end{rubric}

\alleluia{Alleluia, alleluia. ℣. This is the true brotherhood which no conflict could sunder: they who, shedding their blood, followed the Lord. Alleluia.  ℣. The noble army of Martyrs praise thee, O Lord. Alleluia.}

\begin{rubric}
	The Gospel is from the fourth additional Gospel from the Third Common of Many Martyrs (p. \pageref{Luke1147}).
\end{rubric}

\offertory{Our soul is escaped even as a bird out of the snare of the fowler: the snare is broken, and we are delivered.}

\secret\label{MarcusMarcellianusSecret}
\lett{S}{anctify,} O Lord, the gifts which we dedicate unto thee: that, at the intercession of thy holy Martyrs Marcus and Marcellianus, they may obtain for us thy gracious favour. Through.

\communion{Verily I say unto you, whatsoever ye have done unto one of the least of mine, ye have done it unto me: come, ye blessed of my Father, inherit the kingdom prepared for you from the foundation of the world.}

\postcommunion\label{MarcusMarcellianusPostcommunion}
\lett{O}{Lord,} who hast satisfied us with the gift of thy salvation, we humbly beseech thee: that as we joyfully taste thereof, so at the intercession of thy holy Martyrs Marcus and Marcellianus, we may effectually be renewed by the same. Through.


\subby{Sts. Gervase \& Protase}
\feastday{{Sts. Gervase \& Protase}}
\fancyhead[RE,LO]{19 June}
\begin{inhead}
    {Memorial\\
19 June}
\end{inhead}

\begin{rubric}
	In Eastertide, the propers are from the Common of Many Martyrs in Eastertide (p. \pageref{CommonMartyrsEaster}), with the Prayers as followeth and the Greater Alleluia from the Feast of Sts. Primus \& Felician (p. \pageref{PrimusFelician}).
\end{rubric}

\begin{rubric}
	Outside of Eastertide, the propers are as followeth.
\end{rubric}

\introit
\lett{T}{he} Lord shall speak peace unto his people: and to his saints, that they turn not again. \textit{Ps.} Lord, thou art become gracious unto thy land, thou hast turned away the captivity of Jacob.

\collect
\lett{O}{God,} who makest us glad with the yearly solemnity of thy holy Martyrs Gervasius and Protasius: mercitully grant; that as we do rejoice in their merits, so we may be enkindled by their example. Through.

\begin{rubric}
	The Epistle is from the second additional Epistle from the Second Common of a Martyr not a Bishop (p. \pageref{1Peter413}).
\end{rubric}

\gradall{God is glorious in his holy ones: fearful in praises, doing wonders. ℣. Thy right hand, O Lord, is become glorious in power: thy right hand hath dashed in pieces the enemy.}{Alleluia, alleluia. ℣. This is the true brotherhood, which overcame the wickedness of the world: which followed Christ, gaining heaven's glorious realms. Alleluia.}

\begin{rubric}
	The Gospel is from the Second Common of Many Martyrs (p. \pageref{CommonMartyrsII}).
\end{rubric}

\offertory{Be glad, O ye righteous, and rejoice in the Lord: and be joyful, all ye that are true of heart.}

\secret
\lett{W}{e} beseech thee, O Lord, mercifully to accept these our oblations: that, at the intercession of thy holy Martyrs Gervasius and Protasius, we may be defended against all adversities. Through.

\communion{The dead bodies of thy servants, O Lord, have they given to be meat unto the fowls of the air, and the flesh of thy Saints unto the beasts of the land: according to the greatness of thy power, preserve thou those that are appointed to die.}

\postcommunion
\lett{M}{ay} this communion, O Lord, cleanse us from guilt: and at the intercession of thy holy Martyrs Gervasius and Protasius, make us partakers of thy heavenly healing. Through.


\subby{St. Silverius}
\feastday{{St. Silverius}}
\fancyhead[RE,LO]{20 June}
\begin{inhead}
    {Memorial\\
20 June}
\end{inhead}

\begin{rubric}
	The propers are from the First Common of a Martyr Bishop (p. \pageref{CommonMartyrBishopI}), with the Epistle as followeth.
\end{rubric}

\readingcitation{Epistle}{Jude 17}
\lett{D}{early beloved:} Remember ye the words which were spoken before of the apostles of our Lord Jesus Christ; How that they told you there should be mockers in the last time, who should walk after their own ungodly lusts. These be they who separate themselves, sensual, having not the Spirit. But ye, beloved, building up yourselves on your most holy faith, praying in the Holy Ghost, Keep yourselves in the love of God, looking for the mercy of our Lord Jesus Christ unto eternal life.


\subby{St. Alban}
\feastday{{St. Alban}}
\fancyhead[RE,LO]{22 June}
\begin{inhead}
    {Double\\
22 June}
\end{inhead}

\begin{rubric}
	The Daily Office propers are from the First Common of a Martyr not a Bishop (p. \pageref{CommonMartyrNotBishopI}).
\end{rubric}

\introit
\lett{T}{he} just shall rejoice in thy strength, O Lord: exceeding glad shall he be of thy salvation: thou hast given him his heart's desire.\textit{Ps.} For thou hast prevented him with the blessings of goodness: thou hast set a crown of pure gold upon his head.

\collect
\lett{O}{God,} who hast hallowed this day by the martyrdom of blessed Alban: grant, we beseech thee; that as year by year we rejoice to pay him honour, so we may be defended by his continual help. Through.

\begin{rubric}
	Commemoration of St. Paulinus (p. \pageref{PaulinusCollect}).
\end{rubric}

\readingcitation{Epistle}{Wisdom 10:10}
\lett{W}{isdom} guided the righteous man in straight paths; She shewed him God’s kingdom, and gave him knowledge of holy things; She prospered him in his toils, and multiplied the fruits of his labour; When in their covetousness men dealt hardly with him, She stood by him and made him rich; She guarded him from enemies, And from those that lay in wait she kept him safe, And over his sore conflict she watched as judge, That he might know that godliness is more powerful than all. When a righteous man was sold, wisdom forsook him not, But from sin she delivered him; She went down with him into a dungeon, And in bonds she left him not, Till she brought him the sceptre of a kingdom, And authority over those that dealt tyrannously with him; She shewed them also to be false that had mockingly accused him, And gave him eternal glory.

\gradall{Blessed is the man that feareth the Lord: he hath great delight in his commandments. ℣. His seed shall be mighty upon earth: the generation of the faithful shall be blessed.}{Alleluia, alleluia. ℣. Thou hast set, O Lord, a crown of pure gold upon his head. Alleluia.}

\readingcitation{Gospel}{Matthew 16:24}
\lett{A}{t that time:} Jesus said unto his disciples: If any man will come after me, let him deny himself, and take up his cross, and follow me. For whosoever will save his life shall lose it: and whosoever will lose his life for my sake shall find it. For what is a man profited, if he shall gain the whole world, and lose his own soul? or what shall a man give in exchange for his soul? For the Son of man shall come in the glory of his Father with his angels; and then he shall reward every man according to his works.

\offertory{Thou hast crowned him with glory and worship, and hast made him to have dominion of the works of thy hands, O Lord.}

\secret
\lett{W}{e} beseech thee, O Lord, that like as in the veneration of blessed Alban, thy Martyr, we do shew forth thy wonders: so through this bounden service of propitiation he may be a faithful intercessor for us in the sight of thy mercy. Through.

\begin{rubric}
	Commemoration of St. Paulinus (p. \pageref{PaulinusSecret}).
\end{rubric}

\communion{If any man will come after me, let him deny himself, and take up his cross and follow me.}

\postcommunion
\lett{L}{et} blessed Alban, thy Martyr, we beseech thee, O Lord, ever implore thy holy majesty: that these thy sacraments may cleanse us from guilt, and preserve in us the fervour of thy charity. Through.

\begin{rubric}
	Commemoration of St. Paulinus (p. \pageref{PaulinusPostcommunion}).
\end{rubric}


\subby{St. Paulinus}
\feastday{{St. Paulinus}}
\fancyhead[RE,LO]{22 June}
\begin{inhead}
    {Memorial\\
22 June}
\end{inhead}

\introit
\lett{L}{et} thy priests, O Lord, be clothed with righteousness, and let thy saints sing with joyfulness: for thy servant David's sake, turn not away the presence of thine anointed. \textit{Ps.} Lord, remember David, and all his trouble.

\collect\label{PaulinusCollect}
\lett{O}{God,} who to them that forsake all things in this world for thee hast promised a hundredfold in the world to come, and life eternal: mercifully grant; that, following in the footsteps of thy holy Bishop, Paulinus, we may be enabled to despise earthly things, and to seek only after things heavenly: Who livest.

\begin{rubric}
	If to-day be Saturday, a Commemoration is made of the anticipated Vigil of St. John Baptist, as below.
\end{rubric}

\readingcitation{Epistle}{2 Corinthians 8:9}
\lett{B}{rethren:} Ye know the grace of our Lord Jesus Christ, that, though he was rich, yet for your sakes he became poor, that ye through his poverty might be rich. And herein I give my advice: for this is expedient for you, who have begun before, not only to do, but also to be forward a year ago. Now therefore perform the doing of it; that as there was a readiness to will, so there may be a performance also out of that which ye have. For if there be first a willing mind, it is accepted according to that a man hath, and not according to that he hath not. For I mean not that other men be eased, and ye burdened: But by an equality, that now at this time your abundance may be a supply for their want, that their abundance also may be a supply for your want: that there may be equality: As it is written, He that had gathered much had nothing over; and he that had gathered little had no lack.

\gradall{Behold a great priest who in his days pleased God. ℣. There was none found like unto him, who kept the law of the Most High.}{Alleluia, alleluia. ℣. Thou art a priest for ever after the order of Melchisedech. Alleluia.}

\begin{rubric}
	The Gospel is from the Second Common of a Confessor not a Bishop (p. \pageref{CommonConfessorNotBishopII}).
\end{rubric}

\offertory{I have found David my servant, with my holy oil have I anointed him: my hand shall hold him fast, and my arm shall strengthen him.}

\secret\label{PaulinusSecret}
\lett{G}{rant,} O Lord, that, after the example of thy holy Bishop Paulinus, we may join the sacrifice of perfect charity to the oblation of the altar: and by zeal in well-doing be found worthy of thine everlasting mercy. Through.

\communion{A faithful and wise servant whom the lord hath made ruler over his household, to give them their portion of meat in due season.}

\postcommunion\label{PaulinusPostcommunion}
\lett{G}{rant} us, O Lord, through these holy things that love of piety and humility which thy holy Bishop Paulinus drew from this divine fountain: and by his intercession graciously pour forth on all who entreat thee the riches of thy grace. Through.


\subby{Vigil of St. John Baptist}
\feastday{{St. John Baptist Vigil}}
\fancyhead[RE,LO]{23 June}
\begin{inhead}
    {Vigil\\
23 June}
\end{inhead}

\introit
\lett{F}{ear} not, Zacharias, thy prayer is heard: and thy wife Elisabeth shall bear thee a son, and thou shalt call his name John: he shall be great in the sight of the Lord: and shall be filled with the Holy Ghost, even from his mother's womb: and many shall rejoice at his birth. \textit{Ps.} The king shall rejoice in thy strength, O Lord: exceeding glad shall he be of thy salvation.

\collect
\lett{G}{rant,} we beseech thee, almighty God: that thy family may walk in the way of salvation; and, following the teachings of blessed John the Forerunner, may attain in safety unto him whom he fore-told, Jesus Christ thy Son, our Lord: Who liveth.

\begin{rubric}
The \nth{2} is of St. Mary (p. \SPMaryEaster) and the \nth{3} against the persecutors of the Church (p. \SPAgainst) or for the Chief Bishop (p. \SPChiefBishop).
\end{rubric}

\readingcitation{Epistle}{Jeremiah 1:4}
\lett{I}{n those days:} The word of the \divineName{Lord} came unto me, saying, Before I formed thee in the belly I knew thee; and before thou camest forth out of the womb I sanctified thee, and I ordained thee a prophet unto the nations. Then said I, Ah, Lord \divineName{God}! behold, I cannot speak: for I am a child. But the \divineName{Lord} said unto me, Say not, I am a child: for thou shalt go to all that I shall send thee, and whatsoever I command thee thou shalt speak. Be not afraid of their faces: for I am with thee to deliver thee, saith the \divineName{Lord}. Then the \divineName{Lord} put forth his hand, and touched my mouth. And the \divineName{Lord} said unto me, Behold, I have put my words in thy mouth. See, I have this day set thee over the nations and over the kingdoms, to root out, and to pull down, and to destroy, and to throw down, to build, and to plant: saith the Lord almighty.

\gradual{There was a man sent from God, whose name was John. ℣. The same came to bear witness of the light, to make ready a people prepared for the Lord.}

\readingcitation{Gospel}{Luke 1:5}
\lett{T}{here} was in the days of Herod, the king of Judaea, a certain priest named Zacharias, of the course of Abia: and his wife was of the daughters of Aaron, and her name was Elisabeth. And they were both righteous before God, walking in all the commandments and ordinances of the Lord blameless. And they had no child, because that Elisabeth was barren, and they both were now well stricken in years. And it came to pass, that while he executed the priest's office before God in the order of his course, According to the custom of the priest's office, his lot was to burn incense when he went into the temple of the Lord. And the whole multitude of the people were praying without at the time of incense. And there appeared unto him an angel of the Lord standing on the right side of the altar of incense. And when Zacharias saw him, he was troubled, and fear fell upon him. But the angel said unto him, Fear not, Zacharias: for thy prayer is heard; and thy wife Elisabeth shall bear thee a son, and thou shalt call his name John. And thou shalt have joy and gladness; and many shall rejoice at his birth. For he shall be great in the sight of the Lord, and shall drink neither wine nor strong drink; and he shall be filled with the Holy Ghost, even from his mother's womb. And many of the children of Israel shall he turn to the Lord their God. And he shall go before him in the spirit and power of Elias, to turn the hearts of the fathers to the children, and the disobedient to the wisdom of the just; to make ready a people prepared for the Lord.

\offertory{Thou hast crowned him with glory and worship: thou hast made him to have dominion of the works of thy hands, O Lord.}

\secret
\lett{S}{anctify,} O Lord, the gifts which we ofter: and, at the intercession of blessed John Baptist, cleanse us thereby from the defilements of our iniquities. Through.

\begin{rubric}
The \nth{2} is of St. Mary (p. \SPMaryEaster) and the \nth{3} against the persecutors of the Church (p. \SPAgainst) or for the Chief Bishop (p. \SPChiefBishop).
\end{rubric}

\communion{His honour is great in thy salvation: glory and great worship shalt thou lay upon him, O Lord.}

\postcommunion
\lett{O}{Lord,} let the glorious prayer of blessed John Baptist ever be with us: and may he implore for us the mercy of him, whose coming he fore-told, Jesus Christ thy Son our Lord: Who liveth.

\begin{rubric}
The \nth{2} is of St. Mary (p. \SPMaryEaster) and the \nth{3} against the persecutors of the Church (p. \SPAgainst) or for the Chief Bishop (p. \SPChiefBishop).
\end{rubric}


\subby{St. Etheldreda}
\feastday{{St. Etheldreda}}
\fancyhead[RE,LO]{23 June}
\begin{inhead}
    {Memorial\\
23 June}
\end{inhead}

\begin{rubric}
	The propers are from the First Common of a Virgin (p. \pageref{CommonVirginOnlyI}), with the Collect as followeth.
\end{rubric}

\collect
\lett{O}{God,} who makest us glad with the yearly solemnity of blessed Etheldreda, thy Virgin: mercifully grant; that as we are enlightened by the example of her purity, so we may be succoured by her prayers. Through.

%MANUAL ADJUSTMENT:
\clearpage
\bcpfeast{24 June. St. John Baptist}{St. John Baptist}{24 June}
%\supplement{24 June}{St. John Baptist}{}

\subbysub{I Evensong}\label{JohnBaptistEvensong}

\gregorioscore{resources/gabc/ProperTime/JohnBaptistEvensong.gabc}

	℣. There was a man sent from God.
	
	℟. Whose name was John.

\properantiphon{Mag.}{When Zacharias {\dag} went into the temple, there appeared unto him the Angel Gabriel, standing on the right side of the altar of incense.}

\subbysub{Mattins}

\invitatoryhymn\label{JohnBaptistInvitatory}

\gregorioscore{resources/gabc/ProperTime/JohnBaptistInvitatory.gabc}

\officehymn\label{JohnBaptistMattins}

\gregorioscore{resources/gabc/ProperTime/JohnBaptistMattins.gabc}

	℣. This child shall be great in the sight of the Lord.
	
	℟. For the hand of the Lord is with him.

\properantiphon{Ben.}{The mouth of Zacharias {\dag} was opened, and he prophesied, saying: Blessed be the God of Israel.}

\subbysub{II Evensong}

\begin{rubric}
	The Office Hymn is of I Evensong, with the following Versicle \& Antiphon.
\end{rubric}

	℣. This child shall be great in the sight of the Lord.
	
	℟. For the hand of the Lord is with him.

\properantiphon{Mag.}{The child {\dag} that is born unto us is more than a prophet; for this is he of whom the Saviour saith: Among them that are born of women, there hath not risen a greater than John the Baptist.}


\subby{Sts. John \& Paul}
\feastday{{Sts. John \& Paul}}
\fancyhead[RE,LO]{26 June}
\begin{inhead}
    {Double\\
26 June}
\end{inhead}

\begin{rubric}
	The Daily Office propers are from the First Common of Many Martyrs (p. \pageref{CommonMartyrsI}).
\end{rubric}

\introit
\lett{G}{reat} are the troubles of the righteous, but the Lord delivereth him out of all: the Lord keepeth all his bones: so that not one of them is broken. \textit{Ps.} I will alway give thanks unto the Lord: his praise shall ever be in my mouth.

\collect
\lett{W}{e} beseech thee, almighty God: that, as their faith and passion did cause blessed John and Paul to be brothers indeed; so this day's festival may bestow upon us a twofold gladness in their glory. Through.

\begin{rubric}
	Commemoration of the Octave of St. John Baptist, as followeth.
\end{rubric}

\lett{A}{lmighty} God, by whose providence thy servant John Baptist was wonderfully born, and sent to prepare the way of thy Son our Saviour by preaching repentance; Make us so to follow his doctrine and holy life, that we may truly repent according to his preaching; and after his example constantly speak the truth, boldly rebuke vice, and patiently suffer for the truth's sake. Through the same.

\readingcitation{Epistle}{Ecclesiasticus 44:10}
\lett{T}{hese} were men of mercy, Whose righteous deeds have not been forgotten. With their seed shall remain continually a good inheritance; Their children are within the covenants. Their seed standeth fast, And their children for their sakes. Their seed shall remain for ever, And their glory shall not be blotted out. Their bodies were buried in peace, And their name liveth to all generations. Peoples will declare their wisdom, And the congregation telleth out their praise. 

\gradall{Behold, how good and joyful a thing it is, brethren, to dwell together in unity! ℣. It is like the precious ointment upon the head, that ran down unto the beard, even Aaron's beard.}{Alleluia, alleluia. ℣. This is the true brotherhood, which overcame the wickedness of the world: which followed Christ, gaining heaven's glorious realms. Alleluia.}

\begin{rubric}
	The Gospel is from the Third Common of Many Martyrs (p. \pageref{CommonMartyrsIII}).
\end{rubric}

\offertory{They that love thy name shall be joyful in thee, for thou, Lord, wilt give thy blessing unto the righteous: and with thy favourable kindness wilt thou defend him as with a shield, O Lord.}

\secret
\lett{W}{e} beseech thee, O Lord, mercifully to accept this our sacrifice which we offer unto thee, pleading the merits of thy holy Martyrs John and Paul: that the same may avail for our perpetual succour. Through.

\begin{rubric}
	Commemoration of the Octave of St. John Baptist, as followeth.
\end{rubric}

\lett{W}{e} set upon thine altars, O Lord, these gifts: celebrating with due honour the nativity of him, who sang of the coming and proclaimed the presence of the Saviour of the world, Jesus Christ thy Son our Lord. Who liveth.

\communion{Though they be punished in the sight of men, God proved them: as gold in the furnace hath he tried them, and received them as a burnt-offering.}

%MANUAL ADJUSTMENT:
\clearpage
\postcommunion
\lett{G}{rant,} we beseech thee, O Lord: that we who have received these heavenly sacraments, in celebration of the festival of thy holy Martyrs John and Paul; may attain in everlasting joys the fulfilment of our service in this life. Through.

\begin{rubric}
	Commemoration of the Octave of St. John Baptist, as followeth.
\end{rubric}

\lett{L}{et} thy Church, O God, rejoice at the birth of blessed John Baptist: through whom she hath known the author of her new birth, Jesus Christ thy Son our Lord. Who liveth.


\subby{Vigil of St. Peter}
\feastday{{St. Peter Vigil}}
\fancyhead[RE,LO]{28 June}
\begin{inhead}
    {Vigil\\
28 June}
\end{inhead}

\introit
\lett{T}{he} Lord saith unto Peter: When thou wast young, thou girdedst thyself, and walkedst whither thou wouldest: but when thou shalt be old, thou shalt stretch forth thy hands, and another shall gird thee, and carry thee whither thou wouldest not: this spake he, signifying by what death he should glorify God. \textit{Ps.} The heavens declare the glory of God: and the firmament sheweth his handy-work.

\collect
\lett{G}{rant,} we beseech thee, almighty God: that as thou hast stablished us on the rock of the apostolic confession; so thou wouldest not suffer us to be troubled by any adversities. Through.

\readingcitation{Epistle}{Acts 3:1}
\lett{I}{n those days:} Peter and John went up together into the temple at the hour of prayer, being the ninth hour. And a certain man lame from his mother's womb was carried, whom they laid daily at the gate of the temple which is called Beautiful, to ask alms of them that entered into the temple; Who seeing Peter and John about to go into the temple asked an alms. And Peter, fastening his eyes upon him with John, said, Look on us. And he gave heed unto them, expecting to receive something of them. Then Peter said, Silver and gold have I none; but such as I have give I thee: In the name of Jesus Christ of Nazareth rise up and walk. And he took him by the right hand, and lifted him up: and immediately his feet and ankle bones received strength. And he leaping up stood, and walked, and entered with them into the temple, walking, and leaping, and praising God. And all the people saw him walking and praising God: And they knew that it was he which sat for alms at the Beautiful gate of the temple: and they were filled with wonder and amazement at that which had happened unto him.

\gradual{Their sound is gone out into all lands: and their words into the ends of the world. ℣. The heavens declare the glory of God: and the firmament sheweth his handy-work.}

%RV, due to textual variant issue:
\readingcitation{Gospel}{John 21:15}
\lett{A}{t that time:} Jesus said unto Simon Peter: Simon, son of John, lovest thou me more than these? He saith unto him, Yea, Lord; thou knowest that I love thee. He saith unto him, Feed my lambs. He saith to him again a second time, Simon, son of John, lovest thou me? He saith unto him, Yea, Lord; thou knowest that I love thee. He saith unto him, Tend my sheep. He saith unto him the third time, Simon, son of John, lovest thou me? Peter was grieved because he said unto him the third time, Lovest thou me? And he said unto him, Lord, thou knowest all things; thou knowest that I love thee. Jesus saith unto him, Feed my sheep. Verily, verily, I say unto thee, When thou wast young, thou girdedst thyself, and walkedst whither thou wouldest: but when thou shalt be old, thou shalt stretch forth thy hands, and another shall gird thee, and carry thee whither thou wouldest not. Now this he spake, signifying by what manner of death he should glorify God. 

\offertory{Right honourable are thy friends unto me, O God: right well is their princedom established.}

\secret
\lett{S}{anctify,} the gift of thy people, we beseech thee, O Lord, by the intercession of thine Apostles: and cleanse us from the defilements of our iniquities. Through.

\communion{Simon, son of Jonas, lovest thou me more than these? Lord, thou knowest all things: thou knowest, Lord, that I love thee.}

\postcommunion
\lett{O}{Lord,} who hast satisfied us with heavenly food: defend us by the intercession of thine Apostles against all adversity. Through.



\subby{Pope St. Leo II}
\feastday{{Pope St. Leo II}}
\fancyhead[RE,LO]{28 June}
\begin{inhead}
    {Memorial\\
28 June}
\end{inhead}

\begin{rubric}
	The propers are from the First Common of a Confessor Bishop (p. \pageref{CommonConfessorBishopI}).
\end{rubric}


\bcpfeast{29 June. St. Peter}{St. Peter}{29 June}
%\supplement{29 June}{St. Peter}{}

\subbysub{I Evensong}\label{PeterEvensong}

\gregorioscore{resources/gabc/ProperTime/PeterEvensong.gabc}

	℣. Their sound is gone out into all lands.
	
	℟. And their words into the ends of the world.

\properantiphon{Mag.}{Thou art the shepherd of the sheep, {\dag} O chief of the Apostles: unto thee were given the keys of the kingdom of heaven.}

\subbysub{Mattins}

\begin{rubric}
	The Invitatory Hymn is of the Common of Apostles (p. \pageref{CommonApostles}).
\end{rubric}

\officehymn\label{PeterMattins}

\gregorioscore{resources/gabc/ProperTime/PeterMattins.gabc}

	℣. They declared the work of God.
		
	℟. And wisely considered of his doing.

\properantiphon{Ben.}{Whatsoever {\dag} thou shalt bind on earth shall be bound in heaven: and whatsoever thou shalt loose on earth shall be loosed in heaven: saith the Lord unto Simon Peter.}

\subbysub{II Evensong}

\begin{rubric}
	The Office Hymn is of I Evensong, with the following Versicle \& Antiphon.
\end{rubric}

	℣. They declared the work of God.
		
	℟. And wisely considered of his doing.

\properantiphon{Mag.}{To-day {\dag} Simon Peter ascended the gibbet of the cross, alleluia: to-day the Key-Bearer of the Kingdom joyfully departed to Christ: to-day Paul the Apostle, the light of the whole world, bowed down his head; and for the Name of Christ, received the crown of martyrdom, alleluia.}


\bcpfeast{30 June. Commemoration of St. Paul}{St. Paul Commemoration}{30 June}
%\supplement{30 June}{Commemoration of St. Paul}{}

\subbysub{I Evensong}\label{PaulEvensong}

\gregorioscore{resources/gabc/ProperTime/PaulEvensong.gabc}

	℣. Thou art a chosen vessel, holy Apostle Paul.
		
	℟. A preacher of the truth throughout all the world.

\properantiphon{Mag.}{O holy Apostle Paul, {\dag} thou preacher of the truth and Doctor of the Gentiles, intercede for us unto God, who hath chosen thee.}

\subbysub{Mattins}

\begin{rubric}
	The Invitatory Hymn is as in I Evensong.
\end{rubric}

\begin{rubric}
	The Office Hymn is as in the Common of Apostles (p. \pageref{CommonApostles}).
\end{rubric}

\begin{rubric}
	In Eastertide, the Office Hymn is as in the Common of Apostles in Eastertide (p. \pageref{CommonApostlesEaster}).
\end{rubric}

	℣. Thou art a chosen vessel, holy Apostle Paul.
		
	℟. A preacher of the truth throughout all the world.

\properantiphon{Ben.}{Ye which have followed me {\dag} shall sit upon twelve thrones, judging the twelve tribes of Israel, saith the Lord.}

\subbysub{II Evensong}

\begin{rubric}
	II Evensong is as in I Evensong.
\end{rubric}


\bcpfeast{1 July. Most Precious Blood of Our Lord Jesus Christ}{Precious Blood}{1 July}
%\supplement{1 July}{Most Precious Blood}{of Our Lord Jesus Christ}

\subbysub{I Evensong}

\gregorioscore{resources/gabc/ProperTime/PreciousBloodEvensong.gabc}\label{PreciousBloodEvensong}

	℣. Thou hast redeemed us, O Lord, by thy Blood.
		
	℟. And hast made us unto our God a kingdom.

\properantiphon{Mag.}{But ye are come {\dag} unto mount Sion and unto the city of the living God, the heavenly Jerusalem, and to Jesus, the mediator of the new covenant, and to the Blood of sprinkling, that speaketh better things than that of Abel.}

\subbysub{Mattins}

\invitatoryhymn

\gregorioscore{resources/gabc/ProperTime/PreciousBloodInvitatory.gabc}\label{PreciousBloodInvitatory}

\officehymn

\gregorioscore{resources/gabc/ProperTime/PreciousBloodMattins.gabc}\label{PreciousBloodMattins}

	℣. Being justified by the Blood of Christ.
		
	℟. We shall be saved from wrath through him.

\properantiphon{Ben.}{The Blood of the Lamb {\dag} shall be to you for a token, saith the Lord: and when I see the Blood, I will pass over you, and the plague shall not be upon you to destroy you.}

\subbysub{II Evensong}

\begin{rubric}
	II Evensong is as in I Evensong, except the following Antiphon.
\end{rubric}

\properantiphon{Mag.}{And this day {\dag} shall be unto you for a memorial: and ye shall keep it a feast to the Lord throughout your generations by an ordinance for ever.}


%MANUAL ADJUSTMENT:
\clearpage
\bcpfeast{2 July. Visitation of the Blessed Virgin Mary}{Visitation}{2 July}
%\supplement{2 July}{Visitation}{of the B.V.M.}

\begin{rubric}
	The Hymns are as in the Common of the Blessed Virgin Mary (p. \pageref{CommonBVM}), with the following Versicles \& Antiphons.
\end{rubric}

	℣. Blessed art thou amongst women.
		
	℟. And blessed is the fruit of thy womb.

\properantiphon{Mag.}{Blessed art thou, {\dag} O Mary, for thou hast believed: and there shall be a performance in thee of those things which were told thee from the Lord, alleluia.}\\

	℣. Blessed art thou amongst women.
		
	℟. And blessed is the fruit of thy womb.

\properantiphon{Ben.}{When Elisabeth {\dag} heard the salutation of Mary, she spake out with a loud voice, and said: Whence is this to me, that the mother of my Lord should come to me? Alleluia.}\\

	℣. Blessed art thou amongst women.
		
	℟. And blessed is the fruit of thy womb.

\properantiphon{Mag.}{All generations {\dag} shall call me blessed: for God hath regarded the lowliness of his handmaiden, alleluia.}


\subby{St. John of San Francisco}
\feastday{{St. John San Francisco}}
\fancyhead[RE,LO]{2 July}
\begin{inhead}
    {Memorial\\
2 July}
\end{inhead}

\begin{rubric}
	The propers are from the Second Common of a Confessor Bishop (p. \pageref{CommonConfessorBishopII}).
\end{rubric}


\subby{Sts. Processus \& Martinian}
\feastday{{St. Processus}}
\fancyhead[RE,LO]{2 July}
\begin{inhead}
    {Memorial\\
2 July}
\end{inhead}

\begin{rubric}
	The propers are from the Second Common of a Confessor Bishop (p. \pageref{CommonConfessorBishopII}).
\end{rubric}


\subby{St. Iren{\ae}us of Lyon}
\feastday{{St. Iren{\ae}us}}
\fancyhead[RE,LO]{3 July}
\begin{inhead}
    {Double\\
3 July}
\end{inhead}

\begin{rubric}
	The propers are from the First Common of a Martyr Bishop (p. \pageref{CommonMartyrBishopI}), except for the following Collect.
\end{rubric}

\collect
\lett{O}{God,} who gavest grace to blessed Iren{\ae}us thy Martyr and Bishop to overcome false doctrine by the truth of his teaching, and favourably to establish peace in thy Church: give thy people, we beseech thee, constancy in holy religion, and grant us thy peace all the days of our life. Through.


%Last Day of Eastertide Possible---
\subby{Sts. Cyril \& Methodius}
\feastday{{Sts. Cyril \& Methodius}}
\fancyhead[RE,LO]{7 July}
\begin{inhead}
    {Double\\
7 July}
\end{inhead}

\begin{rubric}
	The propers are from the Second Common of a Confessor Bishop (p. \pageref{CommonConfessorBishopII}), except for the following.
\end{rubric}

\collect
\lett{A}{lmighty} and everlasting God, who by thy blessed Confessors and Bishops, Cyril and Methodius, didst suffer the peoples of Slavonia to come to the knowledge of thy name: vouchsafe; that we, who glory in their festival may be joined unto their fellowship. Through.

\readingcitation{Gospel}{Luke 10:1}
\lett{A}{t that time:} the Lord appointed other seventy also: and sent them two and two before his face into every city and place, whither he himself would come. Therefore said he unto them, The harvest truly is great, but the labourers are few: pray ye therefore the Lord of the harvest, that he would send forth labourers into his harvest. Go your ways: behold, I send you forth as lambs among wolves. Carry neither purse, nor scrip, nor shoes: and salute no man by the way. And into whatsoever house ye enter, first say, Peace be to this house. And if the son of peace be there, your peace shall rest upon it: if not, it shall turn to you again. And in the same house remain, eating and drinking such things as they give: for the labourer is worthy of his hire. Go not from house to house. And into whatsoever city ye enter, and they receive you, eat such things as are set before you: And heal the sick that are therein, and say unto them, The kingdom of God is come nigh unto you.

\offertory{God is wonderful in his holy ones: even the God of Israel, he will give strength and power unto his people: blessed be God.}

\secret
\lett{W}{e} beseech thee, O Lord, to have respect unto the prayers and oblations of thy faithful people: that they may be acceptable unto thee on this festival of thy Saints, and effectually bestow on us the assistance of thy mercy. Through.

\communion{What I tell you in darkness, that speak ye in light, saith the Lord: and what ye hear in the ear, that preach ye upon the house-tops.}

\postcommunion
\lett{W}{e} beseech thee, almighty God: that as thou dost vouchsafe to bestow upon us heavenly gifts, so, at the intercession of thy Saints Cyril and Methodius, thou wouldest grant unto us to despise things earthly. Through.


\subby{Seven Holy Brothers, with Sts. Rufina \& Secunda}
\feastday{{7 Holy Brothers}}
\fancyhead[RE,LO]{10 July}
\begin{inhead}
    {Memorial\\
10 July}
\end{inhead}

\introit
\lett{P}{raise} the Lord, ye servants, O praise the name of the Lord: who maketh the barren woman to keep house, and to be a joyful mother of children. \textit{Ps.} Blessed be the name of the Lord: from this time forth for evermore.

\collect
\lett{G}{rant,} we beseech thee, almighty God: that, like as we have known thy glorious Martyrs to be constant in their confession, so we may perceive their loving intercession. Through.

\begin{rubric}
	The Epistle is from the Common of Neither Virgin Nor Martyr (p. \pageref{CommonNeitherVirginMartyr}).
\end{rubric}

\gradall{Our soul is escaped, even as a bird out of the snare of the fowler. ℣. The snare is broken, and we are delivered : our help is in the name of the Lord, who hath made heaven and earth.}{Alleluia, alleluia. ℣. This is the true brotherhood, which overcame the wickedness of the world: which followed Christ, gaining heaven's glorious realms. Alleluia.}

\readingcitation{Gospel}{Matthew 12:46}
\lett{A}{t that time:} While Jesus spake to the multitudes, behold his Mother and brethren stood without, desiring to speak with him. Then one said unto him, Behold, thy mother and thy brethren stand without, desiring to speak with thee. But he answered and said unto him that told him, Who is my mother? and who are my brethren? And he stretched forth his hand toward his disciples, and said, Behold my mother and my brethren! For whosoever shall do the will of my Father which is in heaven, the same is my brother, and sister, and mother.

\offertory{Our soul is escaped, even as a bird out of the snare of the fowler: the snare is broken, and we are delivered.}

\secret
\lett{W}{e} beseech thee, O Lord, mercifully to have respect unto these our sacrifices: that through the intercession of thy Saints, they may increase our devotion, and set forward our salvation. Through.

\communion{Whosoever shall do the will of my Father which is in heaven: the same is my brother, and sister, and mother, saith the Lord.}

\postcommunion
\lett{W}{e} beseech thee, almighty God: that as in these holy mysteries we have received the pledge of our salvation, so at the intercession of thy Saints, we may be brought unto the fulfilment of the same. Through.


\subby{St. Joseph of Damascus \& Companions}
\feastday{{St. Joseph Damascene \&c.}}
\fancyhead[RE,LO]{10 July}
\begin{inhead}
    {Memorial\\
10 July}
\end{inhead}

\begin{rubric}
	The propers are from the Third Common of Many Martyrs (p. \pageref{CommonMartyrsIII}).
\end{rubric}


\subby{Pope St. Pius I}
\feastday{{Pope St. Pius I}}
\fancyhead[RE,LO]{11 July}
\begin{inhead}
    {Memorial\\
11 July}
\end{inhead}

\begin{rubric}
	The propers are from the First Common of a Martyr Bishop (p. \pageref{CommonMartyrBishopI}).
\end{rubric}


\subby{Sts. Nabor \& Felix}
\feastday{{Sts. Nabor \& Felix}}
\fancyhead[RE,LO]{12 July}
\begin{inhead}
    {Memorial\\
12 July}
\end{inhead}

\begin{rubric}
	The propers come from the Third Common of Many Martyrs (p. \pageref{CommonMartyrsIII}), except for the following.
\end{rubric}

\collect
\lett{G}{rant,} we beseech thee, O Lord: that as the birthday of thy holy Martyrs, Nabor and Felix, faileth not to return for our observance; so they may continually assist us by their prayers. Through.

\secret
\lett{W}{e} beseech thee, O Lord, that the gifts of thy people may, by the prayers of thy holy Martyrs, Nabor and Felix, be made acceptable unto thee: that, as they are offered for their triumph to thy name, so by their merits they may be rendered worthy in thy sight. Through.

\postcommunion
\lett{O}{Lord,} who on the birthday of thy Saints hast quickened us with the gift of thy sacrament: we beseech thee; that, as by thy grace we now receive the comfort of thy blessings, so we may be brought to the everlasting fruition of the same. Through.


\subby{St. Anacletus}
\feastday{{St. Anacletus}}
\fancyhead[RE,LO]{13 July}
\begin{inhead}
    {Memorial\\
13 July}
\end{inhead}

\begin{rubric}
	The propers are from the Second Common of a Martyr Bishop (p. \pageref{CommonMartyrBishopII}), with the Gospel from the First Common of a Martyr Bishop (p. \pageref{CommonMartyrBishopI}).
\end{rubric}


\subby{St. Vladimir of Kiev}
\feastday{{St. Vladimir}}
\fancyhead[RE,LO]{15 July}
\begin{inhead}
    {Double\\
15 July}
\end{inhead}

\begin{rubric}
	The propers are from the First Common of a Confessor not a Bishop (p. \pageref{CommonConfessorNotBishopI}).
\end{rubric}


\subby{Sts. Nicholas \& Habib Khasha}
\feastday{Sts. Nicholas \& Habib Khasha}
\fancyhead[RE,LO]{16 July}
\begin{inhead}
    {Memorial\\
16 July}
\end{inhead}

\begin{rubric}
	The propers are from the Second Common of Many Martyrs (p. \pageref{CommonMartyrsII}).
\end{rubric}


%\subby{Translation of St. Swithun}
%\feastday{{St. Swithun}}
%\fancyhead[RE,LO]{15 July}
%\begin{inhead}
%    {Memorial\\
%15 July}
%\end{inhead}


%\subby{Our Lady of Einsiedeln}
%\feastday{{Einsiedeln}}
%\fancyhead[RE,LO]{16 July}
%\begin{inhead}
%    {Greater Double\\
%16 July}
%\end{inhead}

\subby{St. Alexius}
\feastday{{St. Alexius}}
\fancyhead[RE,LO]{17 July}
\begin{inhead}
    {Memorial\\
17 July}
\end{inhead}

\begin{rubric}
	The propers are from the First Common of a Confessor not a Bishop (p. \pageref{CommonConfessorNotBishopI}), except for the following.
\end{rubric}

\readingcitation{Epistle}{1 Timothy 6:6}
\lett{D}{early beloved:} Godliness with contentment is great gain. For we brought nothing into this world, and it is certain we can carry nothing out. And having food and raiment let us be therewith content. But they that will be rich fall into temptation and a snare, and into many foolish and hurtful lusts, which drown men in destruction and perdition. For the love of money is the root of all evil: which while some coveted after, they have erred from the faith, and pierced themselves through with many sorrows. But thou, O man of God, flee these things; and follow after righteousness, godliness, faith, love, patience, meekness. Fight the good fight of faith, lay hold on eternal life.

\readingcitation{Gospel}{Matthew 19:27}
\lett{A}{t that time:} Peter said unto Jesus: Behold, we have forsaken all, and followed thee; what shall we have therefore? And Jesus said unto them, Verily I say unto you, That ye which have followed me, in the regeneration when the Son of man shall sit in the throne of his glory, ye also shall sit upon twelve thrones, judging the twelve tribes of Israel. And every one that hath forsaken houses, or brethren, or sisters, or father, or mother, or wife, or children, or lands, for my name's sake, shall receive an hundredfold, and shall inherit everlasting life.


\subby{Translation of St. Raphael of Brooklyn}
\feastday{{St. Raphael Brooklyn Translation}}
\fancyhead[RE,LO]{18 July}
\begin{inhead}
    {Double\\
18 July}
\end{inhead}

\begin{rubric}
	The propers are from the First Common of a Confessor Bishop (p. \pageref{CommonConfessorBishopI}).
\end{rubric}


\subby{St. Sergius of Radonezh}
\feastday{{St. Sergius Radonezh}}
\fancyhead[RE,LO]{18 July}
\begin{inhead}
    {Memorial\\
18 July}
\end{inhead}

\begin{rubric}
	The propers are from the Common of Abbots (p. \pageref{CommonAbbots}).
\end{rubric}


%The rubrics from the English Missal are confusing here, so I use the Anglican Missal for rubrics and translation of the Prayers.
\subby{Sts. Symphorosa \& Her Seven Sons}
\feastday{{Sts. Symphorosa \&c.}}
\fancyhead[RE,LO]{18 July}
\begin{inhead}
    {Memorial\\
18 July}
\end{inhead}

\introit
\lett{T}{he} just cry, and the Lord heareth them: and delivereth them out of all their troubles. \textit{Ps.} I will alway give thanks unto the Lord; his praise shall ever be in my mouth.

\collect
\lett{O}{God,} who vouchsafest unto us to keep the heavenly birthday of thy holy Martyrs Symphorosa and her sons: grant, we beseech thee; that we may rejoice in the perpetual felicity of their friendship. Through.

\begin{rubric}
	The Epistle is the fifth additional Epistle of the Third Common of Many Martyrs (p. \pageref{Hebrews1133}).
\end{rubric}

\gradall{Behold, how good and joyful a thing it is, brethren, to dwell together in unity! ℣. It is like the precious ointment upon the head, that ran down unto the beard, even unto Aaron's beard.}{Alleluia, alleluia. ℣. This is the true brotherhood which overcame the wickedness of the world: which followed Christ, gaining heaven's glorious realms. Alleluia.}

\begin{rubric}
	The Gospel is from the Third Common of Many Martyrs (p. \pageref{CommonMartyrsIII}).
\end{rubric}

\offertory{Be glad, O ye righteous, and rejoice in the Lord: and be joyful, all ye that are true of heart.}

\secret
\lett{W}{e} beseech thee, O Lord, that the gifts which we offer unto thee of our bounden duty and service may be acceptable unto thee for the honour of thy Saints: and by thy mercy profitable unto us for our salvation. Through.

\begin{rubric}
	In Lent, Commemoration of the Feria.
\end{rubric}

\communion{Whosoever shall do the will of my Father which is in heaven: the same is my brother, and sister, and mother, saith the Lord.}

\postcommunion
\lett{G}{rant,} we beseech thee, almighty God: that, at the intercession of thy holy Martyrs Symphorosa and her sons, we, who with our outward lips have partaken of this Sacrament, may inwardly receive the same in purity of heart. Through.


\subby{St. Seraphim of Sarov}
\feastday{{St. Seraphim}}
\fancyhead[RE,LO]{19 July}
\begin{inhead}
    {Memorial\\
19 July}
\end{inhead}

\begin{rubric}
	The propers are from the First Common of a Confessor not a Bishop (p. \pageref{CommonConfessorNotBishopI}).
\end{rubric}


\subby{St. Elias the Prophet}
\feastday{{St. Elias}}
\fancyhead[RE,LO]{20 July}
\begin{inhead}
    {Double\\
20 July}
\end{inhead}\par\noindent
%Elijah changed to Elias \& Elisha to Eliseus.
%From the Carmelite Breviary and American Missal:
%Also from the `Saints of Mount Carmel' (https://archive.org/details/saintsofmountcar0000john/page/252/mode/2up):
%%From the Carmelite Breviary (1930s), translation from the KJV:
\textit{Opening Sentence.} Then stood up Elias the prophet as fire, and his word burned like a lamp, alleluia.%\vr{Ecclus. 48:1}\par

\begin{paracol}{2}[]
\sloppy
\begin{inhead}
	I Evensong
\end{inhead}
\begin{hangparas}{1.25em}{1}
The lofty peaks of Carmel

With tuneful praises ring,

The anthems of Elias

'Tis our delight to sing.\\

%Order changed to Churches:
The glory of our Churches,

Our leader, prop, and stay,

From east to west his offspring

Increaseth day by day.\\

When sorely press'd with famine,

A raven serv'd him bread,

With meal and cruse unfailing,

The widow'd hearth was fed.\\

The boy from death delivered

Is to his home restor'd,

And light so much desir'd,

In radiant flood is pour'd.\\

Behold the Heaven closeth,

To open at his voice,

And copious welcome showers

The thirsty lands rejoice.\\

To Father, Son, and Spirit,

Be equal power and praise,

All glory and dominion

Henceforth for endless days. Amen.\\
\end{hangparas}

    ℣. By the word of the Lord he shut up the heaven.

	℟. And he brought down fire from heaven thrice.

\antiphon{Mag.}{Behold, I {\dag}  will send you Elias the prophet before the coming of the great and dreadful day of the Lord: And he shall turn the heart of the fathers to the children, and the heart of the children to their fathers}\par\noindent

\switchcolumn

\begin{inhead}
	Mattins Invitatory Hymn
\end{inhead}
\begin{hangparas}{1.25em}{1}
Creator of the universe, our grateful hearts rejoice,

Exalting thee in this, the mighty Thesbite of thy choice;\\

Whose zeal for thy great glory, enkindled as a flame,

Defied the impious Prophets, and slew them in thy name.\\

His holocaust ascendeth, for thy fire may never fail,

While scorn and deep derision mock the clamorous priests of Baal.\\

The baneful rage of Jezabel he flieth in his dread,

And sleeping `neath a juniper is by an angel fed.\\

Empowered by the vision, with new courage he hath trod

Unto the heights of Horeb, unto the Mount of God.\\

O virtue of this bread divine, imparted by the Lord!

Full forty days he fasteth in the strength it doth afford.\\

All glory be to thee, O Father, Son, and Paraclete,

O undivided Trinity, thy praise all hearts repeat. Amen.
\end{hangparas}

\fussy
\end{paracol}


\begin{paracol}{2}[]
\sloppy

\begin{inhead}
	Mattins Office Hymn
\end{inhead}
\begin{hangparas}{1.25em}{1}
Come, blest companions, let our joy resounding

Extol to heav'n the leader of our line.

`Tis meet the mem'ry of his deeds abounding

Should waken ceaseless canticles divine.\\

He knows the gentle breathing of the Spirit

Cloth'd in the whistling murmur of the air,

By God's command the chastisements they merit

Proud Jezebel and Ahab justly share.\\

The caverns green of Carmel form his dwelling,

With leathern tunic is he rudely clad,

To impious Ahaziah his foretelling

Gives portent of a dissolution sad.\\

Twice at his pray'r the fire from Heav'n descending

Consumeth trembling soldiers in its flame,

The flowing wat'rs met with his mantle rending,

Dry shod he passeth safely through the same.\\

O Father, let thy help and thy protection

Be o'er thy children as they humbly plead,

Entreat the Spirit, by his sweet election,

To multiply his graces in their need.\\

O unbegotten Father, we adore thee,

O Son begotten, rev'rence be to thee,

O glorious Spirit, bow we low before thee,

Thou simple undivided Trinity. Amen.\\
\end{hangparas}

    ℣. Elias was covered with the whirlwind.

	℟. And his spirit was filled up in Eliseus.

\antiphon{Ben.}{Elias was a man {\dag} subject to like passions as we are, and he prayed earnestly that it might not rain: and it rained not on the earth by the space of three years and six months. And he prayed again, and the heaven gave rain, and the earth brought forth her fruit.}\par\noindent

\switchcolumn

\begin{inhead}
	II Evensong
\end{inhead}
\begin{hangparas}{1.25em}{1}
%Order changed to Churches
Thou prop of our Churches, thou pride of our race,

Let thy praises resound far and near,

Let sea and the land and the air give them place,

Rehearsing in gladness thy glory and grace,

Till the earth and the heav'ns give ear.\\

O sun of the heavens, how lovely thy rays!

What power thy wonders unfold,

How fruitful in merits the length of thy days,

Commission'd by God in His manifold ways,

For noble endeavours of old!\\

To regions celestial, in power and might,

Triumphant thy chariot speeds;

Uplifted by angels to marvellous height,

While shining in splendour and dazzling with light,

Thou guidest the fiery steeds.\\

As witness to men of his Sonship divine,

With Jesus thy glory we view;

The Father hath call'd thee on Tabor to shine,

Companion to Moses, and with him to sign

A testament faithful and true.\\

Protect us we pray, ‘neath thy powerful shield,

Incline to our aid from above,

Let thy fost'ring guidance be ever reveal'd

To thy children of Carmel, whose bosoms are seal'd

With the strength of thy fath'rly love\\

All power, dominion, all glory and praise,

Be given to Father and Son,

To thee, Holy Spirit, for numberless days

Our homage eternal we equally raise

All glory to God, Three in One. Amen.\\
\end{hangparas}

    ℣. Blessed are they that saw thee.

	℟. And were honoured with thy friendship.
	
%Based on KJV with changes based off the Latin.
\antiphon{Mag.}{And Elias took {\dag} his mantle, and wrapped it together, and smote the waters of the Jordan, and they were divided hither and thither, so that he and Eliseus went over on dry ground. As they still went on, behold, there appeared a chariot of fire, and horses of fire, and parted them both asunder; and Elias went up by a whirlwind into heaven, and Eliseus saw him no more.}

\fussy
\end{paracol}

%Tulit Elías {\dag} pállium suum, et invólvit illud, et percússit aquas jordanis, qua divisa sunt in utrámque partem, et transiérunt ipse et Eliséus per siccum. Qui incedentes, ecce, currus igneus et equi igner diviserunt utrúmque : et ascéndit Elías per túrbinem in c{\ae}lum, Eliséus autem non vidit eum ámplius.

\introit
\lett{I}{have} been very zealous for the Lord God of hosts: for the children of Israel have forsaken thy covenant, thrown down thine altars, and slain thy prophets with the sword; and I, even I only, am left; and they seek my life, to take it away. \textit{Ps.} I will magnify thee, O Lord, for thou hast set me up: and not made my foes to triumph over me.
\collect
\lett{G}{rant,} we beech thee, Almighty God: that we who believe that thou didst marvellously lift up thy Prophet Elias in a fiery chariot, while yet in this life; may at his intercession, while still alive, be raised to spiritual heights and rejoice in the resurrection of the just. Through.
\begin{rubric}
	Commemoration of St. Margaret of Antioch, from the Second Common of a Virgin Martyr (p. \pageref{CommonVirginMartyrII}).
\end{rubric}

\readingcitation{Epistle}{Ecclesiasticus 48:1}
%RV:
\lett{T}{here} arose Elijah the prophet as fire, And his word burned like a torch:  Who brought a famine upon them, And by his zeal made them few in number. By the word of the Lord he shut up the heaven: Thrice did he thus bring down fire. How wast thou glorified, O Elijah, in thy wondrous deeds! And who shall glory like unto thee? Who did raise up a dead man from death, And from the place of the dead, by the word of the Most High: Who brought down kings to destruction, And honourable men from their bed: Who heard rebuke in Sinai, And judgements of vengeance in Horeb: Who anointed kings for retribution, And prophets to succeed after him: Who was taken up in a tempest of fire, In a chariot of fiery horses: Who was recorded for reproofs in their seasons, To pacify anger, before it brake forth into wrath; To turn the heart of the father unto the son, And to restore the tribes of Jacob. 

%\lett{T}{hen} stood up Elias the prophet as fire, and his word burned like a lamp. He brought a sore famine upon them, and by his zeal he diminished their number. By the word of the Lord he shut up the heaven, and also three times brought down fire. O Elias, how wast thou honoured in thy wondrous deeds! and who may glory like unto thee! Who didst raise up a dead man from death, and his soul from the place of the dead, by the word of the most High: Who broughtest kings to destruction, and honorable men from their bed: Who heardest the rebuke of the Lord in Sinai, and in Horeb the judgment of vengeance: Who annointedst kings to take revenge, and prophets to succeed after him: Who was taken up in a whirlwind of fire, and in a chariot of fiery horses: Who wast ordained for reproofs in their times, to pacify the wrath of the Lord's judgment, before it brake forth into fury, and to turn the heart of the father unto the son, and to restore the tribes of Jacob.

\gradall{It came to pass, when the Lord would take up Elias into heaven by a whirlwind, that Elias went with Eliseus from Gilgal. ℣. And it came to pass, as they still went on, that, behold, there appeared a chariot of fire, and horses of fire, and parted them both asunder; and Elias went up by a whirlwind into heaven.}{Alleluia, alleluia. ℣. Elias, while he was full of zeal for the law, was taken up into heaven. Alleluia.}

\readingcitation{Gospel}{Luke 9:28}
\lett{A}{t that time:} It came to pass about an eight days after these sayings, Jesus took Peter and John and James, and went up into a mountain to pray. And as he prayed, the fashion of his countenance was altered, and his raiment was white and glistering. And, behold, there talked with him two men, which were Moses and Elias: Who appeared in glory, and spake of his decease which he should accomplish at Jerusalem. But Peter and they that were with him were heavy with sleep: and when they were awake, they saw his glory, and the two men that stood with him. And it came to pass, as they departed from him, Peter said unto Jesus, Master, it is good for us to be here: and let us make three tabernacles; one for thee, and one for Moses, and one for Elias: not knowing what he said. While he thus spake, there came a cloud, and overshadowed them: and they feared as they entered into the cloud. And there came a voice out of the cloud, saying, This is my beloved Son: hear him. And when the voice was past, Jesus was found alone. And they kept it close, and told no man in those days any of those things which they had seen.
\offertory{Elias was a man subject to like passions as we are: and he prayed earnestly that it might not rain, and it rained not on the earth for the space of three years and six months: and he prayed again, and the heaven gave rain, and the earth brought forth her fruit.}

\secret
\lett{W}{e} offer unto thee, O Lord, this sacrifice of praise in honour of thy Prophet Elias: that, as thou didst accept his burnt-offering, so thou wouldest vouchsafe to accept our own sacrifice; that through it we may be made worthy to attain unto everlasting gladness. Through.
\begin{rubric}
	Commemoration of St. Margaret of Antioch, from the Second Common of a Virgin Martyr (p. \pageref{CommonVirginMartyrII}).
\end{rubric}

\subbysub{Preface}
\lett{A}{nd} that we, with glad hearts, should praise, bless, and glorify thee in the Solemnity of blessed Elias thy Prophet: Who at thy word stood up like fire; shut the heavens and raised the dead; smote tyrants and slew the blasphemous; and laid the foundation of the monastic profession: Who, being fed with meat and drink served by Angels, walked in the strength of that food, even to the holy mountain: Who, being raised up from earth in a fiery chariot, will return to us as the Forerunner of the second Coming of Jesus Christ our Lord. Therefore.

\communion{Behold, I will send you Elias the Prophet before the coming of the great and terrible day of the Lord: and he shall turn the hearts of the fathers to the children, and the hearts of the children to their fathers.}

\postcommunion
\lett{O}{God,} who by thy holy Angel didst give meat and drink to blessed Elias thy Prophet: grant, by his intercession, that what we have received of thy heavenly table, we may keep undefiled in purity of mind. Through.
\begin{rubric}
	Commemoration of St. Margaret of Antioch, from the Second Common of a Virgin Martyr (p. \pageref{CommonVirginMartyrII}).
\end{rubric}


\subby{St. Margaret of Antioch}
\feastday{{St. Margaret Antioch}}
\fancyhead[RE,LO]{20 July}
\begin{inhead}
    {Memorial\\
20 July}
\end{inhead}

\begin{rubric}
	The propers are from the Second Common of a Virgin Martyr (p. \pageref{CommonVirginMartyrII}).
\end{rubric}


\subby{St. Praxedes}
\feastday{{St. Praxedes}}
\fancyhead[RE,LO]{21 July}
\begin{inhead}
    {Memorial\\
21 July}
\end{inhead}

\begin{rubric}
	The Prayers and Epistle are from the Second Common of a Virgin (p. \pageref{CommonVirginOnlyII}), with the following.
\end{rubric}

\introit
\lett{I}{will} speak of thy testimonies even before kings, and will not be ashamed: and my delight shall be in thy commandments, which I have loved exceedingly. \textit{Ps.} Blessed are those that are undefiled in the way: and walk in the law of the Lord.

\gradall{Thou hast loved righteousness, and hated iniquity. ℣. Wherefore God, even thy God, hath anointed thee with the oil of gladness.}{Alleluia, alleluia. ℣. In thy comeliness and in thy beauty, go forth, proceed prosperously, and reign. Alleluia.}

\readingcitation{Gospel}{Matthew 13:44}
\lett{A}{t that time:} Jesus spake this parable unto his disciples: The kingdom of heaven is like unto treasure hid in a field; the which when a man hath found, he hideth, and for joy thereof goeth and selleth all that he hath, and buyeth that field. Again, the kingdom of heaven is like unto a merchant man, seeking goodly pearls: Who, when he had found one pearl of great price, went and sold all that he had, and bought it. Again, the kingdom of heaven is like unto a net, that was cast into the sea, and gathered of every kind: Which, when it was full, they drew to shore, and sat down, and gathered the good into vessels, but cast the bad away. So shall it be at the end of the world: the angels shall come forth, and sever the wicked from among the just, And shall cast them into the furnace of fire: there shall be wailing and gnashing of teeth. Jesus saith unto them, Have ye understood all these things? They say unto him, Yea, Lord. Then said he unto them, Therefore every scribe which is instructed unto the kingdom of heaven is like unto a man that is an householder, which bringeth forth out of his treasure things new and old.

\offertory{Full of grace are thy lips, because God hath blessed thee for ever and ever.}

\communion{The kingdom of heaven is like unto a merchant man, seeking goodly pearls: who when he had found one pearl of great price, gave all that he had, and bought it.}


\bcpfeast{22 July. St. Mary Magdalene, Penitent}{St. Mary Magdalene}{22 July}
%\supplement{22 July}{St. Mary Magdalene}{}

\subbysub{I Evensong}\label{MaryMagdaleneEvensong}

\gregorioscore{resources/gabc/ProperTime/MaryMagdaleneEvensong.gabc}

    ℣. Mary hath chosen that good part.

	℟. Which shall not be taken away from her.

	\properantiphon{Mag.}{A woman {\dag} in the city which was a sinner, when she knew that Jesus sat at meat in the house of Simon the leper, brought an alabaster box of ointment, and stood behind at the feet of Jesus, and began to wash his feet with tears, and did wipe them with the hairs of her head, and kissed his feet, and anointed them with the ointment.}

%MANUAL ADJUSTMENT:
\clearpage
\subbysub{Mattins}

\invitatoryhymn\label{MaryMagdaleneInvitatory}

\gregorioscore{resources/gabc/ProperTime/MaryMagdaleneInvitatory.gabc}

\officehymn\label{MaryMagdaleneMattins}

\gregorioscore{resources/gabc/ProperTime/MaryMagdaleneMattins.gabc}

    ℣. Her sins, which are many, are forgiven.

	℟. For she loved much.
	
	\properantiphon{Ben.}{Mary therefore {\dag} anointed the feet of Jesus, and wiped them with her hair, and the house was filled with the odour of the ointment.}

\subbysub{II Evensong}

\begin{rubric}
	The Office Hymn is of I Evensong with the following Versicle \& Antiphon.
\end{rubric}

    ℣. God hath chosen her and preferred her.

	℟. He hath made her to dwell in his tabernacle.
	
	\properantiphon{Mag.}{A woman {\dag} in the city which was a sinner, brought an alabaster box of ointment, and stood at the Lord's feet, and began to wash his feet with tears, and did wipe them with the hairs of her head.}


\subby{St. John Cassian}
\feastday{{St. John}}
\fancyhead[RE,LO]{23 July}
\begin{inhead}
    {Memorial\\
23 July}
\end{inhead}

\begin{rubric}
	The propers are from the Common of Abbots (p. \pageref{CommonAbbots}).
\end{rubric}


\subby{St. Apollinaris of Ravenna}
\feastday{{St. Apollinaris Ravenna}}
\fancyhead[RE,LO]{23 July}
\begin{inhead}
    {Memorial\\
23 July}
\end{inhead}

\introit
\lett{O}{ye} priests of the Lord, bless ye the Lord: O ye holy and humble men of heart, bless ye the Lord. \textit{Cant.} O all ye works of the Lord, bless ye the Lord: praise him and magnify him for ever. 

\collect
\lett{O}{God,} the rewarder of faithful souls, who hast hallowed this day by the martyrdom of blessed Apollinaris, thy Priest: grant, we beseech thee, unto us thy servants; that we who keep his solemn festival may by his prayers obtain thy pardon. Through.

\readingcitation{Epistle}{1 Peter 5:1}
\lett{D}{early beloved:} The elders which are among you I exhort, who am also an elder, and a witness of the sufferings of Christ, and also a partaker of the glory that shall be revealed: Feed the flock of God which is among you, taking the oversight thereof, not by constraint, but willingly; not for filthy lucre, but of a ready mind; Neither as being lords over God's heritage, but being ensamples to the flock. And when the chief Shepherd shall appear, ye shall receive a crown of glory that fadeth not away. Likewise, ye younger, submit yourselves unto the elder. Yea, all of you be subject one to another, and be clothed with humility: for God resisteth the proud, and giveth grace to the humble. Humble yourselves therefore under the mighty hand of God, that he may exalt you in due time: Casting all your care upon him; for he careth for you. Be sober, be vigilant; because your adversary the devil, as a roaring lion, walketh about, seeking whom he may devour: Whom resist stedfast in the faith, knowing that the same afflictions are accomplished in your brethren that are in the world. But the God of all grace, who hath called us unto his eternal glory by Christ Jesus, after that ye have suffered a while, make you perfect, stablish, strengthen, settle you. To him be glory and dominion for ever and ever. Amen.

\gradall{I have found David my servant, with my holy oil have I anointed him: my hand shall hold him fast, and my arm shall strengthen him. ℣. The enemy shall not be able to do him violence, the son of wickedness shall not hurt him.}{Alleluia, alleluia. ℣. The Lord sware, and will not repent: Thou art a priest for ever, after the order of Melchisedech. Alleluia.}

\readingcitation{Gospel}{Luke 22:24}
\lett{A}{t that time:} There was a strife among the disciples, which of them should be accounted the greatest. And he said unto them, The kings of the Gentiles exercise lordship over them; and they that exercise authority upon them are called benefactors. But ye shall not be so: but he that is greatest among you, let him be as the younger; and he that is chief, as he that doth serve. For whether is greater, he that sitteth at meat, or he that serveth? is not he that sitteth at meat? but I am among you as he that serveth. Ye are they which have continued with me in my temptations. And I appoint unto you a kingdom, as my Father hath appointed unto me; That ye may eat and drink at my table in my kingdom, and sit on thrones judging the twelve tribes of Israel.

\offertory{My truth and my mercy shall be with him: and in my name shall his horn be exalted.}

\secret
\lett{G}{raciously} look, O Lord, upon these gifts: which in remembrance of blessed Apollinaris, thy Priest and Martyr, we lay before thee, and offer up for our offences. Through.

\communion{Lord, thou deliveredst unto me five talents, behold, I have gained beside them five talents more. Well done, thou good and faithful servant, thou hast been faithful over a few things, I will make thee ruler over many things, enter thou into the joy of thy Lord.}

\postcommunion
\lett{W}{e} who have received thy holy things, beseech thee, O Lord, that the protection of blessed Apollinaris may continually defend us: forasmuch as thou failest not to look with mercy on those to whom thou dost grant the succour of his assistance. Through.


\subby{St. Liborius}
\feastday{{St. Liborius}}
\fancyhead[RE,LO]{23 July}
\begin{inhead}
    {Memorial\\
23 July}
\end{inhead}

\begin{rubric}
	The propers are from the First Common of a Confessor Bishop (p. \pageref{CommonConfessorBishopI}).
\end{rubric}


\subby{Vigil of St. James}
\feastday{{St. James Vigil}}
\fancyhead[RE,LO]{23 July}
\begin{inhead}
    {Vigil\\
24 July}
\end{inhead}

\begin{rubric}
	The propers are from the Common of Vigils of the Apostles (p. \pageref{CommonVigilApostles}), with commemoration of St. Christina from the Second Common of a Virgin Martyr (p. \pageref{CommonVirginMartyrII}), and the \nth{3} Collect of St. Mary in Eastertide (p. \SPMaryEaster).
\end{rubric}


\subby{St. Christina}
\feastday{{St. Christina}}
\fancyhead[RE,LO]{24 July}
\begin{inhead}
    {Memorial\\
24 July}
\end{inhead}


\bcpfeast{25 July. St. James}{St. James}{25 July}
%\supplement{25 July}{St. James}{}

\begin{secrubric}
	The Daily Office propers are of the Common of Apostles (p. \pageref{CommonApostles}).
\end{secrubric}


\subby{St. Christopher}
\feastday{{St. Christopher}}
\fancyhead[RE,LO]{25 July}
\begin{inhead}
    {Memorial\\
25 July}
\end{inhead}

\begin{rubric}
	The propers are from the First Common of a Martyr not a Bishop (p. \pageref{CommonMartyrNotBishopI}).
\end{rubric}


\bcpfeast{26 July. St. Anne}{St. Anne}{26 July}
%\supplement{26 July}{St. Anne}{}

\subbysub{I Evensong}\label{AnneEvensong}

\gregorioscore{resources/gabc/ProperTime/AnneEvensong.gabc}

    ℣. Be glad, O ye righteous and rejoice in the Lord.
    
	℟. And be joyful, all ye that are true of heart.

\properantiphon{Mag.}{The kingdom of heaven {\dag} is likened unto a merchantman seeking goodly pearls; who, when he had found one pearl of great price, went and sold all that he had, and bought it.}

\subbysub{Mattins}

\begin{rubric}
	The Invitatory Hymn is as in I Evensong.
\end{rubric}

\officehymn\label{AnneMattins}

\gregorioscore{resources/gabc/ProperTime/AnneMattins.gabc}

    ℣. Let the Saints be joyful with glory.
    
	℟. Let them rejoice in their beds.
	
\properantiphon{Ben.}{Give her {\dag} of the fruit of her hands; and let her own works praise her in the gates.}

\subbysub{II Evensong}\label{AnneEvensongII}

\gregorioscore{resources/gabc/ProperTime/AnneEvensongII.gabc}

    ℣. Let the Saints be joyful with glory.
    
	℟. Let them rejoice in their beds.

\properantiphon{Mag.}{She stretched out {\dag} her hand to the poor; yea, she reacheth forth her hands to the needy; she eateth not the bread of idleness.}


\subby{St. Pantaleon}
\feastday{{St. Pantaleon}}
\fancyhead[RE,LO]{27 July}
\begin{inhead}
    {Memorial\\
27 July}
\end{inhead}

\begin{rubric}
	The propers are from the Second Common of a Martyr not a Bishop (p. \pageref{CommonMartyrNotBishopII}).
\end{rubric}


\subby{Sts. Nazarius, Celsus, Victor, \& Innocent}
\feastday{{Sts. Nazarius \&c.}}
\fancyhead[RE,LO]{28 July}
\begin{inhead}
    {Memorial\\
28 July}
\end{inhead}

\begin{rubric}
	The propers are from the First Common of Many Martyrs (p. \pageref{CommonMartyrsI}), except for the following.
\end{rubric}

\collect
\lett{O}{Lord,} let the blessed confession of thy Saints, Nazarius, Celsus, Victor, and Innocent, strengthen us: and obtain for our frailty the succour of thy goodness. Through.

\begin{rubric}
	The Epistle is the first additional Epistle of the Third Common of Many Martyrs (p. \pageref{Wisdom1017}).
\end{rubric}

\secret
\lett{G}{rant} to us, almighty God: that we, who present these gifts unto thee to the honour of thy Saints, Nazarius, Celsus, Victor, and Innocent, may propitiate thee by the offering, and be quickened by the receiving of the same. Through.

\postcommunion
\lett{G}{rant,} O Lord, we beseech thee, that the intercession of thy Saints Nazarius, Celsus, Victor, and Innocent, may so make us acceptable unto thee: that those things which we perform in this temporal celebration we may receive unto eternal salvation. Through.

%MANUAL ADJUSTMENT:
\clearpage
\subby{St. Martha of Bethany}
\feastday{{St. Martha}}
\fancyhead[RE,LO]{29 July}
\begin{inhead}
    {Double\\
29 July}
\end{inhead}

\begin{rubric}
	The propers are from the First Common of a Virgin (p. \pageref{CommonVirginOnlyI}), except for the following, with a Commemoration of St. Felix etc.
\end{rubric}

\readingcitation{Gospel}{Luke 10:38}
\lett{A}{t that time:} Jesus entered into a certain village: and a certain woman named Martha received him into her house. And she had a sister called Mary, which also sat at Jesus' feet, and heard his word. But Martha was cumbered about much serving, and came to him, and said, Lord, dost thou not care that my sister hath left me to serve alone? bid her therefore that she help me. And Jesus answered and said unto her, Martha, Martha, thou art careful and troubled about many things: But one thing is needful: and Mary hath chosen that good part, which shall not be taken away from her.


\subby{Sts. Felix II, Simplicius, Faustinus, \& Beatrice}
\feastday{{St. Felix II \&c.}}
\fancyhead[RE,LO]{29 July}
\begin{inhead}
    {Memorial\\
29 July}
\end{inhead}

\begin{rubric}
	The propers are from the Second Common of Many Martyrs (p. \pageref{CommonMartyrsII}), except for the following.
\end{rubric}

\collect
\lett{G}{rant,} we beseech thee, O Lord: that as thy Christian people rejoice in the temporal solemnity of thy Martyrs, Felix, Simplicius, Faustinus, and Beatrice, so they may eternally enjoy the same; that those things which they devoutly celebrate, they may effectually obtain. Through.

\secret
\lett{W}{e} offer our sacrifices unto thee, O Lord, in commemoration of thy Holy Martyrs Felix, Simplicius, Faustinus, and Beatrice: humbly entreating thee; that they may bestow on us both pardon and salvation. Through.

\postcommunion
\lett{G}{rant,} we beseech thee, almighty God: that the solemnity of thy holy Martyrs Felix, Simplicius, Faustinus, and Beatrice: which we have celebrated in these heavenly mysteries, may obtain for us merciful pardon. Through.


\subby{Sts. Abdon \& Sennen}
\feastday{{Sts. Abdon \& Sennen}}
\fancyhead[RE,LO]{30 July}
\begin{inhead}
    {Memorial\\
30 July}
\end{inhead}

\introit
\lett{L}{et} the sorrowful sighing of the prisoners, O Lord, come before thee: reward thou our neighbours sevenfold in their bosom: avenge thou the blood of thy Saints that is shed. \textit{Ps.} O God, the heathen are come into thine inheritance: thy holy temple have they defiled: and made Jerusalem an heap of stones.

\collect
\lett{O}{God,} who on thy Saints Abdon and Sennen didst bestow the bounteous gift of grace to attain unto thy glory: grant unto thy servants the remission of their sins; that, by the intercession of the merits of thy Saints, they may be found worthy to be delivered from all adversities. Through.

\begin{rubric}
	The Epistle is the fourth additional Epistle of the Third Common of Many Martyrs (p. \pageref{2Corinthians64}).
\end{rubric}

\gradall{God is glorious in his holy ones, fearful in praises, doing wonders. ℣. Thy right hand, O Lord, is become glorious in power: thy right hand hath dashed in pieces the enemy.}{Alleluia, alleluia. ℣. The souls of the just are in the hand of God, and there shall no torment of malice touch them. Alleluia.}

\readingcitation{Gospel}{Matthew 5:1}
\lett{A}{t that time:} Jesus, seeing the multitudes, went up into a mountain, and when he was set, his disciples came unto him: And he opened his mouth, and taught them, saying, Blessed are the poor in spirit: for theirs is the kingdom of heaven. Blessed are they that mourn: for they shall be comforted. Blessed are the meek: for they shall inherit the earth. Blessed are they which do hunger and thirst after righteousness: for they shall be filled. Blessed are the merciful: for they shall obtain mercy. Blessed are the pure in heart: for they shall see God. Blessed are the peacemakers: for they shall be called the children of God. Blessed are they which are persecuted for righteousness' sake: for theirs is the kingdom of heaven. Blessed are ye, when men shall revile you, and persecute you, and shall say all manner of evil against you falsely, for my sake. Rejoice, and be exceeding glad: for great is your reward in heaven: 

\offertory{God is wonderful in his holy ones: even the God of Israel, he will give strength and power unto his people: blessed be God.}

\secret
\lett{W}{e} beseech thee, O Lord, that the sacrifice, which we offer in remembrance of the birthday of thy holy Martyrs, may both loose the bonds of our iniquity, and obtain for us the gifts of thy mercy. Through.

\communion{The dead bodies of thy servants, O Lord, have they given to be meat unto the fowls of the air, and the flesh of thy Saints unto the beasts of the land: according to the greatness of thy power preserve thou those that are appointed to die.}

\postcommunion
\lett{M}{ay} the operation of this mystery, O Lord, both cleanse our sins: and, at the intercession of thy holy Martyrs Abdon and Sennen, obtain the fulfilment of our rightful desires. Through.


\bcpfeast{1 August. Chains of St. Peter}{Lammas}{1 August}
%\supplement{1 August}{Chains of St. Peter}{}

\subbysub{I Evensong}\label{LammasEvensong}

\gregorioscore{resources/gabc/ProperTime/LammasEvensong.gabc}

    ℣. Thou art Peter.

	℟. And upon this rock I will build my Church.

\properantiphon{Mag.}{Thou art the shepherd of the sheep, {\dag} O chief of the Apostles: unto thee were given the keys of the kingdom of heaven.}

\subbysub{Mattins}

\invitatoryhymn\label{LammasInvitatory}

\gregorioscore{resources/gabc/ProperTime/LammasInvitatory.gabc}

\officehymn

\begin{rubric}
	The Office Hymn is from the Feast of the Chair of St. Peter at Antioch, 22 February (p. \pageref{PeterAntiochMattins}), with the following Versicle \& Antiphon.
\end{rubric}

    ℣. Thou art Peter.

	℟. And upon this rock I will build my Church.

\properantiphon{Ben.}{Whatsoever {\dag} thou shalt bind on earth shall be bound in heaven: and whatsoever thou shalt loose on earth shall be loosed in heaven: saith the Lord unto Simon Peter.}

\subbysub{II Evensong}

\begin{secrubric}
	The Office Hymn \& Versicle are as in I Evensong, with the following Antiphon.
\end{secrubric}

\properantiphon{Mag.}{Loosen these earthly fetters {\dag} at the command of God, O Peter; and let the kingdom of heaven be opened unto the blessed.}


\subby{Holy Maccabees}
\feastday{{Holy Maccabees}}
\fancyhead[RE,LO]{1 August}
\begin{inhead}
    {Memorial\\
1 August}
\end{inhead}

\introit
\lett{T}{he} just cry, and the Lord heareth them: and delivereth them out of all their troubles. \textit{Ps.} I will alway give thanks unto the Lord; his praise shall ever be in my mouth.

\collect
\lett{O}{Lord,} let the crown of the brethren, thy Martyrs, cause us to rejoice: that we may thereby be strengthened and increased in our faith; and comforted by their manifold intercession. Through.

\begin{rubric}
	The Epistle is the fifth additional Epistle of the Third Common of Many Martyrs (p. \pageref{Hebrews1133}).
\end{rubric}

\gradall{Behold, how good and joyful a thing it is, brethren, to dwell together in unity! ℣. It is like the precious ointment upon the head, that ran down unto the beard, even unto Aaron's beard.}{Alleluia, alleluia. ℣. This is the true brotherhood, which overcame the wickedness of the world: which followed Christ, gaining heaven's glorious realms. Alleluia.}

\begin{rubric}
	The Gospel is from the Third Common of Many Martyrs (p. \pageref{CommonMartyrsIII}).
\end{rubric}

\offertory{Let the Saints be joyful with glory: let them rejoice in their beds: let the praises of God be in their mouth, alleluia.}

\secret
\lett{G}{rant,} O Lord, that we may with devout hearts celebrate thy mysteries in honour of thy holy Martyrs: and thereby obtain an increase both of protection and joy. Through.

\communion{And I say unto you, my friends: Be not afraid of them that persecute you.}

\postcommunion
\lett{G}{rant,} we beseech thee, almighty God: that growing in virtue we may follow the faith of them whose memory we recall by the partaking of this sacrament. Through.


\subby{Pope St. Stephen}
\feastday{{Pope St. Stephen}}
\fancyhead[RE,LO]{2 August}
\begin{inhead}
    {Memorial\\
2 August}
\end{inhead}

\begin{rubric}
	The propers are from the Second Common for Confessor Bishops (p. \pageref{CommonConfessorBishopII}), except for that which followeth.
\end{rubric}

\introit
\lett{I}{will} deck her priests with health, and her saints shall rejoice and sing. \textit{Ps.} Lord, remember David: and all his trouble.

\readingcitation{Epistle}{Acts 20:17}
\lett{I}{n those days:} From Miletus Paul sent to Ephesus, and called the elders of the church. And when they were come to him, he said unto them, Ye know, from the first day that I came into Asia, after what manner I have been with you at all seasons, serving the Lord with all humility of mind, and with many tears, and temptations, which befell me by the lying in wait of the Jews: and how I kept back nothing that was profitable unto you, but have shewed you, and have taught you publickly, and from house to house, testifying both to the Jews, and also to the Greeks, repentance toward God, and faith toward our Lord Jesus Christ.

\gradall{Behold, a great priest, who in his days pleased God. ℣. There was none found like unto him, who kept the law of the Most High.}{Alleluia, alleluia. ℣. Thou art a priest for ever, after the order of Melchisedech. Alleluia.}

\communion{Lord, thou deliveredst unto me five talents: behold, I have gained beside them five talents more. Well done, thou good and faithful servant, thou hast been faithful over a few things. I will make thee ruler over many things, enter thou into the joy of thy Lord.}


\subby{The Invention of St. Stephen}
\feastday{{Invention of St. Stephen}}
\fancyhead[RE,LO]{3 August}
\begin{inhead}
    {Double\\
3 August}
\end{inhead}
\begin{rubric}
	The Office \& Mass as on 26 December, except for the following.
\end{rubric}

\begin{rubric}
	\textsc{Note,} The Creed is not said.
\end{rubric}

\collect
\lett{G}{rant} us, we beseech thee, O Lord, so to imitate that which we revere: that we may learn to love even our enemies; forasmuch as we celebrate the Finding of him, who was able to pray even for his persecutors to our Lord Jesus Christ thy Son. Who liveth.

\readingcitation{Epistle}{Acts 6:8}
\lett{I}{n those days:} Stephen, full of faith and power, did great wonders and miracles among the people. Then there arose certain of the synagogue, which is called the synagogue of the Libertines, and Cyrenians, and Alexandrians, and of them of Cilicia and of Asia, disputing with Stephen. And they were not able to resist the wisdom and the spirit by which he spake. When they heard these things, they were cut to the heart, and they gnashed on him with their teeth. But he, being full of the Holy Ghost, looked up stedfastly into heaven, and saw the glory of God, and Jesus standing on the right hand of God, and said, Behold, I see the heavens opened, and the Son of man standing on the right hand of God. Then they cried out with a loud voice, and stopped their ears, and ran upon him with one accord, and cast him out of the city, and stoned him: and the witnesses laid down their clothes at a young man’s feet, whose name was Saul. And they stoned Stephen, calling upon God, and saying, Lord Jesus, receive my spirit. And he kneeled down, and cried with a loud voice, Lord, lay not this sin to their charge. And when he had said this, he fell asleep in the Lord.


\subby{Dedication of Our Lady of the Snows}
\feastday{{Our Lady of the Snows}}
\fancyhead[RE,LO]{5 August}
\begin{inhead}
    {Greater Double\\
5 August}
\end{inhead}
\begin{rubric}
	The Office and Mass are from the Common of the Blessed Virgin Mary (p. \pageref{CommonBVM}), with Commemoration of St. Oswald from the First Common of a Martyr not a Bishop (p. \pageref{CommonMartyrNotBishopI}).
\end{rubric}

\begin{rubric}
	\textsc{Note,} The Creed is said and the Preface of the B.V.M. \emph{And that in the Festivity.}
\end{rubric}

%\collect
%\lett{A}{lmighty} and everlasting God, who by the martyrdom of the blessed King Oswald hast hallowed this day with holy joy and gladness: grant unto our hearts the increase of thy charity; that we, who honour his glorious battle for the faith, may imitate his constancy even unto death. Through.

%\readingcitation{Epistle}{Wisdom 4:7}
%RV:
%\lett{B}{ut} a righteous man, though he die before his time, shall be at rest. (For honourable old age is not that which standeth in length of time, Nor is its measure given by number of years: But understanding is gray hairs unto men, And an unspotted life is ripe old age.) Being found well-pleasing unto God he was beloved of him, And while living among sinners he was translated: He was caught away, lest †† wickedness should change his understanding, Or guile deceive his soul. (For the bewitching of naughtiness bedimmeth the things which are good, And the giddy whirl of desire perverteth an innocent mind.) Being made perfect in a little while, he fulfilled long years; For his soul was pleasing unto the Lord: Therefore hasted he out of the midst of wickedness. But as for the peoples, seeing and understanding not, Neither laying this to heart, That grace and mercy are with his chosen, And that he visiteth his holy ones.

%\lett{T}{hough} the righteous be prevented with death, yet shall he be in rest. For honourable age is not that which standeth in length of time, nor that is measured by number of years. But wisdom is the gray hair unto men, and an unspotted life is old age. He pleased God, and was beloved of him: so that living among sinners he was translated. Yea speedily was he taken away, lest that wickedness should alter his understanding, or deceit beguile his soul. For the bewitching of naughtiness doth obscure things that are honest; and the wandering of concupiscence doth undermine the simple mind. He, being made perfect in a short time, fulfilled a long time: For his soul pleased the Lord: therefore hasted he to take him away from among the wicked. This the people saw, and understood it not, neither laid they up this in their minds, That his grace and mercy is with his saints, and that he hath respect unto his chosen.

%\readingcitation{Gospel}{Matthew 16:24}
%\lett{A}{t that time:} Jesus said unto his disciples: If any man will come after me, let him deny himself, and take up his cross, and follow me. For whosoever will save his life shall lose it: and whosoever will lose his life for my sake shall find it. For what is a man profited, if he shall gain the whole world, and lose his own soul? or what shall a man give in exchange for his soul? For the Son of man shall come in the glory of his Father with his angels; and then he shall reward every man according to his works.


\subby{St. Oswald}
\feastday{{St. Oswald}}
\fancyhead[RE,LO]{3 August}
\begin{inhead}
    {Double\\
3 August}
\end{inhead}

\begin{rubric}
	The propers are from the First Common of a Martyr not a Bishop (p. \pageref{CommonMartyrNotBishopI}), except for the following.
\end{rubric}

\collect
\lett{A}{lmighty} and everlasting God, who by the martyrdom of the blessed King Oswald hast hallowed this day with holy joy and gladness: grant unto our hearts the increase of thy charity; that we, who honour his glorious battle for the faith, may imitate his constancy even unto death. Through.

%RV:
\readingcitation{Epistle}{Wisdom 4:7}
\lett{B}{ut} a righteous man, though he die before his time, shall be at rest. (For honourable old age is not that which standeth in length of time, Nor is its measure given by number of years: But understanding is gray hairs unto men, And an unspotted life is ripe old age.) Being found well-pleasing unto God he was beloved of him, And while living among sinners he was translated: He was caught away, lest wickedness should change his understanding, Or guile deceive his soul. (For the bewitching of naughtiness bedimmeth the things which are good, And the giddy whirl of desire perverteth an innocent mind.) Being made perfect in a little while, he fulfilled long years; For his soul was pleasing unto the Lord: Therefore hasted he out of the midst of wickedness. But as for the peoples, seeing and understanding not, Neither laying this to heart, That grace and mercy are with his chosen, And that he visiteth his holy ones.

\readingcitation{Gospel}{Matthew 16:24}
\lett{A}{t that time:} Jesus said unto his disciples: If any man will come after me, let him deny himself, and take up his cross, and follow me. For whosoever will save his life shall lose it: and whosoever will lose his life for my sake shall find it. For what is a man profited, if he shall gain the whole world, and lose his own soul? or what shall a man give in exchange for his soul? For the Son of man shall come in the glory of his Father with his angels; and then he shall reward every man according to his works.

%MANUAL ADJUSTMENT:
\clearpage
\bcpfeast{6 August. Transfiguration of Our Lord Jesus Christ}{Transfiguration}{6 August}
%\supplement{6 August}{Transfiguration}{of Our Lord Jesus Christ}

\subbysub{I Evensong}\label{TransfigurationEvensong}

\gregorioscore{resources/gabc/ProperTime/TransfigurationEvensong.gabc}

    ℣. Glorious didst thou appear in the sight of the Lord. 
    
	℟. Because the Lord hath clothed thee with majesty.


\properantiphon{Mag.}{Christ Jesus, {\dag} the brightness of the Father and the express image of his person, who upholdeth all things by the word of his power, while he was by himself purging away our sins, vouchsafed on this day to shew himself in glory upon an high mountain.}

\subbysub{Mattins}

\begin{rubric}
	The Invitatory Hymn is as in I Evensong.
\end{rubric}

\officehymn\label{TransfigurationMattins}

\gregorioscore{resources/gabc/ProperTime/TransfigurationMattins.gabc}

    ℣. A crown of gold is upon his head.
    
	℟. A visible sign of holiness, glory, and honour.

\properantiphon{Ben.}{And behold a voice {\dag} out of the cloud, which said, This is my beloved Son, in whom I am well pleased; hear ye him, alleluia.}

\subbysub{II Evensong}

\begin{rubric}
	The Office Hymn \& Versicle are of I Evensong, with the following Antiphon.
\end{rubric}

\properantiphon{Mag.}{And when the disciples heard it, {\dag} they fell on their face, and were sore afraid: and Jesus came and touched them, and said, Arise, and be not afraid, alleluia.}


\subby{Sts. Sixtus II, Felicissimus, \& Agapitus}
\feastday{{Sts. Sixtus II \&c.}}
\fancyhead[RE,LO]{6 August}
\begin{inhead}
    {Memorial\\
6 August}
\end{inhead}

\begin{rubric}
	The propers are from the Second Common of Many Martyrs (p. \pageref{CommonMartyrsII}).
\end{rubric}

%MANUAL ADJUSTMENT:
\clearpage
\subby{St. Donatus}
\feastday{{St. Donatus}}
\fancyhead[RE,LO]{7 August}
\begin{inhead}
    {Memorial\\
7 August}
\end{inhead}

\introit
\lett{O}{ye} priest of the Lord, bless ye the Lord: O ye holy and humble men of heart, bless ye the Lord. \textit{Cant.} O all ye works of the Lord, bless ye the Lord: praise him and magnify him for ever.

\collect
\lett{O}{God,} the glory of thy priests: grant, we beseech thee; that we may perceive the succour of thy holy Martyr and Bishop Donatus, whose festival we celebrate. Through.

\begin{rubric}
	The Epistle is the first additional Epistle from the Second Common of a Martyr not a Bishop (p. \pageref{James12}).
\end{rubric}

\gradall{The mouth of the just is exercised in wisdom, and his tongue will be talking of judgment. ℣. The law of his God is in his heart: and his goings shall not slide.}{Alleluia, alleluia. ℣. The just shall not be moved, for the Lord strengtheneth his hand. Alleluia.}

\begin{rubric}
	The Gospel is the second additional Gospel from the Second Common of a Confessor Bishop (p. \pageref{Mark1333}).
\end{rubric}

\offertory{I have found David my servant, with my holy oil have I anointed him: my hand shall hold him fast, and my arm shall strengthen him.}

\secret
\lett{G}{rant,} we beseech thee, O Lord: that, as by these gifts which we offer to the praise of thy name, we render honour to thy holy Martyr and Bishop Donatus, so by his intercession we may receive the reward of this our bounden service. Through.

\communion{A faithful and wise servant, whom the lord hath made ruler over his household: to give them their portion of meat in due season.}

\postcommunion
\lett{A}{lmighty} and merciful God, who makest us alike partakers and ministers of thy sacraments: grant, we beseech thee; that at the intercession of blessed Donatus, thy Martyr and Bishop, we, sharing in his faith and worthily serving thee, may be profited by the same. Through.


\subby{Sts. Cyriacus, Largus, \& Smaragdus}
\feastday{{Sts. Cyriacus \&c.}}
\fancyhead[RE,LO]{8 August}
\begin{inhead}
    {Memorial\\
8 August}
\end{inhead}

\introit
\lett{O}{fear} the Lord, ye that are his saints, for they that fear him lack nothing: the lions do lack, and suffer hunger: but they who seek the Lord shall want no manner of thing that is good. \textit{Ps.} I will alway give thanks unto the Lord: his praise shall ever be in my mouth.

\lett{O}{God,} who makest us glad with the yearly solemnity of thy holy Martyrs Cyriacus, Largus, and Smaragdus: mercifully grant; that as we now celebrate their birthday, so we may imitate their constancy in suffering. Through.

\begin{rubric}
	If today be Saturday, Commemoration is made of the anticipated Vigil of St. Lawrence, as on the following day; the \nth{3} Collect of St. Mary, and the Last Gospel of the Vigil.
\end{rubric}

\readingcitation{Epistle}{1 Thessalonians 2:13}
\lett{B}{rethren:} We thank God without ceasing, because, when ye received the word of God which ye heard of us, ye received it not as the word of men, but as it is in truth, the word of God, which effectually worketh also in you that believe. For ye, brethren, became followers of the churches of God which in Jud{\ae}a are in Christ Jesus: for ye also have suffered like things of your own countrymen, even as they have of the Jews: who both killed the Lord Jesus, and their own prophets, and have persecuted us; and they please not God, and are contrary to all men: forbidding us to speak to the Gentiles that they might be saved, to fill up their sins alway: for the wrath is come upon them to the uttermost.

\gradall{O fear the Lord, ye that are his saints: for they that fear him lack nothing. ℣. But they who seek the Lord, shall want no manner of thing that is good.}{Alleluia, alleluia. ℣. The righteous shall shine, and run to and fro like sparks among the stubble for ever. Alleluia.}

\readingcitation{Gospel}{Mark 16:15}
\lett{A}{t that time:} Jesus said to his disciples: Go ye into all the world, and preach the gospel to every creature. He that believeth and is baptized shall be saved; but he that believeth not shall be damned. And these signs shall follow them that believe; In my name shall they cast out devils; they shall speak with new tongues; they shall take up serpents; and if they drink any deadly thing, it shall not hurt them; they shall lay hands on the sick, and they shall recover.

\offertory{Be glad, O ye righteous, and rejoice in the Lord: and be joyful, all ye that are true of heart.}

\secret
\lett{G}{rant,} O Lord, that this our bounden service may be acceptable in thy sight: that these our oblations may, by the prayers of those on whose solemnity they are offered, be made profitable unto our salvation. Through.

\communion{These signs shall follow them that believe in me: they shall cast out devils: they shall lay hands on the sick, and they shall recover.}

\postcommunion
\lett{W}{e} beseech thee, O Lord our God, that like as we, whom thou hast refreshed by the partaking of thy sacred gift, do offer unto thee our worship: so by the intercession of thy holy Martyrs, Cyriacus, Largus and Smaragdus, we may perceive the benefit of the same. Through.


\subby{Vigil of St. Lawrence}
\feastday{{St. Lawrence Vigil}}
\fancyhead[RE,LO]{9 August}
\begin{inhead}
    {Vigil\\
9 August}
\end{inhead}

\introit
\lett{H}{e} hath dispersed abroad, and given to the poor: his righteousness remaineth for ever: his horn shall be exalted with honour. \textit{Ps.} Blessed is the man that feareth the Lord: he hath great delight in his commandments.

\begin{rubric}
	\emph{Gloria in excelsis} is not said.
\end{rubric}

\collect
\lett{A}{ssist} us, O Lord, in these our supplications: and at the intercession of thy blessed Martyr Lawrence, whose festival we prevent, graciously bestow upon us thy perpetual mercy. Through.

\begin{rubric}
	Commemoration of St. Romanus (p. \pageref{RomanusCollect}).
\end{rubric}

%RV:
\readingcitation{Epistle}{Ecclesiasticus 51:1}
\lett{I}{will} give thanks unto thee, O Lord, O King, And will praise thee, O God my Saviour: I do give thanks unto thy name: For thou wast my protector and helper, And didst deliver my body out of destruction, And out of the snare of a slanderous tongue, From lips that forge lies, And wast my helper before them that stood by; And didst deliver me, according to the abundance of thy mercy, and greatness of thy name, From the gnashings of teeth ready to devour, Out of the hand of such as sought my life, Out of the manifold afflictions which I had; From the choking of a fire on every side, And out of the midst of fire which I kindled not; Out of the depth of the belly of the grave, And from an unclean tongue, And from lying words, The slander of an unrighteous tongue unto the king. My soul drew near even unto death, And my life was near to the grave beneath. They compassed me on every side, And there was none to help me. I was looking for the succour of men, And it was not. And I remembered thy mercy, O Lord, And thy working which hath been from everlasting, How thou deliverest them that wait for thee, And savest them out of the hand of the enemies, O Lord our God.

\gradual{He hath dispersed abroad, and given to the poor: and his righteousness remaineth for ever. ℣. His seed shall be mighty upon earth: the generation of the faithful shall be blessed.}

\readingcitation{Gospel}{Matthew 16:24}
\lett{A}{t that time:} Jesus said unto his disciples: If any man will come after me, let him deny himself, and take up his cross, and follow me. For whosoever will save his life shall lose it: and whosoever will lose his life for my sake shall find it. For what is a man profited, if he shall gain the whole world, and lose his own soul? or what shall a man give in exchange for his soul? For the Son of man shall come in the glory of his Father with his angels; and then he shall reward every man according to his works.

\offertory{My prayer is pure: and therefore I ask that a place be given to my voice in heaven: for my witness is in heaven, and my record is on high: let my prayer ascend to the Lord.}

\secret
\lett{O}{Lord,} mercifully regard the sacrifices which we offer unto thee: and at the intercession of blessed Lawrence, thy Martyr, absolve us from the bonds of our sins. Through.

\begin{rubric}
	Commemoration of St. Romanus (p. \pageref{RomanusSecret}).
\end{rubric}

\communion{He that will come after me, let him deny himself, and take up his cross, and follow me.}

\postcommunion
\lett{G}{rant,} we beseech thee, O Lord, our God: that like as we in this life do gladly honour the memory of blessed Lawrence, thy Martyr; so we may rejoice to behold him for ever. Through.

\begin{rubric}
	Commemoration of St. Romanus (p. \pageref{RomanusPostcommunion}).
\end{rubric}


\subby{St. Romanus}
\feastday{{St. Romanus}}
\fancyhead[RE,LO]{9 August}
\begin{inhead}
    {Memorial\\
9 August}
\end{inhead}

\begin{rubric}
	The propers are from Second Common of a Martyr not a Bishop (p. \pageref{CommonMartyrNotBishopII}), except for the following.
\end{rubric}

\collect\label{RomanusCollect}
\lett{G}{rant,} we beseech thee, almighty God: that, at the intercession of blessed Romanus, thy Martyr, we may both be delivered from all adversities which may happen to the body, and from all evil thoughts which may assault and hurt the soul. Through.

\secret\label{RomanusSecret}
\lett{W}{e} beseech thee, O Lord, to accept our prayers and oblations: and graciously hearken unto us, whom thou dost cleanse by thy heavenly mysteries. Through.

\postcommunion\label{RomanusPostcommunion}
\lett{W}{e} beseech thee, almighty God: that we, who have received this heavenly food, may, at the intercession of blessed Romanus, thy Martyr, be thereby defended against all adversities. Through.


\bcpfeast{10 August. St. Lawrence}{St. Lawrence}{10 August}
%\supplement{10 August}{St. Lawrence}{}

\begin{rubric}
	The Daily Office propers are of the First Common of a Martyr Bishop (p. \pageref{CommonMartyrBishopI}), except for the following.
\end{rubric}

\properantiphon{Mag.}{Laurence the Deacon {\dag} hath wrought a pious work; who by the sign of the Cross enlightened the blind: and distributed to the poor the Church's treasures.}\\

    ℣. He hath dispersed abroad and given to the poor.
    
	℟. And his righteousness remaineth for ever.

\properantiphon{Ben.}{On the iron grate, {\dag} O God, I denied thee not; and when fire was kindled beneath me, O Christ, I confessed thee: thou hast proved my heart and hast visited me in the night season; thou hast tried me with fire, and hast found no wickedness in me.}\\

    ℣. Laurence the Deacon hath wrought a pious work.
    
	℟. Who by the sign of the Cross enlightened the blind.

\properantiphon{Mag.}{Blessed Laurence, {\dag} when laid and burning on the iron grating, spake to the impious tyrant, saying: The feast is ready, turn and eat; but the Church's treasures, which thou claimest, have been garnered up in heaven, by the hands of the poor and needy.}


\subby{Sts. Tiburtius \& Susanna}
\feastday{{Sts. Tiburtius \& Susanna}}
\fancyhead[RE,LO]{11 August}
\begin{inhead}
    {Memorial\\
11 August}
\end{inhead}

\begin{rubric}
	The propers are from the Third Common of Many Martyrs (p. \pageref{CommonMartyrsIII}), except for the following.
\end{rubric}

\collect
\lett{O}{Lord,} let the protection of thy holy Martyrs Tiburtius and Susanna continually defend us: forasmuch as thou failest not to look with mercy on those to whom thou dost grant the succour of their assistance. Through.

\begin{rubric}
	The Epistle is from the third additional epistle of the Third Common of Many Martyrs (p. \pageref{Romans818}).
\end{rubric}

\secret
\lett{A}{ssist,} O Lord, the prayers of thy people, assist their oblations: that those things which are offered in these sacred mysteries may, by the intercession of thy Saints, be acceptable unto thee. Through.

\postcommunion
\lett{O}{Lord,} through whom we have received the pledge of eternal redemption: we beseech thee that at the intercession of thy holy Martyrs, it may avail for our succour both in this life and that which is to come. Through.


\subby{St. Maximus of Constantinople}
\feastday{{St. Maximus Constantinople}}
\fancyhead[RE,LO]{13 August}
\begin{inhead}
    {Double\\
13 August}
\end{inhead}

\begin{rubric}
	The propers are from the Second Common of a Confessor not a Bishop (p. \pageref{CommonConfessorNotBishopII}), with Commemoration of Sts. Hippolytus \& Cassian.
\end{rubric}


%MANUAL ADJUSTMENT:
\clearpage
\subby{Sts. Hippolytus \& Cassian}
\feastday{{Sts. Hippolytus \& Cassian}}
\fancyhead[RE,LO]{13 August}
\begin{inhead}
    {Memorial\\
13 August}
\end{inhead}

\begin{rubric}
	The propers are from the Third Common of Many Martyrs (p. \pageref{CommonMartyrsIII}), except for the following.
\end{rubric}

\collect
\lett{G}{rant,} we beseech thee, almighty God: that the venerable solemnity of thy blessed Martyrs, Hippolytus and Cassian, may increase our devotion and set forward our salvation. Through.

\secret
\lett{R}{egard,} O Lord, the gifts of thy people, which we offer on the festival of thy Saints: and let this confession of thy truth be profitable for our salvation. Through.

\postcommunion
\lett{M}{ay} the communion of thy sacraments, O Lord, which we have received, avail for our salvation: stablish us in the light of thy truth. Through.

\begin{rubric}
	If today be Saturday, the anticipated Vigil of the Assumption of the B.V. Mary is kept, as is noted on the following day, but with Commemoration of Sts. Hippolytus and Cassian, instead of St. Eusebius.
\end{rubric}


%The following is from Sarum, to match the BCP. It's mostly similar to the English Missal's text.
\subby{Vigil of the Assumption of the Blessed Virgin Mary}
\feastday{{Assumption Vigil}}
\fancyhead[RE,LO]{14 August}
\begin{inhead}
    {Vigil\\
14 August}
\end{inhead}

\introit
\lett{H}{ail,} Holy Mother, who didst bring forth the King who ruleth over heaven and earth for ever and ever. \textit{Gospel.} Blessed art thou among women, and blessed is the fruit of thy womb.

\collect
\lett{O}{God,} who didst vouchsafe to choose the virgin womb of blessed Mary wherein to make thy dwelling: grant, we beseech thee; that, being defended by her protection, we may by thee be enabled to attain with gladness to her festival. Who livest.

\begin{rubric}
	Commemoration of St. Eusebius (p. \pageref{EusebiusCollect}).
\end{rubric}

\begin{rubric}
	\nth{3} Collect of the Holy Ghost (p. \SPHolyGhost).
\end{rubric}

\readingcitation{Epistle}{Ecclesiasticus 24:9}
%RV:
\lett{H}{e} created me from the beginning before the world; And to the end I shall not fail. In the holy tabernacle I ministered before him; And so was I established in Sion. In the beloved city likewise he gave me rest; And in Jerusalem was my authority. And I took root in a people that was glorified, Even in the portion of the Lord's own inheritance. 

%\lett{H}{e} created me from the beginning before the world, and I shall never fail. In the holy tabernacle I served before him; and so was I established in Sion. Likewise in the beloved city he gave me rest, and in Jerusalem was my power. And I took root in an honourable people, even in the portion of the Lord's inheritance, and my abode is in the full congregation of the saints.

\gradual{Blessed and venerable art thou, O Virgin Mary: who without spot wast found the Mother of the Saviour. ℣. Virgin, Mother of God, he whom the whole world containeth not, being made man lay hid in thy womb.}

\readingcitation{Gospel}{Luke 11:27}
\lett{A}{t that time:} As Jesus spake to the multitudes, a certain woman of the company lifted up her voice, and said unto him, Blessed is the womb that bare thee, and the paps which thou hast sucked. But he said, Yea rather, blessed are they that hear the word of God, and keep it.

\offertory{Happy art thou, O holy Virgin Mary, and most worthy of all praise, for from thee sprang the Sun of Righteousness, Christ our God. ℣. Blessed art thou, Virgin Mary, who didst bear the Lord, didst give birth to the Creator of the world, Who made thee, and ever remainest Virgin.}

\secret
\lett{O}{Lord,} who didst translate the Mother of God from this present life, to the intent that she might faithfully intercede before thee for our sins: grant that her prayers may render these our oblations acceptable in the sight of thy mercy. Through the same.

\begin{rubric}
	Commemoration of St. Eusebius (p. \pageref{EusebiusSecret}).
\end{rubric}

\begin{rubric}
	\nth{3} Secret of the Holy Ghost (p. \SPHolyGhost).
\end{rubric}

\communion{Gentle Mother of God, succour all that pray; we, too, with them, humbly entreat that by the aid of thy prayers we may sing praises unto the Trinity.}

%While the English Missal's text and the Sarum's are very similar, the Sarum is used because of its distinctive use of `requiem'.
\postcommunion
\lett{W}{e} beseech thee, O merciful God, to strengthen our frailty, that we who keep the requiem of the Holy Virgin Mother of God, may by her intercession rise again from our iniquities. Through.

\begin{rubric}
	Commemoration of St. Eusebius (p. \pageref{EusebiusPostcommunion}).
\end{rubric}

\begin{rubric}
	\nth{3} Postcommunion of the Holy Ghost (p. \SPHolyGhost).
\end{rubric}


\subby{St. Eusebius}
\feastday{{St. Eusebius}}
\fancyhead[RE,LO]{14 August}
\begin{inhead}
    {Memorial\\
14 August}
\end{inhead}

\introit
\lett{T}{he} just shall flourish like a palm-tree: and shall spread abroad like a cedar in Libanus: planted in the house of the Lord: in the courts of the house of our God. \textit{Ps.} It is a good thing to give thanks unto the Lord: and to sing praises unto thy name, O Most Highest.

\collect\label{EusebiusCollect}
\lett{O}{God,} who makest us glad with the yearly solemnity of blessed Eusebius thy Confessor: mercifully grant; that we, who celebrate his birthday, may by his example, be drawn nearer unto thee. Through.

\begin{rubric}
	The Epistle is additional Epistle from the Second Common of a Confessor not a Bishop (p. \pageref{Philippians37}).
\end{rubric}

\gradall{The just shall flourish like a palm-tree: and shall spread abroad like a cedar in Libanus in the house of the Lord. ℣. To tell of thy loving-kindness early in the morning, and of thy truth in the night-season.}{Alleluia, alleluia. ℣. The just shall grow as the lily: and flourish for ever before the Lord. Alleluia.}

\readingcitation{Gospel}{Matthew 11:25}
\lett{A}{t that time:} Jesus answered and said: I thank thee, O Father, Lord of heaven and earth, because thou hast hid these things from the wise and prudent, and hast revealed them unto babes. Even so, Father: for so it seemed good in thy sight. All things are delivered unto me of my Father: and no man knoweth the Son, but the Father; neither knoweth any man the Father, save the Son, and he to whomsoever the Son will reveal him. Come unto me, all ye that labour and are heavy laden, and I will give you rest. Take my yoke upon you, and learn of me; for I am meek and lowly in heart: and ye shall find rest unto your souls. For my yoke is easy, and my burden is light.

\offertory{The just shall rejoice in strength, O Lord: exceeding glad shall he be of thy salvation: thou hast given him his heart’s desire.}

\secret\label{EusebiusSecret}
\lett{G}{rant,} we beseech thee, O Lord, that we who, trusting in this our sacrifice of praise, do offer it before thee to the honour of thy Saints: may by the same be delivered from all evils both in this life and in that which is to come. Through.

\communion{The just shall rejoice in the Lord, and put his trust in him: and all they that are true of heart shall be glad.}

\postcommunion\label{EusebiusPostcommunion}
\lett{O}{Lord,} our God, who hast refreshed us with heavenly meat and drink, we humbly beseech thee: that we may be defended by the prayers of him in whose memory we have received the same. Through.


\bcpfeast{15 August. Assumption of the Blessed Virgin Mary}{Assumption}{15 August}
%\supplement{15 August}{Assumption}{of the Blessed Virgin Mary}

\begin{rubric}
	The Daily Office propers are of the Common of the Blessed Virgin Mary (p. \pageref{CommonBVM}), except for the following.
\end{rubric}

    ℣. The holy Mother of God is exalted.
    
	℟. Above choirs of Angels to the heavenly kingdom.

\properantiphon{Mag.}{O most prudent Virgin, {\dag} whither goest thou, shining resplendent like the glowing dawn? Daughter of Sion, thou art all comely and beautiful, fair as the moon, clear as the sun.}\\

    ℣. The holy Mother of God is exalted.
    
	℟. Above choirs of Angels to the heavenly kingdom.

\properantiphon{Ben.}{Who is she {\dag} that riseth up as the morning, fair as the moon, clear as the sun, and terrible as an army with banners?}

\begin{rubric}
	II Evensong is as in I Evensong, except the following Antiphon.
\end{rubric}

\properantiphon{Mag.}{On this day {\dag} the Virgin Mary went up into heaven: rejoice, because she reigneth with Christ for ever and ever.}


\bcpfeast{16 August. St. Joachim}{St. Joachim}{16 August}
%\supplement{16 August}{St. Joachim}{}

\begin{rubric}
	The Daily Office propers are of the First Common of a Confessor not a Bishop (p. \pageref{CommonConfessorNotBishopI}), except for the following Versicle \& Antiphon used for I Evensong and Mattins.
\end{rubric}

    ℣. His seed shall be mighty upon earth.
    
	℟. The generation of the faithful shall be blessed.

\properantiphon{Mag. \& Ben.}{Let us praise a man famous {\dag} in his generation, with whom the Lord established the blessing of all nations, and the covenant, and made it rest upon his head.}


\subby{Octave Day of St. Lawrence}
\feastday{{St. Lawrence Octave Day}}
\fancyhead[RE,LO]{17 August}
\begin{inhead}
    {Simple\\
17 August}
\end{inhead}

\introit
\lett{T}{hou} hast proved and visited mine heart, O Lord, in the night-season: thou hast tried me with fire, and hast found no wickedness in me. \textit{Ps.} Hear the right, O Lord: consider my complaint.

\collect
\lett{S}{tir} up, O Lord, in thy Church that Spirit whom the blessed Levite Lawrence served: that we, being filled with the same, may study to love that which he loved, and to perform in deed that which he taught. Through . . . in the unity of the same Holy Spirit.

\readingcitation{Epistle}{2 Corinthians 9:6}
\lett{B}{rethren:} He which soweth sparingly shall reap also sparingly; and he which soweth bountifully shall reap also bountifully. Every man according as he purposeth in his heart, so let him give; not grudgingly, or of necessity: for God loveth a cheerful giver. And God is able to make all grace abound toward you; that ye, always having all sufficiency in all things, may abound to every good work: (As it is written, He hath dispersed abroad; he hath given to the poor: his righteousness remaineth for ever. Now he that ministereth seed to the sower both minister bread for your food, and multiply your seed sown, and increase the fruits of your righteousness;)

\gradall{Thou hast crowned him with glory and worship, O Lord. ℣. And hast made him to have dominion of the works of thy hands.}{Alleluia, alleluia. ℣. The Levite Lawrence wrought a good work: who by the sign of the cross gave light to the blind. Alleluia.}

\readingcitation{Gospel}{John 12:24}
\lett{A}{t that time:} Jesus said unto his disciples: Verily, verily, I say unto you, Except a corn of wheat fall into the ground and die, it abideth alone: but if it die, it bringeth forth much fruit. He that loveth his life shall lose it; and he that hateth his life in this world shall keep it unto life eternal. If any man serve me, let him follow me; and where I am, there shall also my servant be: if any man serve me, him will my Father honour.

\offertory{The just shall rejoice in thy strength, O Lord: exceeding glad shall he be of thy salvation: thou hast given him his heart's desire.}

\secret
\lett{O}{Lord,} we beseech thee, let the holy prayer of blessed Lawrence commend the sacrifices unto thee: that it may be rendered acceptable by the merits of him in whose honour it is solemnly offered forth. Through.

\communion{He that will come after me, let him deny himself, and take up his cross, and follow me.}

\postcommunion
\lett{W}{e} humbly beseech thee, almighty God: that we, whom thou hast fulfilled with heavenly gifts, may, at the intercession of blessed Lawrence thy Martyr, be defended by thy continual protection. Through.

%Found in the supplement of the 1920 Missale Romanum.
\subby{St. Helen}
\feastday{{St. Helen}}
\fancyhead[RE,LO]{18 August}
\begin{inhead}
    {Memorial\\
18 August}
\end{inhead}

\begin{rubric}
	\textsc{Note,} When celebrated within the Octave of the Assumption, this Memorial is merely commemorated at Mass.
\end{rubric}

\introit
\lett{B}{ut} God forbid that I should glory, save in the Cross of our Lord Jesus Christ: by whom the world is crucified unto me, and I unto the world. \textit{Ps.} Thy rod and thy staff comfort me.

\collect
\lett{O}{Lord} Jesu Christ, who didst reveal unto blessed Helen the place where thy Cross lay hid, that through her thou mightest enrich thy Church with this precious treasure: grant unto us at her intercession; that by the ransom of the life-giving tree we may attain unto the rewards of everlasting life. Who livest.

\readingcitation{Epistle}{Proverbs 31:10}
\lett{W}{ho} can find a virtuous woman? for her price is far above rubies. The heart of her husband doth safely trust in her, so that he shall have no need of spoil. She will do him good and not evil all the days of her life. She seeketh wool, and flax, and worketh willingly with her hands. She is like the merchants' ships; she bringeth her food from afar. She riseth also while it is yet night, and giveth meat to her household, and a portion to her maidens. She considereth a field, and buyeth it: with the fruit of her hands she planteth a vineyard. She girdeth her loins with strength, and strengtheneth her arms. She perceiveth that her merchandise is good: her candle goeth not out by night. She layeth her hands to the spindle, and her hands hold the distaff. She stretcheth out her hand to the poor; yea, she reacheth forth her hands to the needy. She is not afraid of the snow for her household: for all her household are clothed with scarlet. She maketh herself coverings of tapestry; her clothing is silk and purple. Her husband is known in the gates, when he sitteth among the elders of the land. She maketh fine linen, and selleth it; and delivereth girdles unto the merchant. Strength and honour are her clothing; and she shall rejoice in time to come. She openeth her mouth with wisdom; and in her tongue is the law of kindness. She looketh well to the ways of her household, and eateth not the bread of idleness. Her children arise up, and call her blessed; her husband also, and he praiseth her. Many daughters have done virtuously, but thou excellest them all. Favour is deceitful, and beauty is vain: but a woman that feareth the \divineName{Lord}, she shall be praised. Give her of the fruit of her hands; and let her own works praise her in the gates.

\gradall{Like as the rich also among the people shall make their supplication before thee. Kings' daughters were among thy honourable women. ℣. She shall be brought unto the King in raiment of needle-work: the virgins that be her fellows shall bear her company, and shall be brought unto thee. With joy and gladness shall they be brought: and shall enter into the King's palace.}{Alleluia, alleluia. ℣. He hath dispersed abroad, and given to the poor : and his righteousness remaineth for ever. Alleluia.}

\readingcitation{Gospel}{Matthew 13:44}
\lett{A}{t that time:} Jesus spake this parable to his disciples: The kingdom of heaven is like unto treasure hid in a field; the which when a man hath found, he hideth, and for joy thereof goeth and selleth all that he hath, and buyeth that field. Again, the kingdom of heaven is like unto a merchant man, seeking goodly pearls: Who, when he had found one pearl of great price, went and sold all that he had, and bought it. Again, the kingdom of heaven is like unto a net, that was cast into the sea, and gathered of every kind: Which, when it was full, they drew to shore, and sat down, and gathered the good into vessels, but cast the bad away. So shall it be at the end of the world: the angels shall come forth, and sever the wicked from among the just, And shall cast them into the furnace of fire: there shall be wailing and gnashing of teeth. Jesus saith unto them, Have ye understood all these things? They say unto him, Yea, Lord. Then said he unto them, Therefore every scribe which is instructed unto the kingdom of heaven is like unto a man that is an householder, which bringeth forth out of his treasure things new and old.

\offertory{For I determined not to know any thing, save Jesus Christ, and him crucified.}

\secret
\lett{T}{hrough} these sacred mysteries vouchsafe unto us, O Lord: that as in thy mercy thou didst grant unto blessed Helen ever to carry thy Son crucified in her heart; so we may likewise continually hear him in our hearts. Who liveth and reigneth with thee.

\communion{I will go up to the palm tree, I will take hold of the boughs thereof.}

\postcommunion
\lett{G}{rant} unto us, O merciful God: that we who have been refreshed by the benefits of thy life-giving Cross on earth; may through the intercession of blessed Helen attain unto the eternal fruition of the same in heaven. Who livest.


\subby{St. Agapitus}
\feastday{{St. Agapitus}}
\fancyhead[RE,LO]{18 August}
\begin{inhead}
    {Memorial\\
18 August}
\end{inhead}

\begin{rubric}
	The propers come from the Second Common of a Martyr not a Bishop (p. \pageref{CommonMartyrNotBishopII}), except for that which followeth.
\end{rubric}

\begin{rubric}
	\textsc{Note,} When celebrated within the Octave of the Assumption, this Memorial is merely commemorated at Mass.
\end{rubric}

\collect
\lett{O}{Lord,} let thy Church trust with gladness in the advocacy of thy blessed Martyr Agapitus: that by his glorious prayers she may continue in devotion, and abide in safety. Through.

\readingcitation{Gospel}{John 12:24}
\lett{A}{t that time:} Jesus said unto his disciples: Verily, verily, I say unto you, Except a corn of wheat fall into the ground and die, it abideth alone: but if it die, it bringeth forth much fruit. He that loveth his life shall lose it; and he that hateth his life in this world shall keep it unto life eternal. If any man serve me, let him follow me; and where I am, there shall also my servant be: if any man serve me, him will my Father honour.

\secret
\lett{R}{eceive,} O Lord, the gifts which we offer on the solemnity of him, through whose advocacy we trust to be delivered. Through.

\postcommunion
\lett{O}{Lord,} who hast satisfied thy family with sacred gifts: we beseech thee; that we may at all times be comforted by the intercession of him whose festival we celebrate. Through.


\subby{Octave Day of the Assumption of the Blessed Virgin Mary}
\feastday{{Assumption Octave}}
\fancyhead[RE,LO]{22 August}
\begin{inhead}
    {Greater Double\\
22 August}
\end{inhead}
\begin{rubric}
	Mass as on the Feast with commemoration of Sts. Timothy, Hippolytus, \& Symphorian, as in the following Mass.
\end{rubric}
\begin{rubric}
	If today be Saturday, the anticipated Vigil of St. Bartholomew is kept, as is noted on the following day.
\end{rubric}

\subby{Sts. Timothy, Hippolytus, \& Symphorian}
\feastday{{Sts. Timothy \&c.}}
\fancyhead[RE,LO]{22 August}
\begin{inhead}
    {Memorial\\
22 August}
\end{inhead}

\begin{rubric}
	The propers come from the Third Common of Many Martyrs (p. \pageref{CommonMartyrsIII}), except for that which followeth.
\end{rubric}

\collect
\lett{W}{e} beseech thee, O Lord, graciously to impart unto us thy help: and, at the intercession of thy blessed Martyrs Timothy, Hippolytus, and Symphorian, stretch forth upon us the right hand of thy mercy. Through.

\secret
\lett{G}{rant,} O Lord, that like as thy dedicated people do acknowledge that in tribulation they have been succoured by the merits of thy Saints: so this oblation, which they offer unto thee in honour of the same, may be acceptable in thy sight. Through.

\postcommunion
\lett{O}{Lord} our God, who hast fulfilled us with the bounty of thy heavenly gift: we beseech thee, that, at the intercession of thy holy Martyrs, Timothy, Hippolytus, and Symphorian, we may ever live by the partaking of the same. Through.


\subby{Vigil of St. Bartholomew}
\feastday{{St. Bartholomew Vigil}}
\fancyhead[RE,LO]{23 August}
\begin{inhead}
    {Vigil\\
23 August}
\end{inhead}

\begin{rubric}
	The propers come from the Common of the Vigil of Apostles (p. \pageref{CommonVigilApostles}).\par
	\textsc{Note,} \nth{2} Collect of St. Mary in Eastertide (p. \SPMaryEaster) \& \nth{3} against the Persecutors of the Church (p. \SPAgainst) or for the Chief Bishop (p. \SPChiefBishop).
\end{rubric}


\bcpfeast{24 August. St. Bartholomew}{St. Bartholomew}{24 August}
%\supplement{24 August}{St. Bartholomew}{}

\begin{secrubric}
	The Daily Office propers are from the Common of Apostles (p. \pageref{CommonApostles}), with the following Antiphons.
\end{secrubric}

\properantiphon{Mag.}{They will deliver you up {\dag} to the councils, and they will scourge you in their synagogues; and ye shall be brought before governors and kings for my sake, for a testimony against them and the Gentiles.}

\properantiphon{Ben.}{Ye which have forsaken all, {\dag} and followed me, shall receive an hundredfold, and shall inherit everlasting life.}

\properantiphon{Mag.}{Be ye valiant in warfare, {\dag} and contend ye with the old serpent: and ye shall receive an eternal kingdom, alleluia.}


\subby{St. Zephyrinus}
\feastday{{St. Zephyrinus}}
\fancyhead[RE,LO]{26 August}
\begin{inhead}
    {Memorial\\
26 August}
\end{inhead}

\begin{rubric}
	The propers come from the Second Common of a Martyr Bishop out of Eastertide (p. \pageref{CommonMartyrBishopII}), except for that which followeth.
\end{rubric}

\collect
\lett{G}{rant,} we beseech thee, almighty God: that we who rejoice in the merits of blessed Zephyrinus, thy Martyr and Bishop, may be instructed by his example. Through.

\secret
\lett{S}{anctify,} O Lord, the gifts which we dedicate to thee: that at the intercession of blessed Zephyrinus thy Martyr and Bishop they may obtain for us thy gracious favour. Through.

\postcommunion
\lett{M}{ay} this communion, O Lord, cleanse us from guilt: and, at the intercession of blessed Zephyrinus, thy Martyr and Bishop, make us partakers of thy heavenly healing. Through.


\subby{St. Augustine of Hippo}
\feastday{{St. Augustine Hippo}}
\fancyhead[RE,LO]{28 August}
\begin{inhead}
    {Greater Double\\
28 August}
\end{inhead}
\begin{rubric}
	The propers are from the Common of Doctors (p. \pageref{CommonDoctors}), except for that which followeth.
\end{rubric}

\introit
\lett{I}{n} the midst of the Church he opened his mouth: and the Lord filled him with the spirit of wisdom and of understanding: he clothed him with a robe of glory. \textit{Ps.} It is a good thing to give thanks unto the Lord: and to sing praises unto thy name, O most Highest.

\collect
\lett{A}{ssist} us, almighty God, in these our supplications: that as thou dost suffer us to put our trust and confidence in thy mercy, so, at the intercession of blessed Augustine thy Confessor and Bishop, thou wouldest graciously vouchsafe unto us the wonted effects of thy compassion. Through.

\begin{rubric}
    Commemoration is made of St. Hermes (p. \pageref{HermesCollect}).
\end{rubric}

\gradall{The mouth of the righteous is exercised in wisdom, and his tongue will be talking judgment. ℣. The law of his God is in his heart: and his goings shall not slide.}{Alleluia, alleluia. ℣. I have found David my servant, with my holy oil have I anointed him. Alleluia.}

\offertory{The righteous shall flourish like a palm-tree: and shall spread abroad like a cedar in Libanus.}

\secret
\lett{M}{ay} the devout prayers of thy Bishop and Doctor, Saint Augustine, never fail to succour us, O Lord: that they may render our oblations acceptable in thy sight; and may ever obtain for us thy merciful pardon. Through.

\begin{rubric}
    Commemoration is made of St. Hermes (p. \pageref{HermesSecret}).
\end{rubric}

\communion{A faithful and wise servant, whom the lord hath made ruler over his household: to give them their portion of meat in due season.}

\postcommunion
\lett{W}{e} beseech thee, O Lord, that blessed Augustine, thy Bishop and illustrious Doctor, may stand before thee as our advocate: that these thy sacrifices may avail for our salvation. Through.

\begin{rubric}
    Commemoration is made of St. Hermes (p. \pageref{HermesPostcommunion}).
\end{rubric}


\subby{St. Hermes}
\feastday{{St. Hermes}}
\fancyhead[RE,LO]{28 August}
\begin{inhead}
    {Memorial\\
28 August}
\end{inhead}

\begin{rubric}
	The propers are from the Second Common of a Martyr not a Bishop (p. \pageref{CommonMartyrNotBishopII}), except for that which followeth.
\end{rubric}

\collect\label{HermesCollect}
\lett{O}{God,} who didst strengthen blessed Hermes thy Martyr with the virtue of constancy in his passion: Grant unto us by his example; to despise for love of thee the prosperity of the world, and to fear none of its adversities. Through.

\secret\label{HermesSecret}
\lett{W}{e} offer unto thee, O Lord, this sacrifice of praise in commemoration of thy Saints: grant, we beseech thee; that as it hath bestowed glory on them, so may it avail for our salvation. Through.

\postcommunion\label{HermesPostcommunion}
\lett{O}{Lord,} who hast fulfilled us with thy heavenly benediction, we beseech thy mercy: that, at the intercession of thy blessed Martyr Hermes, this service of our lowliness may avail for the comforting of our souls. Through.


\bcpfeast{29 August. Decollation of St. John Baptist}{Decollation}{29 August}
%\supplement{29 August}{Beheading}{of St. John Baptist}

\begin{rubric}
	The Hymns and Versicle are from the Common of Apostles (p. \pageref{CommonApostles}), with the Antiphons as followeth.
\end{rubric}

\properantiphon{Mag. \& Ben.}{Herod sent {\dag} an executioner, and commanded him to behead John in the prison: and when his disciples heard of it, they came, and took up his body, and laid it in a tomb.}

\properantiphon{Mag.}{The king, the unbeliever, {\dag} sent his detestable servants, and commanded them to behead John the Baptist.}


\subby{St. Sabina}
\feastday{{St. Sabina}}
\fancyhead[RE,LO]{29 August}
\begin{inhead}
    {Memorial\\
29 August}
\end{inhead}

\begin{rubric}
	The propers are from the Common of a Martyr not a Virgin (p. \pageref{CommonMartyrNotVirgin}), except for that which followeth.
\end{rubric}


\subby{Sts. Felix \& Adauctus}
\feastday{{Sts. Felix \& Adauctus}}
\fancyhead[RE,LO]{30 August}
\begin{inhead}
    {Memorial\\
30 August}
\end{inhead}

\introit
\lett{L}{et} the people tell of the wisdom of the Saints, and let the church shew forth their praise: their names shall live for evermore. \textit{Ps.} Rejoice in the Lord, O ye righteous: for it becometh well the just to be thankful.

\collect
\lett{O}{Lord,} we humbly entreat thy majesty: that, like as thou dost continually gladden us with the commemoration of thy Saints; so thou wouldest evermore defend us with their supplication. Through.

\begin{rubric}
	The Epistle is the third additional Epistle from the Third Common of Many Martyrs (p. \pageref{Wisdom1017}).
\end{rubric}

\gradall{The souls of the just are in the hand of God, and there shall no torment of malice touch them. ℣. In the eyes of the unwise they seemed to die: but they are in peace.}{Alleluia, alleluia. ℣. The righteous shall shine forth, and run to and fro like sparks among the stubble for ever. Alleluia.}

\begin{rubric}
	The Gospel is the fifth additional Gospel from the Third Common of Many Martyrs (p. \pageref{Luke1016}).
\end{rubric}

\offertory{Be glad, O ye righteous, and rejoice in the Lord: and be joyful, all ye that are true of heart.}

\secret
\lett{L}{ook} down, O Lord, upon the sacrifices of thy people: that, as with devout hearts they celebrate them to the honour of thy Saints, so they may perceive them to be profitable to their salvation. Through.

\communion{What I tell you in darkness, that speak ye in light, saith the Lord: and what ye hear in the ear, that preach ye upon the housetops.}

\postcommunion
\lett{W}{e} beseech thee, O Lord: that we, being filled with the sacred gifts; may at the intercession of thy Saints ever continue in thanksgiving for the same. Through.


\subby{St. Aidan of Lindisfarne}
\feastday{{St. Aidan}}
\fancyhead[RE,LO]{31 August}
\begin{inhead}
    {Memorial\\
31 August}
\end{inhead}

\begin{rubric}
	The propers are from the First Common of a Confessor Bishop (p. \pageref{CommonConfessorBishopI}).
\end{rubric}


\subby{St. Giles}
\feastday{{St. Giles}}
\fancyhead[RE,LO]{1 September}
\begin{inhead}
    {Memorial\\
1 September}
\end{inhead}

\begin{rubric}
	The propers are from the Common of Abbots (p. \pageref{CommonAbbots}), except for that which followeth.\par
	Commemoration is made of the Twelve Brothers.
\end{rubric}


\subby{Twelve Holy Brothers}
\feastday{{12 Holy Brothers}}
\fancyhead[RE,LO]{1 September}
\begin{inhead}
    {Memorial\\
1 September}
\end{inhead}

\introit
\lett{T}{he} just cry, and the Lord heareth them: and delivereth them out of all their troubles. \textit{Ps.} I will alway give thanks unto the Lord; his praise shall ever be in my mouth.

\collect
\lett{O}{Lord,} let the crown of the brethren thy Martyrs, cause us to rejoice: that we may thereby be strengthened and increased in our faith; and comforted by their manifold intercession. Through.

\begin{rubric}
	The Epistle is the fifth additional Epistle of the Third Common of Many Martyrs (p. \pageref{Hebrews1133}).
\end{rubric}

\gradall{Behold, how good and joyful a thing it is, brethren, to dwell together in unity! ℣. It is like the precious ointment upon the head, that ran down unto the beard, even unto Aaron's beard.}{Alleluia, alleluia. ℣. This is the true brotherhood which overcame the wickedness of the world: which followed Christ, gaining heaven's glorious realms. Alleluia.}

\begin{rubric}
	The Gospel is from the Third Common of Many Martyrs (p. \pageref{CommonMartyrsII}).
\end{rubric}

\offertory{Be glad, O ye righteous, and rejoice in the Lord: and be joyful, all ye that are true of heart.}

\secret
\lett{G}{rant,} O Lord, that we may with devout hearts celebrate thy mysteries in honour of thy holy Martyrs: and thereby obtain an increase both of protection and joy. Through.

\communion{Whosoever shall do the will of my Father which is in heaven: the same is my brother, and sister, and mother, saith the Lord.}

\postcommunion
\lett{G}{rant,} we beseech thee, almighty God: that growing in virtue we may follow the faith of them whose memory we recall by the partaking of this sacrament. Through.


\subby{St. Stephen of Hungary}
\feastday{{St. Stephen Hungary}}
\fancyhead[RE,LO]{2 September}
\begin{inhead}
	{Memorial\\
		2 September}
\end{inhead}

\begin{rubric}
	The propers are from the First Common of Confessor not Bishop (p. \pageref{CommonConfessorNotBishopI}), except for the following.
\end{rubric}

\collect
\lett{G}{rant,} we beseech thee, almighty God unto thy Church: that as thy blessed Confessor Stephen, while he reigned on earth, did spread abroad her faith, so she may be found worthy to have him for her glorious defender in the heavens. Through.

\begin{rubric}
	The Gospel is the additional Gospel from the Second Common of a Confessor not a Bishop (p. \pageref{Luke1912}).
\end{rubric}

\secret
\lett{A}{lmighty} God, look upon the sacrifices which we offer: and grant; that we, who celebrate the mysteries of the Passion of the Lord, may imitate that which we perform. Through the same.

\postcommunion
\lett{G}{rant,} we beseech thee, almighty God: that, as thy blessed Confessor Stephen, for the propagation of thy faith, was counted worthy to pass from an earthly kingdom to the glory of the heavenly realm; so we may imitate his faith with due devotion. Through.


\subby{St. Gorazde of Prague}
\feastday{{St. Gorazde}}
\fancyhead[RE,LO]{4 September}
\begin{inhead}
	{Double\\
		4 September}
\end{inhead}

\begin{rubric}
	The propers are from the Second Common of a Martyr Bishop (p. \pageref{CommonMartyrBishopII}.
\end{rubric}


\bcpfeast{8 September. Nativity of the Blessed Virgin Mary}{Nativity B.V.M.}{8 September}
%\supplement{8 September}{Nativity}{of the Blessed Virgin Mary}

\begin{rubric}
	The Hymns are from the Common of the Blessed Virgin Mary (p. \pageref{CommonBVM}), with the following Versicles \& Antiphons.
\end{rubric}

    ℣. To-day is the Nativity of the holy Virgin Mary.
    
	℟. Whose glorious life illumineth all the churches.

\properantiphon{Mag.}{Let us celebrate {\dag} the worshipful Nativity of the blessed and glorious Virgin Mary: for she hath obtained the dignity of Motherhood, and yet lost not her maiden purity.}\\

    ℣. To-day is the Nativity of the holy Virgin Mary.
    
	℟. Whose glorious life illumineth all the churches.

\properantiphon{Ben.}{To-day let us celebrate {\dag} with due solemnity the Nativity of God's Mother, the ever Virgin Mary: from whom the Son of the Highest proceeded, alleluia.}\\

    ℣. To-day is the Nativity of the holy Virgin Mary.
    
	℟. Whose glorious life illumineth all the churches.

\properantiphon{Mag.}{Thy Nativity, {\dag} O Virgin Mother of God, hath proclaimed joyful tidings unto all the world: for out of thee hath arisen the Sun of righteousness, even Christ our God: who, taking away the curse, hath bestowed a blessing; and confounding death, hath given unto us life everlasting.}


\subby{St. Hadrian}
\feastday{{St. Hadrian}}
\fancyhead[RE,LO]{8 September}
\begin{inhead}
	{Memorial\\
		8 September}
\end{inhead}

\begin{rubric}
	The propers are from the First Common of a Martyr not a Bishop (p. \pageref{CommonMartyrNotBishopI}).
\end{rubric}


%MANUAL ADJUSTMENT:
\clearpage
\subby{St. Gorgonius}
\feastday{{St. Gorgonius}}
\fancyhead[RE,LO]{9 September}
\begin{inhead}
	{Memorial\\
		9 September}
\end{inhead}

\begin{rubric}
	The propers are from the Second Common of a Martyr not a Bishop (p. \pageref{CommonMartyrNotBishopII}), except for the following.
\end{rubric}

\collect
\lett{L}{et} thy Saint Gorgonius, O Lord, gladden us by his intercession; and make us to rejoice in this holy solemnity. Through.

\secret
\lett{O}{Lord,} let thy holy Martyr Gorgonius so intercede for us: that this oblation of our service may be acceptable unto thee. Through.

\postcommunion
\lett{G}{rant,} O God, that thy household may be quickened and refreshed by thine eternal goodness: and in thy Martyr Gorgonius be continually nourished with the sweet savour of Christ, thy Son. Who liveth and reigneth with thee.


\subby{Sts. Protus \& Hyacinth}
\feastday{{Sts. Protus \& Hyacinth}}
\fancyhead[RE,LO]{11 September}
\begin{inhead}
	{Memorial\\
		11 September}
\end{inhead}

\begin{rubric}
	The propers are from the Third Common of Many Martyrs out of Eastertide (p. \pageref{CommonMartyrsIII}), except for the following.
\end{rubric}

\collect
\lett{O}{Lord,} let the meritorious confession of thy blessed Martyrs, Protus and Hyacinth, comfort us; and may their loving intercession ever defend us. Through.

\secret
\lett{G}{rant,} we beseech thee, O Lord, that the oblation of our bounden service which we ofier unto thee for the commemoration of thy holy Martyrs, Protus and Hyacinth; may effectually avail for our healing unto everlasting salvation. Through.

\postcommunion
\lett{W}{e} beseech thee, O Lord, that by the supplication of thy blessed Martyrs, Protus and Hyacinth: thy holy mysteries which we have received may avail for our cleansing. Through.


\subby{Most Holy Name of Mary}
\feastday{{Holy Name of Mary}}
\fancyhead[RE,LO]{12 September}
\begin{inhead}
	{Greater Double\\
		12 September}
\end{inhead}

\begin{rubric}
	The Daily Office propers are from the Common of the Blessed Virgin Mary (p. \pageref{CommonBVM}), except for the antiphon for I Evensong, as followeth.
\end{rubric}

%Last clause translated from the Diurnale Monasticum:
\antiphon{Mag.}{O holy Mary, {\dag} help thou the suffering, strengthen the faint-hearted, comfort the sorrowful; pray for the people, entreat for the clergy, intercede for all womankind vowed unto God: may all acknowledge the help of thy prayer, who celebrate the commemoration of thine holy Name.}\par\noindent

\introit
\lett{A}{ll} the rich among the people shall make their supplication before thee: the Virgins that be her fellows shall be brought unto the King: they that bear her company shall be brought unto thee with joy and gladness. \textit{Ps.} My heart is inditing of a good matter: I speak of the things which I have made unto the King.

\collect
\lett{G}{rant,} we beseech thee, almighty God: that thy faithful people, who rejoice in the Name and protection of the most Holy Virgin Mary; may by her loving intercession be delivered from all evils upon earth, and be found worthy to attain unto everlasting joys in heaven. Through.

\readingcitation{Epistle}{Ecclesiasticus 24:17}
%KJV:
%\lett{I}{as} the vine brought forth pleasant savour, and my flowers are the fruit of honour and riches. I am the mother of fair love, and fear, and knowledge, and holy hope: I therefore, being eternal, am given to all my children which are named of him. Come unto me, all ye that be desirous of me, and fill yourselves with my fruits. For my memorial is sweeter than honey, and mine inheritance than the honeycomb. They that eat me shall yet be hungry, and they that drink me shall yet be thirsty. He that obeyeth me shall never be confounded, and they that work by me shall not do amiss.
%RV:
\lett{A}{s} the vine I put forth grace; And my flowers are the fruit of glory and riches. Come unto me, ye that are desirous of me, And be ye filled with my produce. For my memorial is sweeter than honey, And mine inheritance than the honeycomb. They that eat me shall yet be hungry; And they that drink me shall yet be thirsty. He that obeyeth me shall not be ashamed; And they that work in me shall not do amiss.

\gradall{Blessed and venerable art thou, O Virgin Mary: who without spot wast found the Mother of the Saviour. ℣. Virgin, Mother of God, he whom the world containeth not, being made man lay hid in thy womb.}{Alleluia, alleluia. ℣. After child-birth, O Virgin, thou didst remain inviolate: Mother of God, intercede for us. Alleluia.}

\readingcitation{Gospel}{Luke 1:26}
\lett{A}{t that time:} The angel Gabriel was sent from God unto a city of Galilee, named Nazareth, to a virgin espoused to a man whose name was Joseph, of the house of David; and the virgin’s name was Mary. And the angel came in unto her, and said, Hail, thou that art highly favoured, the Lord is with thee: blessed art thou among women. And when she saw him, she was troubled at his saying, and cast in her mind what manner of salutation this should be. And the angel said unto her, Fear not, Mary: for thou hast found favour with God. And, behold, thou shalt conceive in thy womb, and bring forth a son, and shalt call his name \divineName{Jesus}. He shall be great, and shall be called the Son of the Highest: and the Lord God shall give unto him the throne of his father David: and he shall reign over the house of Jacob for ever; and of his kingdom there shall be no end. Then said Mary unto the angel, How shall this be, seeing I know not a man? And the angel answered and said unto her, The Holy Ghost shall come upon thee, and the power of the Highest shall overshadow thee: therefore also that holy thing which shall be born of thee shall be called the Son of God. And, behold, thy cousin Elisabeth, she hath also conceived a son in her old age: and this is the sixth month with her, who was called barren. For with God nothing shall be impossible. And Mary said, Behold the handmaid of the Lord; be it unto me according to thy word.

\offertory{Hail, Mary, full of grace; the Lord is with thee: blessed art thou among women, and blessed is the fruit of thy womb.}

\secret
\lett{T}{hrough} thy mercy, O Lord, and the intercession of blessed Mary ever Virgin, may this oblation avail for our prosperity and peace, both now and for ever. Through.

\communion{Blessed is the womb of the Virgin Mary, that bore the Son of the everlasting Father.}

\postcommunion
\lett{G}{rant,} we beseech thee, O Lord: that we, who have received these aids to our salvation, may at all times and in all places be protected through the advocacy of blessed Maty ever Virgin; in whose honour we have made these offerings to thy majesty. Through.


\bcpfeast{14 September. Exaltation of the Holy Cross}{Exaltation of the Cross}{14 September}
%\supplement{14 September}{Exaltation}{of the Holy Cross}

\begin{rubric}
	The Daily Office propers are as in the Feast of the Invention of the Holy Cross, 3 May (p. \pageref{InventionoftheCross}), omitting the \emph{Alleluia}. If II Evensong be said, the following Antiphon is said.
\end{rubric}

\antiphon{Mag.}{O Cross {\dag} exceeding blessed, which alone wast counted worthy to bear the Lord, the King of heaven, alleluia.}


\bcpfeast{Ember Wednesday in Autumn}{Autumnal Ember Wednesday}{}

\properantiphon{Ben.}{This kind {\dag} of evil spirit can come forth by nothing but by prayer and fasting.}


\bcpfeast{Ember Friday in Autumn}{Autumnal Ember Friday}{}

\properantiphon{Ben.}{A woman {\dag} in the city, which was a sinner, stood at the Lord's feet behind him, and began to wash his feet with tears, and did wipe them with the hairs of her head, and kissed his feet, and anointed them with the ointment.}


\bcpfeast{Ember Saturday in Autumn}{Autumnal Ember Saturday}{}

\properantiphon{Ben.}{Give light, O Lord, {\dag} to them that sit in darkness: and gudie our feet into the way of peace, thou God of Israel.}


\bcpfeast{15 September. Seven Sorrows of the Blessed Virgin Mary}{Seven Sorrows}{15 September}
%\supplement{15 September}{Seven Sorrows}{of the Blessed Virgin Mary}

\subbysub{I Evensong}\label{SevenSorrowsEvensong}

\gregorioscore{resources/gabc/ProperTime/SevenSorrowsEvensong.gabc}

    ℣. Pray for us, O Queen of Martyrs.

	℟. Who didst stand by the Cross of Jesus.

\properantiphon{Mag.}{Look not {\dag} upon me, because I am black, because the sun hath looked upon me: my mother's children were angry with me.}

\subbysub{Mattins}

\invitatoryhymn\label{SevenSorrowsInvitatory}

\gregorioscore{resources/gabc/ProperTime/SevenSorrowsInvitatory.gabc}

\officehymn\label{SevenSorrowsMattins}

\gregorioscore{resources/gabc/ProperTime/SevenSorrowsMattins.gabc}

    ℣. O Virgin Mary, by thy sorrow's might.

	℟. Make us rejoice in heaven's kingdom bright.

\properantiphon{Ben.}{Come ye, {\dag} and let us go up to the mountain of the Lord, and see if there be any sorrow like unto my sorrow.}

\subbysub{II Evensong}

\begin{rubric}
	The Office Hymn \& Versicle are as in I Evensong, with the following Antiphon.
\end{rubric}

\properantiphon{Mag.}{Sorrow oppresseth me, {\dag} my face is swollen with weeping, and on my eyelids is the shadow of death.}


\subby{St. Nicomedes}
\feastday{{St. Nicomedes}}
\fancyhead[RE,LO]{16 September}
\begin{inhead}
	{Memorial\\
		15 September}
\end{inhead}

\begin{rubric}
	The propers are from the First Common of a Martyr not a Bishop (p. \pageref{CommonMartyrNotBishopI}), except for the following.
\end{rubric}

\collect
\lett{A}{ssist,} O Lord, thy people: that as they do profit by the glorious merits of blessed Nicomedes thy Martyr, so his advocacy may at all times succour them to the obtaining of thy mercy. Through.

\secret
\lett{M}{ercifully} receive, O Lord, the gifts which we offer: and let the prayer of the blessed Martyr Nicomedes commend them unto thy majesty. Through.

\postcommunion
\lett{C}{leanse} us, O Lord, by the sacraments which we have received: that through the intercession of blessed Nicomedes, thy Martyr, they may set us free from all our offences. Through.


\subby{Sts. Cornelius \& Cyprian}
\feastday{{Sts. Cornelius \& Cyprian}}
\fancyhead[RE,LO]{16 September}
\begin{inhead}
	{Double\\
		16 September}
\end{inhead}

\begin{rubric}
	The propers are from the First Common of Many Martyrs out of Eastertide (p. \pageref{CommonMartyrsI}), except for the following.
\end{rubric}

\collect
\lett{P}{rotect} us, O Lord, we beseech thee, who observe the feast of thy blessed Martyrs and Bishops Cornelius and Cyprian: and grant that by their meritorious supplication we may find favour in thy sight. Through.

\begin{rubric}
    Commemoration is made of St. Euphemia, Lucy, and Geminian (p. \pageref{EuphemiaCollect}).
\end{rubric}

\secret
\lett{A}{ssist} us mercifully, O Lord, in these our supplications which we make before thee in remembrance of thy Saints: that we who trust not in our own righteousness may be succoured by the merits of them that have found favour in thy sight. Through.

\begin{rubric}
    Commemoration is made of St. Euphemia, Lucy, and Geminian (p. \pageref{EuphemiaSecret}).
\end{rubric}

\postcommunion
\lett{O}{Lord,} who hast fulfilled us with saving mysteries, we beseech thee: that we may be aided by the prayers of those whose festival we celebrate. Through.

\begin{rubric}
    Commemoration is made of St. Euphemia, Lucy, and Geminian (p. \pageref{EuphemiaPostcommunion}).
\end{rubric}


\subby{Sts. Euphemia, Lucy, \& Geminian}
\feastday{{Sts. Euphemia \&c.}}
\fancyhead[RE,LO]{16 September}
\begin{inhead}
	{Memorial\\
		16 September}
\end{inhead}

\begin{rubric}
	The propers are from the First Common of Many Martyrs (p. \pageref{CommonMartyrsI}), except for the following.
\end{rubric}

\collect\label{EuphemiaCollect}
\lett{G}{rant,} O Lord, that our prayers in this time of our rejoicing may be brought to good effect: that as with yearly service we recall the day of the passion of thy holy Martyrs, Euphemia, Lucy, and Geminian, so we may imitate the steadfastness of their faith. Through.

\begin{rubric}
	The Gospel is from the Second Common of Many Martyrs (p. \pageref{CommonMartyrsII}).
\end{rubric}

\secret\label{EuphemiaSecret}
\lett{G}{raciously} hearken, we beseech thee, O Lord, unto the prayers of thy people: and make us to rejoice in the intercession of those, whose festival thou dost suffer us to celebrate. Through.

\postcommunion\label{EuphemiaPostcommunion}
\lett{O}{Lord,} graciously hear our prayers: that we, who solemnly observe the feast of thy holy Martyrs, Euphemia, Lucy, and Geminian, may be succoured by their continual help. Through.


\subby{Sts. Januarius \& Companions}
\feastday{{Sts. Januarius \&c.}}
\fancyhead[RE,LO]{19 September}
\begin{inhead}
	{Double\\
		19 September}
\end{inhead}

\begin{rubric}
	The propers are from the Third Common of Many Martyrs (p. \pageref{CommonMartyrsIII}), except for the first additional Gospel of the same Common (p. \pageref{Matthew243}).
\end{rubric}
\begin{rubric}
	\textsc{Note,} Commemoration is make of St. Theodore of Canterbury.
\end{rubric}


\subby{St. Theodore of Canterbury}
\feastday{{St. Theodore Canterbury}}
\fancyhead[RE,LO]{19 September}
\begin{inhead}
	{Memorial\\
		19 September}
\end{inhead}

\begin{rubric}
	The propers are from the First Common of a Confessor Bishop (p. \pageref{CommonConfessorBishopI}).
\end{rubric}

\begin{rubric}
	If today be Saturday, the anticipated Vigil of St. Matthew is kept.
\end{rubric}


\subby{Sts. Eustace \& Companions}
\feastday{{Sts. Eustace \&c.}}
\fancyhead[RE,LO]{20 September}
\begin{inhead}
	{Double\\
		20 September}
\end{inhead}

\begin{rubric}
	The propers are from the Second Common of Many Martyrs (p. \pageref{CommonMartyrsII}).
\end{rubric}


\subby{Vigil of St. Matthew}
\feastday{{St. Matthew Vigil}}
\fancyhead[RE,LO]{20 September}
\begin{inhead}
	{Vigil\\
		20 September}
\end{inhead}

\begin{rubric}
	The propers are from the Common of Vigils of the Apostles (p. \pageref{CommonVigilApostles}) (with the \nth{2} Prayers of St. Mary in Eastertide (p. \SPMaryEaster) and the \nth{3} against the Persecutors of the Church (p. \SPAgainst) or for the Chief Bishop (p. \SPChiefBishop)), except for the following Gospel.
\end{rubric}

\readingcitation{Gospel}{Luke 5:27}
\lett{A}{t that time:} Jesus saw a publican, named Levi, sitting at the receipt of custom: and he said unto him, Follow me. And he left all, rose up, and followed him. And Levi made him a great feast in his own house: and there was a great company of publicans and of others that sat down with them. But their scribes and Pharisees murmured against his disciples, saying, Why do ye eat and drink with publicans and sinners? And Jesus answering said unto them, They that are whole need not a physician; but they that are sick. I came not to call the righteous, but sinners to repentance.


\bcpfeast{21 September. St. Matthew}{St. Matthew}{21 September}

\begin{secrubric}
	The Daily Office propers are of the Common of Apostles (p. \pageref{CommonApostles}).
\end{secrubric}


\subby{Sts. Maurice \& Companions}
\feastday{{Sts. Maurice \&c.}}
\fancyhead[RE,LO]{22 September}
\begin{inhead}
	{Memorial\\
		22 September}
\end{inhead}

\begin{rubric}
	The propers are from the First Common of Many Martyrs (p. \pageref{CommonMartyrsI}), except for the following.
\end{rubric}

\collect
\lett{G}{rant,} we beseech thee, almighty God: that the solemn festival of thy holy Martyrs, Maurice and his Companions, may in such wise gladden us: that, as we do lean upon their advocacy, so we may glory in their heavenly birth. Through.

\begin{rubric}
	The Epistle is the sixth additional Epistle from the Third Common of Many Martyrs (p. \pageref{Revelation713}).
\end{rubric}

\secret
\lett{R}{egard,} we beseech thee, O Lord, the gifts which we offer unto thee in commemoration of thy holy Martyrs, Maurice and his Companions: and grant; that through the intercession of them, for whose sake they are acceptable unto thee, they may be profitable unto us for evermore. Through.

\postcommunion
\lett{O}{Lord,} who hast refreshed us with the gladness of thy heavenly sacraments: we humbly entreat thee; that we may be protected by the succour of them in whose triumphs we glory. Through.


\subby{Pope St. Linus}
\feastday{{Pope St. Linus}}
\fancyhead[RE,LO]{23 September}
\begin{inhead}
	{Memorial\\
		23 September}
\end{inhead}

\begin{rubric}
	The propers are from the First Common of a Martyr Bishop (p. \pageref{CommonMartyrBishopI}), except for the following.
\end{rubric}

\collect
\lett{O}{God,} who makest us glad with the yearly solemnity of blessed Linus thy Martyr and Bishop: mercifully grant; that, as we now celebrate his birthday, so we may likewise rejoice in his protection. Through.

\begin{rubric}
    Commemoration is made of St. Thecla (p. \pageref{TheclaCollect}).
\end{rubric}

\secret
\lett{S}{anctify,} O Lord, the gifts which we dedicate to thee: that at the intercession of blessed Linus, thy Martyr and Bishop, they may obtain for us thy gracious favour. Through.

\begin{rubric}
    Commemoration is made of St. Thecla (p. \pageref{TheclaSecret}).
\end{rubric}

\postcommunion
\lett{M}{ay} this communion, O Lord, cleanse us from guilt: and, at the intercession of blessed Linus, thy Martyr and Bishop, make us partakers of thy heavenly healing. Through.

\begin{rubric}
    Commemoration is made of St. Thecla (p. \pageref{TheclaPostcommunion}).
\end{rubric}


\subby{St. Thecla}
\feastday{{St. Thecla}}
\fancyhead[RE,LO]{23 September}
\begin{inhead}
	{Memorial\\
		23 September}
\end{inhead}

\begin{rubric}
	The propers are from the First Common of a Virgin Martyr (p. \pageref{CommonVirginMartyrI}), except for the following.
\end{rubric}

\collect\label{TheclaCollect}
\lett{G}{rant,} we beseech thee, almighty God: that we, who celebrate the birthday of blessed Thecla, thy Virgin and Martyr; may both rejoice in her yearly solemnity, and likewise profit by the example of her faith. Through.

\secret\label{TheclaSecret}
\lett{R}{eceive,} O Lord, the gifts which we offer on the solemnity of blessed Thecla, thy Virgin and Martyr: through whose advocacy we trust to be delivered. Through.

\postcommunion\label{TheclaPostcommunion}
\lett{M}{ay} the mysteries which we have received be for our succour, O Lord: and at the intercession of blessed Thecla, thy Virgin and Martyr, cause us to rejoice in thy continual protection. Through.


\subby{Sts. Cyprian \& Justina}
\feastday{{Sts. Cyprian \& Justina}}
\fancyhead[RE,LO]{26 September}
\begin{inhead}
	{Memorial\\
		26 September}
\end{inhead}

\begin{rubric}
	The propers are from the Third Common of Many Martyrs (p. \pageref{CommonMartyrsIII}), except for the following.
\end{rubric}

\collect
\lett{O}{Lord,} let the protection of thy blessed Martyrs, Cyprian and Justina, continually defend us: forasmuch as thou failest not to look with mercy on those to whom thou dost grant the succour of thy assistance. Through.

\secret
\lett{W}{e} beseech thee, O Lord, that the gifts which we offer unto thee of our bounden duty and service may be acceptable unto thee for the honour of thy Just ones: and by thy mercy profitable unto us for our salvation. Through.

\postcommunion
\lett{G}{rant} to us, we beseech thee, O Lord: at the intercession of thy holy Martyrs Cyprian and Justina; that those things which we touch with our mouths we may receive in purity of heart. Through.


\subby{Sts. Cosmas \& Damian}
\feastday{{Sts. Cosmas \& Damian}}
\fancyhead[RE,LO]{27 September}
\begin{inhead}
	{Memorial\\
		27 September}
\end{inhead}

\introit
\lett{L}{et} the people tell of the wisdom of the Saints, and let the Church shew forth their praise: their names shall live for evermore. \textit{Ps.} Rejoice in the Lord, O ye righteous: for it becometh well the just to be thankful.

\collect
\lett{G}{rant,} we beseech thee, almighty God: that we, who observe the heavenly birthday of thy holy Martyrs, Cosmas and Damian, may by their intercession be delivered from all evils that beset us. Through.

\begin{rubric}
	The Epistle is from the Second Common of Many Martyrs (p. \pageref{CommonMartyrsII}).
\end{rubric}

\gradall{The righteous cry, and the Lord heareth them: and delivereth them out of all their troubles. ℣. The Lord is nigh unto them that are of a contrite heart: and will save such as be of an humble spirit.}{Alleluia, alleluia. ℣. This is the true brotherhood, which overcame the wickedness of the world: which followed Christ, gaining heaven’s glorious realms. Alleluia.}

\begin{rubric}
	The Gospel is from the Second Common of Many Martyrs (p. \pageref{CommonMartyrsII}).
\end{rubric}

\offertory{All they that love thy Name shall be joyful in thee: for thou, Lord, wilt give thy blessing unto the righteous: and with thy favourable kindness, O Lord, wilt thou defend us as with a shield.}

\secret
\lett{M}{ay} the devout prayers of thy Saints never fail us, O Lord: that they may both render our oblations acceptable, and ever obtain for us thy merciful pardon. Through.

\communion{The dead bodies of thy servants, O Lord, have they given to be meat unto the fowls of the air, and the flesh of thy Saints unto the beasts of the land: according to the greatness of thy power preserve thou those that are appointed to die.}

\postcommunion
\lett{W}{e} beseech thee, O Lord, that thy people may be protected by the prayers which thy Saints do offer, and by this heavenly banquet, whereof thou hast suffered them to be partakers. Through.


\subby{St. Wenceslaus}
\feastday{{St. Wenceslaus}}
\fancyhead[RE,LO]{28 September}
\begin{inhead}
	{Memorial\\
		28 September}
\end{inhead}

\begin{rubric}
	The propers are from the First Common of a Martyr not a Bishop (p. \pageref{CommonMartyrNotBishopI}), except for the following.
\end{rubric}

\collect
\lett{O}{God,} who through the victory of martyrdom didst translate blessed Wenceslas from an earthly principality to the glory of heaven: defend us through his prayers from all adversity; and grant us likewise to rejoice in his fellowship. Through.


\bcpfeast{29 September. St. Michael}{St. Michael}{29 September}\label{SaintMichael}
%\supplement{29 September}{St. Michael}{}

\subbysub{I Evensong}\label{MichaelEvensong}

\gregorioscore{resources/gabc/ProperTime/MichaelEvensong.gabc}

    ℣. An Angel stood at the altar of the temple.

	℟. Having in his hand a golden censer.

\properantiphon{Mag.}{While John beheld {\dag} the sacred mystery, the Archangel Michael sounded the trumpet: Forgive us, O Lord our God, that openest the book, and loosest the seals thereof, alleluia.}

\subbysub{Mattins}

\begin{rubric}
	The Invitatory Hymn is as in I Evensong.
\end{rubric}

\officehymn\label{MichaelMattins}

\gregorioscore{resources/gabc/ProperTime/MichaelMattins.gabc}

    ℣. An Angel stood at the altar of the temple.

	℟. Having in his hand a golden censer.

\properantiphon{Ben.}{There was silence in heaven {\dag} while the dragon waged war: and Michael fought against him, and had the victory, alleluia.}

\subbysub{II Evensong}

\begin{rubric}
	The Office Hymn is as in I Evensong, with the following Versicle \& Antiphon.
\end{rubric}

    ℣. In the presence of the Angels I will sing praise unto thee, O my God.

	℟. I will worship toward thy holy temple, and praise thy Name.

\properantiphon{Mag.}{O Prince most glorious, {\dag} Michael the Archangel, keep us in remembrance: here and everywhere, always, entreat the Son of God for us, alleluia, alleluia.}

%MANUAL ADJUSTMENT:
\clearpage
\subby{St. Jerome}
\feastday{{St. Jerome}}
\fancyhead[RE,LO]{30 September}
\begin{inhead}
	{Greater Double\\
		30 September}
\end{inhead}

\begin{rubric}
	The propers are from the First Common of Many Martyrs out of Eastertide (p. \pageref{CommonMartyrsI}), except for the following.\par
	\textsc{Note,} The Creed is said.
\end{rubric}

\collect
\lett{O}{God,} who for the exposition of the sacred Scriptures didst bestow upon thy Church blessed Jerome, thy Confessor and most illustrious Doctor: grant, we beseech thee; that, by the intercession of his merits, we may through thine assistance be enabled to perform those things which he taught both in word and deed. Through.

\begin{rubric}
    Commemoration is made of St. Gregory of Armenia (p. \pageref{ArmeniaCollect}).
\end{rubric}

\secret
\lett{G}{rant} us, we beseech thee, O Lord, through these heavenly gifts to serve thee in freedorn of spirit: that the gifts which we offer may, through the mediation of blessed Jerome, thy Confessor, work in us both healing and glory. Through.

\begin{rubric}
    Commemoration is made of St. Gregory of Armenia (p. \pageref{ArmeniaSecret}).
\end{rubric}

\postcommunion
\lett{W}{e} beseech thee, O Lord, that we whom thou hast fulfilled with heavenly nourishment: may through the mediation of blessed Jerome, thy Confessor, be found worthy to obtain the grace of thy loving-kindness. Through.

\begin{rubric}
    Commemoration is made of St. Gregory of Armenia (p. \pageref{ArmeniaPostcommunion}).
\end{rubric}


\subby{St. Gregory of Armenia}
\feastday{{St. Gregory Armenia}}
\fancyhead[RE,LO]{30 September}
\begin{inhead}
	{Memorial\\
		30 September}
\end{inhead}

\begin{rubric}
	The propers are from the First Common of a Confessor Bishop (p. \pageref{CommonConfessorBishopI}), except for the following.
\end{rubric}

%Propers changed to indicate that he is a Confessor not a Martyr.
\collect\label{ArmeniaCollect}
\lett{O}{God,} who, through thy blessed Bishop and Confessor Gregory, didst grant that the people and King of Armenia to receive the light of true faith: likewise grant unto thy Church to rejoice in such mighty triumphs, and, by the merits and intercession of the same, to be succoured before thee. Through.
%Deus, qui per beátum Gregórium Pontíficem et Mártyrem tuum Arméniæ gentis pópulum régémque veræ fídei lucem recípere tribuísti: da Ecclésiæ tuæ de tantis gaudére triúmphis, et apud te méritis ejusdem et précibus adjuvári. Per Dóminum.

\secret\label{ArmeniaSecret}
\lett{G}{raciously} receive, O Lord, the sacrifices dedicated unto thee, by the merits of thy blessed Bishop and Confessor Gregory: and grant that they may avail us unto everlasting help. Through.
%Hóstias tibi, Dómine, beáti Gregórii Mártyris tui atque Pontíficis dicátas méritis, benígnus assúme: et ad perpétuum nobis tríbue proveníre subsídium. Per Dóminum.


\postcommunion\label{ArmeniaPostcommunion}
\lett{G}{rant,} we beseech thee, O Lord our God; that we, having been refreshed by participation in the holy gifts, may experience the fruit of the intercession of blessed Gregory thy Bishop and Confessor, whose festival we now celebrate. Through.
%Refécti participatióne múneris sacri, quǽsumus, Dómine, Deus noster: ut, cujus exséquimur cultum, intercedénte beáto Gregório Mártyre tuo atque Pontífice, sentiámus efféctum. Per Dóminum nostrum.


\subby{St. Remigius}
\feastday{{St. Remigius}}
\fancyhead[RE,LO]{1 October}
\begin{inhead}
	{Memorial\\
		1 October}
\end{inhead}

\begin{rubric}
	The propers are from the First Common of a Confessor Bishop (p. \pageref{CommonConfessorBishopI}).
\end{rubric}


\bcpfeast{2 October. Holy Guardian Angels}{Guardian Angels}{2 October}
%\supplement{2 October}{Holy Guardian Angels}{}

\subbysub{I Evensong}\label{GuardianAngelsEvensong}

\gregorioscore{resources/gabc/ProperTime/GuardianAngelsEvensong.gabc}

    ℣. In the presence of the Angels I will sing praise unto thee, O my God.

	℟. I will worship toward thy holy temple, and praise thy Name.
	
\properantiphon{Mag.}{They are all {\dag} ministering spirits, sent forth to minister to them who shall be heirs of salvation.}

\subbysub{Mattins}

\begin{rubric}
	The Invitatory Hymn is as in I Evensong.
\end{rubric}

\officehymn\label{GuardianAngelsMattins}

\gregorioscore{resources/gabc/ProperTime/GuardianAngelsMattins.gabc}

    ℣. In the presence of the Angels I will sing praise unto thee, O my God.

	℟. I will worship toward thy holy temple, and praise thy Name.

\properantiphon{Ben.}{And the Angel {\dag} that talked with me came again, and waked me, as a man that is wakened out of his sleep.}

\subbysub{II Evensong}

\begin{rubric}
	The Office Hymn \& Versicle is as in I Evensong, with the following Antiphon.
\end{rubric}

\properantiphon{Mag.}{Ye holy Angels, {\dag} our Guardians, defend us in time of battle, lest we perish in the dreadful day of judgment.}


\subby{Sts. Placidus \& Companions}
\feastday{{Sts. Placidus \&c.}}
\fancyhead[RE,LO]{5 October}
\begin{inhead}
	{Double\\
		5 October}
\end{inhead}

\begin{rubric}
	The propers are from the Third Common of Many Martyrs (p. \pageref{CommonMartyrsIII}), with the Prayers from the Second Common of Many Martyrs (p. \pageref{CommonMartyrsII}).
\end{rubric}


\bcpfeast{7 October. Holy Rosary of the Blessed Virgin Mary}{Holy Rosary}{7 October}
%\supplement{7 October}{Holy Rosary}{of the Blessed Virgin Mary}

\subbysub{I Evensong}\label{HolyRosaryEvensong}

\gregorioscore{resources/gabc/ProperTime/HolyRosaryEvensong.gabc}

    ℣. Queen of the holy Rosary, pray for us.

	℟. That we may be made worthy of the promises of Christ.

\properantiphon{Mag.}{Blessed art thou, {\dag} O Virgin Mary, Mother of God, for thou hast believed the Lord: for those things are accomplished in thee which were told thee: intercede for us unto the Lord our God.}

\subbysub{Mattins}

\invitatoryhymn\label{HolyRosaryInvitatory}

\gregorioscore{resources/gabc/ProperTime/HolyRosaryInvitatory.gabc}

\officehymn\label{HolyRosaryMattins}

\gregorioscore{resources/gabc/ProperTime/HolyRosaryMattins.gabc}

    ℣. God hath chosen her and preferred her.

	℟. He hath made her to dwell in his tabernacle.

\properantiphon{Ben.}{To-day let us celebrate {\dag} devoutly the Solemnity of the holy Rosary of Mary, Mother of God, that she may intercede for us unto Jesus Christ the Lord.}

\subbysub{II Evensong}\label{HolyRosaryEvensongII}

\gregorioscore{resources/gabc/ProperTime/HolyRosaryEvensongII.gabc}

    ℣. Queen of the holy Rosary, pray for us.

	℟. That we may be made worthy of the promises of Christ.

\properantiphon{Mag.}{O blessed Mother {\dag} and spotless Virgin, thou glorious Queen of the world, may all acknowledge the help of thy prayer who celebrate the solemnity of thy holy Rosary.}


\subby{St. Mark of Rome}
\feastday{{St. Mark Rome}}
\fancyhead[RE,LO]{7 October}
\begin{inhead}
	{Memorial\\
		7 October}
\end{inhead}

\begin{rubric}
	The propers are from the Second Common of a Confessor Bishop (p. \pageref{CommonConfessorBishopII}), except for the following.
\end{rubric}

\collect
\lett{G}{raciously} hear our prayers, O Lord: and at the intercession of blessed Mark, thy Confessor and Bishop, mercifully grant us pardon and peace. Through.

\secret
\lett{G}{rant,} O Lord, that like as thy dedicated people do acknowledge that in tribulation they have been succoured by the merits of thy Saints: so this oblation, which they offer unto thee in honour of the same, may be acceptable in thy sight. Through.

\postcommunion
\lett{G}{rant,} we beseech thee, O Lord, that thy faithful people may ever rejoice in the veneration of thy Saints: and be defended by their perpetual supplication. Through.


\subby{Sts. Sergius, Bacchus, Marcellus, \& Apuleius}
\feastday{{Sts. Sergius \&c.}}
\fancyhead[RE,LO]{7 October}
\begin{inhead}
	{Memorial\\
		7 October}
\end{inhead}

\begin{rubric}
The propers are from the Second Common of Many Martyrs (p. \pageref{CommonMartyrsII}), except for the following.
\end{rubric}

\collect\label{SergiusCollect}
\lett{M}{ay} the blessed merits of thy holy Martyrs Sergius, Bacchus, Marcellus, and Apuleius uphold us, O Lord: and ever make us fervent in thy love. Through.

\secret\label{SergiusSecret}
\lett{W}{e} beseech thee, O Lord, that, through the meritorious supplication of thy Saints: this sacrifice, which we offer, may obtain for us the favour of thy majesty. Through.

\postcommunion\label{SergiusPostcommunion}
\lett{G}{rant,} O Lord, that, being strengthened by the sacraments which we have received: we may, at the intercession of thy holy Martyrs, Sergius, Bacchus, Marcellus, and Apuleius, fight against all iniquity, and be defended by thy heavenly armour. Through.


\subby{Sts. Denys, Rusticus, \& Eleutherius}
\feastday{{Sts. Denys \&c.}}
\fancyhead[RE,LO]{9 October}
\begin{inhead}
	{Memorial\\
		9 October}
\end{inhead}

\introit
\lett{L}{et} the people tell of the wisdom of the Saints, and let the Church shew forth their praise: their names shall live for evermore. \textit{Ps.} Rejoice in the Lord, O ye righteous: for it becometh well the just to be thankful.

\collect
\lett{O}{God,} who on this day didst strengthen blessed Denys, thy Martyr and Bishop, with the virtue of constancy in his passion, and didst vouchsafe to join unto him Rusticus and Eleutherius for the preaching of thy glory to the Gentiles: grant us, we beseech thee; by their example, to despise for love of thee the prosperity of the world, and to fear none of its adversities. Through.

\readingcitation{Epistle}{Acts 17:22}
\lett{I}{n those days:} Paul stood in the midst of Mars' hill, and said: Ye men of Athens, I perceive that in all things ye are too superstitious. For as I passed by, and beheld your devotions, I found an altar with this inscription, \textsc{To the Unknown God}. Whom therefore ye ignorantly worship, him declare I unto you. God that made the world and all things therein, seeing that he is Lord of heaven and earth, dwelleth not in temples made with hands; Neither is worshipped with men's hands, as though he needed any thing, seeing he giveth to all life, and breath, and all things; And hath made of one blood all nations of men for to dwell on all the face of the earth, and hath determined the times before appointed, and the bounds of their habitation; That they should seek the Lord, if haply they might feel after him, and find him, though he be not far from every one of us: For in him we live, and move, and have our being; as certain also of your own poets have said, For we are also his offspring. Forasmuch then as we are the offspring of God, we ought not to think that the Godhead is like unto gold, or silver, or stone, graven by art and man's device. And the times of this ignorance God winked at; but now commandeth all men every where to repent: Because he hath appointed a day, in the which he will judge the world in righteousness by that man whom he hath ordained; whereof he hath given assurance unto all men, in that he hath raised him from the dead. And when they heard of the resurrection of the dead, some mocked: and others said, We will hear thee again of this matter. So Paul departed from among them. Howbeit certain men clave unto him, and believed: among the which was Dionysius the Areopagite, and a woman named Damaris, and others with them.

\gradall{Our soul is escaped even as a bird out of the snare of the fowler. ℣. The snare is broken, and we are delivered: our help is in the name of the Lord, who hath made heaven and earth.}{Alleluia, alleluia. ℣. Let the righteous be glad and rejoice before God: let them also be merry and joyful. Alleluia.}

\readingcitation{Gospel}{Luke 12:1}
\lett{A}{t that time:} Jesus said unto his disciples: Beware ye of the leaven of the Pharisees, which is hypocrisy. For there is nothing covered, that shall not be revealed; neither hid, that shall not be known. Therefore whatsoever ye have spoken in darkness shall be heard in the light; and that which ye have spoken in the ear in closets shall be proclaimed upon the housetops. And I say unto you my friends, Be not afraid of them that kill the body, and after that have no more that they can do. But I will forewarn you whom ye shall fear: Fear him, which after he hath killed hath power to cast into hell; yea, I say unto you, Fear him. Are not five sparrows sold for two farthings, and not one of them is forgotten before God? But even the very hairs of your head are all numbered. Fear not therefore: ye are of more value than many sparrows. Also I say unto you, Whosoever shall confess me before men, him shall the Son of man also confess before the angels of God.

\offertory{Let the Saints be joyful with glory, let them rejoice in their beds: let the praises of God be in their mouth, alleluia.}

\secret
\lett{G}{raciously} receive, we beseech thee, O Lord, the oblations of thy people, for the honour of thy Saints: and sanctify us by their intercession. Through.

\communion{And I say unto you, my friends: Be not afraid of them that persecute you.}

\postcommunion
\lett{W}{e} beseech thee, O Lord that we, who have received thy sacraments: may at the intercession of thy blessed Martyrs Denys, Rusticus, and Eleutherius, be profited thereby unto the increase of eternal redemption. Through.


\bcpfeast{11 October. Motherhood of the Blessed Virgin Mary}{Motherhood}{11 October}

\subbysub{I Evensong}

\begin{rubric}
	The Office Hymn is from the Common of the Blessed Virgin Mary (p. \pageref{CommonBVM}), with the following Versicle \& Antiphon.
\end{rubric}

    ℣. Blessed art thou amongst women.

	℟. And blessed is the fruit of thy womb.
	
	%Original translation from the Breviarium Monasticum:
	\antiphon{Mag.}{Let us celebrate with joy {\dag} the Motherhood of the blessed Mary ever Virgin.}
%Cum jucunditáte Maternitátem beáta Maria semper Virginis celebrémus.

%MANUAL ADJUSTMENT:
\clearpage
\subbysub{Mattins}

\invitatoryhymn\label{MotherhoodInvitatory}

\gregorioscore{resources/gabc/ProperTime/MotherhoodInvitatory.gabc}

\officehymn\label{MotherhoodMattins}

\gregorioscore{resources/gabc/ProperTime/MotherhoodMattins.gabc}

    ℣. The root of Jesse hath budded forth: a star hath arisen out of Jacob.

	℟. The Virgin hath brought forth the Saviour: we praise thee, O our God.
	
	%Original translation of the last clause.
	\antiphon{Ben.}{O holy Mary, {\dag} help thou the suffering, strengthen the faint-hearted, comfort the sorrowful; pray for the people, entreat for the clergy, intercede for all womankind vowed unto God: may all experience thy help, who celebrate thine admirable motherhood.}
	
\subbysub{II Evensong}

\begin{rubric}
	The Office Hymn is from the Common of the Blessed Virgin Mary (p. \pageref{CommonBVM}), with the following Versicle \& Antiphon.
\end{rubric}

    ℣. Blessed art thou amongst women.

	℟. And blessed is the fruit of thy womb.
	
	%Original translation from the Breviarium Monasticum:
	\antiphon{Mag.}{Thy motherhood, O Virgin Birthgiver of God, announces joy unto the whole world: for from thee the Sun of Righteousness hath arisen, Christ our God.}


\subby{St. Wilfred}
\feastday{{St. Wilfred}}
\fancyhead[RE,LO]{12 October}
\begin{inhead}
	{Memorial\\
		12 October}
\end{inhead}

\begin{rubric}
	The propers are from the Second Common of a Confessor Bishop (p. \pageref{CommonConfessorBishopII}), except for the following.
\end{rubric}

\collect
\lett{O}{God,} by whose grace the blessed Bishop Wilfrid did wondrously shine forth with the glorious tokens of his merits: mercifully grant unto us; that we, who by his doctrine are taught to seek after things heavenly, may ever be defended by his advocacy. Through.

\secret
\lett{P}{urify,} we beseech thee, almighty God, the hearts of thy family by the enlightening of thy Holy Spirit: that through the intercession of blessed Wilfrid, thy Confessor and Bishop, these gifts of devotion may be rendered acceptable unto thee. Through . . . in the unity of the same.

\postcommunion
\lett{O}{Lord,} who hast fulfilled us with the food of eternal redemption, we humbly entreat thy mercy: that by the intercession of blessed Wilfrid, thy Confessor and Bishop, we may receive the gifts of everlasting salvation. Through.

%MANUAL ADJUSTMENT:
\clearpage
\subby{St. Edward}
\feastday{{St. Edward}}
\fancyhead[RE,LO]{13 October}
\begin{inhead}
	{Memorial\\
		13 October}
\end{inhead}

\begin{rubric}
	The propers are from the First Common of a Confessor not a Bishop (p. \pageref{CommonConfessorNotBishopI}), except for the following.
\end{rubric}

\collect
\lett{O}{God,} who didst bestow upon thy blessed Confessor King Edward the crown of everlasting glory: grant us, we beseech thee; so to venerate him on earth, that we may be enabled to reign with him in heaven. Through.


\subby{St. Callistus}
\feastday{{St. Callistus}}
\fancyhead[RE,LO]{14 October}
\begin{inhead}
	{Double\\
		14 October}
\end{inhead}

\introit
\lett{O}{ye} priests of the Lord, bless ye the Lord: O ye holy and humble men of heart, bless ye the Lord. \textit{Cant.} O all ye works of the Lord, bless ye the Lord: praise him, and magnify him for ever.

\collect
\lett{O}{God,} who seest that we fail by reason of our infirmity: through the examples of thy Saints mercifully restore us to the love of thee. Through.

\readingcitation{Epistle}{Hebrews 5:1}
\lett{B}{rethren:} Every high priest taken from among men is ordained for men in things pertaining to God, that he may offer both gifts and sacrifices for sins: Who can have compassion on the ignorant, and on them that are out of the way; for that he himself also is compassed with infirmity. And by reason hereof he ought, as for the people, so also for himself, to offer for sins. And no man taketh this honour unto himself, but he that is called of God, as was Aaron.

\gradall{I have found David my servant, with my holy oil have I anointed him: my hand shall hold him fast, and my arm shall strengthen him. ℣. The enemy shall not be able to do him violence: the son of wickedness shall not hurt him.}{Alleluia, alleluia. ℣. The Lord loved him, and adorned him: he clothed him with a robe of glory. Alleluia.}

\begin{rubric}
	The Gospel is of the Second Common of a Martyr not a Bishop (p. \pageref{CommonMartyrNotBishopII}).
\end{rubric}

\offertory{My truth and my mercy shall be with him: and in my name shall his horn be exalted.}

\secret
\lett{O}{Lord,} let this mystical oblation be profitable unto us: that it may both deliver us from our offences, and stablish us in everlasting salvation. Through.

\communion{Blessed is the servant whom the lord, when he cometh, shall find watching: verily I say unto you, that he shall make him ruler over all his goods.}

\postcommunion
\lett{W}{e} beseech thee, almighty God: that the gifts which we have hallowed may cleanse us from our iniquities, and bring forth in us the fruit of godly living. Through.


\bcpfeast{15 October. Our Lady of Walsingham}{Walsingham}{15 October}
%\supplement{15 October}{Our Lady of Walsingham}{}

%\gregorioscore{resources/gabc/ProperTime/WalsinghamEvensong.gabc}
	
\begin{paracol}{2}[]
\sloppy
\begin{inhead}
	I Evensong\label{WalsinghamEvensong}
\end{inhead}
\begin{hangparas}{1.25em}{1}
Hail Mary, ever blessed,

Of Walsingham the Queen!

Through vision of Richeldis,

Thy favours there were seen.

When England was thy dowry,

There pilgrims bowed the knee.

At morn and noon and even,

They knelt to honour thee.\\

Hail Mary, ever blessed.

Thy children still delight

To tell abroad thy praises,

Thy miracles, thy might.

Still pilgrim feet are treading

Along the holy way.

Hostess of England's Nazareth,

Receive us home today!\\

Hail Mary, ever blessed.

The wells of water pure

Which mark thy holy places

Are signs that God doth cure

For sick of soul and body.

E’er since Richeldis' day,

They spring in benediction

Beside the Pilgrims' Way.\\

Hail Mary, ever blessed.

Thy name is great indeed;

For Jesus Christ our Saviour

Was in thy womb conceived.

Thy name be ever praised,

Increasing in this place,

And loud the angel's greeting:

`Hail Mary, full of grace!'\\

All glory, laud, and honour be

O Jesu, Virgin-born

Our worship is alone to thee

With Father and the Ghost.

Thou art our Mediator true,

Appealing for our sin;

Send unto us thine own Spirit,

Who makes us thy true kin. Amen.
\end{hangparas}

    ℣. O Mary, Queen of Heaven and of England, praise ye the Lord.

	℟. And bring us unto the majesty of thy Son.
	
	\properantiphon{Mag.}{O all ye Saints, {\dag} come unto the Holy Place of Walsingham, and bless ye the Lord.}

\switchcolumn

\begin{inhead}
	Mattins\label{WalsinghamMattins}
\end{inhead}

\begin{rubric}
		\textsc{Note,} The Invitatory Hymn is the \emph{Salve Regina} (p. \pageref{SalveReginaEnglish}).
	\end{rubric}

\begin{hangparas}{1.25em}{1}
When Christ was born in Bethlehem,\par
The Angels worshipped and adored;\par
And kings and shepherds bowed the knee\par
To him their Saviour and their Lord.\\

For thirty years in Nazareth,\par
His Mother worshipped at the shrine\par
Of him who came in human form\par
To save the world, your Lord and mine.\\

For thirty years in Nazareth\par
He lived a hidden life of prayer;\par
And then went forth to work and die,\par
And every human sorrow share.\\

A thousand years have passed away;\par
Another Nazareth must rise\par
In England’s fair and lovely land,\par
Beneath those distant northern skies.\\

So Mary, God's own Mother blest,\par
Seeks out Richeldis, lady fair;\par
And bids her build at Walsingham\par
A second Nazareth of prayer.\\

Richeldis hastens to obey;\par
And soon the builder’s work is done:\par
The Shrine of England’s Nazareth\par
Proclaims the Mother and the Son.\\

But times did come when wicked hands\par
Were laid upon that holy place;\par
And men who knew not Mary's love\par
Did Walsingham's fair Shrine deface.\\

Four hundred years again have passed;\par
Once more the Mother of our Lord\par
Is shrined and honoured where her Son\par
Upon his Altar is adored.\\

So when you come to Walsingham,\par
Remember this, and mark it well:\par
It is to Nazareth you come,\par
Where Jesus, Mary, Joseph dwell.\\

All glory, laud, and honour be\par
To Jesu, Virgin-born, to thee;\par
All glory, as is ever meet,\par
To Father and to Paraclete. Amen.
\end{hangparas}

℣. Blessed is the holy Virgin Mary, and most worthy of all praise.

℟. Through her hath arisen the Sun of Justice, Christ our God, by whom we are saved and redeemed.

\properantiphon{Ben.}{Wicked men in their lust defied the Church, {\dag} and razed the Shrine of Our Lady.}
\end{paracol}

\begin{rubric}
	In II Evensong, the Office Hymn \& Versicle are of I Evensong, with the following Antiphon.
\end{rubric}

\properantiphon{Mag.}{O God, raise thou up again devotion to thy daughter: {\dag} Our blessed Lady of Walsingham.}


\bcpfeast{18 October. St. Luke}{St. Luke}{18 October}
%\supplement{18 October}{St. Luke}{}

\begin{secrubric}
	The Daily Office propers are of the Common of Apostles (p. \pageref{CommonApostles}).
\end{secrubric}


\subby{St. Frideswide}
\feastday{{St. Frideswide}}
\fancyhead[RE,LO]{19 October}
\begin{inhead}
	{Memorial\\
		19 October}
\end{inhead}

\begin{rubric}
	The propers are from the First Common of a Virgin (p. \pageref{CommonVirginOnlyI}), except for the following.
\end{rubric}

\collect
\lett{A}{lmighty} and everlasting God, the author of virtue and lover of virginity: grant us, we beseech thee; that, like as thy Virgin Frideswide was pleasing unto thee by the merit of chastity of life; so through her merits we may find favour in thy sight. Through.

\secret
\lett{W}{e} offer thee, O Lord, our prayers and gifts with gladness in honour of Saint Frideswide thy Virgin: that we may be enabled worthily to perform the same, and to obtain thine everlasting healing. Through.

\postcommunion
\lett{W}{e} beseech thee, O Lord, that the mysteries which we have received may be profitable unto many: through the intercession of thy blessed Virgin Frideswide, may they both deliver us from our sins, and obtain for us thy gracious protection. Through.


\subby{St. Hilarion}
\feastday{{St. Hilarion}}
\fancyhead[RE,LO]{21 October}
\begin{inhead}
	{Memorial\\
		21 October}
\end{inhead}

\begin{rubric}
	The propers are from the Common of Abbots (p. \pageref{CommonAbbots}).\par
	\textsc{Note,} Commemoration is made of Sts. Ursula \& Companions.
\end{rubric}


\subby{Sts. Ursula \& Companions}
\feastday{{Sts. Ursula \&c.}}
\fancyhead[RE,LO]{21 October}
\begin{inhead}
	{Memorial\\
		21 October}
\end{inhead}

\begin{rubric}
	The propers are from the First Common of a Virgin Martyr (p. \pageref{CommonVirginMartyrI}), with the Prayers from the Common of Many Virgin Martyrs (p. \pageref{CommonVirginMartyrs}).
\end{rubric}


\bcpfeast{24 October. St. Raphael}{St. Raphael}{24 October}
%\supplement{24 October}{St. Raphael}{}

\begin{rubric}
	The Hymns \& Versicles are of Michaelmas Day (p. \pageref{SaintMichael}), with the following Antiphons.
\end{rubric}

\properantiphon{Mag. \& Ben.}{I am Raphael {\dag} the Angel, that stand before the Holy One: wherefore bless ye God for ever, and confess ye all his great and wonderful doings.}

\properantiphon{Mag.}{O Prince most glorious, {\dag} Raphael the Archangel, keep us in remembrance: here and everywhere, always, entreat the Son of God for us.}


\subby{Sts. Chrysanthus \& Daria}
\feastday{{Sts. Chrysanthus \& Daria}}
\fancyhead[RE,LO]{25 October}
\begin{inhead}
	{Memorial\\
		25 October}
\end{inhead}

\begin{rubric}
	The propers are from the First Common of Many Martyrs (p. \pageref{CommonMartyrsI}), except for the following.
\end{rubric}

\collect
\lett{W}{e} beseech thee, O Lord, that the prayers of thy blessed Martyrs Chrysanthus and Daria may assist us: that, as we venerate them with our homage, so we may ever feel their loving help. Through.

\begin{rubric}
	The Epistle is the fourth additional Epistle of the Third Common of Many Martyrs (p. \pageref{2Corinthians64}).
\end{rubric}

\begin{rubric}
	The Gospel is the fourth additional Gospel of the Third Common of Many Martyrs (p. \pageref{Luke1147}).
\end{rubric}

\secret
\lett{W}{e} beseech thee, O Lord, that the sacrifice of thy people, which they solemnly offer on the birthday of thy holy Martyrs Chrysanthus and Daria, may be pleasing unto thee. Through.

\postcommunion
\lett{O}{Lord,} who hast fulfilled us with mystic gifts and joys: grant, we beseech thee; that by the intercession of thy holy Martyrs Chrysanthus and Daria, we may spiritually attain to those things which we temporally perform. Through.

%MANUAL ADJUSTMENT:
\clearpage
\subby{St. Evaristus}
\feastday{{St. Evaristus}}
\fancyhead[RE,LO]{26 October}
\begin{inhead}
	{Memorial\\
		26 October}
\end{inhead}

\begin{rubric}
	The propers are from the First Common of a Martyr Bishop (p. \pageref{CommonMartyrBishopI}).
\end{rubric}

\begin{rubric}
	If to-day be Saturday, the anticipated Vigil of Sts. Simon and Jude is kept, as is noted on the following day: but the \nth{2} Collect is of St. Evaristus, and the \nth{3} of St. Mary in Eastertide (p. \SPMaryEaster).
\end{rubric}

\subby{Vigil of Sts. Simon \& Jude}
\feastday{{Sts. Simon \& Jude Vigil}}
\fancyhead[RE,LO]{27 October}
\begin{inhead}
	{Vigil\\
		27 October}
\end{inhead}

\introit
\lett{L}{et} the sorrowful sighing of the prisoners, O Lord, come before thee: reward thou our neighbours sevenfold into their bosom: avenge thou the blood of thy saints that is shed. \textit{Ps.} O God, the heathen are come into thine inheritance: thy holy temple have they defiled: and made Jerusalem an heap of stones.

\begin{rubric}
	\emph{Gloria in excelsis} is not said.
\end{rubric}

\collect
\lett{G}{rant,} we beseech thee, almighty God: that as we prevent the glorious birthday of thine Apostles Simon and Jude; so they may prevent us in the sight of thy majesty, for the obtaining of thy blessings. Through.

\begin{rubric}
	\nth{2} Collect of St. Mary in Eastertide (p. \SPMaryEaster). \nth{3} against the Persecutors of the Church (\SPAgainst) or for the Chief Bishop (p. \SPChiefBishop).
\end{rubric}

\readingcitation{Epistle}{1 Corinthians 4:9}
\lett{B}{rethren:} We are made a spectacle unto the world, and to Angels, and to men. We are fools for Christ's sake, but ye are wise in Christ; we are weak, but ye are strong; ye are honourable, but we are despised. Even unto this present hour we both hunger, and thirst, and are naked, and are buffeted, and have no certain dwellingplace; And labour, working with our own hands: being reviled, we bless; being persecuted, we suffer it: Being defamed, we intreat: we are made as the filth of the world, and are the offscouring of all things unto this day. I write not these things to shame you, but as my beloved sons I warn you.

\gradual{Avenge thou, O Lord, the blood of thy Saints that is shed. ℣. The dead bodies of thy servants, O Lord, have they given to be meat unto the fowls of the air: and the flesh of thy Saints unto the beasts of the land.}

\readingcitation{Gospel}{John 15:1}
\lett{A}{t that time:} Jesus said unto his disciples: I am the true vine, and my Father is the husbandman. Every branch in me that beareth not fruit he taketh away: and every branch that beareth fruit, he purgeth it, that it may bring forth more fruit. Now ye are clean through the word which I have spoken unto you. Abide in me, and I in you. As the branch cannot bear fruit of itself, except it abide in the vine; no more can ye, except ye abide in me. I am the vine, ye are the branches: He that abideth in me, and I in him, the same bringeth forth much fruit: for without me ye can do nothing. If a man abide not in me, he is cast forth as a branch, and is withered; and men gather them, and cast them into the fire, and they are burned. If ye abide in me, and my words abide in you, ye shall ask what ye will, and it shall be done unto you.

\offertory{Let the Saints be joyful with glory: let them rejoice in their beds: let the praises of God be in their mouth.}

\secret
\lett{W}{e} humbly beseech thee, O Lord: that, although our consciences be burdened by reason of our sins: the gifts, wherewith we do prevent the festival of thy holy Apostles Simon and Jude, may through their merits be rendered acceptable in thy sight. Through.

\begin{rubric}
	\nth{2} Secret of St. Mary in Eastertide (p. \SPMaryEaster). \nth{3} against the Persecutors of the Church (\SPAgainst) or for the Chief Bishop (p. \SPChiefBishop).
\end{rubric}

\communion{The dead bodies of thy servants, O Lord, have they given to be meat unto the fowls of the air, and the flesh of thy Saints unto the beasts of the land: according to the greatness of thy power, preserve thou those that are appointed to die.}

\postcommunion
\lett{W}{e} who have received thy sacrament, humbly entreat thee, O Lord: that, by the intercession of thy blessed Apostles Simon and Jude, those things which we temporally perform we may receive unto life eternal. Through.

\begin{rubric}
	\nth{2} Postcommunion of St. Mary in Eastertide (p. \SPMaryEaster). \nth{3} against the Persecutors of the Church (\SPAgainst) or for the Chief Bishop (p. \SPChiefBishop).
\end{rubric}


\bcpfeast{28 October. Sts. Simon and Jude}{Sts. Simon \& Jude}{28 October}
%\supplement{28 October}{Sts. Simon and Jude}{}

\begin{secrubric}
	The Daily Office propers are of the Common of Apostles (p. \pageref{CommonApostles}).
\end{secrubric}


\bcpfeast{Christ the King Sunday}{Christ the King}{}
%\supplement{}{Christ the King}{Sunday}

\subbysub{I Evensong}\label{ChristTheKingEvensong}

\gregorioscore{resources/gabc/ProperTime/ChristTheKingEvensong.gabc}

℣. All power is given unto me.

℟. In heaven and in earth.

\properantiphon{Mag.}{The Lord God shall give unto him {\dag} the throne of his father David: and he shall reign over the house of Jacob for ever; and of his kingdom there shall be no end, alleluia.}

\subbysub{Mattins}

\invitatoryhymn\label{ChristTheKingInvitatory}

\gregorioscore{resources/gabc/ProperTime/ChristTheKingInvitatory.gabc}

\officehymn\label{ChristTheKingMattins}

\gregorioscore{resources/gabc/ProperTime/ChristTheKingMattins.gabc}

℣. His kingdom shall be increased.

℟. And of his peace there shall be no end.

\properantiphon{Ben.}{Unto God and his Father {\dag} hath he made us to be a kingdom, who is the first begotten of the dead, and the Prince of the kings of the earth, alleluia.}

\subbysub{II Evensong}

\begin{rubric}
	The Office Hymn is as in I Evensong, with the following Versicle \& Antiphon.
\end{rubric}

℣. His kingdom shall be increased.

℟. And of his peace there shall be no end.

\properantiphon{Mag.}{He hath on his vesture {\dag} and on his thigh a name written, King of kings, and Lord of lords: to him be glory and dominion for ever and ever.}


\subby{Vigil of All Hallows}
\feastday{{Hallowe'en}}
\fancyhead[RE,LO]{31 October}
\begin{inhead}
	{Vigil\\
		31 October}
\end{inhead}

\introit
\lett{T}{he} Saints and judges of the nations, and have dominion over the people: and the Lord their God shall reign for ever. \textit{Ps.} Rejoice in the Lord, O ye righteous: for it becometh well the just to be thankful.

\collect
\lett{O}{Lord,} our God, multiply upon us thy grace: and grant, that by our holy profession we may follow after the gladness of them, whose glorious festival we prevent. Through.

\begin{rubric}
	\nth{2} Collect of the Holy Ghost (p. \SPHolyGhost). \nth{3} Collect against the Persecutors of the Church (p. \SPAgainst) or for the Chief Bishop (p. \SPChiefBishop).
\end{rubric}

%RV:
\readingcitation{Epistle}{Revelation 5:6}
\lett{I}{n those days:} Lo, I, John, saw in the midst of the throne and of the four living creatures, and in the midst of the elders, a Lamb standing, as though it had been slain, having seven horns, and seven eyes, which are the seven Spirits of God, sent forth into all the earth. And he came, and he taketh it out of the right hand of him that sat on the throne. And when he had taken the book, the four living creatures and the four and twenty elders fell down before the Lamb, having each one a harp, and golden bowls full of incense, which are the prayers of the saints. And they sing a new song, saying, Worthy art thou to take the book, and to open the seals thereof: for thou wast slain, and didst purchase unto God with thy blood men of every tribe, and tongue, and people, and nation,  and madest them to be unto our God a kingdom and priests; and they reign upon the earth. And I saw, and I heard a voice of many angels round about the throne and the living creatures and the elders; and the number of them was ten thousand times ten thousand, and thousands of thousands; saying with a great voice, Worthy is the Lamb that hath been slain to receive the power, and riches, and wisdom, and might, and honour, and glory, and blessing. 

\gradual{Let the Saints be joyful with glory: let them rejoice in their beds. ℣. O sing unto the Lord a new song: let the congregation of saints praise him.}

\readingcitation{Gospel}{Luke 6:17}
\lett{A}{t that time:} Jesus came down from the mountain, and stood in the plain, and the company of his disciples, and a great multitude of people out of all Judaea and Jerusalem, and from the sea coast of Tyre and Sidon, which came to hear him, and to be healed of their diseases; And they that were vexed with unclean spirits: and they were healed. And the whole multitude sought to touch him: for there went virtue out of him, and healed them all. And he lifted up his eyes on his disciples, and said, Blessed be ye poor: for yours is the kingdom of God. Blessed are ye that hunger now: for ye shall be filled. Blessed are ye that weep now: for ye shall laugh. Blessed are ye, when men shall hate you, and when they shall separate you from their company, and shall reproach you, and cast out your name as evil, for the Son of man's sake. Rejoice ye in that day, and leap for joy: for, behold, your reward is great in heaven.

\offertory{Let the Saints be joyful with glory, let them rejoice in their beds: let the praises of God be in their mouth.}

\secret
\lett{W}{e} set upon thine altars, O Lord, the gifts which we offer: grant, we beseech thee; that they may be profitable for our salvation through the supplication of all thy Saints, whose coming festival we prevent. Through.

\begin{rubric}
	\nth{2} Secret of the Holy Ghost (p. \SPHolyGhost). \nth{3} Collect against the Persecutors of the Church (p. \SPAgainst) or for the Chief Bishop (p. \SPChiefBishop).
\end{rubric}

\communion{The souls of the righteous are in the hand of God, and there shall no torment of malice touch them: in the sight of the unwise they seemed to die, but they are in peace.}

\postcommunion
\lett{W}{e} beseech thee, O Lord: that, having accomplished with joy the mysteries of the coming feast; we may be aided by the prayers of them in whose memory they are shewn forth. Through.

\begin{rubric}
	\nth{2} Postcommunion of the Holy Ghost (p. \SPHolyGhost). \nth{3} Collect against the Persecutors of the Church (p. \SPAgainst) or for the Chief Bishop (p. \SPChiefBishop).
\end{rubric}


\bcpfeast{1 November. All Hallows}{All Hallows}{1 November}
%\supplement{1 November}{All Hallows}{}

\subbysub{I Evensong}\label{AllHallowsEvensong}

\gregorioscore{resources/gabc/ProperTime/AllHallowsEvensong.gabc}

℣. Be glad, O ye righteous, and rejoice in the Lord.

℟. And be joyful, all ye that are true of heart.

\properantiphon{Mag.}{O ye Angels and Archangels, {\dag} O ye Thrones and Dominions, Principalities and Powers, Virtues of the heavens, Cherubim and Seraphim, O ye Patriarchs and Prophets, ye holy Doctors of the law, O all ye Apostles, ye Martyrs of Christ, ye holy Confessors, ye Virgins of the Lord, ye holy Hermits, and all Saints: offer for us your intercessions.}

\subbysub{Mattins}

\begin{rubric}
	The Invitatory Hymn is as in I Evensong.
\end{rubric}

\officehymn\label{AllHallowsMattins}

\gregorioscore{resources/gabc/ProperTime/AllHallowsMattins.gabc}

℣. Let the Saints be joyful with glory.

℟. Let them rejoice in their beds.

\properantiphon{Ben.}{The glorious company {\dag} of the Apostles praise thee; the goodly fellowship of the Prophets praise thee; the white-robed army of Martyrs praise thee; with one heart and voice do all the elect acknowledge thee: O blessed Trinity, one only God.}

\subbysub{II Evensong}

\begin{rubric}
	The Office Hymn is as in I Evensong, with the following Versicle \& Antiphon.
\end{rubric}

℣. Let the Saints be joyful with glory.

℟. Let them rejoice in their beds.

\properantiphon{Mag.}{O how glorious {\dag} is the kingdom wherein all the Saints rejoice with Christ; arrayed in white robes, they follow the Lamb whithersoever he goeth.}

\begin{rubric}
	The Office is of All Hallows through its Octave, unless there occur a Sunday or Feast Day (Double or higher), any concurring Memorial being commemorated.
\end{rubric}


\bcpfeast{2 November. All Souls}{All Souls}{2 November}

\begin{rubric}
	The Office of the Day is the Office of the Dead.
\end{rubric}


\subby{St. Winifred}
\feastday{{St. Winifred}}
\fancyhead[RE,LO]{3 November}
\begin{inhead}
	{Memorial\\
		3 November}
\end{inhead}

\begin{rubric}
	The propers are from the First Common of a Virgin Martyr (p. \pageref{CommonVirginMartyrI}).
\end{rubric}


\subby{Sts. Vitalis \& Agricola}
\feastday{{Sts. Vitalis \& Agricola}}
\fancyhead[RE,LO]{4 November}
\begin{inhead}
	{Memorial\\
		4 November}
\end{inhead}

\begin{rubric}
	The propers are from the Second Common of Many Martyrs (p. \pageref{CommonMartyrsII}), except for the following.
\end{rubric}

\collect
\lett{G}{rant,} we beseech thee, almighty God: that we, who devoutly celebrate the festival of thy holy Martyrs Vitalis and Agricola, may be aided by their intercession with thee. Through.

\begin{rubric}
	The Epistle is from the Third Common of Many Martyrs (p. \pageref{CommonMartyrsIII}).
\end{rubric}

\begin{rubric}
	The Gospel is from the Second Common of a Martyr Bishop (p. \pageref{CommonMartyrBishopII}).
\end{rubric}

\secret
\lett{W}{e} beseech thee, O Lord, mercifully to accept these our oblations: that, at the intercession of thy holy Martyrs Vitalis and Agricola, we may be defended against all adversities. Through.

\postcommunion
\lett{M}{ay} this communion, O Lord, cleanse us from guilt: and at the intercession of thy holy Martyrs, Vitalis and Agricola, make us partakers of thy heavenly healing. Through.


\subby{St. Elizabeth}
\feastday{{St. Elizabeth}}
\fancyhead[RE,LO]{5 November}
\begin{inhead}
	{Memorial\\
		5 November}
\end{inhead}

\begin{rubric}
	The propers are from the Common of Neither Virgin nor Martyr (p. \pageref{CommonNeitherVirginMartyr}).
\end{rubric}


\subby{St. Willibrord}
\feastday{{St. Willibrord}}
\fancyhead[RE,LO]{7 November}
\begin{inhead}
	{Memorial\\
		7 November}
\end{inhead}

\begin{rubric}
	The propers are from the First Common of a Confessor Bishop (p. \pageref{CommonConfessorBishopI}).
\end{rubric}


\subby{Octave Day of All Hallows}
\feastday{{All Hallows Octave Day}}
\fancyhead[RE,LO]{8 November}
\begin{inhead}
	{Greater Double\\
		8 November}
\end{inhead}

\begin{rubric}
	The propers as on the Feast, with Commemoration of the Four Crowned Martyrs.
\end{rubric}


\subby{Four Crowned Martyrs}
\feastday{{Four Crowned Martyrs}}
\fancyhead[RE,LO]{8 November}
\begin{inhead}
	{Memorial\\
		8 November}
\end{inhead}

\begin{rubric}
	The propers are from the First Common of Many Martyrs (p. \pageref{CommonMartyrsI}), except for the following.
\end{rubric}

\begin{rubric}
	\textsc{Note,} The Epistle and Gospel are of All Hallows' Day.
\end{rubric}

\collect
\lett{G}{rant,} we beseech thee, almighty God: that, like as we have known thy glorious Martyrs to be constant in their confession, so we may perceive their loving intercession for us with thee. Through.

\secret
\lett{L}{et} thy plenteous benediction descend, O Lord: and both render our gifts acceptable unto thee at the intercession of thy holy Martyrs, and make them to us a sacrament of redemption. Through.

\postcommunion
\lett{O}{Lord,} who hast refreshed us with the gladness of thy heavenly sacraments: we humbly entreat thee; that we may be protected by the succour of them in whose triumphs we glory. Through.


\subby{Dedication of the Basilica of St. Saviour}
\feastday{{St. Saviour Basilica}}
\fancyhead[RE,LO]{9 November}
\begin{inhead}
	{Greater Double\\
		9 November}
\end{inhead}

\begin{rubric}
	The propers are from the Common of a Dedication of a Church (p. \pageref{CommonDedication}).
\end{rubric}

\begin{rubric}
	\textsc{Note,} The Creed is said, and Commemoration is made of St. Theodore Tyro.
\end{rubric}


\subby{St. Theodore Tyro}
\feastday{{St. Theodore Tyro}}
\fancyhead[RE,LO]{9 November}
\begin{inhead}
	{Memorial\\
		9 November}
\end{inhead}

\begin{rubric}
	The propers are from the Second Common of a Martyr not a Bishop (p. \pageref{CommonMartyrNotBishopII}), except for the following.
\end{rubric}

\collect
\lett{O}{God,} who dost encompass and protect us with the glorious confession of blessed Theodore, thy Martyr: grant us both to profit by his example, and to be supported by his intercession. Through.

\secret
\lett{A}{ccept,} O Lord, the prayers of thy faithful people, and the oblations of their sacrifices: and at the intercession of blessed Theodore, thy Martyr, may we through these offices of godly devotion enter into heavenly glory. Through.

\postcommunion
\lett{G}{rant} to us, we beseech thee, O Lord: at the intercession of blessed Theodore thy Martyr; that those things which we touch with our lips we may receive in purity of heart. Through.


\subby{Sts. Tryphon, Respicius, \& Nympha}
\feastday{{Sts. Tryphon \&c.}}
\fancyhead[RE,LO]{10 November}
\begin{inhead}
	{Memorial\\
		10 November}
\end{inhead}

\introit
\lett{T}{he} righteous cry, and the Lord heareth them: and delivereth them out of all their troubles. \textit{Ps.} I will alway give thanks unto the Lord: his praise shall ever be in my mouth.

\collect
\lett{G}{rant,} we beseech thee, O Lord, that we, ever keeping the feast of thy holy Martyrs Tryphon, Respicius, and Nympha: may through their intercession enjoy the benefit of thy protection. Through.

\begin{rubric}
	The Epistle is the third additional Epistle from the Third Common of Many Martyrs (p. \pageref{Romans818}).
\end{rubric}

\gradall{Avenge, O Lord, the blood of thy Saints that is shed. ℣. The dead bodies of thy servants have they given to be meat unto the fowls of the air: the flesh of thy Saints unto the beasts of the land.}{Alleluia, alleluia. ℣. Right dear in the sight of the Lord is the death of his Saints. Alleluia.}

\begin{rubric}
	The Gospel is from the Third Common of Many Martyrs (p. \pageref{CommonMartyrsIII}).
\end{rubric}

\offertory{Be glad, O ye righteous, and rejoice in the Lord: and be joyful, all ye that are true of heart.}

\secret
\lett{W}{e} beseech thee, O Lord, that the gifts which we offer unto thee of our bounden duty and service may be acceptable unto thee for honour of thy Just ones: and by thy mercy profitable unto us for our salvation. Through.

\communion{Whosoever shall do the will of my Father who is in heaven: the same is my brother, and sister, and mother, saith the Lord.}

\postcommunion
\lett{G}{rant} to us, we beseech thee, O Lord: at the intercession of thy holy Martyrs Tryphon, Respicius and Nympha; that those things which we touch with our lips we may receive in purity of heart. Through.


\subby{St. Martin of Tours}
\feastday{{St. Martin Tours}}
\fancyhead[RE,LO]{11 November}
\begin{inhead}
	{Greater Double\\
		11 November}
\end{inhead}

\begin{rubric}
	The Hymns and Versicles are from the First Common of a Confessor Bishop (p. \pageref{CommonConfessorBishopI}), with the following Antiphons.
\end{rubric}

\antiphon{Mag. \& Ben.}{O blessed Martin, {\dag} whose righteous soul possesseth Paradise; whereat the Angels triumph, the Archangels are jubilant: thee the choir of Saints proclaimeth, the throng of Virgins inviteth, saying, Abide with us for ever.}

\antiphon{Mag.}{O blessed Bishop, {\dag} who with all his heart loved Christ the King, and feared not the dominion of princes! O holy soul, which, although withheld from the persecutor's sword, lacked not the palm of martyrdom!}

\introit
\lett{T}{he} Lord hath established a covenant of peace with him, and made him a prince: that he should have the dignity of the priesthood for ever. \textit{Ps.} Lord, remember David: and all his trouble.

\collect
\lett{O}{God,} who seest that we stand not in our own strength: mercifully grant; that, by the intercession of blessed Martin, thy Confessor and Bishop, we may be defended against all adversities. Through.

\begin{rubric}
    Commemoration is made of St. Mennas, from the First Common of a Martyr not a Bishop (p. \pageref{CommonMartyrNotBishopI}).
\end{rubric}

\readingcitation{Epistle}{Ecclesiasticus 44:16}
\lett{E}{noch} pleased the Lord, and was translated, Being an example of repentance to all generations. Noah was found perfect and righteous; In the season of wrath he was taken in exchange for the world; Therefore was there left a remnant unto the earth, When the flood came. Everlasting covenants were made with him, That all flesh should no more be blotted out by a flood. Abraham was a great father of a multitude of nations; And there was none found like him in glory; Who kept the law of the Most High, And was taken into covenant with him: In his flesh he established the covenant; And when he was proved, he was found faithful. Therefore he assured him by an oath, That the nations should be blessed in his seed; That he would multiply him as the dust of the earth, And exalt his seed as the stars, And cause them to inherit from sea to sea, And from the River unto the utmost part of the earth. In Isaac also did he establish likewise, for Abraham his father’s sake, The blessing of all men, and the covenant: And he made it rest upon the head of Jacob; He acknowledged him in his blessings, And gave to him by inheritance, And divided his portions; Among twelve tribes did he part them. And he brought out of him a man of mercy, Which found favour in the sight of all flesh.

\gradall{Behold a great priest, who in his days pleased God. ℣. There was none found like unto him, who kept the law of the Most High.}{Alleluia, alleluia. ℣. The blessed man Saint Martin, Bishop of the city of Tours, entered into rest: whom Angels and Archangels, Thrones, Dominations, and Virtues received. Alleluia.}

\readingcitation{Gospel}{Luke 11:33}
\lett{A}{t that time:} Jesus said unto his disciples: No man, when he hath lighted a candle, putteth it in a secret place, neither under a bushel, but on a candlestick, that they which come in may see the light. The light of the body is the eye: therefore when thine eye is single, thy whole body also is full of light; but when thine eye is evil, thy body also is full of darkness. Take heed therefore that the light which is in thee be not darkness. If thy whole body therefore be full of light, having no part dark, the whole shall be full of light, as when the bright shining of a candle doth give thee light.

\offertory{My truth and my mercy shall be with him: and in my name shall his horn be exalted.}

\secret
\lett{S}{anctify,} we beseech thee, O Lord God, these gifts which we offer on the solemnity of thy holy Bishop Martin: that our life may ever thereby be directed both in prosperity and adversity. Through.

\begin{rubric}
    Commemoration is made of St. Mennas, from the First Common of a Martyr not a Bishop (p. \pageref{CommonMartyrNotBishopI}).
\end{rubric}

\communion{Blessed is the servant, whom the lord when he cometh shall find watching: verily I say unto you, that he shall make him ruler over all his goods.}

\postcommunion
\lett{G}{rant,} we beseech thee, O Lord our God: that through the intercession of them, on whose festival they are offered, these sacraments may avail for our salvation. Through.

\begin{rubric}
    Commemoration is made of St. Mennas, from the First Common of a Martyr not a Bishop (p. \pageref{CommonMartyrNotBishopI}).
\end{rubric}


\subby{St. Mennas}
\feastday{{St. Mennas}}
\fancyhead[RE,LO]{11 November}
\begin{inhead}
	{Memorial\\
		11 November}
\end{inhead}

\begin{rubric}
	The propers are from the Second Common of a Martyr not a Bishop (p. \pageref{CommonMartyrNotBishopII}), with the Prayers from the First Common of a Martyr not a Bishop (p. \pageref{CommonMartyrNotBishopI}).
\end{rubric}


\subby{Pope St. Martin I}
\feastday{{Pope St. Martin I}}
\fancyhead[RE,LO]{12 November}
\begin{inhead}
	{Memorial\\
		12 November}
\end{inhead}

\begin{rubric}
	The propers are from the Second Common of a Martyr Bishop (p. \pageref{CommonMartyrBishopII}), with the second additional Epistle from the Second Common of a Martyr not a Bishop (p. \pageref{1Peter413}) and the Gospel from the First Common of a Martyr Bishop (p. \pageref{CommonMartyrBishopI}).
\end{rubric}


\subby{St. Britius of Tours}
\feastday{{St. Britius Tours}}
\fancyhead[RE,LO]{13 November}
\begin{inhead}
	{Memorial\\
		13 November}
\end{inhead}

\begin{rubric}
	The propers are from the First Common of a Confessor Bishop (p. \pageref{CommonConfessorBishopI}).
\end{rubric}


\subby{St. Gregory Palamas}
\feastday{{St. Gregory Palamas}}
\fancyhead[RE,LO]{14 November}
\begin{inhead}
	{Memorial\\
		14 November}
\end{inhead}

\begin{rubric}
	The propers are from the First Common of a Bishop Confessor (p. \pageref{CommonConfessorBishopI}).
\end{rubric}


\subby{St. Gregory the Wonder-worker}
\feastday{{St. Gregory Wonder-worker}}
\fancyhead[RE,LO]{17 November}
\begin{inhead}
	{Memorial\\
		17 November}
\end{inhead}

\begin{rubric}
	The propers are from the First Common of a Bishop Confessor (p. \pageref{CommonConfessorBishopI}), except for the following.
\end{rubric}

\readingcitation{Gospel}{Mark 11:22}
\lett{A}{t that time:} Jesus answering his disciples, saith unto them: Have faith in God. For verily I say unto you, That whosoever shall say unto this mountain, Be thou removed, and be thou cast into the sea; and shall not doubt in his heart, but shall believe that those things which he saith shall come to pass; he shall have whatsoever he saith. Therefore I say unto you, What things soever ye desire, when ye pray, believe that ye receive them, and ye shall have them.


\subby{St. Hilda of Whitby}
\feastday{{St. Hilda Whitby}}
\fancyhead[RE,LO]{17 November}
\begin{inhead}
	{Memorial\\
		17 November}
\end{inhead}

\begin{rubric}
	The propers are from the First Common of a Virgin (p. \pageref{CommonVirginOnlyI}), except for the following.
\end{rubric}

\collect
\lett{G}{rant,} we beseech thee, almighty God: that we, who rejoice in the yearly solemnity of blessed Hilda thy Virgin, may by her intercession be led from our old nature to newness of life. Through.

\gradall{Thou hast loved justice and hated iniquity. ℣. Wherefore God, even thy God, hath anointed thee with the oil of gladness.}{Alleluia, alleluia. ℣. For I am jealous over you with godly jealousy: for I have espoused you to one husband, that I may present you as a chaste virgin to Christ. Alleluia.}

\secret
\lett{W}{e} bring unto thee, O Lord, our offerings for the sacrifice: that we, being reconciled to thy mercy through the merits of blessed Hilda thy Virgin, may be made thereby a living sacrifice acceptable unto thee. Through.

\postcommunion
\lett{O}{Lord,} through whom we have received the benediction of this heavenly banquet, we entreat thee: that as unto us it is thy sacrament, so at the intercession of blessed Hilda thy Virgin, it may effectually avail for our salvation. Through.


\subby{Dedication of the Basilica of the Holy Apostles Peter \& Paul}
\feastday{{Basilica of Peter \& Paul}}
\fancyhead[RE,LO]{18 November}
\begin{inhead}
	{Greater Double\\
		18 November}
\end{inhead}

\begin{rubric}
	The propers are from the Dedication of a Church (p. \pageref{CommonDedication}).
\end{rubric}

\begin{rubric}
	\textsc{Note,} The Creed is said.
\end{rubric}


\subby{Pope St. Pontianus}
\feastday{{Pope St. Pontianus}}
\fancyhead[RE,LO]{19 November}
\begin{inhead}
	{Memorial\\
		19 November}
\end{inhead}

\begin{rubric}
	The propers are from the First Common of a Martyr Bishop (p. \pageref{CommonMartyrBishopI}), except for the Gospel from the Second Common of a Martyr not a Bishop (p. \pageref{CommonMartyrNotBishopII}).
\end{rubric}


\subby{St. Edmund}
\feastday{{St. Edmund}}
\fancyhead[RE,LO]{20 November}
\begin{inhead}
	{Double\\
		20 November}
\end{inhead}

\begin{rubric}
	The propers are from the First Common of a Martyr not a Bishop (p. \pageref{CommonMartyrNotBishopI}), except for the following.
\end{rubric}

\collect
\lett{O}{God} of unspeakable mercy, who didst give grace to the most blessed King Edmund to overcome the enemy by dying for thy name: mercifully grant unto this thy family; that, by his intercession they may be found worthy to vanquish and destroy in themselves the temptations of their ancient foe. Through.

\secret
\lett{W}{e} beseech thee, almighty God, mercifully regard this sacrifice of our redemption: and at the intercession of blessed Edmund, thy King and Martyr, graciously accept it for this thy family. Through.

\postcommunion
\lett{L}{et} the homage of our service be pleasing unto thee, almighty God: that these holy things which we have received may, at the intercession of blessed Edmund, thy King and Martyr, be profitable unto us for the obtaining of the rewards of everlasting life. Through.


\subby{Presentation of the Blessed Virgin Mary}
\feastday{{Presentation B.V.M.}}
\fancyhead[RE,LO]{21 November}
\begin{inhead}
	{Greater Double\\
		21 November}
\end{inhead}

\begin{rubric}
	The Hymns and Versicles are from the Common of the Blessed Virgin Mary (p. \pageref{CommonBVM}), except for the Antiphon for I Evensong and II Evensong.
\end{rubric}

\antiphon{Mag.}{O ever blessed Mother of God, {\dag} Mary ever Virgin, temple of the Godhead, hallowed shrine of the Holy Spirit: thou only, above all others, wast acceptable to our Lord Jesus Christ, alleluia.}

\begin{rubric}
	The Mass propers are from the Common of the Blessed Virgin Mary (p. \pageref{CommonBVM}), except for the following.
\end{rubric}

\begin{rubric}
	\textsc{Note,} The Creed is said.
\end{rubric}

\collect
\lett{O}{God,} who didst will that blessed Mary ever Virgin, the dwelling-place of Holy Ghost, should this day be presented in the temple: grant, we beseech thee; that through her intercession we may be found worthy to be presented in the temple of thy glory. Through . . . in the unity of the same.

\secret
\lett{T}{hrough} thy mercy, O Lord, and the intercession of blessed Mary ever Virgin, may this oblation avail for our prosperity and peace, both now and for ever. Through.

\postcommunion
\lett{G}{rant,} we beseech thee, O Lord: that we, who have received these aids to our salvation, may at all times and in all places be protected through the advocacy of blessed Mary ever Virgin; in whose honour we have made these offerings to thy majesty. Through.

\begin{rubric}
	When this Feast Day is only commemorated, the following is said for the Last Gospel.
\end{rubric}

\readingcitation{Last Gospel.}{Luke 11:27}
\lett{A}{t that time:} As Jesus spake unto the multitudes, a certain woman of the company lifted up her voice, and said unto him: Blessed is the womb that bare thee, and the paps which thou hast sucked. But he said: Yea rather, blessed are they that hear the word of God, and keep it.


\subby{St. Gelasius}
\feastday{{St. Gelasius}}
\fancyhead[RE,LO]{21 November}
\begin{inhead}
	{Memorial\\
		21 November}
\end{inhead}

\begin{rubric}
	The propers are from the First Common of a Confessor Bishop (p. \pageref{CommonConfessorBishopI}).
\end{rubric}


\subby{St. Columbanus}
\feastday{{St. Columbanus}}
\fancyhead[RE,LO]{21 November}
\begin{inhead}
	{Memorial\\
		21 November}
\end{inhead}

\begin{rubric}
	The propers are from the Common of Abbots (p. \pageref{CommonAbbots}).
\end{rubric}


\subby{St. Cecilia}
\feastday{{St. Cecilia}}
\fancyhead[RE,LO]{22 November}
\begin{inhead}
	{Greater Double\\
		22 November}
\end{inhead}

\begin{rubric}
	The Hymns \& Versicles are from the First Common of a Virgin Martyr (p. \pageref{CommonVirginMartyrI}), with the following Antiphons.
\end{rubric}

\antiphon{Mag.}{It is secret, {\dag} O Valerian, what now I wish to tell thee: I have an Angel of God who loves me, and guards my body with exceeding zeal.}

\antiphon{Ben.}{When the dawn of day was breaking, {\dag} Cecilia the Martyr cried, saying: Forward, soldiers of Christ! Cast off the works of darkness, and put on the armour of light.}

\antiphon{Mag.}{This noble maiden {\dag} alway bare the glorious Gospel of Christ in her bosom: without ceasing, she continued in prayer and supplication, and celestial converse, night and day.}

\introit
\lett{I}{will} speak of thy testimonies, even before kings, and will not be ashamed: and my delight shall be in thy commandments, which I have loved exceedingly. \textit{Ps.} Blessed are those that are undefiled in the way: and walk in the law of the Lord.

\collect
\lett{O}{God,} who makest us glad with the yearly solemnity of blessed Cecilia thy Virgin and Martyr: grant, that we do venerate her in our service, so we may follow the example of her godly conversation. Through.

\readingcitation{Epistle}{Ecclesiasticus 51:9}
\lett{O}{Lord my God,} I lifted up my supplication from the earth, And prayed for deliverance from death. I called upon the Lord, the Father of my Lord, That he would not forsake me in the days of affliction, In the time when there was no help against the proud. I will praise thy name continually, And will sing praise with thanksgiving; And my supplication was heard: For thou savedst me from destruction, And deliveredst me from the evil time: Therefore will I give thanks and praise unto thee, And bless thy name, O Lord our God.

\gradall{Hearken, O daughter, and consider, incline thine ear: so shall the King have pleasure in thy beauty. ℣. In thy comeliness and in thy beauty, go forth, proceed prosperously, and reign.}{Alleluia, alleluia. ℣. The five wise virgins took oil in their vessels with their lamps: and at midnight there was a cry made: Behold, the bridegroom cometh: go ye out to meet Christ the Lord. Alleluia.}

\begin{rubric}
	The Gospel is from the First Common of a Virgin Martyr (p. \pageref{CommonVirginMartyrI}).
\end{rubric}

\offertory{The Virgins that be her fellow shall be brought unto the King: they that bear her company shall be brought unto thee, with joy and gladness: and shall enter into the palace of the Lord the King.}

\secret
\lett{W}{e} beseech thee, O Lord: that by the intercession of blessed Cecilia, thy Virgin and Martyr, this sacrifice of atonement and praise may ever render us worthy of thy loving-kindness. Through.

\communion{Let the proud be confounded, for they go wickedly about to destroy me: but I will be occupied in thy commandments and in thy statures, that I be not ashamed.}

\postcommunion
\lett{O}{Lord,} who hast satisfied thy family with sacred gifts: we beseech thee, that we may at all times be comforted by the intercession of her whose festival we celebrate. Through.


\subby{St. Clement}
\feastday{{St. Clement}}
\fancyhead[RE,LO]{23 November}
\begin{inhead}
	{Double\\
		23 November}
\end{inhead}

\begin{rubric}
	The Hymns are from the First Common of Many Martyrs (p. \pageref{CommonMartyrsI}), with the following Versicles \& Antiphons.
\end{rubric}

℣. Thou hast crowned him with glory and honour, O Lord.

℟. And madest him to have dominion of the works of thy hands.

\antiphon{Mag.}{Let us all pray {\dag} to our Lord Jesus Christ, that he would open a fountain for his Confessors.}\\

℣. Let the Saints be joyful with glory.

℟. Let them rejoice in their beds.

\antiphon{Ben.}{When he had started going to the sea, {\dag} the people cried out with loud voices, O Lord Jesus Christ, save him: and Clement responded, weeping, Heavenly Father, receive my spirit.}\\

℣. Let the Saints be joyful with glory.

℟. Let them rejoice in their beds.

\antiphon{Mag.}{In the heavenly kingdom {\dag} rejoice the souls of the Blessed, who followed the footsteps of Christ their Master: and since for love of him they poured forth their life-blood, therefore with Christ do they exult for ever.}

\introit
\lett{T}{he} Lord saith: My words which I have put in thy mouth, shall not depart out of thy mouth: and thy gifts shall be accepted upon mine altar. \textit{Ps.} Blessed is the man that feareth the Lord: he hath great delight in his commandments.

\collect
\lett{O}{God,} who makest us glad with the yearly solemnity of blessed Clement thy Martyr and Bishop: mercifully grant; that as we now celebrate his birthday, so we may imitate his constancy in suffering. Through.

\begin{rubric}
    Commemoration is made of St. Felicitas (p. \pageref{FelicityCollect}).
\end{rubric}

\readingcitation{Epistle}{Philippians 3:17}
\lett{B}{rethren:} Be followers together of me, and mark them which walk so as ye have us for an ensample. (For many walk, of whom I have told you often, and now tell you even weeping, that they are the enemies of the cross of Christ: Whose end is destruction, whose God is their belly, and whose glory is in their shame, who mind earthly things.) For our conversation is in heaven; from whence also we look for the Saviour, the Lord Jesus Christ: Who shall change our vile body, that it may be fashioned like unto his glorious body, according to the working whereby he is able even to subdue all things unto himself. Therefore, my brethren dearly beloved and longed for, my joy and crown, so stand fast in the Lord, my dearly beloved. I beseech Euodias, and beseech Syntyche, that they be of the same mind in the Lord. And I intreat thee also, true yokefellow, help those women which laboured with me in the gospel, with Clement also, and with other my fellowlabourers, whose names are in the book of life.

\gradall{The Lord sware, and will not repent: Thou art a priest for ever, after the order of Melchisedech. ℣. The Lord said unto my Lord: Sit thou on my right hand.}{Alleluia, alleluia. ℣. This is a priest whom the Lord hath crowned. Alleluia.}

\begin{rubric}
	The Gospel is of the Second Common of a Confessor Bishop (\pageref{CommonConfessorBishopII}).
\end{rubric}

\offertory{My truth and my mercy shall be with him: and in my name shall his horn be exalted.}

\secret
\lett{S}{anctify,} O Lord, the gifts which we offer unto thee: and, at the intercession of blessed Clement, thy Martyr and Bishop, cleanse us thereby from the defilement of our sins. Through.

\begin{rubric}
    Commemoration is made of St. Felicitas (p. \pageref{FelicitySecret}).
\end{rubric}

\communion{Blessed is the servant, whom the Lord when he cometh shall find watching: verily I say unto you, that he shall make him ruler over all his goods.}

\postcommunion
\lett{O}{Lord} our God, who hast fulfilled us with the partaking of the sacred Body, and the precious Blood, we beseech thee: that those things which we perform with godly devotion, we may at the intercession of blessed Clement, thy Martyr and Bishop, attain in the assurance of our redemption. Through.

\begin{rubric}
    Commemoration is made of St. Felicitas (p. \pageref{FelicityPostcommunion}).
\end{rubric}

%MANUAL ADJUSTMENT:
\clearpage
\subby{St. Felicitas}
\feastday{{St. Felicitas}}
\fancyhead[RE,LO]{23 November}
\begin{inhead}
    {Memorial\\
23 November}
\end{inhead}

\begin{rubric}
	The propers are from the Common of a Martyr not a Virgin (p. \pageref{CommonMartyrNotVirgin}), except for the following.
\end{rubric}

\collect\label{FelicityCollect}
\lett{G}{rant,} we beseech thee, almighty God: that we, celebrating the festival of blessed Felicitas, thy Martyr, may be protected by her merits and prayers. Through.

\secret\label{FelicitySecret}
\lett{G}{raciously} hearken, O Lord, to the prayers of thy people: and make us to rejoice in the intercession of her whose festival thou dost grant unto us to celebrate. Through.

\postcommunion\label{FelicityPostcommunion}
\lett{W}{e} humbly beseech thee, almighty God: that by the intercession of thy Saints thou wouldest both multiply in us thy gifts, and likewise dispose our times according to thy will. Through.


\subby{St. Chrysogonus}
\feastday{{St. Chrysogonus}}
\fancyhead[RE,LO]{24 November}
\begin{inhead}
    {Memorial\\
24 November}
\end{inhead}

\begin{rubric}
	The propers are from the First Common of a Martyr not a Bishop (p. \pageref{CommonMartyrNotBishopI}, except for the following.
\end{rubric}

\collect
\lett{A}{ssist} us, O Lord, in these our supplications: that we, who acknowledge ourselves to be guilty by reason of our iniquity, may be delivered by the intercession of thy blessed Martyr Chrysogonus. Through.

\secret
\lett{W}{e} beseech thee, O Lord, mercifully to accept these our oblations: that at the intercession of thy blessed Martyr Chrysogonus, we may be defended against all perils. Through.

\postcommunion
\lett{O}{Lord,} let the partaking of thy sacrament both cleanse us from our secret faults, and deliver us from the snares of our enemies. Through.


\subby{St. Catherine of Alexandria}
\feastday{{St. Catherine Alexandria}}
\fancyhead[RE,LO]{25 November}
\begin{inhead}
    {Double\\
25 November}
\end{inhead}

\begin{rubric}
	The propers are from the First Common of a Virgin Martyr (p. \pageref{CommonVirginMartyrI}), except for the following.
\end{rubric}

\collect
\lett{O}{God,} who didst give the law to Moses on the height of Mount Sinai, and in the same place didst through thy holy Angels wondrously bestow the body of blessed Catherine, thy Virgin and Martyr: grant, we beseech thee; that by her merits and intercession we may be enabled to attain unto that mount which is Christ. Who liveth.

\secret
\lett{R}{eceive,} O Lord, the gifts which we offer on the solemnity of blessed Catherine, thy Virgin and Martyr: through whose advocacy we trust to be delivered. Through.

\postcommunion
\lett{M}{ay} the mysteries which we have received be for our succour, O Lord: and at the intercession of blessed Catherine, thy Virgin and Martyr, cause us to rejoice in thy continual protection. Through.


\subby{St. Peter of Alexandria}
\feastday{{St. Peter Alexandria}}
\fancyhead[RE,LO]{26 November}
\begin{inhead}
    {Memorial\\
26 November}
\end{inhead}

\begin{rubric}
	The propers are from the First Common of a Martyr Bishop (p. \pageref{CommonMartyrBishopI}).
\end{rubric}






